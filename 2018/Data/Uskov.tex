\begin{center}{ \bf  АСИМПТОТИЧЕСКОЕ РЕШЕНИЕ ЗАДАЧИ КОШИ ДЛЯ УРАВНЕНИЯ ПЕРВОГО ПОРЯДКА С МАЛЫМ ПАРАМЕТРОМ И НЕКОТОРЫМ ПАРАМЕТРОМ В ПРАВОЙ ЧАСТИ}\\
{\it В.И. Усков } \\
(Воронеж; {\it vum1@yandex.ru} )
\end{center}
\addcontentsline{toc}{section}{Усков В.И.\dotfill}

Рассматривается следующая задача:
\[\varepsilon\frac{dx}{dt}=(A+c\cdot I)x(t,\varepsilon)+h(t),
\eqno{(1)}\]
\[x(0,\varepsilon)=x^0(\varepsilon)\in E,
\eqno{(2)}\]
где $A$ --- замкнутый линейный оператор, $E$ --- банахово пространство, $\overline{\operatorname{dom}}\,A=E$; $A$ --- фредгольмов оператор с нулевым индексом (далее, Ф-оператор); $h(t)$ --- заданная достаточно гладкая функция со значениями в $E$; $x^0(\varepsilon)$ --- голоморфная в окрестности точки $\varepsilon=0$ функция; $c\in\mathbb{C}$; $t\in[0,T]$; $\varepsilon\in(0,\varepsilon_0]$.

Под \textit{решением} задачи (1), (2) подразумевается дифференцируемая функция $x(t,\varepsilon)$, удовлетворяющая (1), (2) при каждых $t$, $\varepsilon$.

Исследуется поведение решения при $\varepsilon\to0$; устанавливается возможность наблюдения явления погранслоя [1]; изучается влияние параметра $c$ на качественные свойства решения; строится асимптотическое разложение решения по степеням параметра $\varepsilon$; доказывается асимптотичность этого разложения.

Определение Ф-свойства некоторого оператора $A$ и решение линейного уравнения с таким оператором, использущиеся при решении задачи, приведены в [2].

Приложением поставленной задачи может быть начально-краевая задача для уравнения
\[\varepsilon^2\frac{\partial^2 u}{\partial x^2}-2\varepsilon\frac{\partial^2 u}{\partial x\partial t}+\frac{\partial^2 u}{\partial t^2}-2c\varepsilon\frac{\partial u}{\partial t}+2c\frac{\partial u}{\partial x}+(\gamma^2+c^2)u(x,t,\varepsilon)=\varphi(x,t),\]
встречающаяся в задачах математической физики, связанных с потенциальными барьерами квантовой физики, в теории дифракции, в теории тонких упругих оболочек, в задачах на собственные функции типа волн Релея и т.д. [3].

В работе [2] было построено асимптотическое разложение решения задачи (1), (2) в случае Ф-оператора, имеющего произвольную размерность ядра и жордановы цепочки разной длины. Поскольку оператор $A+c\cdot I$ вообще говоря не является Ф-оператором, результаты этой работы перенести на эту задачу не представляется возможным. О поставленной задаче делался доклад на конференции [4]; приводился в качестве примера матрично-дифференциальный оператор (о нем см. [5]).

В настоящей работе методом Васильевой-Вишика-Люс\-тер\-ни\-ка, разработанным в [6, 7], строится асимптотическое разложение решения по степеням малого параметра $\varepsilon$ в виде
$$ x(t,\varepsilon)=\bar{x}_m(t,\varepsilon)+\bar{v}_m(t,\varepsilon)+R_m(t,\varepsilon),$$ 
$$ \bar{x}_m(t,\varepsilon)=\sum\limits_{k}{\varepsilon^kx_k(t)}, \,\, \bar{v}_m(t,\varepsilon)=\sum\limits_{k}{\varepsilon^kv_k(\tau)}, \,\, \tau=\tau(t,\varepsilon),\eqno{(4)}$$
где $\bar{x}_m(t,\varepsilon)$ --- регулярная часть, $\bar{v}_m(t,\varepsilon)$ --- погранслойная часть, $R_m(t,\varepsilon)$ --- остаточный член разложения.

\noindent Доказывается асимптотичность этого разложения.

%%%%  ОФОРМЛЕНИЕ СПИСКА ЛИТЕРАТУРЫ %%%
\smallskip \centerline{\bf Литература}\nopagebreak

1. {\it Зубова С.П.} О роли возмущений в задаче Коши для уравнения с фредгольмовым оператором при производной / С. П. Зубова // Доклады РАН. --- 2014. --- Т. 454, № 4. --- C. 383--386.

2. {\it Зубова С.П.} Асимптотическое решение сингулярно возмущенной задачи Коши для уравнения первого порядка в банаховом пространстве / С.П. Зубова, В.И. Усков // Вестник Воронежского госуниверситета. Серия: Физика. Математика. --- 2016. --- № 3. --- С. 147--155.

3. {\it Вишик М.И.} Регулярное вырождение и пограничный слой для линейных дифференциальных уравнений с малым параметром / М.И. Вишик, Л.А. Люстерник // Успехи мат. наук. -- 1957. -- Т. 12, вып. 5 (77). --- С. 3--122.

4. {\it Усков В.И.} Задача Коши для уравнения первого порядка с малым параметром / В.И. Усков // Современные проблемы анализа динамических систем, приложения в технике и технологиях: материалы II Международной открытой конференции. --- Воронеж. --- 2017.

5. {\it Зубова С.П.} О свойствах вырожденности некоторого матричного дифференциального оператора. --- Естественные и математические науки: научные приоритеты учёных. Сборник научных трудов по итогам международной научно-практической конференции / С.П. Зубова, В.И. Усков // Пермь. --- 2016. --- Вып. 1. --- С. 9--12.

6. {\it Васильева А.Б.} Асимптотические разложения решений сингулярно возмущенных уравнений / А.Б. Васильева, В.Ф. Бутузов. --- М. Наука. --- 1973. --- 272 с.

7. {\it Треногин В.А.} Развитие и приложения асимптотического метода Люстерника-Вишика / В.А. Треногин // Успехи мат. наук. --- 1970, июль-август. --- Т. 25, вып. 4 (154). --- С. 123--156.
