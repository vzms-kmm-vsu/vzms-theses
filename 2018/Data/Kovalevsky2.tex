\vzmstitle[
	\footnote{
		Работа выполнена при финансовой поддержке Министерства образования и науки
		РФ (проект 14.Z50.31.0037).
	}
]{О разрешимости начально-краевой задачи для вырождающегося
параболического уравнения высокого порядка}

\vzmsauthor{Ковалевский}{Р.\,А.}

\vzmsinfo{Воронеж}

\vzmscaption

Рассмотрим функцию $\alpha (t),\,\,t \in R_ + ^1 $, для которой $\alpha ( +
0) = {\alpha }'( + 0) = 0$, $\alpha \mbox{(}t\mbox{) > 0}$при$t > 0$,
$\alpha \mbox{(}t\mbox{) = const}$ для $t \geqslant d$ при некотором $d\mbox{ >
0}$. Рассмотрим интегральное преобразование
\[
F_\alpha [u(t)](\eta ) = \int\limits_0^{ + \infty } {u(t)\exp (i\eta }
\int\limits_t^d {\frac{d\rho }{\alpha (\rho )}} )\frac{dt}{\sqrt {\alpha
(t)} },
\]
определённое, например, на функциях
$u(t) \in C_0^\infty (R_ + ^1 )$.
Свойства этого преобразования доказаны в [1]--[5].

В $R_{ + + }^{n + 1} = \{(x,t,y):x \in R^{n - 1},\,\,0 < t < + \infty ,\,\,0
< y < + \infty \}$ рассматривается линейное дифференциальное уравнение вида
\begin{equation}
\label{eq4700}
A(y,\partial _t ,D_x ,D_{\alpha ,y} ,\partial _y )v(t,x,y) = F(t,x,y),
\end{equation}
где
\begin{equation}
\label{eq4701}
A(y,\partial _t ,D_x ,D_{\alpha ,y} ,\partial _y )v = \sum\limits_{\left|
\tau \right| + j_1 + qj_2 + rj_3 \leqslant 2m} {a_{\tau j_1 j_2 j_3 } (y)D_x^\tau
D_{\alpha ,y}^{j_1 } } \partial _y^{j_2 } \partial _t^{j_3 } v.
\end{equation}



Здесь $m,k,l$~--- натуральные числа, $q = \frac{2m}{k} > 1,\,\,r = \frac{2m}{l} >
1,\,\,a_{\tau j_1 j_2 j_3 } (y)$~--- некоторые ограниченные на $\bar {R}_ + ^1
$ функции,
\linebreak
$a_{00k0} (y) \ne 0$, $a_{000l} (y) \ne 0_{ }$при всех$y \in
\bar {R}_ + ^1 .$ Без ограничения общности будем считать, что $a_{00k0} (y)
= 1$ при всех $t \in \bar {R}_ + ^1 .$

На границе $t = 0_{ }$множества $R_{ + + }^{n + 1} $ задаются граничные
условия вида$B_j (\partial _t ,D_x ,\partial _y )\left. v \right|_{t = 0} =
\sum\limits_{\left| \tau \right| + rj_3 + qj_2 \leqslant m_j }^ {b_{\tau j_2 j_3 }
\partial _t^{j_3 } D_x^\tau \partial _y^{j_2 } } \left. v \right|_{y = 0} =
G_j (t,x),j = 1,2,...,\mu .$ (3)

На границе $y = 0$ множества $R_{ + + }^{n + 1} $ задаются начальные условия
вида
\begin{equation}
\label{eq4702}
\partial _t^\mu \left. v \right|_{t = 0} = \Phi _\mu (x,y),\,\,\mu =
1,2,...,l - 1.
\end{equation}



Рассмотрим абстрактную функцию $u(t)$ со значениями в ги\-ль\-бе\-р\-то\-вом
пространстве $Y_1 $, такую, что $u(t) = 0$ при $t < 0$. Предположим, что
существует преобразование Фурье функции $e^{ - \gamma t}u(t)$ ($\gamma \geqslant
0)$, принадлежащее гильбертову пространству $Y_0 \supset Y_1 $. Будем
говорить, что функция $V(p)$ принадлежит пространству $W_{2,\gamma }^a (Y_1
,Y_0 ),\,\,a \geqslant 0,\,\,\gamma \geqslant 0$ если конечна норма
\begin{multline*}
\left\| u \right\|_{W_{2,\gamma }^a (Y_1 ,Y_0 )} =
\\=
\{\int\limits_{R^1}
{\left\| {e^{ - \gamma t}u(t)} \right\|_{Y_1 }^2 dy} + \int\limits_{R^1}
{\left| {\gamma + i\tau } \right|^{2a}\left\| {F_{y \to \tau } [e^{ - \gamma
t}u(t)]} \right\|_{Y_0 }^2 d\tau \}^{\frac{1}{2}}} .
\end{multline*}



Обозначим через $E_{2,\gamma }^a (Y_1 ,Y_0 )_{ }$множество функций $V(p)$,
где $p$ - комплексное число, для которого $Rep > \gamma $, таких что функции
$V(p)$ принимают значения в гильбертовом пространстве $Y_1 \subset Y_0 $,
являются аналитическими функциями в полуплоскости $Rep > \gamma $ и для них
конечна норма
\[
\left\| u \right\|_{E_{2,\gamma }^a (Y_1 ,Y_0 )} = \mathop {\sup
}\limits_{\rho > \gamma } \{\int\limits_{Rep = \rho } {(\left\| {V(p)}
\right\|_{Y_1 }^2 + \left| p \right|^{2a}\left\| {V(p)} \right\|_{Y_0 }^2
)dp} \}^{\frac{1}{2}}.
\]

Преобразование Лапласа устанавливает взаимно однозначное и взаимно
непрерывное соответствие между пространствами $W_{2,\gamma }^a (Y_1 ,Y_0 )$
и $E_{2,\gamma }^a (Y_1 ,Y_0 )$ .

Рассмотрим также однородные условия (\ref{eq4702})
\begin{equation}
\label{eq4703}
\partial _t^\mu \left. v \right|_{t = 0} = 0,\,\,\mu = 1,2,...,l - 1.
\end{equation}



\textbf{Условие 1.} Уравнение
\begin{equation}
\label{eq4704}
\sum\limits_{\left| \tau \right| + j_1 + qj_2 + rj_3 = 2m}^ {a_{\tau j_1 j_2
j_3 } (y)\xi ^\tau \eta ^{j_1 }} z^{j_2 }p^{j_3 } = 0.
\end{equation}
не имеет $z$ -- корней, лежащих на мнимой оси при всех $y \geqslant 0\,\,(\xi
,\eta ) \in R^n\,\,$, $p \in Q = \{p \in C,\,\,\left| {\arg p} \right| <
\frac{\pi }{2},\,\,\left| p \right| > 0\}$, $\left| p \right| + \left| \eta
\right| + \left| \xi \right| > 0$.

Пусть $z_1 (p,y,\xi ,\eta ),...,z_{r_1 } (p,y,\xi ,\eta )\,\,\,(1 \leqslant r_1
\leqslant k)$ - корни, лежащие в левой полуплоскости, а $z_{r_1 + 1} (p,y,\xi
,\eta ),...,z_k (p,y,\xi ,\eta )$ лежат в правой полуплоскости.

\textbf{Условие 2.} Функции $z_j (p,y,\xi ,\eta )$, $j = 1,\,2,...,\,k,$ при
всех $\xi \in R^{n - 1}$ являются бесконечно дифференцируемыми функциями по
переменным $y \in \Omega \subset \bar {R}_ + ^1 $ и $\eta \in R^1$. Причём,
при всех $p \in Q = \{p \in C,\,\,\left| {\arg p} \right| < \frac{\pi
}{2},\,\,\left| p \right| > 0\}$, $j_1 = 0,\,\,1,\,\,2,\,...,l =
0,\,\,1,\,\,2,\,...,\xi \in R^{n - 1}$, $y \in \Omega \subset \bar {R}_ +
^1 $, $\eta \in R^1$ справедливы оценки
\[
\left| {(\alpha (y)\partial _y )^{j_1 }\partial _\eta ^{j_2 } z_j (y,\xi
,\eta )} \right| \leqslant c_{j_1 ,l} (\left| p \right| + \left| \xi \right| +
\left| \eta \right|)^{q - j_2 },\,\,\,\,\,\left| p \right| + \left| \xi
\right| + \left| \eta \right| > 0,
\]
с константами $c_{j_1 ,l} > 0$, не зависящими от$p,y,\,\,\xi ,\,\,\eta .$

Из условия 4 следует, что при всех $p \in Q$, $\xi \in R^{n - 1}$, $y \in
\Omega \subset \bar {R}_ + ^1 $, $\eta \in R^1$ справедливы оценки
\begin{equation}
\label{eq4705}
Rez_j (p,y,\xi ,\eta ) \leqslant - c_1 (\left| p \right| + \left| \xi \right| +
\,\left| \eta \right|)^q,\,\,\,j = 1,...,r_1 ;
\end{equation}
\begin{equation}
\label{eq4706}
Rez_j (p,y,\xi ,\eta ) \geqslant c_2 (\left| p \right| + \left| \xi \right| +
\,\left| \eta \right|)^q,\,\,\,j = r_1 + 1,...,k,
\end{equation}
с некоторыми константами $c_1 >0$
и $c_2 > 0$, не зависящими
от $p,\,\,y,\,\,\xi ,\,\,\eta $.

\textbf{Условие 3.} Число граничных условий (22) равно числу $z$ - корней
уравнения (23), лежащих в левой полуплоскости, и при всех $\xi \in R^{n -
1},\,\,\,\,\left| \xi \right| > 0$ многочлены $$B_j^0 (\xi ,z) =
\sum\limits_{\left| \tau \right| + qj_2 + rj_3 = m_j } {b_{\tau j_2 j_3 }
p^{j_3 }} \xi ^\tau z^{j_2 }\,$$ линейно независимы по модулю многочлена
$$P(\xi ,z) = \prod\limits_{j_1 = 1}^{r_1 } {(z - z_{j_1 } (0,\xi ,0))} .$$

Доказаны следующие утверждения.

\textbf{Теорема 1. }
Пусть $s \geqslant \max \{2m,\,\,\mathop {\max }\limits_{1 \leqslant
j \leqslant r_1 } m_j + \frac{1}{2}\max \{q,r\},\,\,s$- кратно $2m$. Пусть
выполнены условия 4--6. Тогда существует такое число $\gamma _0 > 0$ , что
при всех $\gamma > \gamma _0 $ и любых $F(t,x,y) \in H_{r,\gamma ,\alpha
.q}^{s - 2m} (R_{ + + }^{n + 1} ) \equiv W_{2,\gamma }^{\frac{s - 2m}{r}}
(H_{s - 2m,\alpha ,q} (R_ + ^n ),L_2 (R_ + ^n ))$ и
\linebreak
$G_j (x,y) \in
H_{r,\gamma }^{\sigma _j } (R_ + ^n ) \equiv W_{2,\gamma }^{\frac{s -
2m}{r}} (H_{\sigma _j } (R^{n - 1}),L_2 (R^{n - 1}),\,\,\sigma _j = s - m_j
- \frac{1}{2}q,\,\,j = 1,2,...,r_1 $ задача (\ref{eq4700}), (3), (\ref{eq4703}) имеет единственное
решение, принадлежащее пространству $H_{r,\gamma ,\alpha .q}^s $, причём
справедлива априорная оценка
\[
\left\| v \right\|_{H_{r,\gamma ,\alpha ,q}^s } \leqslant C(\left\| F
\right\|_{H_{r,\gamma ,\alpha ,q}^{s - 2m} } + \sum\limits_{j = 1}^{r_1 }
{\left\langle {\left\langle {G_j } \right\rangle } \right\rangle }
_{H_{r,\gamma ,}^{\sigma _j } } ) \quad .
\]






Рассмотрим теперь задачу (\ref{eq4700}), (3), (\ref{eq4702}). Введём наряду с пространствами
$W_{2,\gamma }^a (Y_1 ,Y_0 )$ пространства $\hat {W}_{2,\gamma }^a (Y_1 ,Y_0
)$ абстрактных функций $t \to u(t),t \in R_ + ^1 $ со значениями в$Y_1
\subset Y_0 $ с конечной нормой
\[
\left\| u \right\|_{\hat {W}_{2,\gamma }^a (Y_1 ,Y_0 )} = \{\int\limits_{R_
+ ^1 } {\left\| {e^{ - \gamma t}u(t)} \right\|_{Y_1 }^2 dy} + \sum\limits_{j
= 0}^a {\int\limits_{R_ + ^1 } {\left\| {\partial _t^j (e^{ - \gamma
t}u(t)]} \right\|_{Y_0 }^2 dt\}^{\frac{1}{2}}} } .
\]

Здесь $a \geqslant 0$ - целое число.

В дальнейшем используются пространства $\hat {W}_{2,\gamma }^a (Y_1 ,Y_0 )$
при следующем конкретном выборе пространств $Y_1 ,\,\,\,Y_0 $ :
\begin{gather*}
\hat {H}_{r,\gamma ,\alpha .q}^s (R_{ + + }^{n + 1} ) \equiv \hat
{W}_{2,\gamma }^{\frac{s}{r}} (H_{s,\alpha ,q} (R_ + ^n ),L_2 (R_ + ^n )),
\\
\hat {H}_{r,\gamma }^\beta (R_ + ^n ) \equiv \hat {W}_{2,\gamma
}^{\frac{\beta }{r}} (H_{\sigma _j } (R^{n - 1}),L_2 (R^{n - 1}).
\end{gather*}



Задача (\ref{eq4700}), (3), (\ref{eq4702}) сводится к задаче (\ref{eq4700}), (3), (\ref{eq4703}), если выполнено
следующее условие согласования.

\textbf{Условие 4. }Для набора функций $F(t,x,y) \in \hat {H}_{r,\gamma
,\alpha .q}^{s - 2m} (R_{ + + }^{n + 1} )$, $G_j (x,y) \in \hat
{H}_{r,\gamma }^{\sigma _j } (R_ + ^n ),\,\,\,\,\sigma _j = s - m_j -
\frac{1}{2}q,\,\,j = 1,2,...,r_1 ;\,\,\,\Phi _\mu (x,t) \in \hat {H}_{\alpha
,q}^{\beta _\mu } (R_ + ^n )$
$\beta _\mu = s - \mu r - \frac{1}{2}r,\,\,\mu = 1,2,...,l -1$
существует такая функция
$v_0 (y,x,t) \in \hat {H}_{r,\gamma ,\alpha ,q}^s (R_{ + + }^{n + 1}
)$,
что
выполняется условие $\partial _t^\mu \left. v \right|_{t = 0} = \Phi _\mu
(x,y),\,\,\mu = 1,2,...,l - 1$ ;
после продолжения функций $F - Av_0 ,\,\,G_j - B_j v_0 \vert _{y = 0}
\,\,\,j = 1,2,...,r_1 $ нулём при $y < 0$ справедливы включения $F - Av_0 \in
\hat {H}_{r,\gamma ,\alpha ,q}^{s - 2m} ,\,G_j - B_j v_0 \vert _{y = 0} \,
\in \,\hat {H}_{r,\gamma }^{\sigma _j } \,,\,\,j = 1,2,...,r_1 $;
существует постоянная $c > 0$ , что справедлива оценка
\[
\left\| {v_0 } \right\|_{\hat {H}_{r,\gamma ,\alpha ,q}^s } \leqslant
c\sum\limits_{\mu = 0}^{l - 1} {\left\| {\Phi _\mu } \right\|} _{\hat
{H}_{r,\gamma }^{\beta _\mu } } \quad .
\]







Справедлива следующая теорема.

\textbf{Теорема 2. }Пусть выполнено условие 7 и условия теоремы 1. Тогда
существует такое $\gamma _0 > 0$, что при всех $\gamma > \gamma _0 $ и любых
$F(t,x,y) \in \hat {H}_{r,\gamma ,\alpha .q}^{s - 2m} (R_{ + + }^{n + 1} )$,
$G_j (x,y) \in \hat {H}_{r,\gamma }^{\sigma _j } (R_ + ^n ),\,\,\sigma _j =
s - m_j - \frac{1}{2}q,\,\,j = 1,2,...,r_1 $,
\[
\Phi _\mu (x,t) \in \hat {H}_{\alpha ,q}^{\beta _\mu } (R_ + ^n ),\,\,\beta
_\mu = s - \mu r - \frac{1}{2}r,\,\,\mu = 0,1,...,l - 1
\]
задача (\ref{eq4700}), (3), (\ref{eq4702}) имеет единственное решение, принадлежащее пространству
$\hat {H}_{r,\gamma ,\alpha .q}^s (R_{ + + }^{n + 1} )$, причём справедлива
априорная оценка
\[
\left\| v \right\|_{\hat {H}_{r,\gamma ,\alpha ,q}^s } \leqslant C(\left\| F
\right\|_{\hat {H}_{r,\gamma ,\alpha ,q}^{s - 2m} } + \sum\limits_{j =
1}^{r_1 } {\left\langle {\left\langle {G_j } \right\rangle } \right\rangle }
_{\hat {H}_{r,\gamma }^{\sigma _j } } + \sum\limits_{\mu = 0}^{l - 1}
{\left\| {\Phi _\mu } \right\|} _{\hat {H}_{\alpha .q}^{\beta _\mu } } ) .
\]



\litlist

1. {\it Баев А. Д.} Вырождающиеся эллиптические уравнения высокого порядка и
связанные с ними псевдодифференциальные операторы / А. Д. Баев // Доклады
Академии наук. -- 1982. - Т. 265, № 5. - С. 1044 -- 1046.

2. {\it Баев А.Д.} Об общих краевых задачах в полупространстве для вырождающихся
эллиптических уравнений высокого порядка /А.Д. Баев// Доклады Академии наук,
2008, т. 422, №6, с. 727 -- 728.

3. {\it Баев А.Д.} Об одном классе псевдодифференциальных операторов с вырождением
/А.Д. Баев, Р.А. Ковалевский// Доклады академии наук. -- 2014. - T. 454.- №
1. - С. 7-10.

4. {\it Баев А.Д.} Теоремы об ограниченности и композиции для одного класса весовых
псевдодифференциальных операторов / А.Д. Баев, Р.А. Ковалевский// Вестник
Воронежского государственного университета Серия: Физика. Математика.--
2014, №1. -- С. 39- 49.

5.{\it Баев А.Д.} Краевые задачи для одного класса вырождающихся
псевдодифференциальных уравнений /А.Д. Баев, Р.А. Ковалевский// Доклады
академии наук. -- 2015. - T. 461.- № 1. - С. 1-3.
