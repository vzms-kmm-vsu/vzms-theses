\vzmstitle{Кобордизмы свободных узлов и графов}

\vzmsauthor{Федосеев}{Д.\,А.}

\vzmsinfo{Москва: МГУ им. Ломоносова, ИПУ им. Трапезникова РАН; {\it denfedex@yandex.ru}}

\vzmscaption


В 2009 году В.О. Мантуров ввёл в маломерной топологии понятие {\em чётности} [1], позволившее усилить множество
инвариантов виртуальных узлов, свободных узлов и др. и построить много новых инвариантов. Одним из главных отличительных свойств чётности является возможность строить инварианты со значениями в {\em картинках} --- диаграммах или линейных комбинациях диаграмм, которые позволяют доказывать утверждения о том, что {\em если диаграмма достаточно сложна, то она воспроизводит сама себя}, см. также [2, 3].

Компактно такое утверждение записывается в виде формулы $$[K]=K,$$ где в левой части $K$ обозначает узел или зацепление
(класс эквивалентности диаграмм по движениям), а в правой части $K$ --- одна достаточно сложная диаграмма этого же узла. Скобка $[K]$ строится комбинаторным образом и такое равенство означает, что для любой диаграммы $K'$, эквивалентной диаграмме $K$, имеет место $[K']=K$, что по построению значит, что диаграмма $K$ может быть получена из диаграммы $K'$ посредством некоторых операций {\em разведения}.

Таким образом, о многих свойствах {\em всех} диаграмм узла (например, $K'$) можно судить по {\em одной-единственной} его диаграмме $K$. В частности, глядя на единственную диаграмму (свободного) узла или зацепления, можно сказать о его нетривиальности, о минимальном количестве перекрёстков у диаграмм этого узла.

Центральный результат, изложенный в настоящем докладе, использует технику двумерных узлов и чётности на них (см. [4]) и позволяет для некоторых (нечётных) диаграмм судить ещё об одном свойстве --- их {\em срезанности}.
Важнейшим отношением эквивалентности классических узлов является отношение конкордантности --- возможности затянуть два узла цилиндрической плёнкой в четырёхмерном пространстве $\mathbb{R}^{3}\times [0,1]$, в компонентах края которого $\mathbb{R}^{3}\times \{0\}$ и $\mathbb{R}^{3}\times \{1\}$ лежат исследуемые узлы. В классической теории узлов это понятие особенно важно в связи с наличием различных типов конкордантности и срезанности --- топологической и гладкой, которые связаны с различными проблемами четырёхмерной топологиии.

Понятие конкордантности легко переносится на свободные узлы. Узел (свободный) называется {\em срезанным}, если он конкордантен тривиальному узлу.

Свободные узлы представляют собой класс эквивалентности оснащённых $4$-графов с одной компонентой по движениям Рейдемейстера, которые мы приводить не будем, поскольку эквивалентные оснащённые $4$-графы являются кобордантными. Под {\em кобордантностью} мы понимаем следующее. Скажем, что два оснащённых  $4$-графа
 $\Gamma_{1},\Gamma_{2}$ {\em кобордантны}, если существует {\em затягивающая поверхность}, представляющая собой двумерный комплекс --- образ цилиндра при непрерывном отображении общего положения, такой что образами краёв цилиндра являются графы $\Gamma_{1},\Gamma_{2}$, при этом в окрестности прообраза каждой вершины графа $\Gamma_{i}$ образ компоненты края цилиндра принадлежит объединению двух противоположных полуребер.

Ясно, что это отношение действительно является отношением эквивалентности. Эквивалентность оснащённого $4$-графа $\Gamma$ тривиальному узлу (задаваемому окружностью без вершин) называется {\em срезанностью}. Заклеивая тривиальную окружность диском, мы получаем {\em срезывающий диск} для $\Gamma$.

Далее, пусть $\Gamma$ --- оснащённый $4$-граф, $D(\Gamma)$ --- соответствующая этому графу хордовая диаграмма. Назовём {\em спариванием} ${\cal P}$ хорд разбиение всех хорд $\{d_{i}\}$ диаграммы $D(\Gamma)$ на попарно непересекающиеся множества ${\cal P}_{i}$, состоящие из одного или двух элементов, у которого для каждого множества ${\cal P}_{i}$, состоящего из двух хорд $c_{i},d_{i}$, каждому концу хорды $c_{i}$ взаимно-однозначно соответствует один конец хорды $d_{i}$. Хордовая диаграмма $C({\cal P})$ состоит из множества хорд, концы которых совпадают с концами хорд диаграммы $D$, а каждая хорда либо совпадает с хордой диаграммы $C$, составляющей одно семейство, либо соединяет два соответствующих друг другу конца двух различных хорд из одного семейства.

Основная теорема, изложенная в настоящем докладе, состоит в следующем: \\

{\bf Теорема.} {\it Если диаграмма $K$ свободного узла является {\em нечётной}, то она является {\em срезанной} тогда и только тогда, когда её хорды допускают {\em спаривание без пересечений}. } \\

Отметим, что эта теорема даёт {\em конечную процедуру} для определения, является ли данный
свободный узел со всеми нечётными перекрёстками срезанным.

%%%%  ОФОРМЛЕНИЕ СПИСКА ЛИТЕРАТУРЫ %%%
\litlist

1. {\it Мантуров В.О.} Чётность в теории узлов. Математический Сборник. 201:5, 2010. с. 65--110.

2. {\it Мантуров В.О.} Почти классификация свободных зацеплений. Доклады РАН. 452:4, 2013. с. 1--4.

3. {\it Kauffman L.H., Manturov V.O.} A graphical construction of the $sl(3)$ invariant for virtual knots. Quantum Topology. 5:4, 2014. с. 523--539.

4. {\it Fedoseev D.A., Manturov V.O.} Parities on 2-knots and 2-links. J. Knot Theory Ramifications. 25, 2016.

