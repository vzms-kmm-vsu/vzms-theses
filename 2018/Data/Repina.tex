
\begin{center}
\bf   \textbf{Анализ различных методов интерполяции для восстановления сигналов}
\end{center}

Современная вычислительная математика ориентирована на использование компьютеров для прикладных отчетов. Любые математические приложения начинаются с построения модели явления, к которому относится изучаемый вопрос. Основополагающими средствами изучения математических моделей являются аналитические методы: получение точных решений в частных случаях, разложения в ряды. Компьютер дает возможность запоминать большие (но конечные) массивы чисел и производить над ними арифметические операции и сравнения с большой (но конечной) скоростью по заданной вычислителем программе. Поэтому на компьютере можно изучать только те математические модели, которые описываются конечными наборами чисел, и использовать конечные последовательности арифметических действий, а также сравнений чисел по величине (для автоматического управления дальнейшими вычислениями). Целью данной работы является сравнительный анализ двух методов интерполяции с последующей реализацией в компьютерной программе TurboDelphiLite, а именно тригонометрического метода и метода Лагранжа для восстановления физических спектров.
