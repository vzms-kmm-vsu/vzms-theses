\vzmstitle{
	ОБРАТНЫЕ  ЭКСТРЕМАЛЬНЫЕ
	ЗАДАЧИ  ТЕПЛОМАССООБМЕНА
	НА  ПРОНИЦАЕМЫХ  ПОВЕРХНОСТЯХ
	ПРИ  ГИПЕРЗВУКОВЫХ  РЕЖИМАХ  ПОЛЁТА
}

\vzmsauthor{Бильченко}{Г.\,Г.}
\vzmsauthor{Бильченко}{Н.\,Г.}

\vzmsinfo{Казань; {\it ggbil2@gmail.com, bilchnat@gmail.com}}

\vzmscaption



%%  {Обратные  экстремальные  задачи
%%   тепломассообмена  на  проницаемых
%%   поверхностях  при  гиперзвуковых
%%   режимах  полёта}
%%% ===========================================================



Работа  является  продолжением
[1, 2].



\textbf{1.}\;%
Рассмотренную
в
[3{\textbf{--}}5]
прямую  задачу:
%% =============================== \
\[
(m,\tau,s) \to
\left (
q,\; f;\;\; Q,\; F,\; N
\right )
{,}
\eqno(1)
\]
%% =============================== /
рассмотренные  в
[5]
обратные  задачи  по  вдуву:
%% =============================== \\
\begingroup\belowdisplayskip=\belowdisplayshortskip
\[
q^{\vee}  \to  m^{\sim}
\,{,}
\qquad
\left(m^{\sim}, \tau, s\right)
\to
\left(q^{\sim} \approx q^{\vee} {,}\; f^{\sim}\right)
{,}
\eqno(2)
\]
\endgroup
%% =============================== //
%% =============================== \\
\[
f^{\vee}  \to  m^{\sim}
\,{,}
\qquad
\left(m^{\sim}, \tau, s\right)
\to
\left(q^{\sim} {,}\; f^{\sim} \approx f^{\vee}\right)
{,}
\eqno(3)
\]
%% =============================== //
а также  рассмотренную  в
[6]
двумерную  обратную  задачу:
%%% =============================== \\
\[
\left(q^{\vee}, f^{\vee}\right)
\!\to\!
\left(m^{\sim}, \tau^{\sim}\right)
{,}
\;\;
\left(m^{\sim}, \tau^{\sim}, s\right)
\!\to\!
\left(q^{\sim} \approx q^{\vee}
{,}\;
f^{\sim} \approx f^{\vee}\right)
\eqno(4)
\]
%%% =============================== //
обозначим
$\text{ПЗ}$\,{,}
$\text{ОЗ}_{m}^{q}$\,{,}
$\text{ОЗ}_{m}^{f}$\,{,}
$\text{ОЗ}_{\left(m,\tau\right)}^{\left(q,f\right)}$\,{,}
соответственно.



В
[5,  6]
управления:
\textit{вдув}
$m\left(x\right)
${,}
\textit{температурный  фактор}
$\tau\left(x\right)
\!=\!{T_{w}\left(x\right)}/{T_{e_{0}}}
\,
${,}
\textit{магнитное поле}
$s\left(x\right)
\!=\!\sigma B_{0}^{2}\left(x\right)$
задавались
для
$X=[0; 1]$
или
разыскивались
($m^{\sim}
\!=\!\left(m_j^{\sim}\right)_{j=1,\ldots,n_1}\,$,
$\tau^{\sim}
\!=\!\left(\tau_j^{\sim}\right)_{j=1,\ldots,n_1}$)
в  виде  элементов
[7]
для  сетки  \textit{управления}
%% =============================== \\
\[
X_1\,{:}
\quad
x_{0}^{\wedge}=0 <
x_{1}^{\wedge} < \ldots <
x_{n_1}^{\wedge} = 1
\,{;}
\eqno(5)
\]
%% =============================== //
контрольные  значения
$q^{\vee}
\!=\!\left(q_j^{\vee}\right)_{j=0,\ldots,n_2}
\,${,}
$
f^{\vee}
\!=\!\left(f_j^{\vee}\right)_{j=0,\ldots,n_2}
\,$
задавались,
а  значения
$q^{\sim}
\!=\!\left(q^{\sim}_{j}\right)_{j=0,\ldots,n_2}
\,${,}
$f^{\sim}
\!=\!\left(f^{\sim}_{j}\right)_{j=0,\ldots,n_2}
\,$
вычислялись
для  всех  узлов
сетки  \textit{наблюдения}
%% =============================== \\
\[
X_2\,{:}
\quad
x_{0}^{\vee}=0 < x_{1}^{\vee} < \ldots < x_{n_2}^{\vee} = 1
\,{.}
\eqno(6)
\]
%% =============================== //
%%% =====================================
%%% page 2 start here
%%% =====================================



Будем  предполагать,  что:  \\
\noindent
1)\;%
выполнены  условия  существования  решения
[5, 6],
в  частности,
$X_{1}\subseteq X_{2}$;


\noindent
2)\;%
согласованы
\textit{ограничения}
[7]
на  вдув
$I_{j,k}^{m}
=\left[b_{j,k}^{m}; t_{j,k}^{m}\right]$
и  на  температурный  фактор
$I_{j,k}^{\tau}
=\left[b_{j,k}^{\tau}; t_{j,k}^{\tau}\right]$,
где  $j=1,\ldots,n_{1}$:
%% =============================== \\
\begingroup\belowdisplayskip=\belowdisplayshortskip
\[
\left(m^{\sim}\right)^{\left(k\right)}\left(x\right)
\!\in\!
I_{j,k}^{m}
\:\:
\text{для}
\:\:
x
\!\in\!
\left[x_{j-1}^{\wedge};x_{j}^{\wedge}\right]
\:\!\!{,}
\;\:
k
\:\!\!=\:\!\!
0,\ldots,\nu_1^{m}
{,}
\;\:
\nu_1^{m}\geqslant 0
\;\!{;}
\eqno(7)
\]
\endgroup
%% =============================== //
%% =============================== \\
\[
\left(\tau^{\sim}\right)^{\left(k\right)}\left(x\right)
\!\in\!
I_{j,k}^{\tau}
\:\:
\text{для}
\:\:
x
\!\in\!
\left[x_{j-1}^{\wedge};x_{j}^{\wedge}\right]
\:\!\!{,}
\;\:
k
\:\!\!=\:\!\!
0, \ldots , \nu_1^{\tau}
{,}
\;\:
\nu_1^{\tau}\geqslant 1
\;\!{.}
\eqno(8)
\]
%% =============================== //



ПЗ
(1)
является  частью
\textit{прямой  экстремальной  задачи}
(далее~{\textbf{--}}  ПЭЗ),
рассмотренной
в
[8{\textbf{--}}10].
%%%
В
$\text{ПЭЗ}_{m}^{Q}$
для  заданных
$\tau(x)$
и
$s(x)$
требуется  в  ограничениях
(7)
найти  управление
$m^{\sim}$
как  приближённое  решение  экстремальной  задачи
%% =============================== \
\[
Q^{*}\left ( \tau,s; N_{\max} \right )
=
\inf\limits_{m^{\sim}}
Q(m^{\sim},\tau,s)
\eqno(9)
\]
%% =============================== /
с  ограничением
на  мощность  обеспечивающей  вдув  системы
%% =============================== \
\[
N(m^{\sim},\tau,s) \leqslant N_{\max}
\,{.}
\eqno(10)
\]
%% =============================== /
В
$\text{ПЭЗ}_{m}^{F}$
требуется  найти
$m^{\sim}$
как  приближённое  решение
%% =============================== \
\[
F^{*}\left ( \tau,s; N_{\max} \right )
=
\inf\limits_{m^{\sim}}
F(m^{\sim},\tau,s)
\eqno(11)
\]
%% =============================== /
в  условиях  (7),  (10)
при  заданных
$\tau(x)$
и
$s(x)$.


Обратные  задачи
(2),  (3),  (4)
с  условиями  (7)  или/и  (8)
в  аппроксимационной  постановке
[5, 6]
с  дополнительным   условием
(10)
назовём
\textit{обратными  экстремальными  задачами}
и  обозначим  их
$\text{ОЭЗ}_{m}^{q}$\,{,}
$\text{ОЭЗ}_{m}^{f}$\,{,}
$\text{ОЭЗ}_{\left(m,\tau\right)}^{\left(q,f\right)}$\,{,}
соответственно.
%%% ===========================================================



\textbf{2.}\;%
Для
некоторых
$\text{ОЭЗ}$
обсуждаются  результаты
вычислительных  экспериментов.
%%% ===========================================================



\textbf{3.}\;%
Случаю  использования
$\text{ОЭЗ}$
в
\textit{задачах  на  фрагментах}
[1, 2]
посвящено  продолжение
[11]
данной  работы.
%%% ===========================================================
%%% =====================================
%%% page 3 start here
%%% =====================================



%%%%  ОФОРМЛЕНИЕ СПИСКА ЛИТЕРАТУРЫ %%%
\smallskip \centerline{\bf Литература}\nopagebreak



%% ==============================
1.~%
\textit{Бильченко~Г.~Г.,  Бильченко~Н.~Г.}\;
{%
  {Смешанные  обратные  задачи
   тепломассообмена  на  проницаемых
   поверхностях  при  гиперзвуковых
   режимах  полёта}~/$\!$/
   Международная  конференция,  посвящённая
  100-летию
  со  дня  рождения
  С.~Г.~Крейна
  (Воронеж,
  13{\textbf{--}}19  ноября  2017~г.):
  сборник  материалов.~{\textbf{---}}
  Воронеж:  Изд.  дом  ВГУ,
  2017.~{\textbf{---}}
  С.~52{\textbf{--}}54.
}
%% ==============================

%% ==============================
2.~%
\textit{Бильченко~Г.~Г.,  Бильченко~Н.~Г.}\;
{%
  {Комбинированные  обратные  задачи
   тепломассообмена  на  проницаемых
   поверхностях  при  гиперзвуковых
   режимах  полёта}~/$\!$/
  Международная  конференция,  посвящённая
  100-летию
  со  дня  рождения
  С.~Г.~Крейна
  (Воронеж,
  13{\textbf{--}}19  ноября  2017~г.):
  сборник  материалов.~{\textbf{---}}
  Воронеж:  Изд.  дом  ВГУ,
  2017.~{\textbf{---}}
  С.~50{\textbf{--}}51.
}
%% ==============================

%% ==============================
3.~%
\textit{Бильченко~Н.~Г.}\;
{%
  {Метод  А.~А.~Дородницына
   в  задачах  оптимального  управления
    тепломассообменом  на  проницаемых  поверхностях
    в  ламинарном  пограничном  слое
    электропроводящего  газа}~/$\!$/
  Вестник  Воронеж.  гос.  ун-та.
  Сер.  Системный  анализ
  и  информационные  технологии.~{\textbf{---}}
  2016.~{\textbf{---}}
  \No~1.~{\textbf{---}}
  С.~5{\textbf{--}}14.
  }
%% ==============================

%% ==============================
4.~%
\textit{Бильченко~Н.~Г.}\;
{%
  {Вычислительные  эксперименты
    в  задачах  оптимального  управления  тепломассообменом
    на
    %%% Overfull start
    проницаемых\,  поверхностях\,
    в\,  ламинарном\,  пограничном\,  слое
    %%% Overfull end
    электропроводящего  газа}~/$\!$/
  Вестник  Воронеж.  гос.  ун-та.
  Сер.  Системный  анализ
  и  информационные  технологии.~{\textbf{---}}
  2016.~{\textbf{---}}
  \No~3.~{\textbf{---}}
  С.~5{\textbf{--}}11.
  }
%% ==============================

%% ==============================
5.~%
\textit{Бильченко~Г.~Г.,  Бильченко~Н.~Г.}\;
{%
  {Обратные  задачи  тепломассообмена
    на  проницаемых  поверхностях
    гиперзвуковых  летательных  аппаратов.
    I.  О  некоторых  постановках
    и  возможности  восстановления  управления}~/$\!$/
  Вестник  Воронеж.  гос.  ун-та.
  Сер.  Системный  анализ
  и  информационные  технологии.~{\textbf{---}}
  2016.~{\textbf{---}}
  \No~4.~{\textbf{---}}
  С.~5{\textbf{--}}12.
  }
%% ==============================

%% ==============================
6.~%
\textit{Бильченко~Г.~Г.,  Бильченко~Н.~Г.}\;
{%
  {Обратные  задачи  тепломассообмена
    на  проницаемых  поверхностях
%%% =====================================
%%% page 4 start here
%%% =======v=============================
    гиперзвуковых
%%% ========^============================
%%% page 4 start here
%%% =====================================
    летательных  аппаратов.
    III.  О  постановке  двумерных  задач
    и  областях  допустимых  значений
    <<тепло~{\textbf{--}}  трение>>}~/$\!$/
  Вестник  Воронеж.  гос.  ун-та.
  Сер.  Системный  анализ
  и  информационные  технологии.~{\textbf{---}}
  2017.~{\textbf{---}}
  \No~1.~{\textbf{---}}
  С.~18{\textbf{--}}25.
  }
%% ==============================

%% ==============================
7.~%
\textit{Бильченко~Г.~Г., Бильченко~Н.~Г.}\;
{%
  {О  некоторых  классах  функций,
   применяемых  для  решения  задач
   эффективного  управления  тепломассообменом
   на  проницаемых  поверхностях
   гиперзвуковых  летательных  аппаратов}~/$\!$/
  Вестник  Воронеж.  гос.  ун-та.
  Сер.  Системный  анализ
  и  информационные  технологии.~{\textbf{---}}
  2017.~{\textbf{---}}
  \No~3.~{\textbf{---}}
  С.~5{\textbf{--}}15.
  }
%% ==============================

%% ==============================
8.~%
\textit{Бильченко~Н.~Г.}\;
{%
 {Вычислительные  эксперименты
    в  задачах  оптимального  управления  тепломассообменом
    на  проницаемых  поверхностях
    при  гиперзвуковых  режимах  полёта}~/$\!$/
  Вестник  Воронеж.  гос.  ун-та.
  Сер.  Физика.  Математика.~{\textbf{---}}
  2015.~{\textbf{---}}
  \No~1.~{\textbf{---}}
  С.~83{\textbf{--}}94.
}
%% ==============================

%% ==============================
9.~%
\textit{Бильченко~Н.~Г.}\;
{%
  {Вычислительные  эксперименты
    в  задачах  оптимального  управления
    тепломассообменом  на  проницаемых  поверхностях
    тел  вращения
    при  гиперзвуковых  режимах  полёта}~/$\!$/
  Вестник  Воронеж.  гос.  ун-та.
  Сер.  Системный  анализ
  и  информационные  технологии.~{\textbf{---}}
  2015.~{\textbf{---}}
  \No~1.~{\textbf{---}}
  С.~5{\textbf{--}}8.
  }
%% ==============================

%% ==============================
10.~%
\textit{Бильченко~Н.~Г.}\;
{%
  {Вычислительные  эксперименты
    в
    %%% Overfull start
    задачах\,  оптимального\,  управления\,
    тепломассообменом\,
    на
    %%% Overfull end
    проницаемых  поверхностях
    при  гиперзвуковых  режимах  полёта:\,
    сравнительный  анализ
    применения  ``простых''  законов  вдува}~/$\!$/
  Вестник  Воронеж.  гос.  ун-та.
  Сер.  Физика.  Математика.~{\textbf{---}}
  2015.~{\textbf{---}}
  \No~1.~{\textbf{---}}
  С.~95{\textbf{--}}102.
}
%% ==============================

%% ==============================
11.~%
\textit{Бильченко~Г.~Г.,  Бильченко~Н.~Г.}\;
{%
  {Экстремальные  и  неэкстремальные
   обратные  задачи  на  фрагментах}~/$\!$/
  <<Воронежская  зимняя  математическая  школа
  С.~Г.~Крейна~{\textbf{--}}  2018>>:
  Материалы  международной  конференции
  (26{\textbf{--}}31  января  2018~г.).~{\textbf{---}}
  Воронеж:  ИПЦ  <<Научная  книга>>,
  2018.%~{\textbf{---}}
  %С.~???{\textbf{--}}???.
  }
%% ==============================
