\vzmstitle{ОБРАТИМОСТЬ РАВНОМЕРНО ИНЪЕКТИВНЫХ И СЮРЪЕКТИВНЫХ ДИФФЕРЕНЦИАЛЬНЫХ ОПЕРАТОРОВ ВТОРОГО ПОРЯДКА}

\vzmsauthor{Кабанцова}{Л.\,Ю.}

\vzmsinfo{Воронеж; {\it dlju@yandex.ru}}

\vzmscaption

Пусть $X$~--- комплексное
банахово пространство.
Рассматриваются следующие функциональные пространства:
$C_b=
$\linebreak $=
C_b(\mathbb{R},X)$~--- банахово пространство непрерывных
ограниченных на $\mathbb{R}$ функций со значениями в пространстве $X;$
$C_\omega=
$\linebreak $=
C_\omega(\mathbb{R},X)$~--- замкнутое подпространство $\omega-$периодических фу\-н\-к\-ций из $C_b(\mathbb{R},X);$
$C_0=C_0(\mathbb{R},X)$~--- замкнутое подпространство функций из $C_b(\mathbb{R},X),$ стремящихся к нулю на бесконечности;
$C_{b,u}=C_{b,u}(\mathbb{R},X)$~--- замкнутое подпространство равномерно непрерывных функций из $C_b(\mathbb{R},X);$
$AP=AP(\mathbb{R},X)$~--- замкнутое подпространство почти периодических функций из
$C_b(\mathbb{R},X);$ $AP_{\infty}=AP_{\infty}(\mathbb{R},X)$ -- замкнутое подпространство почти периодических
на бесконечности функций из $C_{b,u}(\mathbb{R},X).$

Далее через $\mathcal{F}=\mathcal{F}(\mathbb{R},X)$ обозначается одно из перечисленных выше пространств
и под $\mathcal{F}^{(k)}, k\in\mathbb{N},$ понимается линейное подпространство функций из $\mathcal{F},$
имеющих $k$ непрерывных производных, причём $x^{(k)}\in \mathcal{F}$.

Пространство $\mathcal{F}$ называется {\it спектрально полным},
если ему принадлежат все функции вида $x_0\exp(i\lambda t),$ $\lambda\in\mathbb{R},$ $x_0 \in X.$
Такими пространствами являются  $C_b,\;C_{b,u},\;AP,\;AP_{\infty}.$

В банаховом пространстве $\mathcal{F}=\mathcal{F}(\mathbb{R}, X)\subseteq C_b (\mathbb{R}, X)$
рассматривается линейный дифференциальный оператор

$$L: \mathcal{F}^{(2)}\subset \mathcal{F}\rightarrow\mathcal{F}, \quad L=\frac {d^2}{dt^2}+B_1\frac {d}{dt}+B_2$$
с областью определения $D(L)=\mathcal{F}^{(2)}.$ Операторы $B_1, B_2$ принадлежат банаховой алгебре $\mathrm{End}\, X.$

Рассматриваемому оператору $L$ соответствует характеристический многочлен (пучок операторов)
$$H:\mathbb{C}\rightarrow \mathrm{End}\,X, \quad H(\lambda)=\lambda^2 I+B_1\lambda+B_2,\;\lambda \in\mathbb{C}.$$

Следующие две теоремы доказаны для спектрально полного пространства $\mathcal{F}.$

\textbf{Теорема~1.} {\it
Равномерно инъективный оператор \linebreak $L:\mathcal{F}^{(2)}\subset \mathcal{F}\rightarrow \mathcal{F}$ обратим.}

\textbf{Теорема~2.} {\it
Сюръективный оператор $L:\mathcal{F}^{(2)}\subset \mathcal{F}\rightarrow \mathcal{F}$ обратим.}

При этом для спектрально неполного пространства \linebreak $C_0(\mathbb{R}, X)$  имеют место аналогичные теоремы:

\textbf{Теорема~3.} {\it
Равномерно инъективный оператор \linebreak $L : C_0^{(2)}(\mathbb{R}, X)\subset C_0(\mathbb{R}, X)\rightarrow C_0(\mathbb{R}, X)$ обратим.}

\textbf{Теорема~4.} {\it
Сюръективный оператор \\$L:C_0^{(2)}(\mathbb{R},X)\subset C_0(\mathbb{R},X)\rightarrow C_0(\mathbb{R},X)$ обратим.}

Для спектрально неполного пространства $C_{\omega}$  теоремы 1 и 2 неверны.
Точнее, существуют операторные коэффициенты $B_1,$ $B_2\in \mathrm{End}X$ такие,
что оператор $L:C^{(2)}_\omega\subset C_\omega\rightarrow C_\omega$ является односторонне обратимым.

 Построим соответствующие примеры в пространстве \linebreak $C_\omega(\mathbb{R},l_2),$ где $\omega=1.$
   В $l_2$ рассматриваются операторы односторонних сдвигов $U$, $V$ вида $Ue_n=e_{n+1},$  $Ve_1=0,\; Ve_n=e_{n-1},\,n\in \mathbb{N},$ где $\{e_n\}$ -- стандартный базис в $l_2$.
Определим оператор $A=\ln(I-V/2).$
 Рассмотрим функцию $\mathcal{H}:\mathbb{R}\rightarrow End\;l_2,$ определяемую равенством $\mathcal{H}(t)=2\exp(At)U.$
   Коэффициенты оператора $L:C_{1}^{(2)}(\mathbb{R},l_2)\subset C_{1}(\mathbb{R},l_2)\rightarrow C_{1}(\mathbb{R},l_2)$
   задаются следующим образом: $B_1=-2A,$ $B_2=A^2.$

Вводится операторнозначная функция $G:\mathbb{R}\rightarrow \mathrm{End}\;l_2,$ которая на $[0,1]$ задаётся по формуле
$$G(t)=\int_0^t \mathcal{H}(t-s)\mathcal{H}(s)\,ds+\int_t^{1} \mathcal{H}(1+t-s)\mathcal{H}(s)\,ds,\;t\in [0,1],
$$ и периодически продолжается на $\mathbb{R}.$

\textbf{Теорема~5.} {\it Оператор $L:C_{1}^{(2)}(\mathbb{R},l_2)\rightarrow C_{1}(\mathbb{R},l_2)$
обратим справа, но не является обратимым. Одним из правых обратных является интегральный оператор}
 $$((L^{-1})_ry)(t)=\int_0^{1}G(t-s)y(s)\,ds,\; t\in \mathbb{R},\;y\in C_{1}(\mathbb{R},l_2).$$

В условиях следующей теоремы полагается \linebreak $A=\ln(I-U/2),$
$\mathcal{H}(t)=2V\exp(At),\;t\in \mathbb{R}.$

\textbf{Теорема~6.} {\it Оператор $L:C_{1}^{(2)}(\mathbb{R},l_2)\rightarrow C_{1}(\mathbb{R},l_2)$
 обратим слева, но не является обратимым. Одним из левых обратных является интегральный оператор}
$$((L^{-1})_l\;y)(t)=\int_0^{1}G(t-s)y(s)\,ds,\; t\in \mathbb{R},\;y\in C_{1}(\mathbb{R},l_2).$$

\textbf{Теорема~7.} {\it Оператор $L:C^{(2)}_\omega\subset C_\omega\rightarrow C_\omega$ сюръективен точно тогда,
когда операторы $H (in\kappa)=-(n\kappa)^2I+in\kappa B_1+B_2,\;\kappa=2\pi/\omega,\;n\in\mathbb{Z},$
 сюръективны.}

 \textit{Оператор $L:C^{(2)}_\omega\subset C_\omega\rightarrow C_\omega$ инъективен точно тогда,
 когда операторы $ H (in\kappa)=-(n\kappa)^2I+in\kappa B_1+B_2,\;\kappa=2\pi/\omega,$ $n\in\mathbb{Z},$}\textit{ инъективны.}

\textit{Оператор $L:C^{(2)}_\omega\subset C_\omega\rightarrow C_\omega$ обратим точно тогда, когда
выполнено условие
$$\sigma(H)\cap (i\kappa\mathbb{Z})=\emptyset,
$$
(т.е. спектр пучка $H$ не содержит точек мнимой оси $i\mathbb{R}$ вида $i2\pi n/\omega,\;n\in \mathbb{Z}$).}

%%%  ОФОРМЛЕНИЕ СПИСКА ЛИТЕРАТУРЫ %%%
\litlist

1. {\it Баскаков А.\,Г., Кабанцова Л.\,Ю., Коструб И.\,Д., Смагина Т.\,И. }
Линейные дифференциальные операторы и операторные матрицы второго порядка. Дифференц. уравнения. 2017. Т.~53. Вып.~1.
С.~10-19.
