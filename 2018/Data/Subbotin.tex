\begin{center}{\bf СПЕКТРАЛЬНО ОБРАТИМЫЕ\\ И ГАМИЛЬТОНОВЫ ДИНАМИЧЕСКИЕ СИСТЕМЫ}\\
{\it А.В. Субботин, Ю.П. Вирченко}\\
(Белгород; {\it virch@bsu.edu.ru})
\end{center}
\addcontentsline{toc}{section}{Субботин А.В., Вирченко Ю.П.}

Введено общее понятие о спектрально обратимых автономных динамических системах. Показано, что всякая автономная гамильтонова система является спектрально обратимой. Доказана теорема о том, что генераторы  линейных систем, получаемых линеаризацией каждой данной спектрально обратимой системы, связаны линейным преобразованием на основе дифференцируемой  матриц-функции  ${\sf U}(X)$ с генераторами линейных гамильтоновых систем.

Пусть ${\sf F} : {\Bbb R}^{2n} \mapsto {\Bbb R}^{2n}$ -- биекция ${\Bbb R}^{2n}$, $n \in {\Bbb N}$. Система дифференциальных уравнений
$${\dot X}(t) = {\sf F}(X(t))\eqno (1) $$
определяет траектории $X(t) = \langle x_1 (t), ..., x_{2n} (t) \rangle$ в пространстве ${\Bbb R}^{2n}$ состояний  $X = \langle x_1, ..., x_{2n} \rangle$. Для этой динамической системы в каждой точке $X \in {\Bbb R}^{2n}$ определена  $2n \times 2n$-матрица ${\sf G}(X) = (G_{ij} (X); i,j = 1 \div 2n)$,
$$ G_{ij} (X)  = \frac {\partial {\sf F}_i (X)}{\partial x_j}\,,\ \ \det {\sf G}(X) \ne 0. \eqno (2) $$


\textbf{Определение 1.} {\it Пусть ${\sf G}$ -- невырожденная чётномерная матрица размерности $2n$, $\det G \ne 0$ простой структуры, не имеющая кратных точек спектра
$\{\lambda_1, ..., \lambda_{2n}\}$, $\lambda_i \ne \lambda_j$, если $i \ne j$, $i,j = 1 \div 2n$. Матрицу ${\sf G}$ называется  спектрально-обратимой, если её спектр обладает следующим свойством: для каждого $j = 1 \div 2n$ существует единственный номер $j'$ такой, что $\lambda_{j'} = - \lambda_j$.}
\smallskip

Вводится понятие спектрально-обратимых динамических систем вида (1).
\smallskip

\textbf{Определение 2.} {\it Динамическую систему (2) назовём спектрально-обратимой, если в каждой точке $X \in {\Bbb R}^{2n}$ соответствующая ей матрица ${\sf G}(X)$ является спектрально-обратимой.}
\smallskip

Примером спектрально-обратимых систем являются т.н. гамильтоновы системы.
Для гамильтоновой системы соответствующий генератор ${\sf G}(X)$ в каждой точке $X \in {\Bbb R}^{2n}$ имеет  блочную структуру
$${\sf G}(X) = \begin{pmatrix} - {\sf B}^{\rm T} (X) & - {\sf C}(X) \\ {\sf A}(X) & {\sf B}(X) \end{pmatrix}\eqno (3)$$
с блоками в виде $n \times n$-матриц ${\sf A}(X)$, ${\sf B} (X)$, ${\sf C}(X)$, где матрицы ${\sf A}(X)$ и ${\sf C}(X)$ симметричны.

Матрицы, имеющие блочную структуру (3), мы называем {\it гамильтоновыми матрицами}.

\textbf{Теорема 1.} {\it Каждая гамильтонова матрица (4) является спектрально-обратимой.}
\smallskip

Справедлива следующая теорема, обратная к Теореме 1.
\smallskip

\textbf{Теорема 2.} {\it Каждая вещественная спектрально-обратимая матрица ${\sf G}$ размерности $2n$ вещественно подобна гамильтоновой матрице той же размерности.}
\smallskip

В настоящем сообщении  по заданной матриц-функ\-ции ${\sf G}(X)$ со значениями в виде вещественных спектрально обратимых матриц построена соответствующей ей  матриц-функции ${\sf H}(X)$, со значениями в виде  гамильтоновых матриц.

Доказано следующее утверждение.
\smallskip

\textbf{Теорема 3.} {\it Пусть для фиксированного чётного числа $d  = 2n$ задана ${\sf G}: {\Bbb R}^d \mapsto {\Bbb R}^d\times {\Bbb R}^d$ -- дифференцируемая по $X \in {\Bbb R}^d$ матриц-фун\-к\-ция со значениями в виде вещественных  невырожденных спектрально-обратимых матриц  ${\sf G}(X)$, $\det {\sf G}(X) \ne 0$ простой структуры, которые не имеют кратных собственных значений. Тогда существует дифференцируемая по $X$ матриц-фун\-к\-ция ${\sf U}: {\Bbb R}^{d} \mapsto {\Bbb R}^d \times {\Bbb R}^d$ со значениями в виде вещественных невырожденных матриц ${\sf U}(X)$,  $\det {\sf U}(X) \ne 0$, такая, что матрицы ${\sf U}(X){\sf  G}(X){\sf U}^{-1}(X) =  {\sf H} (X)$ являются гамильтоновыми при каждом $X \in {\Bbb R}^d$ .}
\smallskip

Теорема доказывается в два этапа. Сначала  доказывается утверждение о существовании дифференцируемой  по $\xi \in {\Bbb R}$ матриц-функции ${\sf U}: {\Bbb R} \mapsto {\Bbb R}^d \times {\Bbb R}^d$ с указанными в утверждении теоремы свойствами при наличии дифференцируемой матриц-функции ${\sf G}(\xi)$, которую она трансформирует в матриц-функцию  ${\sf U}(\xi){\sf  G}(\xi){\sf U}^{-1}(\xi) =  {\sf H} (\xi)$. Для этого доказывается разрешимость гомологического уравнения $[{\sf R}(\xi), {\sf H}(\xi)] + {\sf D}(\xi) = {\sf S}(\xi)$, где
$${\sf R}(\xi) = \frac {d {\sf U}(\xi)}{d \xi}\,{\sf U}^{-1}(\xi) \,, \qquad {\sf T}(\xi) = \frac {d {\sf G}(\xi)}{d \xi}\,, \qquad {\sf S}(\xi) = \frac {d {\sf H}(\xi)}{d \xi}\,. $$

Затем матриц-функция ${\sf U}(X)$ строится в виде ${\sf U}(X) = {\sf U}(x_1, x_2, ..., x_d) \cdot ... \cdot {\sf U}(x_1, x_2) {\sf U}(x_1)$, где последовательность $\langle {\sf U}(x_1), {\sf U}(x_1, x_2), ..., {\sf U}(x_1, x_2, ..., x_d)\rangle$  дифференцируемых  мат\-риц-функций  составляется пошагово, полагая $\xi = x_j$ на каждом шаге $j = 1, 2, ..., d$ построения.
\smallskip

\centerline{Литература}\nopagebreak

1. {\it Вирченко Ю.П., Субботин А.В.}  Характеризация линейных гамильтоновых систем~// Материалы международной конференции <<Дифференциальные уравнения и их приложения>> 26-31 мая 2013, Белгород~/ Белгород: Политерра, 2013.~-- C.180-181.

2. {\it Гантмахер Ф.Р.} Теория матриц~/ М.: Наука, 1966.~-- 576~c.




