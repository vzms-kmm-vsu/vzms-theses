
\begin{center}{ \bf  НЕКОТОРЫЕ МЕТОДЫ ЧИСЛЕННОГО РЕШЕНИЯ
ЧАСТИЧНО ИНТЕГРАЛЬНОГО УРАВНЕНИЯ РОМАНОВСКОГО}\\
{\it  В.А. Калитвин } \\
(Липецк; {\it  kalitvin@mail.ru} )
\end{center}
\addcontentsline{toc}{section}{Калитвин В.А.\dotfill}


Будем рассматривать частично   интегральное уравнение
$$
x(t,s)\!=\!\int_a^bm(t,s,\sigma)x(\sigma,t)d\sigma\!+\!f(t,s)\!\equiv\! (Mx)(t,s)\!+\!f(t,s)
\eqno(1)
$$
с непрерывными заданными функциями $f(t,s)$ и $m(t,s,\sigma),$ где $t,s,\sigma\in[a,b].$

Следующая схема обоснована В.И. Романовским в [1] и применима для численного  решения  уравнения (1).

Отрезок $[a,b]$ разобъём на части длины $\delta$ точками \break
$
t_i=s_i=\sigma_i\ (i=0,\dots,n; t_0=s_0=\sigma_0=a, t_n=s_n=\sigma_n=b).
$
Положим
$
x_{kl}=x(t_k,s_l), f_{kl}=f(t_k,s_l), m_{hkl}=m(t_k,s_l,\sigma_h)
$
и через $\Delta$ обозначим определитель системы
$$
x_{kl}=f_{kl}+\delta\sum_{h=1}^nx_{hk}m_{hkl} \ (k,l=1,\dots,n).
\eqno(2)
$$
Этот определитель имеет размер $n^2\times n^2,$ а его строки и столбцы могут быть занумерованы двойными индексами

$
ih=11,12,\dots,1n, 21,22,\dots,2n,\dots, n1,n2,\dots,nn,
$

$
fg=11,12,\dots,1n, 21,22,\dots,2n,\dots, n1,n2,\dots,nn.
$

Применение обозначений
$
a_{ih|kk}=m_{ihk}, a_{ih|fg}=0  (h\not=f),
$
$
e_{ih|fg}=\left\{
\begin{array}{l}
1,\hbox{если} i=f,h=g,\\[2pt]
0, \hbox{если} i\not=f \hbox{или} h\not=g\\[2pt]
\end{array}
\right.
$
приводит к равенству $\Delta=|e_{ih|fg}-\delta a_{ih|fg}|=|e_{pq}-\delta a_{pq}|,$ где $p=ih, q=fg.$ Пусть $\Delta_{pq}$ --- минор определителя $\Delta,$ соответствующий элементу $e_{pq}-\delta a_{pq}.$ При $\Delta\not= 0$ система (2) имеет единственное решение
$$
x_p={1\over\Delta}\sum_q\Delta_{pq}f_q={1\over\Delta}\sum_{f,g=1}^n\Delta_{ih|fg}f_{fg}=
$$
$$
={\Delta_{pp}\over\Delta}f_p+{1\over \Delta}\sum_{g=1}^n\Delta_{ih|gi}f_{gi}+{1\over\Delta}\sum_q {'}\Delta_{pq}f_q,
\eqno(3)
$$
где $\sum {'}$ обозначает суммирование по всем $q,$ кроме $q=p$ и $q=gi (g=1,2,\dots,n).$

Если теперь $n\to\infty,$ то  система (2) аппроксимирует уравнение (1), а решение (3) стремится к решению уравнения (1) [1].
Таким образом, численное решение уравнения (1) может быть найдено по формуле (3).

Отметим, что формула (3) получается при условии $\Delta\not= 0.$ При достаточно большом $n$ это условие означает, что
$1\not\in\sigma(M),$ где через $\sigma(M)$ обозначен спектр оператора $M.$

Ещё один метод численного решения уравнения (1) основан на численном решении системы  интегральных уравнений Фредгольма с параметром
$t$
$$
\left\{
\begin{array}{l}
y(t,s)=\int_a^bk(t,s,\sigma)y(t,\sigma)d\sigma\!-\!\int_a^bl(t,s,\sigma)z(t,\sigma)d\sigma + g(t,s),\\[10pt]
z(t,s)=\int_a^bl(t,s,\sigma)y(t,\sigma)d\sigma\!-\!\int_a^bk(t,s,\sigma)z(t,\sigma)d\sigma + h(t,s) \\[10pt]
\end{array}
\right.
\eqno(4)
$$
при  условии $
y(t,s)=y(s,t), \ z(t,s)=-z(s,t).
%\eqno(27)
$	
на неизвестные функции.

Действительно, система (4) с приведённым дополнительным условием получается из уравнения (1) с применением обозначений
$
y(t,s)={1\over 2}(x(t,s)+x(s,t)), z(t,s)={1\over 2}(x(t,s)-x(s,t)),
$
$
g(t,s)\!=\!{1\over 2}(f(t,s)\!+\!f(s,t)),
$
$
h(t,s)\!={1\over 2}\!(f(t,s)-f(s,t)),
$
$
k(t,s,\sigma)={1\over 2}(m(t,s,\sigma)+m(s,t,\sigma)), l(t,s,\sigma)={1\over 2}(m(t,s,\sigma)-m(s,t,\sigma)).
$
%Тогда $x=y+z, f=g+h, m=k+l, y(t,s)=y(s,t), z(t,s)=-z(s,t), g(t,s)=g(s,t), h(t,s)=-h(s,t), k(t,s,\sigma)=k(s,t,\sigma),
%l(t,s,\sigma)=-l(s,t,\sigma),$ а уравнение (3) записывается в виде системы
%$$
%\left\{
%\begin{array}{l}
%y(t,s)=\int\limits_a^bk(t,s,\sigma)y(t,\sigma)d\sigma-\int\limits_a^bl(t,s,\sigma)z(t,\sigma)d\sigma + g(t,s),\\[10pt]
%z(t,s)=\int\limits_a^bl(t,s,\sigma)y(t,\sigma)d\sigma-\int\limits_a^bk(t,s,\sigma)z(t,\sigma)d\sigma + h(t,s) \\[10pt]
%\end{array}
%\right.
%\eqno(26)
%$$
%интегральных уравнений Фредгольма с параметром $t,$ в которой неизвестная функция удовлетворяет дополнительному условию
%$$
%y(t,s)=y(s,t), \ z(t,s)=-z(s,t).
%\eqno(27)
%$$	



С применением квадратурных формул система (4) с дополнительным условием заменяется системой линейных алгебраических уравнений.
Например, при использовании формулы левых прямоугольников отрезок $[a,b]$ разбивается на $n$ равных частей точками
$t_i=s_i=\sigma_i=a+ih,$ где $h=(b-a)/n, i=0,1,\dots,n,$ а система (4) заменяется системой
$$
\left\{
\begin{array}{l}
y_{ij}(n)=h\Biggl(\sum\limits_{p=0}^{n-1}k_{ijp}y_{ip}(n)-\sum\limits_{p=0}^{n-1}l_{ijp}z_{ip}(n)\Biggr) + g_{ij},\\[10pt]
z_{ij}(n)=h\Biggl(\sum\limits_{p=0}^{n-1}l_{ijp}y_{ip}(n)-\sum\limits_{p=0}^{n-1}k_{ijp}z_{ip}(n)\Biggr) + h_{ij},\\[10pt]
\end{array}
\right.
\eqno(5)
$$
где $k_{ijp}=k(t_i,s_j,\sigma_p), l_{ijp}=l(t_i,s_j,\sigma_p),$ $g_{ij}=g(t_i,s_j),$  $h_{ij}=h(t_i,s_j)$  $(i,j,p=0,1,\dots,n-1).$

Система (5) решается при каждом фиксированном $i=0,1,\dots,n-1,$ т.е. её решение сводится к решению $n$ систем линейных алгебраических уравнений [3]. Так как при каждом фиксированном $t\in[a,b]$ система (4) есть система линейных интегральных уравнений с вполне непрерывными интегральными операторами, то при $n\to\infty$ решение $(y_{ij}^{(n)},z_{ij}^{(n)})$ системы (5) стремится к
$(y_{ij},z_{ij}),$ где $y_{ij}=y(t_i,s_j),z_{ij}=z(t_i,s_j).$

Проверка дополнительного условия сводится к оценке малости числа
$
\delta=\max_{ij}(|y_{ij}-y_{jj}|+|z_{ij}+z_{jj}|.
$
Если $\delta$ достаточно мало, то приближённые  значения решения уравнения (3) в точках $(t_i,s_j) (i,j=0,1,\dots,n-1)$ вычисляются по формуле
$
x(t_i,s_j)=y(t_i,s_j)+z(t_i,s_j) (i,j=0,1,\dots,n-1).
$

Отметим, что непосредственное применение квадратурных формул к уравнению (1) с непрерывными заданными функциями $f(t,s)$ и
$m(t,s,\sigma)$ вызывает трудности, связанные с тем, что оператор $M$ в уравнении (1) не является вполне непрерывным, а известные обоснования метода механических квадратур для интегральных уравнений Фредгольма используют полную непрерывность интегральных операторов, определяющих такие уравнения.

Однако, если $1\not\in\sigma(M^2),$ то метод механических квадратур применяется не к уравнению (1), а к эквивалентному ему обратимому уравнению
$$
x(t,s)=(M^2x)(t,s)+(Mf)(t,s)+f(t,s)
\eqno(6)
$$
с вполне непрерывным интегральным оператором $M^2.$ При этом используется  формула
$$
\int_a^b\int_a^bz(t,s)dtds=\sum_{i=1}^P\sum\limits_{j=1}^Q\gamma_{ijPQ}z(t_i,s_j)+r_{PQ}(z),
\eqno(7)
$$
где $a\leqslant t_1<t_2<\dots<t_P\leqslant b, a\leqslant s_1<s_2<\dots<s_Q\leqslant b.$ Предполагается, что квадратурный процесс (7) сходится, т.е. для любой непрерывной функции $f\in C(D)$  выполняется условие
$
r_{PQ}(z)=\int_a^b\int_a^bz(t,s)dtds\!-\!\sum\limits_{i=1}^P\sum\limits_{j=1}^Q\gamma_{ijPQ}z(t_i,s_j)\to 0$ при
$
P,Q\to\infty.
$

Уравнение (6) запишем в виде
$$
x(t,s)=\int_a^b\int_a^bk(t,s,\sigma,\sigma_1)x(\sigma_1,\sigma)d\sigma_1d\sigma+g(t,s),
\eqno(8)
$$
где $k(t,s,\sigma,\sigma_1)=m(t,s,\sigma)m(\sigma,t,\sigma_1),$ а  функция $g(t,s)$ определяется равенством
$
g(t,s)=\int_a^bm(t,s,\sigma)f(\sigma,t)d\sigma+f(t,s).
$



%Интегральное уравнение (32) с вполне непрерывным оператором $M^2$ однозначно разрешимо в $C(D).$ Для приближенного решения уравнения
%(32)  может быть использован метод механических квадратур, рассмотренный в [9].
%
%Пусть
%$$
%t_p=a+ph\  (p=0,1,\dots ,P,\  a+Ph\le b< (P+1)h),
%$$
%$$
%s_q=a+qg \ (q=0,1,\dots ,Q,\  a+Qg\le b< (Q+1)g).
%^$$
%\noindent
Полагая в (7) $t=t_p,$  $s=s_q$ и заменяя интеграл по формуле
$
\int_a^b\int_a^bk(t_p,s_q,\sigma,\sigma_1)x(\sigma_1,\sigma)d\sigma_1d\sigma\! =\!
\sum\limits_{i=1}^P\sum\limits_{j=1}^Q\gamma_{ijPQ}k_{pqij}x(t_i,s_j)+
$

\noindent
$
r_{pqPQ},
$
\noindent где $k_{pqij}=k(t_p,s_q,t_i,s_j),$ а $r_{pqPQ}$ --- остаток, получим систему, после отбрасывания остатков в уравнениях которой будем иметь систему уравнений для приближенных значений $x_{pq}$ функции $x$ в точках
$(t_p,s_q)\  (p=1,\dots ,P; q=1,\dots ,Q):$
$$
x_{pq}=\sum\limits_{i=1}^P\sum\limits_{j=1}^Q\gamma_{ijPQ}k_{pqij}x_{ij}+g(t_p,s_q)
(p=1,\dots ,P; q=1,\dots ,Q),
\eqno(9)
$$
где $x_{ij}=x(t_i,s_j).$ В силу  [2] справедлива

{\bf Теорема~1.} {\it Пусть выполнены условия:

1) при каждых P и Q коэффициенты $\gamma_{ijPQ}$ формулы (7) положительны и существует такое число $G,$ что $\gamma_{ijPQ}\leqslant G;$

2) процесс (7) сходится;

3) $x_0\in C$ --- решение уравнения (8).

Тогда при достаточно больших $P$ и $Q$ система (9) имеет решение $x_{pq} (p=1,\dots,P; q=1,\dots,Q),$

\noindent
\centerline{
$
\max\limits_{1\leqslant p\leqslant P, 1\leqslant q\leqslant Q}|x_{pq}-x_0(t_p,s_q)|\to 0\ \hbox {при} P,Q\to\infty,
$}

\noindent
а скорость сходимости оценивается неравенствами

\noindent
$
c_1R_{PQ}\leqslant\max\limits_{1\leqslant p\leqslant P, 1\leqslant q\leqslant Q}|x_{pq}-x_0(t_p,s_q)|\leqslant c_2R_{PQ},
$
где $c_1$ и $c_2$ --- положительные постоянные,


\centerline{$
R_{PQ}=\max\limits_{1\leqslant p\leqslant P, 1\leqslant q\leqslant Q}|r_{PQ}(z_{pqPQ})|,
$}

\smallskip
\centerline{$\ z_{pqPQ}(t,s)=k(t_p,s_q,t,s)x_0(t,s).
$}}



%Если при каждых $P$ и $Q$ коэффициенты $\gamma_{pqij}$
%формулы  (33) положительны и  процесс (33) сходится, т.е.
%$r_{pq}$ стремятся к нулю равномерно относительно $p,q$ при $h,g\to 0$ и существует такое число $G,$ что $|\gamma_{pqij}|\le G$ для любой функции $x\in C,$ и если $\tilde x\in C$ --- решение уравнения (2),
%то при достаточно больших $P$ и $Q$ система  (34) имеет решение $x_{p,q}$ $(p=1,\dots ,P; q=1,\dots ,Q)$ и
%$$
%\max\limits_{1\le p\le P, 1\le q\le Q}|x_{p,q}-\tilde x(t_p,s_q)|\to 0 \ \hbox {при}\  P,Q\to\infty.
%$$}

Аналитическое приближение $x_{pq}(t,s)$ к решению $\tilde x(t,s)$ уравнения (8) естественно определить равенством
$$
x_{pq}(t,s)=hg\sum\limits_{i=1}^P\sum\limits_{j=1}^Q\gamma_{ijPQ}k(t,s,t_i,s_j)x_{ij}+g(t,s).
$$

\smallskip \centerline{\bf Литература}\nopagebreak

1. {\it Romanovskij V.I.} Sur une classe d'equations integrales lineares// Acta Math., 1932. - Vol. 59. - P. 99-208.

2. {\it Вайникко Г.М.} Возмущенный метод Галёркина и общая теория приближённых методов для нелинейных уравнений// Журнал вычислительной математики и математической физики, 1967. - Т. 7, № 4. - С. 723-751.

3. {\it Калитвин А.С., Калитвин В.А.} Один метод численном решения интегрального уравнения Романовского двухсвязных марковских цепей// Обозрение прикладной и промышленной математики, 2009. - Т. 16, вып. 1. - С. 115-116.
