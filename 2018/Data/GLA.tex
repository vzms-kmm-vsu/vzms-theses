\begin{center}{ \bf  МНОГОУРОВНЕВОЕ МАТЕМАТИЧЕСКОЕ МОДЕЛИРОВАНИЕ ТЕПЛООБМЕНА В ЖИДКОСТНЫХ КАНАЛАХ ТЕРМОЭЛЕКТРИЧЕСКОГО БЛОКА ОХЛАЖДЕНИЯ\footnote{Работа выполнена при финансовой поддержке Министерства образования и науки Российской Федерации в рамках Федеральной целевой программы (Соглашение №14.577.21.0202, уникальный идентификатор RFMEFI57715X0202).}}\\
{\it Д.Н. Галдин, А.В. Кретинин, Е.Е. Спицына Е.Е., И.В. Рощупкина И.В.} \\
\end{center}
\addcontentsline{toc}{section}{Галдин Д.Н., Кретинин А.В., Спицына Е.Е., Рощупкина И.В.\dotfill}

Моделирование компонента термоэлектрического блока охлаждения, по которому циркулирует жидкость, проводилось с использованием модуля вычислительной гидродинамики Ansys CFX.
В расчёте были использованы следующие модули Ansys Workbench:

-	ANSYS Design Modeler - для работы с трёхмерной геометрической моделью каналов блока охлаждения. Данный модуль является универсальным CAD-редактором с широким набором инструментов для создания новой геометрии, а также для разбиения и упрощения импортированной геометрии. Данный модуль в своей основе имеет ядро Parasolid, обладает надежным, отказоустойчивым генератором геометрии и соответствует производственным стандартам.

-	ANSYS Meshing – для генерации сеточной модели по исходной модели проточной части. Модуль содержит широкий набор методов, удовлетворяющих специфическим требованиям той или иной области физики, а также мощный функционал управления глобальными параметрами и локальными сгущениями расчетной сетки (функции автоматического изменения размеров; локальные измельчения сетки по ребру, поверхности, в объеме; переменная плотность сетки).

-	Ansys CFX - для расчётов гидрогазодинамики.

-	Parameter Set – для задание входных параметров и сбора результатов расчётов.
Исходная модель моделируемого блока охлаждения представлена на рисунке 1. Для отображения внутренних компонентов из модели исключён внешний корпус. Блок охлаждения состоит из элементов Пельтье, жидкостного теплообменника и воздушного теплообменника.

\begin{figure}
	\centering
	\includegraphics[width=0.9\textwidth]{gla1.jpg}
	\caption{Модель блока охлаждения}
\end{figure}

Далее приведен пример моделирования жидкостного теплообменника с граничными условиями для одной точки из плана вычислительного эксперимента. Моделирование проводилось в стационарной постановке. В качестве граничных и начальных условий в моделируемом варианте постановки задачи использованы следующие значения:

- на входе в проточную часть (Рисунок 2) задавался массовый расход жидкости 0,01861 кг/с;
температура жидкости задавалась равной 46,7~$^\circ$С;

- на выходе из проточной части – среднее статическое давление 2 атм;

- температура на поверхностях теплообмена задавалась равной 30~$^\circ$С.

Суммарная площадь поверхности теплообмена 0,184433 м$^2$.

\begin{figure}
	\centering
	\includegraphics[width=0.9\textwidth]{gla2.jpg}
	\caption{Граничные условия на входе в проточную часть}
\end{figure}

В результате моделирования полученное среднее значение коэффициента теплоотдачи 369,082 [Вт/(м2•К)]. Распределение значения коэффициента теплоотдачи по поверхности теплообмена представлено на рисунке 3.

\begin{figure}
	\centering
	\includegraphics[width=0.9\textwidth]{gla3.jpg}
	\caption{Распределение значения коэффициента теплоотдачи по поверхности теплообмена}
\end{figure}


На основе обработки экспериментальных данных получены значения коэффициента теплоотдачи во всех точках плана эксперимента, которые обеспечивают приемлемую точность определения основных параметров функционирования ТЭМО. На основе технологии Response Surface модуля ANSYS DesignXplore установлено, что коэффициент теплоотдачи от жидкости в стенку зависит от 4 параметров: \textit{I} - силы тока установку, $T_{f} \cdot T_{x} \cdot T_{g} $ - средних температур жидкости, холодной и горячей сторон ТЭМО на данном режиме.

Получена зависимость $\alpha _{6} (I,T_{f} ,T_{x} ,T_{g} )$, которая обеспечивает точность не менее 1.5 \% по определению коэффициента теплоотдачи от жидкости на холодной стороне ТЭМО.

\[\begin{array}{l} {\alpha _{6} =-25.981755+73.2158396\cdot T_{x} -22.644839\cdot T_{f} +2.61022986\cdot T_{x}^{2} +} \\ {+1.28292285\cdot T_{f}^{2} -4.1723389\cdot T_{x} \cdot T_{f} +74.2391083\cdot I-25.621\cdot T_{g} -} \\ {-0.14647315\cdot I^{2} +0.104869597\cdot T_{g}^{2} -2.34\cdot I\cdot T_{f} +1.76037\cdot I\cdot T_{x} -} \\ {-0.49132\cdot I\cdot T_{g} +0.56715222\cdot T_{f} \cdot T_{g} -0.96986872\cdot T_{x} \cdot T_{g} +} \\ {+0.015616455\cdot I\cdot T_{f} \cdot T_{x} +0.017657\cdot I\cdot T_{f} \cdot T_{g} -0.00252416\cdot I\cdot T_{x} \cdot T_{g} +} \\ {+0.005895081\cdot T_{x} \cdot T_{f} \cdot T_{g} -0.000297\cdot I\cdot T_{f} \cdot T_{x} \cdot T_{g} } \end{array}\]

Здесь \textit{I} - сила тока на всю установку, а $T_{f} \cdot T_{x} \cdot T_{g} $ - средние температуры жидкости, холодной и горячей сторон ТЭМО на данном режиме.

Данная зависимость может быть использована в расчетных моделях теплообмена, полученных в виде критериальных соотношений Nu=\textit{f}(Re,Pr), что в дальнейшем позволяет проводить расчеты без привлечения инструментария ANSYS. При процедуре моделирования большое значение имеет выбор уровня «точности». Быстрый расчет, основанный на одномерных методах и эмпирических данных, позволяет оперативно реагировать на изменения на системном уровне, однако имеет невысокую точность. С другой стороны, полноценные 3D-методы обеспечивают более высокий уровень точности и снижение уровня эпистемических неопределенностей, связанных с инструментами моделирования. Ключевым моментом, определяющим эффективность процесса управления неопределенностями, является применение высокоточных методов моделирования, основанных на совокупности применяемых физических законов с последующим их использованием при выполнении оптимизации в рамках робастного проектирования. В связи с этим все рабочие процессы в ТЭБО моделировались с использованием математических моделей самого высокого иерархического уровня на основе инструментария платформы ANSYS Workbench, а затем результаты моделирования обобщались с использованием технологии Response Surface до вида аппроксимационных полиномиальных зависимостей.

\smallskip \centerline{\bf Литература}\nopagebreak

1. {\it Kamil Lubikowski, Stanislaw Radkowskia, Krzysztof Szczurowskia, Michal Wikarya} Seebeck phenomenon, calculation method comparison.Journal of Power Technologies 95 (Polish Energy Mix), 2015. - 63–67.

2. {\it D. Enescu, E.O. Virjoghe, M. Ionel, M.F. Stan} Electro-thermal analysis of peltier cooling using FEM.Scientific Bulletin of the Electrical Engineering Faculty – Year 10 No. 1 (12)

3. Development of an experimental and analytical model of an active cooling method for high-power three-dimensional integrated circuit (3d-ic) utilizing multidimensional configured thermoelectric modules, HUY NGOC PHAN, PhD Thesis, the university of Texas at Arlington, 2011.
