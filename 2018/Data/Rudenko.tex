\begin{center}
\textbf{Третья краевая задача о колебаниях струны с пружиной}
\end{center}
\addcontentsline{toc}{section}{Руденко А.}


Известно, что решение многих задач из курса физики напрямую зависит от владения аппаратом математического анализа. Так, например, и уравнения колебания струны, которые рассматриваются как в математическом анализе, так и в курсе физики, но с разными подходами к их решению.Существует несколько способов решения этих уравнений.В курсе математического анализа же решение уравнений колебания струны легко решить методом Даламбера. Изучению задач управления распределенными системами и их оптимизации посвящены работы многих математиков, среди которых особо можно отметить публикации Е.И. Моисеева, В.А.Ильина, Л.Н. Знаменской, А.И. Егорова и т.д. Основная цель исследования - это получение условий, при которых процесс колебаний распределенной системы под воздействием некоторого граничного локального или нелокального управления может быть переведен из одного состояния, которое задано начальными смещениями и скоростями системы, в наперед заданное финальное состояние.

Сегодня в теории граничной управляемости появились новые направления, открытые работами В.А.Ильина - конструктивная управляемость, где не просто обосновывается существование управления, а предъявляется его явная формула и относительная управляемость, когда полной управляемости нет, но она возникает при выполнении некоторых выписываемых явно отношений между параметрами задачи.
