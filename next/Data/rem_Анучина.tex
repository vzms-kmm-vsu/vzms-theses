
\vzmstitle[]{
	О разрешимости самосопряжённой краевой
	задачи на линейно упорядоченном множестве
}
\vzmsauthor{Анучина}{Ю.\,А.}
\vzmsinfo{Воронеж, ВГУ; {\it Julia.anuchina05@yandex.ru}}

\vzmscaption


{Рассмотрим отрезок $\left[ 0,l\right]  $, $0<l<+\infty  $,
вещественной оси. Пусть задан набор из $n+2 $
точек отрезка: $a_{0} =0<a_{1} <a_{2} <...<a_{n} <a_{n+1} =l $. Введём
обозначение $\Im _{n} =\cup  _{i=0}^{n} \left( a_{i} ,a_{i+1} \right)  $.}

{Пусть заданы функции $p_{i} \left( x\right)  $, $i=\overline{0,2}  $,
причём каждая из которых соответственно принадлежит
пространствам $2-i $  непрерывно дифференцируемых
функций на объединении интервалов $\Im  $, а функция
 $f\left( x\right)  $  "--- пространству непрерывных функций,
то есть $f\in C\left( \Im \right)  $.}

{Рассмотрим самосопряжённую краевую задачу
для дифференциального уравнения}
$$
\left( \left( p_{0} \left( x\right) u'' \left( x\right) \right) ^{\prime } -p_{1} \left( x\right) u' \left( x\right) \right) ^{\prime } +p_{2} \left( x\right) u\left( x\right) =f\left( x\right) ,\eqno(1)
$$
заданного на $\Im  $  при граничных условиях
$$
	\alpha _{k1}^{\delta } u''' \left( \delta \right) +\alpha _{k2}^{\delta } u'' \left( \delta \right) +\alpha _{k3}^{\delta } u' \left( \delta \right) +\alpha _{k4}^{\delta } u\left( \delta \right) =0,
	\quad
	k=1,2,\eqno(2)
$$
{где $\delta  $, соответственно, принимает значения
0 или $l $, и условиях согласования}
\setcounter{equation}{2}
\begin{multline}
	\alpha _{k1}^{a_{\eta } } d^{3} u\left( a_{\eta } -0\right) +\alpha _{k2}^{a_{\eta }  } d^{3} u\left( a_{\eta } +0\right) +\alpha _{k3}^{a_{\eta } } d^{2} u\left( a_{\eta } -0\right) +
	\\+
	\alpha _{k4}^{a_{\eta } } d^{2} u\left( a_{\eta } +0\right) +
	\alpha _{k5}^{a_{\eta } } u' \left( a_{\eta } -0\right) +\alpha _{k6}^{a_{\eta } } u' \left( a_{\eta } +0\right) +
	\\+
	\alpha _{k7}^{a_{\eta } } u\left( a_{\eta } -0\right) +\alpha _{k8}^{a_{\eta } } u\left( a_{\eta } +0\right) =0
	,
	\quad
	k=\overline{1,4},
	%\eqno{(3)}
\end{multline}
{заданных в каждой внутренней вершине $a_{\eta }  $,
 $\eta =\overline{1,n}  $, где $d^{3} u\left( x\right) =\left( p_{0} \left( x\right) u'' \left( x\right) \right) ^{\prime } -p_{1} \left( x\right) u' \left( x\right)  $,
 $d^{2} u\left( x\right) =-p_{0} \left( x\right) u'' \left( x\right)  $.}


{Нашей целью является нахождение критерия
невырожденности данной краевой задачи. А именно,
будут интересовать условия на коэффициенты
 $\alpha  $  граничных условий и условий согласования.}

{Для простоты изложения будем полагать,
что $n=1 $, то есть задана всего лишь одна внутренняя
точка $a $.}

{И граничные условия, и условия согласования
являются самосопряжёнными условиями. Самосопряжённые
условия, приведённые к каноническому виду,
могут иметь несколько различных видов: в граничных
точках три вида и девять во внутренней вершине.}

{Приступим к рассмотрению вопросов о невырожденности
краевых задач. Получим условия невырожденности
для трёх случаев. Для остальных невырожденность
доказывается аналогично.}

{\textbf{Теорема 1.}}{\textit{ Пусть в точке
 $x=0 $  и $x=l $  канонические условия (2) имеют вид}}
$$
\left\{
\begin{array}{c}
u\left( \delta \right) =0, \\
u' \left( \delta \right) =0,
\end{array}
\right. \eqno(4)
$$
{\textit{где $\delta  $, соответственно, принимает
значения 0 или $l $, или канонические условия
в точке $x=a $  имеют вид}}
$$
\left\{
\begin{array}{c}
u\left( a-0\right) =0. \\
u\left( a+0\right) =0. \\
u' \left( a-0\right) =0. \\
u' \left( a+0\right) =0.
\end{array}
\right. \eqno(5)
$$

{\textit{Тогда краевая задача (1)-(3) является
невырожденной на $\Im  $  при любых условиях, соответственно,
в точке $x=a $  или в точках $x=0 $  и $x=l $.}}

{\textbf{Доказательство. }}{Как известно
[1, с. 130], решение однородной краевой задачи
(1)-(3) (при $f\left( x\right) =0 $) является многочлен не
выше первой степени, то есть }
$$
\varphi \left( x\right) =\left\{
\begin{array}{cc}
c_{11} x+c_{10} , & x\in \left( 0,a\right)  \\
c_{21} x+c_{20} , & x\in \left( a,l\right)
\end{array}
\right.
$$
{В силу условий (4) или условий (5) решение
на отрезках $\left( 0,a\right)  $  и $\left( a,l\right)  $ }{\textit{
}}{тождественны нулю, что означает невырожденность
краевой задачи. }{\textbf{Теорема доказана.}}

{Рассмотрим ещё два частных случая краевой
задачи для дифференциального уравнения (1) при
граничных условиях (2) и условиях согласования
(3).}

{\textbf{Теорема 2.}}{\textit{ Пусть канонические
условия (2) в точке $x=0 $  имеют вид (4), а в точке
 $x=l $  }}
$$
\left\{
\begin{array}{c}
u' \left( l\right) -\beta _{1}^{l} u\left( l\right) =0 \\
d^{3} u\left( l\right) +\beta _{1}^{l} d^{2} u\left( l\right) +\beta _{2}^{l} u\left( l\right) =0
\end{array}
\right. \eqno(6)
$$
{\textit{и $\beta _{2}^{l} \neq 0 $, тогда краевая задача
(1)-(3) является невырожденной на $\Im  $  при любых
условиях в точке $x=a $.}}


{\textbf{Доказательство. }}{Составим матрицу
из коэффициентов условий. Ранг такой матрицы
равен 2, поэтому задача будет являться невырожденной.
Т}{\textbf{еорема доказана.}}

{\textbf{Теорема 3. }}{\textit{Пусть канонические
условия (2) в точке $x=0 $  имеют вид (4), а в точке
 $x=l $  - (6) и $\beta _{2}^{l} =0 $, тогда краевая задача
(1)-(3) является вырожденной на $\Im  $, если выполняются
следующие условия:}}

{\textit{1) Вектор $\left( \alpha _{11} ,\alpha _{14} ,\alpha _{15} ,\alpha _{16} ,-\beta _{1}^{l} \right)  $
пропорционален вектору $\left( \alpha _{13} ,\alpha _{16} ,\alpha _{18} ,\alpha _{19} ,1-\beta _{1}^{l} \left( l-a\right) \right)  $,
если в точке $x=a $  канонические условия имеют
вид}}
$$
\left\{
\begin{array}{l}
	u' \left( a-0\right) -\alpha _{11} u' \left( a+0\right) -\alpha _{12} u\left( a-0\right) -
		\\\hspace{16.4em}
		-\alpha _{13} u\left( a+0\right) =0, \\
	d^{2} u\left( a+0\right) +\alpha _{11} d^{2} u\left( a-0\right) +\alpha _{14} u' \left( a+0\right) +
		\\\hspace{10em}
		+\alpha _{15} u\left( a-0\right) +\alpha _{16} u\left( a+0\right) =0, \\
	d^{3} u\left( a-0\right) +\alpha _{12} d^{2} u\left( a-0\right) +\alpha _{15} u' \left( a+0\right) +
		\\\hspace{10em}
		+\alpha _{17} u\left( a-0\right) +\alpha _{18} u\left( a+0\right) =0, \\
	d^{3} u\left( a+0\right) +\alpha _{13} d^{2} u\left( a+0\right) +\alpha _{16} u' \left( a+0\right) +
		\\\hspace{10em}
		+\alpha _{18} u\left( a-0\right) +\alpha _{19} u\left( a+0\right) =0,
\end{array}
\right. \eqno(7)
$$
{\textit{2) Вектор $\left( \alpha _{24} ,\alpha _{26} ,\alpha _{28} ,\alpha _{29} ,-\beta _{1}^{l} \right)  $
пропорционален вектору $\left( \alpha _{22} ,\alpha _{25} ,\alpha _{24} ,\alpha _{26} ,1-\beta _{1}^{l} \left( l-a\right) \right)  $,
если в точке $x=a $  канонические условия имеют
вид}}
$$
\left\{
\begin{array}{l}
	d^{2} u\left( a-0\right) +\alpha _{21} u' \left( a-0\right) +\alpha _{22} u' \left( a+0\right) +\alpha _{23} u\left( a-0\right) +
		\\\hspace{16.4em}
		+\alpha _{24} u\left( a+0\right) =0, \\
	d^{2} u\left( a+0\right) +\alpha _{22} u' \left( a-0\right) +\alpha _{25} u' \left( a+0\right) +\alpha _{24} u\left( a-0\right) +
		\\\hspace{16.4em}
		+\alpha _{26} u\left( a+0\right) =0, \\
	d^{3} u\left( a-0\right) +\alpha _{23} u' \left( a-0\right) +\alpha _{24} u' \left( a+0\right) +\alpha _{27} u\left( a-0\right) +
		\\\hspace{16.4em}
		+\alpha _{28} u\left( a+0\right) =0, \\
	d^{3} u\left( a+0\right) +\alpha _{24} u' \left( a-0\right) +\alpha _{26} u' \left( a+0\right) +\alpha _{28} u\left( a-0\right) +
		\\\hspace{16.4em}
		+\alpha _{29} u\left( a+0\right) =0,
\end{array}
\right. \eqno(8)
$$
{\textit{3) Либо $\alpha _{31} =0 $, либо $\alpha _{33} =\alpha _{35} =\alpha _{36} =0 $
и $\beta _{1}^{l} =\frac{1}{l-a}  $, если в точке $x=a $  канонические
условия имеют вид}}
$$
\left\{
\begin{array}{l}
	\!\!\!u\left( a-0\right) -\alpha _{31} u\left( a+0\right) =0, \\
	\!\!\!d^{2} u\left( a-0\right) +\alpha _{32} u'\! \left( a-0\right) +\alpha _{33} u'\! \left( a+0\right) +\alpha _{34} u\left( a+0\right) {=}0, \\
	\!\!\!d^{2} u\left( a+0\right) +\alpha _{33} u'\! \left( a-0\right) +\alpha _{35} u'\! \left( a+0\right) +\alpha _{36} u\left( a+0\right) {=}0, \\
	\!\!\!d^{3} u\left( a+0\right) +\alpha _{31} d^{3} u\left( a-0\right) +\alpha _{34} u' \left( a-0\right) +
	\\\quad
	+
	\alpha _{36} u' \left( a+0\right) +\alpha _{37} u\left( a+0\right) =0,
\end{array}
\right. \eqno(9)
$$
{\textit{4) $\alpha _{41} =0 $, если в точке $x=a $  канонические
условия имеют вид}}
$$
\left\{
\begin{array}{l}
u\left( a-0\right) -\alpha _{41} u\left( a+0\right) =0, \\
u' \left( a-0\right) -\alpha _{42} u\left( a+0\right) =0, \\
u' \left( a+0\right) -\alpha _{43} u\left( a+0\right) =0, \\
d^{3} u\left( a+0\right) +\alpha _{41} d^{3} u\left( a-0\right) +\alpha _{42} d^{2} u\left( a-0\right) +
\\\hspace{6em}
+\alpha _{43} d^{2} u\left( a+0\right) +\alpha _{44} u\left( a+0\right) =0,
\end{array}
\right. \eqno(10)
$$
{\textit{5) $\alpha _{51} =\alpha _{52} =0 $  и $\beta _{1}^{l} =\frac{1}{l-a}  $,
если в точке $x=a $  канонические условия имеют
вид}}
$$
\left\{
\begin{array}{l}
u\left( a-0\right) =0, \\
u\left( a+0\right) =0, \\
u' \left( a-0\right) -\alpha _{51} u' \left( a+0\right) =0, \\
d^{2} u\left( a+0\right) +\alpha _{51} d^{2} u\left( a-0\right) +\alpha _{52} u' \left( a+0\right) =0,
\end{array}
\right. \eqno(11)
$$
{\textit{6) $\alpha _{62} =\alpha _{63} =0 $  и $\beta _{1}^{l} =\frac{1}{l-a}  $,
если в точке $x=a $  канонические условия имеют
вид}}
$$
\left\{
\begin{array}{c}
u\left( a-0\right) =0. \\
u\left( a+0\right) =0. \\
d^{2} u\left( a-0\right) +\alpha _{61} u' \left( a-0\right) +\alpha _{62} u' \left( a+0\right) =0. \\
d^{2} u\left( a+0\right) +\alpha _{62} u' \left( a-0\right) +\alpha _{63} u' \left( a+0\right) =0.
\end{array}
\right. \eqno(12)
$$
{\textit{7) либо $\alpha _{71} =0 $, либо $\beta _{1}^{l} =\frac{1}{l-a}  $
и $2a\alpha _{73} -1=0 $, если в точке $x=a $  канонические
условия имеют вид}}
$$
\left\{
\begin{array}{l}
	u\left( a-0\right) -\alpha _{71} u\left( a+0\right) =0, \\
	u' \left( a+0\right) -\alpha _{72} u' \left( a-0\right) -\alpha _{73} u\left( a+0\right) =0, \\
	d^{2} u\left( a-0\right) +\alpha _{72} d^{2} u\left( a+0\right) +\alpha _{74} u' \left( a-0\right) +
	\\\hspace{15.6em}+
	\alpha _{75} u\left( a+0\right) =0, \\
	d^{3} u\left( a+0\right) +\alpha _{71} d^{3} u\left( a-0\right) +\alpha _{73} d^{2} u\left( a+0\right) +
	\\\hspace{9em}+
	\alpha _{75} u' \left( a-0\right) +\alpha _{76} u\left( a+0\right) =0,
\end{array}
\right. \eqno(13)
$$
{\textit{8) $\alpha _{82} =0 $, если в точке $x=a $  канонические
условия имеют вид}}
$$
\left\{
\begin{array}{l}
u' \left( a-0\right) -\alpha _{81} u\left( a-0\right) -\alpha _{82} u\left( a+0\right) =0, \\
u' \left( a+0\right) -\alpha _{82} u\left( a-0\right) -\alpha _{83} u\left( a+0\right) =0, \\
d^{3} u\left( a-0\right) +\alpha _{81} d^{2} u\left( a-0\right) +\alpha _{82} d^{2} u\left( a+0\right) +
\\\hspace{8em}+
\alpha _{84} u\left( a-0\right) +\alpha _{85} u\left( a+0\right) =0, \\
d^{3} u\left( a+0\right) +\alpha _{82} d^{2} u\left( a-0\right) +\alpha _{83} d^{2} u\left( a+0\right) +
\\\hspace{8em}+
\alpha _{85} u\left( a-0\right) +\alpha _{86} u\left( a+0\right) =0.
\end{array}
\right. \eqno(14)
$$
{\textbf{Доказательство. }}{Пусть условия
в точке $x=a $  имеют вид (7). }

{Составим матрицу из коэффициентов условий
и приведём её к треугольному виду. Если задача
вырожденная, то строки приведённой матрицы
пропорциональны и последняя строка будет равна
нулю. Тогда вектор $\left( \alpha _{11} ,\alpha _{14} ,\alpha _{15} ,\alpha _{16} ,-\beta _{1}^{l} \right)  $
пропорционален вектору $\left( \alpha _{13} ,\alpha _{16} ,\alpha _{18} ,\alpha _{19} ,1-\beta _{1}^{l} \left( l-a\right) \right)  $
с коэффициентом пропорциональности $k$. }

{Для случаев, когда условия в точке $x=a $
имеют вид (8)-(14), доказательство аналогично.
}{\textbf{Теорема доказана.}}


\litlist


{1. Завгородний М. Г. Краевые задачи для дифференциальных
уравнений на графе: учебник / М. Г. Завгородний,
С. П. Майорова // Воронеж. гос. ун-т. "--- Воронеж:
Издательский дом ВГУ, 2015 - 147 с.}




{Анучина Юлия Алексеевна, аспирант математического
факультета Воронежского государственного
университета}

