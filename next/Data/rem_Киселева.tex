\vzmstitle[]{
	О смешанной задаче для системы Дирака
}
\vzmsauthor{Киселева}{А.\,В.}
\vzmsinfo{Воронеж, ВГУ; {\it kiseleva@math.vsu.ru }}

\vzmscaption



\paragraph{Аннотация.}
В   работе   рассматривается смешанная задача для  системы
уравнений в частных производных, исследование которой приводит к
одномерной системе Дирака.
Приводится точное решение задачи в случае нулевого потенциала и схема решения задачи в произвольном случае.

\paragraph{Ключевые слова:}
смешанная задача, метод Фурье, система Дирака


\subsection{ Задачи для системы Дирака }

Рассмотрим следующую смешанную задачу:
 \begin{equation}\label{JVM18-Burl_eq1}
\frac{\partial u(x,t)}{\partial t}=B\frac{\partial u(x,t)}{\partial
x}+Q(x)u(x,t),
\end{equation}
$$  x\in [0,1], \quad t\in (-\infty ,+\infty),$$
\begin{equation}\label{JVM18-Burl_eq2}
     {{u}_{1}}(0,t)={{u}_{2}}(0,t),   \qquad {{u}_{1}}(1,t)={{u}_{2}}(1,t),
     \end{equation}
\begin{equation}\label{JVM18-Burl_eq3}
     u(x,0)=\varphi (x),
\end{equation}
где  $u(x,t)={{\left( {{u}_{1}}(x,t),{{u}_{2}}(x,t) \right)}^{T}}$,
$\varphi (x)={{({{\varphi }_{1}}(x),{{\varphi }_{2}}(x))}^{T}}$ ($T$
"--- знак транспонирования),  ${{u}_{j}}(x,t)$ и ${{\varphi
}_{j}}(x)$ скалярные функции,\\ $B=\left( \begin{matrix}
   1 & 0  \\
   0 & -1  \\
\end{matrix} \right)$,  $Q(x)=\left( \begin{matrix}
   0 & {{q}_{2}}(x)  \\
   -{{q}_{1}}(x) & 0  \\
\end{matrix} \right)$,  ${{q}_{j}}(x)\in C[0,1]$, все функции  комплекснозначные.

Уравнение в \eqref{JVM18-Burl_eq1} представляет собой простейшую
систему уравнений в частных производных. Однако интересна такая
система тем, что она порождает (например, при применении метода
Фурье в  данной задаче)   систему обыкновенных дифференциальных
уравнений, которую называют системой Дирака. Такие системы
достаточно активно исследуются (см. \cite{Burl2}--\cite{Burl6}).
Возможные приложения приводятся далее.


\subsection{ Волновое уравнение  и система Дирака}

Важнейшим уравнением математической физики является волновое
уравнение
$$\dfrac{{{\partial }^{2}}u(x,t)}{\partial
{{t}^{2}}}=\dfrac{{{\partial }^{2}}u(x,t)}{\partial
{{x}^{2}}}-q(x)u(x,t),$$ и начально"=краевые задачи для него.

Оказывается, что между задачей для волнового уравнения и изучаемой
нами задачей можно установить эквивалентность \cite{burl-VSU19-Kis}.

Рассмотрим, например,    такую смешанную задачу:
\begin{equation}\label{g5-sv-eq1}
\begin{array}{c}
\dfrac{{{\partial }^{2}}v(x,t)}{\partial
{{t}^{2}}}=\dfrac{{{\partial }^{2}}v(x,t)}{\partial
{{x}^{2}}}-q(x)v(x,t), \quad x\in [0,1], \  t\in [0,+\infty),
\\
v(0,t)=v(1,t)=0, \quad  \\
 v(x,0)=\varphi (x),  \quad {{{v}'_{t}}}(x,0)=0,\quad x\in [0,1],
 \\
 \end{array}
\end{equation}
Если уравнение в \eqref{g5-sv-eq1}   представить в виде
$$
  \left(\frac{\partial}{\partial t }+ \frac{\partial}{\partial x
}\right)\left(\frac{\partial}{\partial t }- \frac{\partial}{\partial
x }\right)v(x,t)=-q(x)v(x,t),
$$
то, обозначив  $u_1(x,t)=v(x,t)$,  $u_2(x,t)=
\left(\frac{\partial}{\partial t }- \frac{\partial}{\partial x
}\right)v(x,t)$,  отсюда   придём к системе
$$ \begin{array}{l}
\dfrac{\partial u_1(x,t)}{\partial t}= \dfrac{\partial
u_1(x,t)}{\partial x}+u_2(x,t),   \\
\dfrac{\partial u_2(x,t)}{\partial t}= -\dfrac{\partial
u_2(x,t)}{\partial x}-q(x)u_1(x,t),
 \end{array}
$$
 решение которой должно удовлетворять начальным и
краевым условиям:
$$\begin{array}{c}
u_1(0,t)=u_1(1,t)=0, \quad     \\
 u_1(x,0)= {\varphi} (x),   \quad u_2(x,0)=-\varphi' (x), \quad x\in
 [0,1],
\end{array}$$
Таким образом, получаем: если $v(x,t)$ есть  решение задачи
\eqref{g5-sv-eq1}, то\\
   $u(x,t)=(u_1(x,t), u_2(x,t))^T$  является решением задачи
$$
	\begin{array}{c}
		\dfrac{\partial u(x,t)}{\partial t}=
		B\dfrac{\partial u(x,t)}{\partial x}+Q(x)u(x,t),
		\ \ \, x\in [0,1],\  t\in [0,+\infty),
	\\
	u_1(0,t)=u_1(1,t)=0,
	\\
	u  (x,0)= \widetilde{\varphi} (x),     \quad x\in [0,1],
	\end{array}
$$
 где
$B={\rm diag}(1,-1)$, $$Q(x)=\begin{pmatrix}
  0 & 1 \\
  -q(x) & 0 \\
\end{pmatrix}, \ \ \widetilde{\varphi} (x)=({\varphi} (x),-\varphi' (x))^T.$$







\subsection{Пример модели, приводящей к системе Дирака }


Рассмотрим модель, приводящую к системе дифференциальных уравнений
рассматриваемого типа и описывающей  взаимодействие встречных
световых пучков. Предположим, что надо определить распределение
интенсивности оптического излучения в пространстве между источником
(в нашем случае это будет лазер) и
зеркалом, заполненном некоторой средой. % (смотреть рисунок).
Будем считать, что от зеркала отражается $R$-я часть падающего
излучения (т.~е., его коэффициент отражения равен $R$), а среда как
поглощает излучение с коэффициентом ослабления $a (x)$, так и
рассеивает его. Причём коэффициент излучения света рассеяния при
движении в обратную сторону равен $r (x)$. В этом случае закон
изменения интенсивности $y_{0} (x)$ излучения, распространяющегося
вправо, и интенсивности $y_{1} (x)$ излучения влево определяется
системой двух обыкновенных дифференциальных уравнений первого
порядка:
\begin{eqnarray} \label{eq00-1}
  & & { \frac{dy_{0} (x)}{dx} = -a(x) y_{0} (x) + r(x)
    y_{1}(x),}\\
   \nonumber
  & & {\frac{dy_{1} (x)}{dx} = a(x) y_{1} (x) - r(x)
y_{0}(x). }
\end{eqnarray}

Для правильной постановки задачи требуется помимо уравнений задать
такое же количество граничных условий. Одно из них будет выражать
известную интенсивность излучения $I_{0}$, падающего с левой
границы, когда $x=0$, а второе "--- закон отражения на его правой
границе, когда $x=1$:
\begin{eqnarray} \label{eq01-1}%\label{eq01}
 & & { y_{0} (0) = I_{0},}\\
 \nonumber
 & & { y_{1} (1) = R y_{0} (1). }
\end{eqnarray}

%Рисунок - модель для постановки краевой задачи.




\textit{Замечание.} Модель, которая здесь представлена, % на рисунке,
приводит к краевой задаче для системы линейных обыкновенных
дифференциальных уравнений. Она будет нелинейной, если сделать
коэффициенты ослабления и рассеивания зависящими от интенсивности
излучения. Физически это будет соответствовать изменению оптических
свойств среды под действием мощного излучения.

Если, считать %использовали следующие значения параметров: когда
$R=1$,  $a(x)=1$ и $r(x)=const$, то мы приходим к изучаемой далее
системе Дирака.

 Если рассматривать более сложные эффекты
рассеивания в стороны, то мы придём к уравнениям в частных
производных.

Отметим также, что уравнение вида
$$\frac{\partial u(x,t)}{\partial t}+a(x,t)\frac{\partial u(x,t)}{\partial
x}=0,
$$
называемое одномерным уравнением переноса и описывающее изменение
скалярной величины в пространстве и времени, является модельным
уравнением в механике сплошных сред.




\subsection{ Смешанная задача для системы Дирака в случае нулевого потенциала}

Рассмотрим сначала смешанную задачу (1)--(3) в случае $Q(x)=0$:
\begin{equation}\label{eq01}
 \frac{\partial u(x,t)}{\partial t}= B\frac{\partial u(x,t)}{\partial x},
\end{equation}
 \begin{equation}
 \label{02}
 u_{1}(0,t)=u_{2}(0,t), \quad u_{1}(1,t)=u_{2}(1,t),
\end{equation}
\begin{equation}
 \label{03}
\quad u(x,0)= \varphi(x),
\end{equation}
%
 $u(x,t)=\begin{pmatrix}
  u_{1}(x,t) \\
  u_{2}(x,t) \\
\end{pmatrix}$, \quad $B=\begin{pmatrix}
  1 & 0 \\
  0 & -1 \\
\end{pmatrix}$, \quad
$\varphi(x)=\begin{pmatrix}
  \varphi_{1}(x)\\
  \varphi_{2}(x)\\
\end{pmatrix}$.

Под классическим решением понимаем  непрерывно дифференцируемую
вектор"=функцию $u=(u_{1}(x,t), u_{2}(x,t))^T$, которая
 удовлетворяет уравнению (\ref{eq01}) и условиям (\ref{02})--(\ref{03}).
Для существования классического решения  требуем выполнение
следующих необходимых условий:
\begin{equation}\label{fi-usl}
\begin{array}{c}
\varphi_{1}(x), \varphi_{2}(x) \in C^1[0,1], \quad
\varphi_{1}(0)=\varphi_{2}(0),  \quad \varphi_{1}(1)=\varphi_{2}(1),
\\
\varphi'_{1}(0)=- \varphi'_{2}(0),\quad
 \varphi'_{1}(1)=- \varphi'_{2}(1).
\end{array}
\end{equation}


Уравнение (\ref{eq01}) является простейшей системой двух уравнений в
частных производных:
\begin{equation*}
\begin{cases}
   u'_{1t}(x;t) = u'_{1x}(x;t)\\
   u'_{2t}(x;t) = -u'_{2x}(x;t)
 \end{cases}.
\end{equation*}
 каждое из которых содержит только одну
неизвестную функцию.  Однако, решения этих уравнений связаны между
собой условием (\ref{02}).

Решение  системы (\ref{eq01}) есть
\begin{equation} \label{eq05}
 \begin{cases}
   u_{1}(x;t) = g_{1}(x+t),\\
   u_{2}(x;t) =g_{2} (x-t), \\
 \end{cases}
\end{equation}
где $g_{1}(x), g_{2}(x)$ "--- любые непрерывно дифференцируемые функции. %, являющиеся решениями системы (\ref{eq01}).

Подставляя теперь (\ref{eq05}) в (\ref{03}), (\ref{02}), получим,
что решением задачи \eqref{eq01}--\eqref{03} являются функции
\begin{equation*}
   u_{1}(x;t) = g_{1}(x+t), \quad u_{2}(x;t) =g_{2} (x-t),
\end{equation*}
где $g_{1}(x)$ и $g_{2}(x)$ непрерывно дифференцируемые
2-периодические функции, причём
$$g_{1}(x)=
\begin{cases}
    \varphi_{1}(x), \text{при } x\in [0,1],\\
     \varphi_{2} (-x), \text{при } x\in [-1,0],\\
 \end{cases}
$$
$$
g_{2}(x)=
\begin{cases}
    \varphi_{2}(x), \text{при } x\in [0,1],\\
     \varphi_{1} (-x), \text{при } x\in [-1,0].\\
 \end{cases}
 $$
Отметим, что    $g_2(x-t)=g_1(t-x)$, и решение принимает вид:
\begin{equation}\label{reshD}
u (x,t)=(g_1(x+t), g_1(t-x))^T.
\end{equation}

Для дальнейших исследований, решим теперь данную задачу методом
Фурье. Введём замену:
\begin{equation}\label{zamena1}
   u_{1}(x,t)=y_{1}(x)T(t), \quad u_{2}(x,t)=y_{2}(x)T(t).
\end{equation}
Подставив в   уравнение \eqref{eq01}, мы получим:
\begin{equation*}
   \frac{T'(t)}{T(t)}= \frac{y'_{1}(x)}{y_{1}(x)}=\lambda, \qquad
\frac{T'(t)}{T(t)}=
-\frac{y'_{2}(x)}{y_{2}(x)}=\lambda.\end{equation*} Из условий
\eqref{02} имеем:
\begin{equation*}
   y_{1}(0)= y_{2}(0), \quad y_{1}(1)= y_{2}(1),
\end{equation*}
\begin{equation*}
y_{1}(x)T(0)= \varphi_{1}(x),\quad y_{2}(x)T(0)= \varphi_{2}(x).
\end{equation*}


Таким образом, мы получаем для $T(t)$ уравнение с разделяющимися
переменными
$$T'(t)=\lambda T(t),$$ решение которого есть $T(t)=c e^{\lambda t}$, а для вектор"=функции\\ $y(x)=(y_1(x), y_2(x))^T$  спектральную
задачу:
\begin{equation}\label{sz1}
y'_1(x)=\lambda y_1(x), \quad y'_2(x)=-\lambda y_2(x),
\end{equation}
\begin{equation}\label{sz1-ku}
   y_{1}(0)= y_{2}(0), \quad y_{1}(1)= y_{2}(1).
\end{equation}
Решением уравнений \eqref{sz1} являются функции
$$
   y_{1}(x)= c_{1}e^{\lambda x   }, \quad y_{2}(x)= c_{2} e^{-\lambda x  }.
$$
 Подставляя найденные решения в \eqref{sz1-ku}, мы получим:
$$c_{1}=c_{2}, \quad  c_{1}e^{\lambda    }=c_{2}e^{-\lambda    }, $$
откуда имеем уравнение на собственные значения
$$ e^{\lambda} = e^{-\lambda    }   \quad \Rightarrow  \quad  e^{2\lambda   }=1 \quad \Rightarrow  \quad 2\lambda=2\pi
k i, \ k\in \mathrm{Z}.
   $$
Таким образом, мы получили решение спектральной задачи
\eqref{sz1}--\eqref{sz1-ku}: собственными значениями являются числа
$\lambda_k={\pi k i}, \  k\in \mathrm{Z}$, а собственные функции
есть $ y_{k}(x)=\left(y_{1k}(x), y_{2k}(x)\right)^T= \left(e^{\pi
kix}, e^{-\pi kix}\right)^T$.



Тогда $ T_{k}(t)=c_{k}e^{\pi i kt},$ и подставляя найденные значения
в формулу~\eqref{zamena1}, получим:
\begin{equation*}
   u_{1}(x;t)= \sum\limits_{k=-\infty}^{+\infty}c_{k}e^{\pi kt}e^{\pi
   kix},\quad
   u_{2}(x;t)= \sum\limits_{k=-\infty}^{+\infty}c_{k}e^{\pi kt}e^{-\pi kix}.\\
\end{equation*}
Из начального условия \eqref{03} при $t=0$ и $x\in[0;1]$   имеем:
 \begin{equation}\label{fi01}
        \varphi_{1}(x)= \sum\limits_{k=-\infty}^{+\infty}c_{k}e^{\pi
        kix}, \quad
    \varphi_{2}(x)= \sum\limits_{k=-\infty}^{+\infty}c_{k}e^{-\pi
    kix}
    \end{equation}
Чтобы найти коэффициенты $c_{k}$, мы должны $\varphi_k(x)$ разложить
в ряд по системе $\{e^{\pi kix}\}$. Известно, что это
экспоненциальная форма тригонометрической  системы, и она является
ортогональной в пространстве $L_2[-1,1]$~\cite{Bari}. Но на отрезке
$[0;1]$   система $\{e^{\pi kix}\}$ не ортогональна.


{\bf Определение. } {\it Система ${y_{k}(x)}$ "--- ортогональна на
отрезке $x\in[a;b]$ в пространстве функций $L_{2} [a;b]$, если
выполнено:
\begin{eqnarray*}
 &  (y_{k};y_{m})=\int\limits^b_a{y_{k}}(x){ \overline{y_{m}}(x)}dx=0 \ \ (k \neq m)  & \\
  & (y_{k};y_{k}) =
   \parallel y_{k}\parallel^2=a \neq 0. &
\end{eqnarray*}
}

{\bf Определение. } {\it Рядом Фурье функции ${\varphi (x)}$ по
ортогональной системе $\{y_{k}(x)\}$ называется ряд
$\sum\limits_{k=-\infty}^{+\infty}c_{k} y_{k}(x),$ где
$c_k=(\varphi,y_{k})/(y_{k},y_{k})$ "--- коэффициенты Фурье.

}

Покажем    ортогональность системы вектор"=функций $\{y_{k}(x)\} $ на
отрезке~$[0;1]$. Будем обозначать $\langle y_n,y_k\rangle$
"--- скалярное произведение   вектор"=функций
$y_{k}(x)=(y_{1k}(x),y_{2k}(x))^T$, которое вычисляется по формуле
\\
$\langle y_n,y_k \rangle
=(y_{1n}(x),y_{1k}(x))+(y_{2n}(x),y_{2k}(x))$, где $(\cdot,\cdot)$
означает скалярное произведение в $L_2$. Имеем:
\begin{multline*}
\langle y_{n}(x),y_{k}(x)\rangle=(e^{ \pi nix}; e^{ \pi
kix})+(e^{-\pi nix}; e^{- \pi kix}) = \\ = \int\limits^1_0{e^{ \pi
nix}e^{- \pi kix}}dx +\int\limits^1_0{e^{ -\pi nix}e^{+\pi kix}}dx =
\\ = \int\limits^1_0{e^{ \pi nix}e^{- \pi kix}}dx +\int\limits_{-1}^0{e^{
 \pi nix}e^{-\pi kix}}dx =
\\=
  \int\limits_{-1}^1  e^{ \pi (n-k)ix} dx = 0, \ \
  k \neq n,
\end{multline*}
$$
   \langle y_{n}(x),y_{k}(x)\rangle=\int\limits_{-1}^1   e^{ \pi (k-k)ix} dx = 2.
$$

Итак, на отрезке $[0,1]$ наши собственные функции образуют
ортогональную систему, и из \eqref{03} (см. также \eqref{fi01})
получаем
$$\begin{pmatrix}
   \varphi_1(x) \\
    \varphi_2(x) \\
\end{pmatrix}= \sum\limits_{k=-\infty}^{+\infty}c_{k} \begin{pmatrix}
   y_{1k}(x) \\
    y_{2k}(x) \\
\end{pmatrix},  $$
а для коэффициентов Фурье:
\begin{multline*}
	c_k=
	\frac{\langle \varphi, y_{k}(x)\rangle}{\langle y_{k}(x), y_{k}(x)\rangle}
	= \frac12\left(\left(\varphi_1,y_{1k}\right) +
	\left(\varphi_2,y_{2k}\right)\right)=
	\\=
	\frac12\left( \int\limits_{0}^{1} {\varphi_1}(x)e^{-\pi
	k i x}dx+ \int\limits_{0}^{1} {\varphi_2}(x)e^{ \pi
	k i x}dx\right) =
	\\=
	\left| \begin{array}{c}
	\text{во втором интеграле }\\ \text{  заменим } x \ \text{на }-x
	\end{array}\right|=
	\\=
	\frac12\left( \int\limits_{0}^{1} {\varphi_1}(x)e^{-\pi
	k i x}dx + \int\limits_{-1}^{0} {\varphi_2}(-x)e^{ -\pi
	k i x}dx\right) =
	\\=
	\frac12\int\limits_{-1}^{1} {\widetilde{\varphi} }(x)e^{-\pi
	k i x}dx,
\end{multline*}
где
%%%%%%%%%% конец правок
%Введём замену
%$\varphi_{2}(-x)=\sum\limits_{k=-\infty}^{+\infty}c_{k}e^{\pi kix}$,
%тогда получим, что;
 \begin{equation}\label{phi}
    \begin{matrix}
    \tilde{\varphi}(x) & =
    \left\{
    \begin{matrix}
    \varphi_{1}(x), x\in [0;1]\\
    \varphi_{2}(-x), x\in [-1;0]\\
    \end{matrix} \right.
    \end{matrix}
\end{equation}
Отсюда мы видим, что $c_k$ есть коэффициенты Фурье скалярной функции
$\widetilde{\varphi}(x)$  по системе
 {$\{e^{\pi kix}\}$} на отрезке
$[-1;1]$. Продолжая $\widetilde{\varphi}(x)$ периодически с периодом
2 на всю ось, получаем, что сумма ряда
  $ \sum\limits_{k=-\infty}^{+\infty}c_{k}e^{\pi
kix}$ при всех $x\in (-\infty;+\infty)$ есть
$\widetilde{\varphi}(x)$. Тогда
\begin{equation*}
    \begin{matrix}
    \left\{
    \begin{matrix}
    \sum\limits_{k=-\infty}^{+\infty}c_{k}e^{\pi
kix}= \widetilde{\varphi}(x)=\varphi_{1}(x), x\in [0;1]\\
   \sum\limits_{k=-\infty}^{+\infty}c_{k}e^{\pi
kix}=\widetilde{\varphi}(x)=\varphi_{2}(-x), x\in [-1;0]\\
    \end{matrix} \right.
    \end{matrix}
\end{equation*}
%Из вышенайденного следует, что
%$\sum\limits_{k=-\infty}^{+\infty}c_{k}e^{-\pi kix}=\varphi_{2}(x)$.

Теперь получим для решения исходной задачи:
\begin{equation*}
   u_{1}(x;t)= \sum\limits_{k=-\infty}^{+\infty}c_{k}e^{\pi kt}e^{\pi kix}= \sum\limits_{k=-\infty}^{+\infty}c_{k} e^{\pi ki(x+t)}= \widetilde{\varphi}(x+t),\\
\end{equation*}
\begin{equation*}
   u_{2}(x;t)= \sum\limits_{k=-\infty}^{+\infty}c_{k}e^{\pi kt}e^{-\pi kix}=\sum\limits_{k=-\infty}^{+\infty}c_{k}e^{\pi k(t-x)}
   =\widetilde{\varphi}(-x+t),\\
\end{equation*}
что совпадает с \eqref{reshD} (т.е. $\widetilde{\varphi}(x)$
совпадает с $g_1(x)$ из \eqref{reshD}).

\subsection{ Решение задачи  в случае ненулевого потенциала}



Задача \eqref{JVM18-Burl_eq1}--\eqref{JVM18-Burl_eq3} и её решение
методом Фурье  в случае   $Q(x)\not \equiv 0$ рассматривалась в
\cite{Burl-SGU-16,burl-JVM19}. Большие трудности возникают с
нахождением формул для собственных значений и собственных функций, с
доказательством гладкости полученного решения (см., например
\cite{Burl-SGU-16}). Приведём схему решения этой задачи.



В [1] предполагаются выполненными     следующие условия:
\begin{equation}\label{JVM18-Burl-eq4}
 \begin{array}{c}
\varphi_j(x) \text{--- абсолютно непрерывны}, \ \  \varphi_j' \in
L_2[0,1], \quad j=1,2,\\
\varphi_1(0)=\varphi_2(0), \quad \varphi_1(1)=\varphi_2(1).
 \end{array}
\end{equation}



При этих условиях под классическим решением задачи понимается
 вектор"=функция  \linebreak
$u(x,t)= (u_{1}(x,t), u_{2}(x,t))^{T} $, компоненты которой
абсолютно непрерывны по $x$ и $t$, и которая удовлетворяет уравнению
\eqref{JVM18-Burl_eq1} почти всюду, и
условиям~\eqref{JVM18-Burl_eq2}--\eqref{JVM18-Burl_eq3}.

\subsection{Спектральная задача}\label{Burl_sec1}

    Метод Фурье  для задачи \eqref{JVM18-Burl_eq1}--\eqref{JVM18-Burl_eq3}
связан со спектральной задачей для оператора $L$:
    $$(Ly)(x)=B{y}'(x)+Q(x)y(x),$$
    $${{y}_{1}}(0)={{y}_{2}}(0), \quad   {{y}_{1}}(1)={{y}_{2}}(1),$$
где  $y(x)={{({{y}_{1}}(x),{{y}_{2}}(x))}^{T}}$. Оператор $L$ есть
оператор Дирака с условиями Дирихле.

Для реализации метода Фурье требуется найти собственные значения и
собственные функции оператора $L$, что представляет собой непростую
задачу. Обычно для них удаётся найти лишь асимптотические формулы. В
\cite{Burl7}
  доказано, что собственные значения оператора $L$,
достаточно большие по модулю, простые с асимптотикой
    $${{\lambda }_{n}}=\pi ni+{{\beta }_{n}}, \quad   (n=\pm {{n}_{0}},\pm ({{n}_{0}}+1),\ldots ),$$
а для собственных функций
${{y}_{n}}(x)={{({{y}_{n1}}(x),{{y}_{n2}}(x))}^{T}}$ оператора $L$
имеют место асимптотические формулы:
\begin{multline}\label{Burl_eq4}
	{{y}_{nj}}(x)={{e}^{{{p}_{j}}\pi nix}}\left( 1+{{\beta }_{n}}x
	\right)+\int\limits_{0}^{x}{b(x,\tau ){{e}^{\pi ni\tau }}d\tau}+
	\\+
	\int\limits_{0}^{x}{b(x,\tau ){{e}^{-\pi ni\tau }}d\tau
	}+
	%\\+
	{{\beta }_{n}}\int\limits_{0}^{x}{b(x,\tau ){{e}^{\pi ni\tau
	}}d\tau }
	+
	\\+
	{{\beta }_{n}}\int\limits_{0}^{x}{b(x,\tau ){{e}^{-\pi
	ni\tau }}d\tau }
		+O(\beta _{n}^{2}),
\end{multline}
  ($j=1,2$), где $n=\pm {{n}_{0}},\pm ({{n}_{0}}+1),\ldots $,
${{p}_{1}}=1$, ${{p}_{2}}=-1$, и оценка $O(\ldots )$ равномерна по
$x\in [0,1]$ (здесь и в дальнейшем через ${{\beta }_{n}}$ обозначаем
различные числа  такие, что $\sum{{{\left| {{\beta }_{n}}
\right|}^{2}}}<\infty $). Норма $\left\| \cdot  \right\|$ есть либо
норма в ${{L}_{2}}[0,1]$, либо в пространстве $L_{2}^{2}[0,1]$
вектор"=функций размерности~2. Через $b(x,t)$ будем обозначать
различные непрерывные функции из некоторого конечного набора.





\textbf{Замечание.} В \cite{Burl7}  приводятся точные выражения для
${{\beta }_{n}}$ и $b(x,t)$, но здесь они не требуются, а вид
\eqref{Burl_eq4} функций ${{y}_{nj}}$ упрощает рассуждения.

Сопряжённый оператор ${{L}^{*}}$ есть
    $$({{L}^{*}}z)(x)=-B{z}'(x)+{{Q}^{*}}(x)z(x),$$
    $${{z}_{1}}(0)={{z}_{2}}(0), \quad  {{z}_{1}}(1)={{z}_{2}}(1),$$
где  $z(x)={{({{z}_{1}}(x),{{z}_{2}}(x))}^{T}}$,
${{Q}^{*}}(x)=\left( \begin{matrix}
   0 & {{{\bar{q}}}_{1}}(x)  \\
   -{{{\bar{q}}}_{2}}(x) & 0  \\
\end{matrix} \right)$.


Для собственных функций
${{z}_{n}}(x)={{({{z}_{n1}}(x),{{z}_{n2}}(x))}^{T}}$ оператора
${{L}^{*}}$ имеют место асимптотические формулы:
\begin{multline*}
    {{z}_{nj}}(x)=
    %\\=
    {{e}^{{{p}_{j}}\pi nix}}\left( 1+{{\beta }_{n}}x \right)
    +\int\limits_{0}^{x}{b(x,\tau ){{e}^{\pi ni\tau }}d\tau }+
    \\+
    \int\limits_{0}^{x}{b(x,\tau ){{e}^{-\pi ni\tau }}d\tau }+
    %\\+
    {{\beta }_{n}}\int\limits_{0}^{x}{b(x,\tau ){{e}^{\pi ni\tau }}d\tau }+
    \\+
    {{\beta }_{n}}\int\limits_{0}^{x}{b(x,\tau ){{e}^{-\pi ni\tau }}d\tau }+O(\beta _{n}^{2}),
\end{multline*}
 ($j=1,2$), где $n=\pm {{n}_{0}},\pm ({{n}_{0}}+1),\ldots $, ${{p}_{1}}=1$, ${{p}_{2}}=-1$, (${{\beta }_{n}}$ и $b(x,t)$ другие, отличные от \eqref{Burl_eq4}).


Как и в \cite{Burl-SGU-16} доказывается, что множество с.п.ф.
операторов $L$ и $L^*$ полны в $L_2^2[0,1]$.





На базе этих асимптотик  в [1] доказано, что справедливы
асимптотические формулы:
$$({{y}_{n}},{{z}_{n}})=2+{{\beta }_{n}},$$
$$(\varphi ,{{z}_{n}})= {{\beta }_{n}}+{{\beta }_{n}^2},$$
\begin{multline*}
	\frac{(\varphi ,{{z}_{n}})}{({{y}_{n}},{{z}_{n}})}{{y}_{nj}}(x){{e}^{{{\lambda }_{n}}t}}=
	\\=
	{{\beta }_{n}}\left[ {{e}^{{{p}_{j}}\pi nix}}+
	\int\limits_{0}^{x}{b(x,\tau ){{e}^{\pi ni\tau }}d\tau
	}+\int\limits_{0}^{x}{b(x,\tau ){{e}^{-\pi ni\tau }}d\tau }
	\right]e^{\pi nit}+
	\\+
	O({{\beta }_{n}^2})
\end{multline*}
 ($j=1,2$),
где $x\in [0,1]$, $t\in [-T,T]$, $T>0$ любое фиксированное число, и
оценка $O(\ldots )$ равномерна по $x$ и $t$. Здесь   $(\cdot ,\cdot
)$ "--- скалярное произведение в $L_{2}^{2}[0,1]$ (это же обозначение
сохраняется и для скалярного произведения в ${{L}_{2}}[0,1]$).



Дальнейшее исследование основано на представлении формального
решения задачи \eqref{JVM18-Burl_eq1}--\eqref{JVM18-Burl_eq3} по
методу Фурье   в виде (см.~\cite{burl-JVM19,Vagabov94})
\begin{equation}\label{Burl_eq5}
u(x,t)=-\frac{1}{2\pi i}\left( \int\limits_{|\lambda |=r}{{}}+
    \sum\limits_{n\geqslant {{n}_{0}}}{\int\limits_{{{\gamma }_{n}}}{{}}} \right)
    ({{R}_{\lambda }}\varphi )(x){{e}^{\lambda t}}d\lambda ,
\end{equation}
где $r>0$ достаточно велико и фиксировано, \\
${{R}_{\lambda }}={{(L-\lambda E)}^{-1}}$ "--- резольвента оператора
$L$  ($E$ "--- единичный оператор, $\lambda $ "--- спектральный
параметр), \\
${{\gamma }_{n}}=\left\{ \lambda : |\lambda -\lambda
_{n}^{0}|=\delta  \right\},$ $\lambda _{n}^{0}=\pi ni$, $\delta
>0$ и достаточно мало, чтобы собственные значения ${{\lambda }_{n}}$
попадали по одному внутрь ${{\gamma }_{n}}$. \\
При этом
$$-\frac{1}{2\pi i}\int\limits_{{{\gamma }_{n}}}{({{R}_{\lambda }}\varphi ){{e}^{\lambda t}}d\lambda  }=\frac{(\varphi
,{{z}_{n}})}{({{y}_{n}},{{z}_{n}})}{{y}_{n}}(x){{e}^{{{\lambda
}_{n}}t}}.$$




\subsection{Классическое решение}\label{Burl_sec3}



В этом пункте получим классическое решение в предположении, что
$\varphi (x)\in C^1[0,1]$.

{\bf Лемма. } {\it Пусть $\mu_0$ не является собственным значением
оператора~$L$, $|\mu_0|>r$, и находится вне $\gamma_n$, $\varphi
(x)\in {{D}_{{{L}^{2}}}}$,  $g=(L-\mu_0 E)^2\varphi$. Тогда
$$
  \int\limits_{|\lambda |=r}
        ({{R}_{\lambda }}\varphi )(x){{e}^{\lambda t}}d\lambda =
  \int\limits_{|\lambda |=r} \frac{(R_{\lambda}g) (x)}{(\lambda-\mu_0)^2}{{e}^{\lambda t}
  }d\lambda,
$$
$$
 \frac{(\varphi ,{{z}_{n}})}{({{y}_{n}},{{z}_{n}})}=
 \frac{(g ,{{z}_{n}})}{(\lambda_n-\mu_0)^2({{y}_{n}},{{z}_{n}})}.
$$
 }

\textsl{ Доказательство.}  Утверждение следует из представления,
справедливого для любой функции $\varphi\in D_{L^2}$:
$$ R_{\lambda}\varphi = - \frac{\varphi}{\lambda-\mu_0} -\frac{g_1}{(\lambda-\mu_0)^2}
+\frac{R_{\lambda}g}{(\lambda-\mu_0)^2}, $$ где
 $g_1=(L-\mu_0 E)\varphi$.~$\Box$



{\bf Теорема. } {\it  Если $\varphi (x)\in C^1[0,1]$ и удовлетворяет
краевым условиям в~\eqref{JVM18-Burl-eq4}, то формальное решение
$u(x,t)$ является классическим, т.е. $u(x,t)$ непрерывно
дифференцируема по $x$ и $t$ и удовлетворяет
\eqref{JVM18-Burl_eq1}--\eqref{JVM18-Burl_eq3}. }

 \textsl{
Доказательство.}
 По приведённой выше лемме и   из \eqref{Burl_eq5}
имеем
\begin{equation}\label{Burl_eq7}
    u(x,t)=-\frac1{2\pi i}\!\!\int\limits_{\,\,|\lambda |=r}\!\!\! \frac{(R_{\lambda}g) (x)}{(\lambda-\mu_0)^2}{{e}^{\lambda t}
  }d\lambda +\!\! \sum_{|n|\geqslant n_0}
 \frac{(g
 ,{{z}_{n}})y_n(x)}{(\lambda_n-\mu_0)^2({{y}_{n}},{{z}_{n}})}e^{\lambda_nt}.
\end{equation}
Отсюда следует, что ряд \eqref{Burl_eq7} и ряды, получающиеся из
него почленным дифференцированием по $x$ и $t$ равномерно сходятся в
$Q_T$ при любом $T>0$. Далее, из  \eqref{Burl_eq5} имеем при $t=0$:
\begin{equation}\label{Burl_eq8}
 u(x,t)=-\frac1{2\pi i}\int\limits_{|\lambda |=r} (R_{\lambda}\varphi)
 (x)d\lambda + \sum_{|n|\geqslant n_0}
 \frac{(\varphi
 ,{{z}_{n}})y_n(x)}{ ({{y}_{n}},{{z}_{n}})}.
\end{equation}
Правая часть \eqref{Burl_eq8} представляет собой ряд Фурье функции
$\varphi(x)$, абсолютно и равномерно сходящийся к функции
$\varphi(x)$, следовательно $u(x,t)$ удовлетворяет
\eqref{JVM18-Burl_eq3}. Условия \eqref{JVM18-Burl_eq2}
очевидны.~$\Box$







\subsection{ О модификации метода Фурье}

Исследование исходной задачи можно провести с помощью представленной
в работах А.П. Хромова \cite{Kh1,Kh2}  модификации метода Фурье,
которая заключается в следующем:

1) используется представление формального ряда Фурье исходной задачи
не в виде ряда по собственным функциям (а во многих случаях кроме
собственных функций есть и присоединённые), а в виде контурного
интеграла от резольвенты спектрального оператора (в нашем случае это
представление \eqref{Burl_eq5}); при этом контуры ${\gamma }_{n}$
могут быть выбраны так, что они содержат только одно или только
близкие серии собственных значений (если собственные значения будут
кратными)

2) используется преобразование формального  ряда:
\begin{equation}\label{Burl_eq5-1}
\begin{array}{c}
u(x,t)=U_0(x,t)+U_1(x,t), \quad \text{где }\\
 U_0(x,t)=-\frac{1}{2\pi i}\left(
\int\limits_{|\lambda |=r}{{}}+
    \sum\limits_{n\geqslant {{n}_{0}}}{\int\limits_{{{\gamma }_{n}}}{{}}} \right)
    ({{R}_{\lambda }^0}\varphi )(x){{e}^{\lambda t}}d\lambda, \\
 U_1(x,t)=-\frac{1}{2\pi i}\left( \int\limits_{|\lambda |=r}{{}}+
    \sum\limits_{n\geqslant {{n}_{0}}}{\int\limits_{{{\gamma }_{n}}}{{}}} \right)
    ({{R}_{\lambda }}\varphi- {R}_{\lambda }^0 \varphi)(x){{e}^{\lambda
    t}}d\lambda.
    \end{array}
\end{equation}
Здесь $R_{\lambda}^0=(L_0-\lambda E)^{-1}$ "--- резольвента оператора
$L_0$, который получается из $L$ при $q_j(x)=0$. Соответственно,
$U_0(x,t)$ есть формальное решение задачи
\eqref{JVM18-Burl_eq1}--\eqref{JVM18-Burl_eq3} в случае $q_j(x)=0$,
  рассмотренной в параграфе 2. Этот ряд суммируется и его сумма выражается в
  явном виде:
$$ U_0(x,t)=(\widetilde{\varphi}(x+t),\widetilde{\varphi}(-x+t))^T,
$$
 где $\widetilde{\varphi}(x)$ получается из функции ${\varphi}(x)$  продолжением на всю числовую ось   с помощью соотношений
\eqref{phi}.


3) В отличие от трудоёмкого анализа возможности почленного
дифференцирования ряда $U_1(x,t)$, аналогично тому, как это
приводится в параграфе 3, проводится следующий итерационный процесс.
Строится новая задача
\begin{equation}\label{JVM18-Burl_eq1-m}
\frac{\partial u(x,t)}{\partial t}=B\frac{\partial u(x,t)}{\partial
x}+Q(x)u(x,t) + F_0(x,t),
\end{equation}
\begin{equation}\label{JVM18-Burl_eq2-m}
     {{u}_{1}}(0,t)={{u}_{2}}(0,t),   \qquad {{u}_{1}}(1,t)={{u}_{2}}(1,t),
     \end{equation}
\begin{equation}\label{JVM18-Burl_eq3-m}
     u(x,0)=0,
\end{equation}
где $F^0(x,t)= Q(x)U_0(x,t)$, $(x,t)\in Q=[0,1]\times\in
(-\infty,+\infty )$.

Формальное решение этой задачи снова разбивается на сумму двух
рядов, и тем самым  представляется в виде:
$$
U_{1}(x,t) = A_1 (x,t)+ U_2(x,t),$$
$$\text{где }
 A_{1}(x,t)=\frac{1}{2}
\int\limits_0^t\,d\tau\int\limits_{x-t+\tau}^{x+t-\tau}
 \widetilde{F_0}(\eta,\tau)\,d\eta, $$
$\widetilde{F}_0(x,\tau)={F}_0(x,\tau)$ при $x\in [0,1]$, и на всю
числовую ось продолжается с помощью соотношений \eqref{phi},
$U_{2}(x,t)$
"--- решение задачи  \eqref{JVM18-Burl_eq1-m}--\eqref{JVM18-Burl_eq3-m} с~$F_1(x,t)=
Q(x)A_1(x,t)$ вместо $F^0(x,t)$.

Продолжая этот процесс до бесконечности, и выполняя необходимые
обоснования, аналогично работам \cite{Kh1, Kh2}, получим, что
справедлив следующий результат.

{\bf Теорема 2.} {\it Классическое решение задачи
\eqref{JVM18-Burl_eq1}--\eqref{JVM18-Burl_eq3} существует и имеет
вид
$$ u(x,t)=A(x,t)=\sum_{n=0}^{\infty}A_n(x,t),$$
$$
\text{где }\quad
A_{0}(x,t)=U_0(x,t)=(\widetilde{\varphi}(x+t),\widetilde{\varphi}(-x+t))^T,
$$
$$A_n(x,t)=\frac{1}{2} \int\limits_0^t\,d\tau\int\limits_{x-t+\tau}^{x+t-\tau}
 \widetilde{F}_{n-1}(\eta,\tau)\,d\eta,\,\,\, n\geqslant 1$$
 и $F_{n}(x,t)= Q(x)A_n(x,t)$ при $x\in[0,1]$, $\widetilde{F}_{n-1}$ есть продолжения
 функций $F_{n-1}$ с помощью соотношений \eqref{phi}.}



Отметим, что приём разбиения ряда на два, один из которых быстро
сходится и сумма которого вычисляется в точном виде, а второй
допускает почленное дифференцирование, называется приёмом ускорения
сходимости рядов (см. А.Н. Крылов, \cite{Krylov1}). Эти приёмы
используются в том числе и для численного решения различных задач
(см., например, \cite{Ch1,Ch2}).

\begin{thebibliography}{99}


\bibitem{Burl-SGU-16} Бурлуцкая М.Ш. Смешанная задача для системы дифференциальных
уравнений первого порядка с непрерывным потенциалом~/ М.Ш. Бурлуцкая
// Изв. Сарат. ун-та. Нов. сер. Сер. Математика. Механика.
Информатика. "--- 2016. "--- Т. 16, вып. 2. "---  С.~145--151.

\bibitem{burl-JVM19}  Бурлуцкая М.Ш. Классическое и обобщённое  решение смешанной задачи для   системы уравнений первого порядка  с непрерывным потенциалом
/ М.Ш. Бурлуцкая // Журнал вычислительной математики и
математической физики. "--- 2019. "--- Т. 59, № 3. "--- С. 380--390.

\bibitem{burl-VSU19-Kis} Бурлуцкая М.Ш. О некоторых свойствах дифференциальных уравнений и
смешанных задач  с инволюцией / М.Ш. Бурлуцкая //  Вестник Воронеж.
гос. ун-та. Сер.: Физика. Математика. "---   2019. "--- № 1. "---
с.~60--68.

\bibitem{Bari}   Бари Н.К. Тригонометрические ряды / Н.К. Бари  "--- М.: Физматгиз,  1961. "--- 936~с.


\bibitem{Vagabov94}  Вагабов А.И. Введение в спектральную теорию дифференциальных  операторов / А.И. Вагабов. "---  Ростов"=на-Дону:
 Изд-во Ростовского госуниверситета, 1994. "--- 160 с.


\bibitem{Burl2}  Джаков П.В.  Зоны неустойчивости одномерных  периодических операторов  Шрёдингера и Дирака / П.В. Джаков,  Б.С. Митягин
// Успехи мат. наук. "--- 2006. "--- Т. 61,  № 4. "--- С. 77--182.


\bibitem{Burl3} Djakov P.  Bari-Markus property for Riesz projections  of 1D periodic Dirac operators /  P. Djakov, B. Mityagin //
 Math. Nachr.  "--- 2010. "--- 283:3.  "--- P.  443--462.



\bibitem{Burl4}
  Баскаков А.Г. Метод подобных  операторов в спектральном анализе несамосопряженного оператора
 Дирака с негладким потенциалом / А.Г. Баскаков, А.В. Дербушев, А.О.  Щербаков // Изв. РАН. Серия матем.  "---  2011.  "---  Т.  75,  № 3.  "--- С. 3--28.



\bibitem{Burl5}   Савчук А.М. Асимптотические формулы для фундаментальных решений  системы Дирака с комплекснозначным суммируемым потенциалом
 /  А.М. Савчук, И.В. Садовничая //  Дифференциальные уравнения. "--- 2013. "--- Т. 49, № 5. "--- С. 573--584.

\bibitem{Burl6} Savchuk A.M. Dirac operator with complex-valued summable  potential /  A.M. Savchuk, A.A. Shkalikov //  Mathematical Notes. "---
2014. "--- V.~96, № 5. "--- P. 777--810.

\bibitem{Burl7}    Бурлуцкая М.Ш.   Уточнённые  асимптотические формулы для собственных значений и собственных  функций системы
Дирака / М.Ш.~Бурлуцкая, В.П. Курдюмов, А.П. Хромов // Доклады
Академии наук.
"--- 2012. "--- Т. 443, № 4. "---  С.~414--417.




\bibitem{Kh1}   Хромов А. П.
      Необходимые и достаточные условия существования классического решения смешанной задачи для однородного
    волнового уравнения в случае суммируемого потенциала~/   А.П. Хромов
    //
    Диф. уравнения.~--- 2019.~--- Т.~55, №~5.~--- С.~717--731.

\bibitem{Kh2}  Корнев В. В.
    Классическое и обобщённое решения смешанной задачи
    для неоднородного волнового уравнения~/ В. В. Корнев, А. П. Хромов //
    ЖВМ и МФ ~--- 2019.~--- Т.~59, №~2.~--- С.~286--300.

\bibitem{Krylov1}       Крылов А.Н.  О некоторых дифференциальных уравнениях  математической физики, имеющих приложения в технических вопросах / А.Н.
Крылов. "---
 ГИТТЛ. Л, 1950. "--- 368 с.


\bibitem{Ch1}  Чернышов  А.Д.  Метод быстрых разложений для решения нелинейных
дифференциальных уравнений /  А.Д. Чернышов  // Журнал вычислит.
математики и матем. физики.  "--- 2014. "--- Т.54,  №1. "--- С. 13--24.

\bibitem{Ch2}  Чернышов  А.Д.  Аналитическое решение задачи об
изгибе упругой консольной балки методом быстрых разложений /
  А.Д.Чернышов, В.В.  Горяйнов // Вестник ЧГПУ им. И.Я. Яковлева Серия:
Механика предельного состояния. "--- 2014 "--- №3(21). "--- С.96--104.



\end{thebibliography}




Сведения об авторе:

\noindent {\bfseries Киселева Анна Викторовна},
студент математического факультета Воронежского государственного университета
