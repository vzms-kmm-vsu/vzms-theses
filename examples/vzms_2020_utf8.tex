\documentclass{vzmsthesis}

\begin{document}


\vzmstitle[
	\footnote{Работа выполнена при моральной поддержке Урюпинского районного военкомата}
]{
	Название статьи
}
\vzmsauthor{Иванов}{И.\,И.}
\vzmsinfo{Воронеж, ВГУ; {\it ivanoff@example.org}}
\vzmsauthor{Петров}{П.\,П.}
\vzmsinfo{Урюпинск, УрГУ; {\it tractorist\_petr@example.org}}

\vzmscaption


Текст статьи


Объем статей "--- 3 страницы (для лекторов "--- до 5 стр.)
в редакторе LaTeX по образцу, который можно скачать на
сайте \href{https://vzms.kmm-vsu.ru}{vzms.kmm-vsu.ru}.
Принимаются работы на русском и английском языках.
При наборе текста на английском языке просьба использовать команду
\\\verb`\selectlanguage{english}`

\paragraph{Требования к вёрстке статей.}
При оформлении текста тезисов просим не переопределять команды и не
вводить свои макросы, а также не использовать автоматическую
нумерацию теорем, лемм и пр.
Допускается автоматическая нумерация формул, рисунков и таблиц
при условии, что в имени ссылки используется не менее 5 букв из фамилии автора
(например, \verb`\label{Ivanov_eq1}`).
Для нумерации формул допускается также использовать команду \verb"\eqno".
Автоматическая нумерация библиографии допускается в случае большого количества источников,
но в целом не приветствуется.

Утверждения типа теорем и лемм следует оформлять по следующему
образцу.

\paragraph{Теорема~1.} {\it
	Пример оформления определений и утверждений типа теорем, лемм.
}

\begin{center}
	Файл должен компилироваться без ошибок \\
	и переполнений (<<overfull>>).
\end{center}

По умолчанию используется кодировка \verb`utf-8`.
Для выбора кодировки \verb`cp1251` в начале файла следует указать:
\verb`\inputencoding{cp1251}`


\begin{center}
	\textbf{Электронную версию тезисов необходимо выслать по электронному адресу vzms@mail.ru.}
\end{center}

% Оформление списка литературы
\litlist

1. {\it Крейн С.Г.} Линейные дифференциальные уравнения в банаховом пространстве. М.: Наука, 1987. 408 с.

2. {\it Львовский С. М.} Набор и верстка в системе \LaTeX. М.: МЦНМО, 2003. — 448 с.

\end{document}
