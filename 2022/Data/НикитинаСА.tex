\vzmstitle[
	%\footnote{Работа выполнена за счёт гранта РНФ, проект}
]
{
	Об одном способе анализа состояния материальных запасов предприятия на основе нечёткой информации о затратах
}
\vzmsauthor{Никитина}{С.\,А.}
\vzmsinfo{Челябинск, ЧелГУ; {\it nikitina@csu.ru}}

\vzmscaption

Построена система анализа, которая позволяет сделать вывод об уровне материальных запасов на предприятии на основе нечёткой информации о затратах.

Рассмотрим лингвистическую переменную $H$"--- <<уровень материальных запасов>>. Значения (термы) этой лингвистической переменной следующие:
$H_{1}$ "--- <<Очень высокий уровень материальных запасов>>;
$H_{2}$ "--- <<Высокий уровень материальных запасов>>;
$H_{3}$ "--- <<Средний уровень материальных запасов>>;
$H_{4}$ "--- <<Низкий уровень материальных запасов>>;
$H_{5}$ "--- <<Очень низкий уровень материальных запасов>>.


Каждый терм $H_{j}, j = 1, \ldots, 5$ является нечётким числом, заданным на отрезке $[0; 1]$. Будем рассматривать в качестве значений лингвистической переменной $H$ трапециевидные нечёткие числа.

Изложим метод по шагам.

Шаг 1. Отобрать показатели $D_{1}, D_{2}, \ldots, D_{n}$ деятельности складской системы предприятия, которые имеют наибольшее отношение к оценке уровня запаса. При выборе показателей нужно учитывать следующее условие [1]: рост показателя должен приводить к снижению уровня запасов.

Согласно [2] на уровень запаса на складе значительное влияние оказывают затраты на хранение, на выполнение заказа (организацию) и на потери от дефицита товара. При увеличении доли указанных затрат в суммарных затратах, происходит снижение финансовых вложений в приобретение продукции. Следствием этого является уменьшение величины пополнения товаров, то есть происходит снижение уровня материальных запасов.

В качестве показателей будем рассматривать следующие:
$D_{1}$ "--- доля затрат на хранение;
$D_{2}$ "--- доля затрат на выполнение заказа;
$D_{3}$ "--- доля затрат на потери от дефицита.

Пусть $d_{i}, i = 1, \ldots, 3$ "--- значения уровня показателя $i$ в текущий период времени.

Шаг 2.
Определить, какие из отобранных показателей являются наиболее, а какие наименее значимыми при влиянии на уровень запаса материальных ресурсов. Расположить показатели по убыванию их значимости.
Присвоить показателям весовые коэффициенты $\omega_{i}, i = 1, \ldots, n$ в зависимости от их значимости.
Весовые коэффициенты можно определить по правилу Фишберна [3].
Если выбранные показатели не могут быть расположены в порядке убывания их значимости, то весовые коэффициенты для них выбирают одинаковыми.

Шаг 3.
Фаззифицировать выбранные показатели, то есть ввести лингвистические переменные.
Предположим, что терм"=множество этих лингвистических переменных содержит следующие значения: <<Очень низкий показатель, Низкий показатель, Средний показатель, Высокий показатель, Очень высокий показатель>>. Каждому значению лингвистической переменной необходимо сопоставить функцию принадлежности. Для этого удобно применить систему нечётких чисел трапециевидного типа, удовлетворяющую условию серой шкалы Поспелова [4].

Шаг 4.
Установить уровень принадлежности $i$-го показателя к $j$-ой лингвистической классификации по результатам значений имеющихся входных параметров $d_{i}$. Обозначим этот уровень принадлежности через $\mu_{ij}, i = 1, \ldots, n,  j = 1, \ldots 5$.

Шаг 5.
Найти интегральный показатель уровня материальных запасов
\begin{center}
$ h = \sum_{j = 1}^{5} h_{j} \sum_{i = 1}^{n} \omega_{i} \cdot \mu_{ij},$
\end{center}
где $h_{j} = 0,9 - 0,2(j - 1)$ "--- середина промежутка носителя терма $H_{j}, j = 1, \ldots 5$.

Из приведённой формулы следует, что  интегральный показатель уровня материальных запасов есть результат двумерной свёртки показателей принадлежности, с двумя системами весов – весов для показателей и весов характерных точек для заданной  лингвистической классификации. Чем больше значение этого показателя, тем ниже уровень запасов.



\litlist

1. {\it Батыршин~И.~З., Недосекин~А.~О., Стецко~А.~А. и др.} Нечёткие гибридные системы. Теория и
практика. //М: ФИЗМАТЛИТ, 2007. — 208 с.

2. {\it Шрайбфедер~Дж.} Эффективное управление запасами. //М: Альпина Бизнес Букс, 2008. — 304 с.

3. {\it Абдулаева~З.~И., Недосекин~А.~О.} Стратегический анализ инновационных рисков. //СПб: Изд. СПбГГУ, 2013. — 146 с.

4. {\it Поспелов Д.~А.} «Серые» и/или «черно"=белые» // Прикладная эргономика. Специальный выпуск  «Рефлексивные процессы». – 1994. – Т.1. – C. 29–33.
