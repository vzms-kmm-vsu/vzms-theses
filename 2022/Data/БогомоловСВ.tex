\vzmstitle[
	%\footnote{Работа выполнена за счёт гранта РНФ, проект 16-11-10125}
]{
	Стохастическое уравнение Больцмана и газодинамическая иерархия
}
\vzmsauthor{Богомолов}{С.\,В.}
\vzmsinfo{Москва, МГУ; {\it bogomo@cs.msu.ru}}
\
\vzmscaption



Ясная с точки зрения физики вероятностная модель газа из твёрдых сфер рассматривается как с помощью теории случайных процессов, так и в терминах классической кинетической теории для плотностей функций распределения в фазовом пространстве: из системы нелинейных  стохастических дифференциальных уравнений (СДУ) выводится сначала обобщённое, а затем "--- случайное и неслучайное интегро"=дифференциальное уравнение Больцмана с учётом корреляций и флуктуаций. Главной особенностью исходной модели является случайный характер интенсивности скачкообразной меры и её зависимость от самого процесса.

Для полноты картины  отметим, что переход ко всё более грубым приближениям  в соответствии с уменьшением параметра обезразмеривания,  числа Кнудсена. В результате получаются стохастические и неслучайные мезоскопические (Колмогорова--Фоккера--Планка), а затем "--- макроскопические уравнения, отличающиеся от  системы уравнений Навье--Стокса. Ключевым отличием этого вывода является  более точное осреднение по скорости  благодаря  аналитическому решению стохастических дифференциальных уравнений по винеровской мере, в виде которых  представлена промежуточная мезо"=модель в фазовом пространстве. Такой подход существенно отличается от традиционного, использующего не сам случайный процесс, а его функцию распределения.

Рассмотрим систему из $N$ взаимодействующих частиц. В качестве наглядного примера возьмём газ из неустанно перемещающихся твёрдых сфер, взаимодействие между которыми происходит благодаря упругим столкновениям. Чтобы избежать чрезмерной информации  о нашей системе, выражаемой положениями и скоростями входящих в неё частиц, самым продуктивным способом её исследования является привлечение вероятностных представлений.


Положения и скорости частиц будем считать случайными величинами, что совершенно оправдано по  физическим соображениям.
Модель газа, молекулы которого представляются абсолютно упругими шариками, является простейшей, но не
тривиальной. Людвиг Больцман выводил своё
уравнение, опираясь на этот образ и начиная с детерминированной системы, вводя случайность на этапе принятия гипотезы молекулярного хаоса – \foreignlanguage{ngerman}{Stossanzahlanzatz}.
А.В. Скороход изначально рассматривает системы, состоящие из большого числа случайно взаимодействующих
частиц, и исследует поведение таких систем при неограниченном возрастании их числа.

Начнём с формализации \textbf{столкновений} частиц, которые порождают процессы диффузии, вязкости и теплопроводности.  Изменение скорости сталкивающихся частиц называется функцией скачка, которая легко получается из решения задачи о столкновении двух твёрдых сфер диаметра $ D $, в качестве которого возьмём диаметр эффективного сечения рассеяния (Рис.1 слева):
%Our starting point is\textbf{Collisions.}
%Jump function for hard sphere collisions is
$\textbf{f}(\textbf{$\upsilon_{i}$},\textbf{$\upsilon_{j}$},\textbf{$\omega$}) = \textbf{$\omega$} (\textbf{$\omega$},\textbf{$\upsilon_{i}$} - \textbf{$\upsilon_{j}$})$.\\
\begin{figure}[h!]%\label{geom_coll}
\centering{\includegraphics[scale=0.2]{StochasticMolecularDynamics5.png}}
	\caption{Геометрия столкновений двух частиц с двумя вариантами вектора $\omega$ (слева) и цилиндр для подсчёта $N_{ij}^{reached}$ (справа)}
 % Collisions with different $\omega$ and (to the right) a cylinder for $N_{ij}^{reached}$ counting\label{fig4}
\end{figure}
% \begin{figure}[h!]
 %	\centering{\includegraphics[scale=1]{StochasticMolecularDynamics1.png}}
 	%\caption{} \label{fig5}
 %\end{figure}

Эволюция набора из $N$ частиц описывается следующей системой  уравнений Стохастической Молекулярной Динамики:
\begin{eqnarray}\label{SMD}
	 dx_{i}(t) = \upsilon_{i}(t) dt ,\ \ \  \nonumber \\
	 d\upsilon_{i}(t) = \sum\limits_{j=1}^N\int \limits_\Omega f(\upsilon_{i},\upsilon_{j},\omega)b_{ij}(d\omega\times dt) %+[v_i(t),B] dt,
	 \\
	 \lambda_{ij} =N_{ij}^{collisions}/N , \ \ \ \ , \ i=1,...,N,\nonumber
  \end{eqnarray}
  где $x_{i}(t)$ "--- положения, и $\upsilon_{i}(t)$ "--- скорости частиц, являются 3{D} случайными процессами,
 $f(\cdot)$ "--- функция скачка, или приращение скорости $\upsilon_{i}$ из-за столкновения с частицей скорости $\upsilon_{j}$;

 $b_{ij}$ "--- считающие (с результатом розыгрыша 0 или 1) независимые бернуллиевские меры (с интенсивностями, или вероятностями выпадения ``1'',  $\lambda_{ij}$, которые  очень малы "--- редкие столкновения, разреженный газ); они указывают на факт столкновения или его отсутствие.



Для подсчёта числа столкновений $N_{ij}^{collisions}$  воспользуемся  (совершенно так же, как и при выводе уравнения Больцмана)
идеей о том, что число ударивших по $ i $-частице  $ j $-частиц равно их числу $N_{ij}^{reached}$, успевших за  время $ dt $ до неё долететь,
а оно равно числу частиц, содержащихся в  цилиндре, изображённом справа на Рис. 1: %\ref{geom_coll}

 \begin{equation}\label{Nreached}
 N_{ij}^{collisions} = N_{ij}^{reached}.
\end{equation}




\litlist

1. {\it Богомолов С. В., ЗахароваТ.В.}
 Уравнение Больцмана без гипотезы молекулярного хаоса //Математическое моделирование. – 2021. – Т. 33. – №. 1. – С. 3-24.
