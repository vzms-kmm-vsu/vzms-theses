\vzmstitle[
]{УПРАВЛЕНИЕ ДИНАМИЧЕСКОЙ СИСТЕМОЙ С
ЧАСТНЫМИ ПРОИЗВОДНЫМИ}

\vzmsauthor{Зубова }{С.\,П.}
\vzmsinfo{Воронеж, ВГУ; {\it spzubova@mail.ru}}

\vzmsauthor{Раецкая }{Е.\,В.}
\vzmsinfo{Воронеж, ВГЛТУ {\it raetskaya@inbox.ru}}

\vzmscaption


Исследуется динамическая система
\[
\frac{\partial x}{\partial t}=B\frac{\partial x}{\partial
s}+x(t,s)+Du(t,s), \quad t \in T,\quad s \in S,  \eqno(1)
\]
где    $x(t,s)\in \mathbb{R}^{n}$,   $u(t,s)\in \mathbb{R}^{m}$;
$B:\, n\times n$, $D:\, n\times m$, $\, \not\!\exists D^{-1}$;
 $T= [t_0,t_{k}]$, $S=[s_0,s_k)$.

Систему  (1) называем полностью управляемой, если существует функция
управления (управление) $u(t,s)$,
 под воздействием которого состояние $x(t,s)$ системы (1), траектория   переводится из
 начального состояния
$ x(t_0,s)=x_0(s) $ в  конечное состояние $ x(t_k,s)=x_k(s)
 $ за время $T, \, \forall T > 0,\, \forall
x_0(s),\, x_k(s)\in \mathbb{R}^{n}$.

Известно  [1], что система
\[
\frac{\partial x}{\partial t}=B\frac{\partial x}{\partial s}+Du(t,s)
\eqno(2)
\]
является полностью управляемой в том и только том случае, когда
выполняется условие Калмана [2] rank $(D\, BD\, ...\, B^{n-1}) =n$.

Доказывается
\paragraph{Теорема~1.} {\it Система (1) полностью управляема тогда и
только тогда, когда выполняется условие Калмана. }

Таким образом, системы (1), (2) и
\[
\frac{d x}{d t}=Bx(t)+Du(t) \eqno(3)
\]
полностью управляемые или не являются таковыми одновременно.

Для полностью управляемых систем (1) и (2) находятся $u(t,s)$, при
воздействии которых состояния систем $x(t,s)$ удовлетворяют не
только заданным начальным и конечным условиям, но и принимают в
заданные моменты $ t_i$, $ \, t_0<t_1<\, ...\, < t_k$,   любые
значения $ x_i(s) \in \mathbb{R}^n$, $\, i=0,\, 1,\, ...\,,\,  k$.

Вектор"=функции $u(t,s)$ и $x(t,s)$ строятся в аналитическом виде.
для чего применяется метод каскадной декомпозиции [1].


% Оформление списка литературы
%\litlist

1. {\it Zubova S.P, Raetskaya E.V.} Solution of the  multi-point
control problem for a dynamic system in partial derivatives //
Mathematical Methods in the Applied Sciences, AIMS, New York. -2021.
- V. 44, N 15.   -  P. 11998-12009.

 2. {\it Калман Р.Е. и др.} Очерки по
математической теории систем //  М. : Едиториал.  2004.  400 c.



