


\vzmstitle[
	%\footnote{}
]{
	Задача типа Дарбу для одной гиперболической системы с кратными характеристиками
}
\vzmsauthor{Миронов}{А.\,Н.}
\vzmsinfo{Елабуга, КФУ; Самара, СамГТУ; {\it miro73@mail.ru}}
\vzmsauthor{Волков}{А.\,П.}
\vzmsinfo{Самара, СамГТУ; {\it alex.volkov85@gmail.com}}

\vzmscaption

Рассматривается система уравнений с кратными характеристиками с двумя независимыми переменными
\begin{equation*}
\left\{
\begin{array}{c}
u_{xx}=a_{1}(x,y)v_{x}+b_{1}(x,y)u+c_{1}(x,y)v+f_{1}(x,y),\\
v_{yy}=a_{2}(x,y)u_{y}+b_{2}(x,y)u+c_{2}(x,y)v+f_{2}(x,y).
\end{array}
\right.\eqno{(1)}
\end{equation*}
Считаем, что в замыкании рассматриваемой области $G$ плоскости
$(x,y)$ выполняются включения $a_{i}\in C^{2}$, $b_{i}$,
$c_{i}$, $f_{i}\in C^{1}$, $i=\overline{1,2}$.
Решение системы (1) класса $u$, $v\in C^{1}(G)$,
$u_{xx}$, $v_{yy}\in C(G)$ назовём регулярным в $G$.

Различные задачи с условиями на характеристиках для системы (1) исследовались в работах [1], [2] с использованием метода Римана
[3]. Рассмотрим следующую задачу.

\paragraph{Задача $D$:} {\it найти в области $D_0:\, 0<y<x<T$,  регулярное решение системы (1), удовлетворяющее граничным условиям}
\begin{equation*}
\begin{array}{l}
u(y,y)=\varphi_{1}(y),\quad
(u_{x}-a_{1}v)(y,y)=\varphi_{2}(y),\\
v(x,{0})=\psi_{1}(x),\quad
(v_{y}-a_{2}u)(x,{0})=\psi_{2}(x),
\end{array}
\end{equation*}
где $\varphi_{1}(y)$, $\varphi_{2}(y)
\in C^{1}([{0},T])$, $\psi_{1}(x)$, $\psi_{2}(x)\in
C^{1}([0,T])$.


Доказаны существование и единственность решения, предложен способ определения матрицы Римана--Адамара задачи $D$, опирающийся на определение матрицы Римана [1]--[3]. Построено решение задачи $D$ в терминах матрицы Римана--Адамара.




% Оформление списка литературы

% \litlist


1. {\it Миронова Л. Б.} О характеристических задачах для одной системы с двукратными старшими частными производными //Вестн. Сам. гос. техн. ун-та. Сер. Физ.--мат. науки. -- 2006. -- Вып. 43. -- С. 31--37.

2. {\it Жегалов В. И., Миронова Л. Б.} Об одной системе уравнений с двукратными старшими частными производными //Изв. вузов. Матем.
-- 2007. -- № 3. -- С. 12--21.

3. {\it Миронова Л. Б.} О методе Римана в $R^n$ для одной системы с кратными характеристиками //Изв. вузов. Матем. -- 2006. --
№ 1. -- С. 34--39.



