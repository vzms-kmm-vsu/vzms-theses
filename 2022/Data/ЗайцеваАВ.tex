


\vzmstitle[
]{
	Топология слоения Лиувилля биллиарда в параболе
}
\vzmsauthor{Зайцева}{А.\,В.}
\vzmsinfo{Москва, МГУ; {\it AnastasiaZay12@mail.ru}}

\vzmscaption

Математический биллиард "--- это динамическая система, описывающая движение материальной точки внутри замкнутой ограниченной области с некоторым законом отражения на границе.
\\ В работе [4] был проведён топологический анализ биллиарда внутри параболической области, при этом биллиардный шар движется свободно прямолинейно. Рассмотрим теперь такой биллиард, но с добавленным гравитационным потенциалом.
А именно, биллиард задан в области, ограниченной параболами из семейства софокусных парабол. На шар действует потенциал $V = gy$.
Сила тяжести направлена перпендикулярно директрисам парабол.
\par Гамильтонианом этой системы является:
\begin{center}
$H = \frac{\dot{x}^2 + \dot{y}^2}{2} + gy$;
\end{center}
а  дополнительным первым интегралом:
\begin{center}
$G = \dot{x}(x \dot{y} - \dot{x}y) + \frac{gx^2}{2}$;
\end{center}

Более того, такой биллиард допускает разделение переменных Харламова [3].

\par Выразим значения функций через $\lambda_1$ и $\lambda_2$ и импульсы $\mu_1$ и $\mu_2$.

\textbf{Теорема}: Формулы разделения переменных имеют следующий  вид:
\begin{center}
$\mu_1 =\frac{ 2 \dot{\lambda_1}(\lambda_1 - \lambda_2)}{\lambda_1}$, $\mu_2 =\frac{ 2 \dot{\lambda_2}(\lambda_2 - \lambda_1)}{\lambda_2}$
\end{center}

\par В работе И.Ф.Кобцева [5] был рассмотрен биллиард внутри эллипса с потенциалом Гука.
Также в этой работе сравниваются эллиптический биллиард в классическом прямолинейном случае и в случае с потенциалом.
Проведём аналогичное сравнение нашей системы и такой же прямолинейной системы.
А именно, используя метод Харламова, опишем слоения Лиувилля в терминах инвариантов Фоменко---Цишанга.

В докладе будут приведены инварианты Фоменко (грубые молекулы) для некоторых изоэнергетических многообразий $Q_3$.


% Оформление списка литературы
\litlist

1. {\it Болсинов~А.В.,  Фоменко~А.Т. } Интегрируемые гамильтоновы системы.  Геометрия, топология, классификация.
Т.~{I}. --- Ижевск: РХД, 1999
\par 2. {\it  Козлов В.В. } Некоторые интегрируемые обобщения задачи Якоби о геодезических на эллипсоиде. //Прикладная математика и механика, том 59, вып. 1 1995
\par 3. {\it  Харламов М.П. } Топологический анализ и булевы функции: {I}. Методы и приближения к классическим системам //Нелинейная динамика, 2010, том 6, №4, с. 769---805.
\par 4. {\it  Фокичева В.В.  } Топологическая классификация биллиардов в локально плоских областях, ограниченных дугами софокусных квадрик, Матем. сб., 206:10 (2015), 127-176
\par 5. {\it Кобцев И.Ф.  } Эллиптический биллиард в поле потенциальных сил: классификация движений, топологический анализ, Матем. сб., 211:7 (2020), 93-120

