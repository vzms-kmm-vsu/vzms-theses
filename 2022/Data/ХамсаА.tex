\vzmstitle[
	% \footnote{Работа выполнена за счёт гранта РНФ, проект № 20-71-00155}
]{
	О сигнальных задачах с дробной волной
}
\vzmsauthor{Алкади}{Хамса}
\vzmsinfo{Воронеж, ВГУ; {\it hamsaphd.hassan44@gmail.com}}
\vzmsauthor{Костин}{В.\,А.}
\vzmsinfo{Воронеж, ВГУ; {\it vlkostin@mail.ru}}

\vzmscaption

В заметки приводится результат развивающий подход А. Н. Тихонова и А. А. Самарского изложенные в \cite{tehonov} для описания тепловых волн на случай уравнения с дробными производными теория которых активно применяется при изучении распространения сигнала во фрактальных средах 

Рассматривается задача отыскания решения уравнения
\begin{equation}\label{eq:1}
	\frac{\partial ^2 u(t,x)}{\partial x^2}= \frac{\partial ^{\alpha} u(t,x)}{\partial t^{\alpha}},\quad x> 0,\, t\in (-\infty, \infty)
\end{equation}
Где $\frac{\partial ^{\alpha} u(t,x)}{\partial t^{\alpha}}-$ дробная производная Римана-Лиувилля порядка $\alpha \in (0,2)$ для $t\in (-\infty, \infty),\, u(t,x)$ удовлетворяет условиям
	\begin{equation}\label{eq:2}
u(t,0)=\varphi(t) 
\end{equation}
	\begin{equation}\label{eq:3}
	\lim\limits_{x\to + \infty } |u(t,x)|=0
\end{equation}
Где $\varphi(t)$ периодическая функция с рядом Фурье
	\begin{equation}\label{eq:4}
	\varphi(t) =\frac{a_0}{2}+ \sum_{n=1}^{\infty}A_n \cos \left[ w_{n}t - \theta_{n}^{0} \right]
\end{equation}
где $w_{n}= \frac{2\pi n}{T}$

\textbf{Теорема.} Если в условии (\ref{eq:3}) $\varphi(t)$ - периодическая функция вида (\ref{eq:4}), то задача (\ref{eq:1})-(\ref{eq:2})-(\ref{eq:3}) имеет единственное решение, и оно представимо в виде
	\begin{equation}\label{eq:5}
	u(t,x)=+ \sum_{n=1}^{\infty}u_{n}(t,x),
\end{equation}
Где
\begin{equation*}
	u_n(t)= A_n e^{-\cos\frac{\alpha \pi}{4}w_n^{\alpha/2}x}\cos \left[(\sin \frac{\alpha \pi}{4})w_n^{\alpha/2}x - w_nt +\frac{2\pi n}{T}\delta_n^0 \right]
\end{equation*}
В случае $\alpha =1$, решение (\ref{eq:5}) совпадает с решением, приведенным в \cite{tehonov} 

Если $\alpha =1$, то задача (\ref{eq:1})-(\ref{eq:3}) не корректна так как функции (\ref{eq:5}) не удовлетворяют условию (\ref{eq:3}).

\litlist

1. {\it Тихонов, А.Н., Самарский А.А}
Уравнения математической физики. Т.1,2, М: Наука, 1966.

2. {\it Зельдович Я.Б., Соколов Д.Д}
Фрактали, подобие, промежуточная асимптотика. Успехи физических наук 146.7, 1985: 493-506.

3. {\it Майнарди Ф.}
Временное уравнение дробной диффузионно"=волновой функции. Т.1,2, Радиофизика и квантовая электроника. Выпуск 38, №1-2, 1995: 20-36.
