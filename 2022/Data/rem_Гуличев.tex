\vzmstitle[]{
	Разрешимость реологического соотношения для обобщённой модели Кельвина--Фойгта
}
\vzmsauthor{Гуличев}{И.\,О.}
\vzmsinfo{Воронеж, ВГУ; {\it igoolichev@gmail.com}}

\vzmscaption




%\Article
%{.} % Здесь пишите название статьи
%{Гуличев И.О.} %Автор(ы)
%{} % Грант если есть;в случае отсутствия ставим {}
%{Работа посвящена изучению разрешимости реологического соотношения для обобщённой модели Кельвина--Фойгта разных порядков} %Аннотация
%{Реологическое соотношение, обобщённая модель Кельвина--Фойгта} % ключевые слова


\paragraph{Аннотация.}
Работа посвящена изучению разрешимости реологического соотношения для обобщённой модели Кельвина--Фойгта разных порядков.

\paragraph{Ключевые слова:}
реологическое соотношение, обобщённая модель Кельвина--Фойгта

\paragraph{Введение.}
Реологическое соотношение для обобщённой математической модели Кельвина--Фойгта имеет вид
$$
\label{e1}
	\left(1+\sum_{i=1}^{L}\lambda_i\frac{\partial^{i}}{\partial t^{i}}\right)\sigma=2\left(\nu+\sum_{i=1}^{L+1}\varkappa_i\frac{\partial^{i}}{\partial t^{i}}\right)\mathcal{E},\,\,\,\, \varkappa_{L+1}, \lambda_L > 0.
	\eqno{(1)}
$$
Здесь $\lambda_i$ "--- времена релаксации, $\nu$ "--- кинематический коэффициент вязкости, $\varkappa_i$ "--- времена запаздывания (ретардации)\cite{ZT}.\\
Предполагается, что корни  многочлена
$$
Q(p)=1+\sum\limits_{i=1}^L\lambda_ip^i
$$
веществены, отрицательны и различны. Требование вещественности и отрицательности исходит из физического смысла, а требование различности корней нужно для простоты и упрощения вычислений.\\
\par Эти модели являются одними из моделей линейных вязкоупругих жидкостей с конечным числом дискретно распределённых времён релаксации и времён запаздывания. Общая феноменологическая теория таких жидкостей Максвелла, Кельвина--Фойгта и Олдройта.
В основе теории линейных вязкоупругих жидкостей лежит предположение о том, что все воздействия на среду независимы и аддитивны, а реакции среды на внешние воздействия линейны, называемое принципом суперпозиции Л. Больцмана. К таким жидкостям относится эмульсии и суспензии одной ньютоновской жидкости в другой, сильно разбавленные суспензии твёрдых частиц в ньютоновской жидкости, некоторые полимерные растворы\cite{VM}.\\
\par В статье не будет рассматриваться случай $L=0$, так как в таком случае $\sigma$ уже выражено.% Текст статьи\ldots


\paragraph{Реологическое соотношение для обобщённой модели Кельвина--Фойгта порядка $L=1$.} %Название первого параграфа
Рассмотрим случай $L=1$.
$$
	\left(1+\lambda\frac{\partial}{\partial t}\right)\sigma=2\left(\nu+\sum_{i=1}^{2}\varkappa_i\frac{\partial^{i}}{\partial t^{i}}\right)\mathcal{E}.
	\eqno{(2)}
$$
Для выражения $\sigma$ из этого уравнения нужно разделить обе части на $\lambda$ и домножить на $e^{\lambda t}$.\\
$$
	e^{\frac{1}{\lambda} t}\left(\frac{1}{\lambda}\sigma+\frac{\partial}{\partial t}\sigma\right)=\frac{2}{\lambda}e^{\lambda t}\left(\nu+\sum_{i=1}^{2}\varkappa_i\frac{\partial^{i}}{\partial t^{i}}\right)\mathcal{E}.
$$
Теперь можно свернуть левую часть уравнения
$$
\frac{\partial e^{\frac{1}{\lambda} t}\sigma}{\partial t}=\frac{2}{\lambda}e^{\frac{1}{\lambda} t}\left(\nu+\sum_{i=1}^{2}\varkappa_i\frac{\partial^{i}}{\partial t^{i}}\right)\mathcal{E}.
$$
Для получения решения осталось только проинтегрировать обе части, после чего умножить на $e^{-\frac{1}{\lambda} t}$.\\
$$
\sigma=e^{-\frac{1}{\lambda} t}\int\limits_{0}^{t}\frac{2}{\lambda}e^{\frac{1}{\lambda} s}\left(\nu+\sum_{i=1}^{2}\varkappa_i\frac{\partial^{i}}{\partial t^{i}}\right)\mathcal{E}ds.
$$
По свойствам определённого интеграла получим окончательный ответ
$$
\sigma(t) = 2\mu_2\mathcal{E}'(t) + 2\mu_1\mathcal{E}(t)+2\int\limits_0^t 2\beta e^{\frac{1}{\lambda}(s-t)}\mathcal{E}(s)ds+2\alpha e^{-\frac{1}{\lambda}t},
\eqno{(3)}
$$
где
\begin{align}
	\mu_2 = \frac{\varkappa_2}{\lambda};\nonumber\\
	\mu_1 = \frac{\lambda\varkappa_1 - \varkappa_2}{\lambda^2};\nonumber\\
	\beta = \frac{\varkappa_2-\lambda\kappa_1 +\lambda^2\nu}{\lambda^3};\nonumber\\
	\alpha = \frac{-\lambda\varkappa_2\mathcal{E}'(0)+(\varkappa_2 - \lambda\varkappa_1)\mathcal{E}(0)}{\lambda^2}.
	\nonumber
\end{align}
{\bf Теорема}{\it (Существования и единственности) Пусть при
	$$
    h = 2\nu\mathcal{E} + 2\varkappa_1\mathcal{E}'+2\varkappa_2\mathcal{E}''
    $$
	есть непрерывная функция $t \in [0,T]$, $\lambda > 0$. Тогда при любых начальных условиях
	$$
		\sigma|_{t=0}=\sigma_0(x);
		\begin{cases}
		\mathcal{E}|_{t=0}=\mathcal{E}_0(v); \\
		\mathcal{E}'|_{t=0}=\mathcal{E}_1(v),
	\end{cases}
	\eqno{(4)}
	$$
	решение уравнения (2) существует и единственно. Это решение определено при всех $t \in [0,T]$ и имеет вид (3).}
\paragraph{Реологическое соотношение для обобщённой модели Кельвина--Фойгта порядка $L=2$.}
Теперь рассмотрим случай, когда $L=2$. В этом случае реологическое соотношение будет иметь вид\\
$$
	\left(1+\sum_{i=1}^{2}\lambda_i\frac{\partial^{i}}{\partial t^{i}}\right)\sigma=2\left(\nu+\sum_{i=1}^{3}\varkappa_i\frac{\partial^{i}}{\partial t^{i}}\right)\mathcal{E}.
	\eqno{(5)}
$$
Раскрыв скобки и записав в порядке уменьшения степени производных $\sigma$, получим соотношение вида
$$
\label{e4}
	\lambda_2\sigma'' + \lambda_1\sigma' + \sigma = 2\nu\mathcal{E} + 2\varkappa_1\mathcal{E}'+2\varkappa_2\mathcal{E}'' +2\varkappa_3\mathcal{E}'''.
	\eqno{(6)}
$$
{Нетрудно видеть, что соотношение (5) "--- не что иное, как неоднородное дифференциальное уравнение второго порядка \cite{ZP}, решение которого имеет вид}
$$
\sigma = \sigma_{\text{oo}} + \sigma_{\text{чн}} = C_1\sigma_1 + C_2\sigma_2 + \sigma_2\int\limits_{0}^{t}\sigma_1 \frac{g}{\lambda_2} \frac{ds}{W} - \sigma_1\int\limits_{0}^{t}\sigma_2 \frac{g}{\lambda_2} \frac{ds}{W}
,
$$
{где}
$$
g = 2\nu\mathcal{E} + 2\varkappa_1\mathcal{E}'+2\varkappa_2\mathcal{E}'' +2\varkappa_3\mathcal{E}''',
$$
W - определитель Вронского, равный
$$
W = \left|
\begin{array}{cc}
	y_1 & y_2\\
	y_1'& y_2'
\end{array}\right|.
$$

Для нахождения общего неоднородного решения сначала найдём общее однородное. Для этого приравняем к нулю правую часть нашего уравнения.
$$
	\lambda_2\sigma'' + \lambda_1\sigma' + \sigma = 0.
	\eqno{(7)}
$$
Предполагается, что корни вещественны, отрицательны и различны.
Перейдём к характеристическому уравнению
$$
\lambda_2\xi^2 + \lambda_1\xi + 1 = 0
$$
$$
D=\lambda_1^2-4\lambda_2>0;\,\, \lambda_2>0 \Rightarrow \lambda_1^2 >4\lambda_2 \Rightarrow
\begin{cases}
	\lambda_1>4\lambda_2\\
	\lambda_1<-4\lambda_2
\end{cases}
$$
$$
\xi_{1,2}={\frac{-\lambda_1\pm\sqrt{\lambda_1^2-4\lambda_2}}{2\lambda_2}}
$$
$$
	\sigma_{\text{oo}}= C_1e^{\xi_1t} + C_2e^{\xi_2t}
$$
{Теперь решим частную неоднородную часть:}

\begin{multline}
	\sigma_{\text{чн}}=
	e^{\xi_2t}\int\limits_{0}^{t} e^{\xi_1s} \frac{2\nu\mathcal{E}(s) + 2\varkappa_1\mathcal{E}'(s)+2\varkappa_2\mathcal{E}''(s) +2\varkappa_3\mathcal{E}'''(s)}{\lambda_2}
	\\
	\frac{ds}{e^{\xi_1s}\xi_2 e^{\xi_2s}- e^{\xi_2s}\xi_1e^{\xi_1s}}-
	\\
	\\
	- e^{\xi_1t}\int\limits_{0}^{t}e^{\xi_2s} \frac{2\nu\mathcal{E}(s) + 2\varkappa_1\mathcal{E}'+2\varkappa_2\mathcal{E}''(s) +2\varkappa_3 \mathcal{E}'''(s)}{\lambda_2}
	\\
	\frac{ds}{e^{\xi_1s}\xi_2 e^{\xi_2s} - e^{\xi_2s}\xi_1e^{\xi_1s}}.
	\nonumber
\end{multline}

В итоге мы получим выражение вида
$$
\label{resh}
	\sigma_{\text{он}} = 2\mu_2\mathcal{E}'(t)+2\mu_1\mathcal{E}(t) +2\int\limits_{0}^{t}\sum_{i=1}^{2}\beta_ie^{\xi_i(t-s)}\mathcal{E}(s)ds
	+2\sum_{i=1}^{2}e^{\xi_it}\alpha_i,
	\eqno{(8)}
$$
где
\begin{multline}
	\begin{gathered}
	\mu_2=\frac{\varkappa_3(\xi_2-\xi_1)}{\lambda_2(\xi_2-\xi_1)}= \frac{\varkappa_3}{\lambda_2} > 0; \\ \\
	\mu_1= \frac{\varkappa_2(\xi_2-\xi_1)+\varkappa_3(\xi_2^2-\xi_1^2)}{\lambda_2(\xi_2-\xi_1)}=\frac{\varkappa_2+\varkappa_3(\xi_2+\xi_1)}{\lambda_2};\\ \\
	\beta_i=(-1)^i\frac{\nu+ \xi_i\varkappa_1+\varkappa_2\xi_i^2+\varkappa_3\xi_i^3}{\lambda_2(\xi_2-\xi_1)},\text{ где }i=1,2;
	\end{gathered}\\ \\
	\alpha_i=2(-1)^{i-1}\Bigl[ \frac{(\varkappa_1+\varkappa_2\xi_i+\varkappa_3\xi_i^2)\mathcal{E}(0)+(\varkappa_2+\varkappa_3\xi_i)\mathcal{E}'(0)}{\lambda_2(\xi_2-\xi_1)}+\\ \\
	+\frac{\varkappa_3\mathcal{E}''(0)+(-1)^{i-1}C_i(\lambda_2(\xi_2-\xi_1))}{\lambda_2(\xi_2-\xi_1)}\Bigl],\text{ где }i=1,2.
	\nonumber
\end{multline}
	{\bf Теорема}{\it (Существования и единственности) Пусть при
	$$
		h = 2\nu\mathcal{E} + 2\varkappa_1\mathcal{E}'+2\varkappa_2\mathcal{E}'' +2\varkappa_3\mathcal{E}'''
    $$
	есть непрерывная функция $t \in [0,T]$, $\lambda_1 > 0$ и корни многочлена
	$$
	\lambda_2\xi^2 + \lambda_1\xi + 1 = 0
	$$
	вещественны, отрицательны и различны. Тогда при любых начальных условиях
	$$
		\begin{cases}
			\sigma|_{t=0}=\sigma_0(x); \\
			\sigma'|_{t=0}=\sigma_1(x);
		\end{cases}
		\begin{cases}
			\mathcal{E}|_{t=0}=\mathcal{E}_0(v); \\
			\mathcal{E}'|_{t=0}=\mathcal{E}_1(v); \\
			\mathcal{E}''|_{t=0}=\mathcal{E}_2(v),
		\end{cases}
	\eqno{(9)}
	$$
	решение уравнения (6) существует и единственно. Это решение определено при всех $t \in [0,T]$ и имеет вид (8).}


\begin{thebibliography}{X}
	\bibitem{ZT}~В.~Г.~Звягин, М.~В.~Турбин Математические вопросы гидродинамики вязкоупругих сред.--- М.~:~КРАСАНД, 2012."---416~с.
	\bibitem{VM}~Г.~В.~Виноградов, А.~Я.~Малкин Реология полимеров.---М.~:~Химия, 1977."---438~с.
	\bibitem{ZP} ~В.~Ф.~Зайцев, А.~Д.~Полянин  Справочник по обыкновенным дифференциальным уравнениям.---М.:Физматлит, 2001."---576~с.

\end{thebibliography}

Сведения об авторе:

\noindent {\bfseries Гуличев Илья Олегович},
студент математического факультета Воронежского государственного университета.

