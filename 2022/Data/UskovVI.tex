


\vzmstitle[
]{
	Решение линейных рекуррентных соотношений второго порядка
}
\vzmsauthor{Усков}{В.\,И.}
\vzmsauthor{Анохина}{В.\,А.}
\vzmsauthor{Московская}{В.\,С.}
\vzmsauthor{Трофименко}{П.\,В.}
\vzmsinfo{Воронеж, ВГЛТУ; {\it vum1@yandex.ru}}

\vzmscaption

Дифференциальными уравнениями и их системами второго порядка описываются физические, экономические процессы [1] и т.д. С помощью метода конечных разностей их можно свести к рекуррентным соотношениям второго порядка. Такой метод применялся к исследованию нелинейной математической \textit{модели} шнекового рабочего органа лесопожарной грунтометательной машины [2].

Рассмотрим скалярное рекуррентное соотношение второго порядка: \[y_{i+2}+ay_{i+1}+by_i=f_i,\eqno{(1)}\]
где $a,b,c$ --- заданные постоянные, $y_i$ --- искомая последовательность, $f_i$ --- заданная последовательность, $i\geqslant0$, и векторное рекуррентное соотношение второго порядка: \[Ay_{i+2}=By_{i+1}+F_i,\eqno{(2)}\]
где $A,B$ --- заданные квадратные матрицы, причём $\det A=0$, $y_i$ --- искомая последовательность, $F_i$ --- заданная последовательность, $i\geqslant0$.

Цель настоящей работы: проиллюстрировать полученные в [2], [3] результаты на примерах.

\paragraph{Решение скалярного соотношения (1).}

Введём характеристическое уравнение $\lambda^2+a\lambda+b=0$. Обозначим $D=a^2-4b$ --- дискриминант, $\lambda_1$, $\lambda_2$ --- корни этого уравнения.

Получена общая формула $y_i$ в зависимости от знака $D$.

\paragraph{Теорема~1.}
{\it При $D>0$ общее решение соотношения {\rm (1)} равно}

\[y_i=c_1\lambda_1^i+c_2\lambda_2^i+\sum_{k=0}^{i-1}\frac{\lambda_2^{i-1-k}-\lambda_1^{i-1-k}}{\lambda_2-\lambda_1}f_k.\]

{\it При $D=0$, в обозначении $\lambda=\lambda_1=\lambda_2$, общее решение соотношения {\rm (1)} равно}

\[y_i=c_1\lambda^i+c_2i\lambda^{i-1}+\sum_{k=0}^{i-2}(i-1-k)\lambda^{i-2-k}f_k,\]

{\it где $c_1$, $c_2$ обозначены произвольные постоянные.}

\paragraph{Решение векторного соотношения (2).}

Рассматривается случай $\dim\,{\rm Ker}\,A=1$. Вводятся проектор $Q$ на ${\rm Coker}\,A$, сужение $\tilde{A}$ оператора $A$ на ${\rm Coim}\,A$, полуобратный оператор $A^{-}=\tilde{A}^{-1}(I-Q):{\rm Im}\,A\to{\rm Coim}\,A$; фиксируются элементы $e\in{\rm Ker}\,A$, $e\ne0$, $\varphi\in{\rm Coker}\,A$. В ${\rm Coim}\,A$ вводится скалярное произведение $<,>$ так, что $<\varphi,\varphi>=1$.

Получен следующий результат.

\paragraph{Теорема~2.}
{\it Пусть $d=<QBe,\varphi>\ne0$. Тогда соотношение (2) равносильно системе
\[y_{i+2}=Ky_{i+1}+\Phi_i,\eqno{(3)}\]
\[<QBy_{i+1},\varphi>+<QF_i,\varphi>=0,\eqno{(4)}\]
где
\[K(\cdot)=A^{-}B(\cdot)-d^{-1}<QBA^{-}B(\cdot),\varphi>e,\]
\[\Phi_i=A^{-}F_i-d^{-1}(<QBA^{-}F_i,\varphi>+<QF_{i+1},\varphi>)e.\]}

Для решения следующих двух примеров применим теорему 1.

\paragraph{Пример~1.} Рассмотрим задачу:
\[y_{i+2}-15y_{i+1}+56y_i={(i+2)}^i, \quad i\geqslant0,\]
\[y_0=0, \quad y_1=4.\]

Имеем для всех $i\geqslant2$:
\[y_i=4\cdot 8^i-4\cdot 7^i+\sum\limits_{k=0}^{i-1} (8^{i-1-k}-7^{i-1-k}){(k+2)}^k.\]

Выпишем первые члены этой последовательности: \\
$0, 4, 61, 7010, 66411$.

\paragraph{Пример~2.} Рассмотрим задачу:
\[y_{i+2}-12y_{i+1}+36y_i=i!, \quad i\geqslant0,\]
\[y_0=-5, \quad y_1=0.\]

Имеем для всех $i\geqslant2$:
\[y_i=-5\cdot 6^i+30\cdot i\cdot 6^{i-1}+\sum\limits_{k=0}^{i-2} (i-1-k)\,6^{i-2-k}\,k!.\]

Выпишем первые члены этой последовательности: \\
$-5, 0, 181, 2173, 19562, 156522$.

К следующему примеру применим теорему 2.

\paragraph{Пример~3.}
Рассмотрим однородное соотношение для (2) с искомой вектор"=последовательностью $y_i=(v_i, w_i, x_i)$ и матрицами
\[A=\begin{pmatrix}
2&2&2\\
1&2&3\\
4&4&4
\end{pmatrix}, \quad B= \begin{pmatrix}
1&2&3\\
1&3&2\\
2&1&3
\end{pmatrix}.\]

Имеем: $d=3\ne0$. Тогда данное соотношение равносильно (3) с оператором
\[K=\begin{pmatrix}
 0.5 &  1 &  1.5\\
-0.5 & -2 & -0.5\\
 0.5 &  2 &  0.5
\end{pmatrix}\]
и (4)
\[w_{i+1}+x_{i+1}=0.\]

% Оформление списка литературы
\litlist

1. {\it Кукленкова А.А.} Применение дифференциальных уравнений в моделировании экономических процессов //Научное обозрение. Педагогические науки. – 2019. – № 4-3. – С. 60–63.

2. {\it Попиков П.И., Поздняков А.К., Усков В.И., Лысыч М.Н., Гнусов М.А.} Теоретическое исследование кинематических и динамических характеристик шнекового рабочего органа лесопожарной грунтометательной машины //Лесотехнический журнал. – 2021. – Т. 11. – №. 3 (43). – С. 140–151.

3. {\it Усков В.И., Анжаурова Т.М.} Решение линейных рекуррентных соотношений второго порядка //Молодой учёный. – 2019. – №. 42 (280). – C. 1–6.

