\vzmstitle[
	\footnote{Работа выполнена за счёт гранта РФФИ, проект 20-31-90114. Автор является стипендиатом Фонда ``Базис''}
]{
	Особенности псевдоевклидовых систем Ковалевской на алгебрах Ли
}
\vzmsauthor{Кибкало}{В.\,А.}
\vzmsinfo{Москва, МГУ; {\it slava.kibkalo@gmail.com}}

\vzmscaption

Многие интегрируемые системы имеют аналоги в псевдоевклидовых пространствах, как известные волчки Эйлера, Лагранжа и Ковалевской [1]. Семейство аналогов последнего на пучке алгебр Ли $so(3,1)-e(3)-so(4)$, открытое И.В.Комаровым [2], переносится на псевдоевклидов случай, порождая на $\mathbb{R}^6(J_1, J_2, J_3, x_1, x_2, x_3)$ семейство систем $H, K = K_{\varkappa}$ с параметром $\varkappa \in \mathbb{R}$ из  скобки Ли-Пуассона.

Динамические и топологические свойства этих систем сильно отличаются от своих классических аналогов. Например, волчок Эйлера имеет и компактные, и некомпактные слои (совместные уровни функций Казимира $f_1 =a, f_2=b$, энергии $H=h$ и интеграла $F = x_1^2 + x_2^2 - x_3^2 =f$), а поток гамильтонова векторного поля интеграла системы бывает неполным [1]. Формулы $f_1, f_2, H$ получаются из формул [2] семейства Ковалевской заменой знака при $x_3^2, x_3 J_3$ и $J_3^2$.

Автором был решен [3] вопрос о компактности слоя Лиувилля для семейства систем Ковалевской при каждом $\varkappa$.
\paragraph{Теорема~1.}
{\it
	Пусть $f_2 = b \ne 0$. Тогда совместный уровень $T_{a, b, h, k}$ функций $f_1, f_2, H, K$ компактен $\Leftrightarrow$ для имеем $0 \le k < (h -\varkappa c_1^2)^2$. При $k <0$ он пуст, а если точка $(h, k)$ лежит над параболой или на ней, то некомпактен.
}

Изучим критические точки отображения $(H, K)$: в них линейно зависимы поля $sH^i = (\mathrm{sgrad}\,H)^i = \omega^{ij} dH_j$ и $sK^i$.

\paragraph{Утверждение~1.}
{\it
	При каждом $\varkappa \in \mathbb{R}$ множество критических точек ранга 1 для системы Ковалевской на алгебре Ли из пучка $so(3,1)-e(3)-so(4)$ и для ее псевдо"=евклидова аналога совпадают. Аналогичное верно для точек ранга 0. Их образы лежат на кривых, задаваемых теми же формулами, что и в случае Ковалевской при таком $\varkappa$.
}

Невырожденные (боттовские) критические точки ранга~1 отвечают думам бифуркационной диаграммы $\Sigma$, а их вырождения ранга 1 и точки ранга 0 --- особым точкам $\Sigma$. В семействе Ковалевской на $so(3,1)-e(3)-so(4)$ они составляют три семейства: первое $(J_1, J_2, 0, \varkappa, 0 , 0)$, второе $(J_1, 0, 0, x_1, 0 , 0)$ и третье $(J_1, 0, J_3, -J_1^2/2 -\varkappa/2, 0 , -J_1 J_3)$.

Формулы их координат [4] не меняются при переходе к псевдо"=евклидову случаю, что легко проверить. У семейств 1 и 2 (см. [4]) также не меняются значения $f_1, f_2, H$, т.к. $x_3 = J_3 = 0$. Для семейства $3$ последнее неверно. Они могут попасть на ``новые'' 4-орбиты $M^4_{a, b}$, пустые ранее при этом $\varkappa$.

Для определения типа особенности слоения, содержащей невырожденные точки $\mathrm{rk}\,0$, требуется изучить вопрос вещественности или мнимости некоторых собственных значений операторов линеаризации полей $sH, sK$. Их формулы, в отличие от координат точек, начинают зависеть от $\sigma = -1$.

\paragraph{Утверждение~2.}
{\it
	Пусть при данных $(a, b, \varkappa)$ слоение Лиувилля системы Ковалевской на орбите $M^4_{a, b}$ имеет критическую точку ранга 0 из семейств 1 и 2, и она невырождена. Тогда она принадлежит той же орбите псевдо"=евклидовой системы с этим $\varkappa$, и тип компонент ее локальной особенности меняется: центр на седло, и наоборот.
}

Орбита $M^4_{a, b}$, куда попадают точки семейства 3, меняется, но их тип несложно определить, зная координаты.

\paragraph{Утверждение~3.}
{\it
	Точки $\mathrm{rk}\,0$ семейства 3 (см. [4]) невырождена, если $J_3 \ne 0$ и $S = \varkappa J_3^2 - J_1^2 + \varkappa \ne 0$ (для $\sigma = 1$ в исходном или $\sigma = -1$ в псевдо"=евклидовом случаях). Ее тип --- центр"=центр при $S<0$ или центр"=седло при $S >0$.
}

Эти утверждения позволяют получить немало информации о топологии псевдо"=евклидовых аналогов системы Ковалевской, о чем подробнее мы расскажем. На рисунке изображены бифуркационные кривые псевдоевклидовой Ковалевской при $\varkappa = 0, b \ne 0$. В прообразе пунктирной параболы --- некритические некомпактные бифуркации слония.

\begin{figure}[h]
			\center{\includegraphics[width=90mm]{bif_curve_pseudo2.png}}
			\label{bif_diagr}
		\end{figure}



Работа выполнена в МГУ за счёт гранта РФФИ, проект 20-31-90114. Автор является стипендиатом Фонда ``Базис''.

\litlist

1. {\it Borisov A. V., Mamaev I. S.} Rigid body dynamics in non-Euclidean spaces  // Rus. J. of
Math. Phys. – 2016. – Т. 23. – №. 4. – С. 431-454.

2. {\it Komarov I. V.} Kowalewski basis for the hydrogen atom
  // Theoret. Math. Phys. – 1981. – Т. 47. – №. 1. – С. 320-324.

3. {\it Кибкало В. А.} Свойство некомпактности слоев и особенностей неевклидовой системы Ковалевской на пучке алгебр Ли  // Вестник МГУ, Сер. 1. – 2020. – №. 6. – С. 56-59.

4. {\it Козлов И. К.} Топология слоения Лиувилля для интегрируемого случая Ковалевской на алгебре Ли $so(4)$
  // Матем. сб. – 2014. – Т. 205. – №. 4. – С. 79-120.
