\documentclass{vzmsthesis}

\begin{document}

\vzmstitle[
	\footnote{Работа выполнена за счёт гранта РФФИ, проект 20-31-90114. Автор является стипендиатом Фонда ``Базис''}
]{
	Особенности псевдоевклидовых систем Ковалевской на алгебрах Ли
}
\vzmsauthor{Кибкало}{В.\,А.}
\vzmsinfo{Москва, МГУ; {\it slava.kibkalo@gmail.com}}

\vzmscaption

Многие интегрируемые системы имеют аналоги в псевдоевклидовых пространствах, как известные волчки Эйлера, Лагранжа и Ковалевской [1]. Cемейство аналогов последнего на пучке алгебр Ли $so(3,1)-e(3)-so(4)$, открытое И.В.Комаровым [2], переносится на псевдоевклидов случай, порождая на $\mathbb{R}^6(J_1, J_2, J_3, x_1, x_2, x_3)$ семейство систем $H, K = K_{\varkappa}$ c параметром $\varkappa \in \mathbb{R}$ из  скобки Ли-Пуассона.

Динамические и топологические свойства этих систем сильно отличаются от своих классических аналогов. Например, волчок Эйлера имеет и компактные, и некомпактные слои (совместные уровни функций Казимира $f_1 =a, f_2=b$, энергии $H=h$ и интеграла $F = x_1^2 + x_2^2 - x_3^2 =f$), а поток гамильтонова векторного поля интеграла системы бывает неполным [1]. Формулы $f_1, f_2, H$ получаются из формул [2] семейства Ковалевской заменой знака при $x_3^2, x_3 J_3$ и $J_3^2$. 

Автором был решен [3] вопрос о компактности слоя Лиувилля для семейства систем Ковалевской при каждом $\varkappa$. 
\paragraph{Теорема~1.}
{\it
	Пусть $f_2 = b \ne 0$. Тогда совместный уровень $T_{a, b, h, k}$ функций $f_1, f_2, H, K$ компактен $\Leftrightarrow$ для имеем $0 \le k < (h -\varkappa c_1^2)^2$. При $k <0$ он пуст, а если точка $(h, k)$ лежит над параболой или на ней, то некомпактен.
}

Изучим критические точки отображения $(H, K)$: в них линейно зависимы поля $sH^i = (\mathrm{sgrad}\,H)^i = \omega^{ij} dH_j$ и $sK^i$.

\paragraph{Утверждение~1.}
{\it
	При каждом $\varkappa \in \mathbb{R}$ множество критических точек ранга 1 для системы Ковалевской на алгебре Ли из пучка $so(3,1)-e(3)-so(4)$ и для ее псевдо-евклидова аналога совпадают. Аналогичное верно для точек ранга 0. Их образы лежат на кривых, задаваемых теми же формулами, что и в случае Ковалевской при таком $\varkappa$. 
}

Невырожденные (боттовские) критические точки ранга~1 отвечают думам бифуркационной диаграммы $\Sigma$, а их вырождения ранга 1 и точки ранга 0 --- особым точкам $\Sigma$. В семействе Ковалевской на $so(3,1)-e(3)-so(4)$ они составляют три семейства: первое $(J_1, J_2, 0, \varkappa, 0 , 0)$, второе $(J_1, 0, 0, x_1, 0 , 0)$ и третье $(J_1, 0, J_3, -J_1^2/2 -\varkappa/2, 0 , -J_1 J_3)$. 

Формулы их координат [4] не меняются при переходе к псевдо-евклидову случаю, что легко проверить. У семейств 1 и 2 (см. [4]) также не меняются значения $f_1, f_2, H$, т.к. $x_3 = J_3 = 0$. Для семейства $3$ последнее неверно. Они могут попасть на ``новые'' 4-орбиты $M^4_{a, b}$, пустые ранее при этом $\varkappa$.

Для определения типа особенности слоения, содержащей невырожденные точки $\mathrm{rk}\,0$, требуется изучить вопрос вещественности или мнимости некоторых собственных значений операторов линеаризации полей $sH, sK$. Их формулы, в отличие от координат точек, начинают зависеть от $\sigma = -1$. 

\paragraph{Утверждение~2.}
{\it
	Пусть при данных $(a, b, \varkappa)$ слоение Лиувилля системы Ковалевской на орбите $M^4_{a, b}$ имеет критическую точку ранга 0 из семейств 1 и 2, и она невырождена. Тогда она принадлежит той же орбите псевдо-евклидовой системы с этим $\varkappa$, и тип компонент ее локальной особенности меняется: центр на седло, и наоборот. 
}

Орбита $M^4_{a, b}$, куда попадают точки семейства 3, меняется, но их тип несложно определить, зная координаты. 

\paragraph{Утверждение~3.}
{\it
	Точки $\mathrm{rk}\,0$ семейства 3 (см. [4]) невырождена, если $J_3 \ne 0$ и $S = \varkappa J_3^2 - J_1^2 + \varkappa \ne 0$ (для $\sigma = 1$ в исходном или $\sigma = -1$ в псевдо-евклидовом случаях). Ее тип --- центр-центр при $S<0$ или центр-седло при $S >0$.
}

Эти утверждения позволяют получить немало информации о топологии псевдо-евклидовых аналогов системы Ковалевской, о чем подробнее мы расскажем. На рисунке изображены бифуркационные кривые псевдоевклидовой Ковалевской при $\varkappa = 0, b \ne 0$. В прообразе пунктирной параболы --- некритические некомпактные бифуркации слония.

\begin{figure}[h]
			\center{\includegraphics[width=90mm]{bif_curve_pseudo2.png}}
			\label{bif_diagr}
		\end{figure}



Работа выполнена в МГУ за счёт гранта РФФИ, проект 20-31-90114. Автор является стипендиатом Фонда ``Базис''.

\litlist

1. {\it Borisov A. V., Mamaev I. S.} Rigid body dynamics in non-Euclidean spaces  // Rus. J. of
Math. Phys. – 2016. – Т. 23. – №. 4. – С. 431-454.

2. {\it Komarov I. V.} Kowalewski basis for the hydrogen atom
  // Theoret. Math. Phys. – 1981. – Т. 47. – №. 1. – С. 320-324.

3. {\it Кибкало В. А.} Свойство некомпактности слоев и особенностей неевклидовой системы Ковалевской на пучке алгебр Ли  // Вестник МГУ, Сер. 1. – 2020. – №. 6. – С. 56-59.

4. {\it Козлов И. К.} Топология слоения Лиувилля для интегрируемого случая Ковалевской на алгебре Ли so(4)
  // Матем. сб. – 2014. – Т. 205. – №. 4. – С. 79-120.

\end{document}








Как следствие, что $a = 0$ во всех критические точки ранга 0 имеем $K = 0$ (в точках семейств 1 и 2 имеем $x_3 = J_3 = 0$,  т.е. они лежат в 4-орбите $a >0$).



На пространстве ограниченных последовательностей $l_\infty$ определяется оператор Чезаро $C$
равенством
$
	(Cx)_n = {1}/{n} \cdot \sum_{k=1}^n x_k.
$
Определим на $l_\infty$ $\alpha$--функцию,
характеризующую асимптотические свойства последовательности,
равенством
$$
	\alpha(x) = \varlimsup_{i\to\infty}\sup_{i < j \leqslant 2i} |x_i - x_j|.
$$
Пусть $A = \{x\in l_\infty | 0 \leqslant x_n \leqslant 1\}$.
Асимптотические свойства оператора Чезаро удобнее сначала изучать на множестве $A$,
а затем перенормировкой распространять на всё $l_\infty$.
В [1] изучается ряд свойств оператора Чезаро.
Можно легко доказать, что для $x\in A$ выполнено соотношение $\alpha(Cx) \leqslant 1/2$
и $\alpha(Cx) \leqslant \alpha(x)$.
Возникает естественный вопрос о справедливости более точной оценки.

\paragraph{Теорема~1.}
{\it
	Не существует такого $\gamma < 1$,
	что для любого $x\in A$ выполнена мультипликативная оценка
	$$
		\alpha(Cx) \leqslant \gamma \cdot \alpha(x)
	,
	$$
или, что то же самое, не существует такого $p\in \mathbb{N}$,
	что для любого $x\in A$ выполнено неравенство
	$
		\alpha(Cx) \leqslant (1-2^{-p+1})\cdot \alpha(x).
	$
}

Для доказательства теоремы~1 потребуются вспомогательные построения.
\begin{equation}\label{AvdSem_summa_drobey}
	\sum_{i=0}^{p-1} \frac{i \cdot 2^i}{p} = \frac{2^p(p-2) + 2}{p}
\end{equation}
Введём вспомогательный оператор $S:l_\infty \to l_\infty$:
\begin{equation*}\label{operator_S}
	(Sy)_k = y_{i+2}, \mbox{ где } 2^i < k \leqslant 2^i+1
\end{equation*}
Нам потребуются следующие свойства оператора $S$.
\begin{equation}\label{AvdSem_alpha_S}
	\alpha(Sx) = \varlimsup_{k\to\infty} |x_{k+1} - x_{k}|
\end{equation}
\begin{equation}\label{AvdSem_summa_S_less}
	\sum_{k=2}^{2^p} (Sy)_k =
	\sum_{i=0}^{p-1} 2^i y_{i+2}
\end{equation}
Здесь и далее $(Tx)_n = x_{n+1}$.
\begin{equation}\label{AvdSem_summa_S}
	\sum_{k=2^i+1}^{2^{i+j+1}} (Sx)_k =
	2^i\sum_{k=2}^{2^{j+1}} (ST^ix)_k
\end{equation}
Введём вспомогательную функцию
%\vspace{-2.28em}
\begin{equation*}\label{def_k_b}
	k_b(x) = (2b)^{-1} \left|
		\sum\nolimits_{k=1}^{b}x_k - \sum\nolimits_{k=b+1}^{2b}x_k
	\right|
\end{equation*}
Тогда
\begin{equation}\label{AvdSem_alpha_greater_k_b}
	\alpha (Cx) \geqslant \varlimsup_{i\to \infty} k_i(x)
\end{equation}

\paragraph{Схема доказательства теоремы~1.}
Зафиксируем $p$ и построим $y\in l_\infty$:
\begin{equation*}\label{y_construction}
	y = \left\{
		0, 0, \frac{1}{p}, \frac{2}{p}, %\frac{3}{p},
		...,
		\frac{p-1}{p}, 1, \frac{p-1}{p},
		...,
		\frac{1}{p},
		0, ..., 0,
		\frac{1}{p}, ...
	\right\}
\end{equation*}
так, что
\begin{equation}\label{AvdSem_T_y}
	T^{5p}y = y
\end{equation}
Положим $x = Sy$, тогда с учётом (\ref{AvdSem_alpha_S})
$
	\alpha (x) = \alpha (Sy) = \frac{1}{p}
$.

Оценим $\alpha(Cx)$, принимая во внимание (\ref{AvdSem_summa_drobey}) и (\ref{AvdSem_summa_S_less}) --- (\ref{AvdSem_alpha_greater_k_b}):
\begin{multline*}
	\alpha (Cx) \mathop{\geqslant}^{(\ref{AvdSem_alpha_greater_k_b})}
	\varlimsup_{b\to \infty} k_b(x) \geqslant
	\!\!\!
	\varlimsup_{
		i\to \infty,~
		b=2^i~
	}\frac{1}{2^{i+1}}\left|
		\sum_{k=1}^{2^i}(Sy)_k -
		\!\!\!\!\!
		\sum_{k=2^i+1}^{2^{i+1}}(Sy)_k
	\right|
	\!
	\geqslant
	\\ \geqslant
	\varlimsup_{
		m\to \infty,~
		i=5pm+p~
	}\left|
		\frac{1}{2^{5pm+p+1}}\sum_{k=1}^{2^{5pm+p}}(Sy)_k - \frac{y_{5pm+p+2}}{2}
	\right| =
	\\=
	\varlimsup_{m\to \infty}\left|
		\frac{1}{2^{5pm+p+1}}\sum_{k=1}^{2^{5pm}}(Sy)_k
		+
		\frac{1}{2^{5pm+p+1}}\sum_{k=2^{5pm}+1}^{2^{5pm+p}}(Sy)_k
		- \frac{1}{2}
	\right|
	\mathop{=}^{(\ref{AvdSem_summa_S})}
	\\=
	\varlimsup_{m\to \infty}\left|
		\frac{1}{2^{5pm+p+1}}\sum_{k=1}^{2^{5pm}}(Sy)_k
		+
		\frac{2^{5pm}}{2^{5pm+p+1}} \sum_{k=2}^{2^p}(ST^{5pm}y)_k
		- \frac{1}{2}
	\right|
	\mathop{=}^{(\ref{AvdSem_T_y})}
	\\=
	\varlimsup_{m\to \infty}\left|
		\frac{1}{2^{5pm+p+1}}\sum_{k=1}^{2^{5pm}}(Sy)_k
		+
		\frac{1}{2^{p+1}} \sum_{k=2}^{2^p}(Sy)_k
		- \frac{1}{2}
	\right|
	\mathop{=}^{(\ref{AvdSem_summa_S_less})}
	\\=
	\varlimsup_{m\to \infty}\left|
		\frac{1}{2^{5pm+p+1}}\sum_{k=1}^{2^{5pm}}(Sy)_k
		+
		\frac{1}{2^{p+1}} \sum_{i=0}^{p-1}2^i \cdot \frac{i}{p}
		- \frac{1}{2}
	\right|
	\mathop{=}^{(\ref{AvdSem_summa_drobey})}
%\end{multline*}
%\begin{multline*}
	\\=
	\varlimsup_{m\to \infty}\left|
		\frac{1}{2^{5pm+p+1}}\sum_{k=1}^{2^{5pm-2p}}(Sy)_k
		-\frac{1}{p} + \frac{1}{p 2^p}
	\right| \geqslant
	\\ \geqslant
	\varlimsup_{m\to \infty} \left(
		\frac{1}{p} (1-2^{-p})
		- \frac{1}{2^{3p+1}}
	\right) >
	\frac{1}{p} (1-2^{-p+1})
\end{multline*}


Таким образом,
$
	\alpha(Cx) >
	(1-2^{-p+1}) \cdot \alpha(x)
$.

