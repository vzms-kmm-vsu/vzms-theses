

\vzmstitle[
    \footnote{}
]

\sloppy
 {\textbf{МОДЕЛИРОВАНИЕ СОСТОЯНИЯ ДИНАМИЧЕСКОЙ СИСТЕМЫ МЕТОДОМ
 НЕОПРЕДЕЛЕННЫХ КОЭФФИЦИЕНТОВ} } \vzmsauthor{Раецкий }{К.\,А.} \vzmsinfo{Воронеж, ВГУ; {\it
kraetsky@mail.ru}}


\vzmscaption

\sloppy


 Рассматривается полностью управляемая
динамическая система
  \[\dot{x}(t)=Ax(t)+Bu(t), \eqno(1)\]
 где $x(t)\in \mathbb{R}^n$, $u(t)\in \mathbb{R}^m$,
  $A$ и  $B$  "--- матрицы соответствующих размеров, $\not\!\exists B^{-1}$; $t\in [t_0,
  t_k]$.

Решается задача моделирования траектории движения, или состояния
системы $x(t)$, удовлетворяющего условиям
\[x(t_i)=x_i, \quad i=0,\, 1,\, ...\,,\,  k,   \eqno(2)\]
$ t_i\in [t_0, t_k]$, $t_0<t_1<\, ...\, < t_k$, $\forall x_i\in
\mathbb{R}^n$.

То есть строится множество $X=\{x(t)\}$ различных траекторий,
удовлетворяющих условиям (2), для каждой из которых существует
соответствующее $u(t)$, такое, что $x(t)\in X$ является решением
уравнения (1) с этим $u(t)$.

Цель такого моделирования: предложить заказчику  различные варианты
траектории, затем для выбранного заказчиком <<подходящего>> варианта $x(t)$ вычислить соответствующее управление $u(t)$. При
практическом решении задачи  для реальной динамической системы
следует реализовать начальное положение $x(t_0)=x_0$ и полученное
управление $u(t)$. В силу единственности решения начальной задачи
для системы (1) с фиксированным $u(t)$ траектория (состояние)
динамической  системы будет именно выбранной <<подходящей>>.

Если же <<подходящего>> варианта заказчик не выбрал, то
управление $u(t)$ строить нецелесообразно, и следует искать
траектории (состояния) в другом классе вектор"=функций, или другим
методом. В этом состоит отличие задачи моделирования траектории  от
задачи управления.

Предлагается построение $x(t)$ и $u(t)$ методом неопределённых
коэффициентов [1,2], состоящем в формировании $x(t)$ и $u(t)$ в виде
линейных комбинаций линейно независимых скалярных функций с
векторными коэффициентами; подстановке этих комбинаций  в (1) и (2);
нахождении векторных коэффициентов из полученных алгебраических
уравнений.

В данной работе предлагается
\[x(t)=\sum_{j=1}^r\, a_jcosjt+b_jsinjt, \, u(t)=\sum_{j=1}^r\, c_jcosjt+d_jsinjt,    \eqno(3)\]
количество $r$ определяется позже. После подстановки (3) в (1) и
приравнивания коэффициентов при $cosjt$, $sinjt$ получаются
уравнения
\[
 \begin{array}{l}
 j\cdot b_j=Aa_j+Bc_j,\\
-j\cdot a_j=Ab_j+Bd_j,\, j=1,\, 2,\, ...\,,\,  r. \end{array}
\eqno(4)\]

Далее используется свойство отображения $B: \mathbb{R}^m\rightarrow
\mathbb{R}^n$:
\[\mathbb{R}^m=\mbox{Coim} B\dot{+} \operatorname{Ker} B,\quad
\mathbb{R}^n=\mbox{Im} B\dot{+}\mbox{Coker} B,  \eqno(5)
\]
где  $\operatorname{Coker} B$ "--- дефектное подпространство,  $\operatorname{Coim} B$ "--- прямое
дополнение к подпространству $\operatorname{Ker} B$ в $\mathbb{R}^m$. Сужение
$\widetilde{B}$ на  $\operatorname{Coim} B$ имеет обратное отображение
$\widetilde{B}^{-1}$. Через $P$ обозначен проектор на$\operatorname{Ker} B$, через
$Q$ "--- проектор на  $\operatorname{Coker} B$, отвечающие разложениям (5);  через
$B^{-}$ "--- полуобратный оператор к $B$, то есть $B^{-} =
\widetilde{B}^{-1} (I-Q)$.

Расщепление соотношений (4) на уравнения в подпространствах $\operatorname{Im} B$ и
$\operatorname{Coker} B$ и переход от $\operatorname{Im} B$ в $\operatorname{Coim} B$ приводит к равенствам
\[c_j=B^-(j\cdot b_j-Aa_j)+\alpha_j,\,  \,  d_j=B^-(-j\cdot
a_j-Ab_j)+\beta_j, \eqno(6)\] $ \forall \alpha_j,\, \beta_j\in
\operatorname{Ker} B$, и уравнениям для нахождения $a_j$ и $b_j$:
\[
 \begin{array}{l}
 jQb_j=(QAQ)Qa_j+(QA(I-Q))(I-Q)a_j,\\
-jQa_j=(QAQ)Qb_j+(QA(I-Q))(I-Q)b_j.
\end{array} \eqno(7)\]
Вводятся обозначения
\[
 \begin{array}{c}
QAQ=A_1, \, QA(I-Q)=B_1, \, Qa_j=a_j^1, \, (I-Q)a_j=a_j^2, \\
Qb_j=b_j^1, \, (I-Q)b_j=b_j^2. \end{array}  \eqno(8)\]
 Из (7), (8) следует:
 \[a_j=a_j^1+a_j^2,\,b_j=b_j^1+b_j^2, \eqno(9)\]
\[
 \begin{array}{c}
jb_j^1=A_1a_j^1+B_1a_j^2,\\
-ja_j^1=A_1b_j^1+B_1b_j^2.  \end{array}  \eqno(10)\] Система (10) по
виду аналогична системе (4), но содержит меньшее количество
скалярных уравнений  за счёт отщепления  от (4) равенств (6). Далее
следует воспользоваться свойством отображения $B_1: \mbox{Im}
B\rightarrow \mbox{Coker} B$: \[\mbox{Im}B=\mbox{Coim} B_1\dot{+}
\mbox{Ker} B_1,\quad \mbox{Coker}B=\mbox{Im} B_1\dot{+}\mbox{Coker}
B_1
\] и отщепить от (10) формулы для выражения $a_j^2$  и  $b_j^2$
через $a_j^1$ и $b_j^1$, и оставшиеся соотношения привести к виду,
аналогичному (4) или (10),  и так далее... .

Ограничимся случаем $\operatorname{Coker} B_1=\{0\}$. Из (10) получаем:
\[
 a_j^2=B_1^-(j\cdot b_j^1-A_1a_j^1)+\gamma_j,\,  \,
b_j^2=B_1^-(-j\cdot a_j^1-A_1b_j^1)+\delta_j, \eqno(11)\] $  \,
\forall \gamma_j,\, \delta_j\in \mbox{Ker} B_1$. Коэффициенты
$a_j^1$ и   $b_j^1$ находятся из (2), для чего в (2) подставляются
выражения (3) и полученные соотношения расщепляются на соотношения в
подпространствах  $\operatorname{Coker} B$ и $\operatorname{Im} B$:
\[\sum_{j=1}^r\, a_j^1 cosjt_i+b_j^1 sinjt_i=Qx_i,     \eqno(12)\]
\[\sum_{j=1}^r\, a_j^2cosjt_i+b_j^2sinjt_i=(I-Q)x_i.     \eqno(13)\]
Умножив (13) слева на $B_1$ (здесь $B_1$ полуобратимый), и
воспользовавшись равенствами (10) и (12). получаем вместе с (12)
систему
\[ \begin{array}{l}
\sum_{j=1}^r\, a_j^1 cosjt_i+b_j^1 sinjt_i=Qx_i,\\
-\sum_{j=1}^r\,j a_j^1 sinjt_i+jb_j^1 cosjt_i=A_1Qx_i+B_1(I-Q)x_i
\end{array} \eqno(14)\]
для нахождения $a_j^1$, $ b_j^1$. Здесь $2(k+1)$ уравнений с $2r$
неизвестными, следовательно, $r=k+1$.  Значения $t_i$ следует взять
такими, чтобы определитель $\triangle$ системы (14) был отличен от
нуля. Тогда из (14) определяются $a_j^1$ и $ b_j^1$, из (11) -
$a_j^2$ и $ b_j^2$, из (9) -  $a_j$ и $ b_j$, из (3) - $x(t)$. Для
построения <<подходящей>>  траектории варьируются векторы
$\gamma_j,\, \delta_j\in \mbox{Ker} B_1$, и если таковые находятся,
то рассчитывается $u(t)$ по формулам (4), (6). Коэффициенты
$\alpha_j,\, \beta_j$ в (7) следует фиксировать, исходя из
дополнительных условий (субоптимальное управление, ...).

Заметим [3], что $\operatorname{Coker} B_1=\{0\}\rightleftarrows\operatorname{rank}  (B\,
AB)=n$.
\paragraph{Теорема~1.} {\it Если $\operatorname{rank} (B\, AB)=n$ и $t_i$ таковы, что  $\triangle\neq
0$, то существуют $x(t)$ и $u(t)$, удовлетворяющие $(1)$ и $(2)$ в
виде $(3)$.}

Примером динамической системы со свойством $\operatorname{rank} (B\, AB)=n$
является модель движения материальной точки в вертикальной плоскости
под действием реактивной силы.

% Оформление списка литературы
\litlist

1. {\it Zubova S.P., Raetskiy K.A.}  Modeling the trajectory of
motion of a linear dynamic system with multi-point conditions //
Mathematical Biosciences and Engineering. - 2021.  - T. 18. - V. 6.
- P. 7861-7876.

2. {\it Раецкий К.А.} Построение модели движения линейной
динамической системы с многоточечными условиями // Таврический
вестник информатики и математики. - 2021. -  № 1. -  С. 65-80.

3. {\it Zubova S.P, Raetskaya E.V.}  Solution of the  multi-point
control problem for a dynamic system in partial derivatives //
Mathematical Methods in the Applied Sciences, AIMS, New York. -
2021. - V. 44. - N.15. -  P. 11998-12009.
