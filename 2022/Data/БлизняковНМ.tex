\vzmstitle[

]{
	метод ДИСКРЕТНОГО ФУНКЦИОНАЛА И ПРОЕКЦИЙ ФУНКЦИЙ ДРОБНОГО ПОРЯДКА
ЦЕНТРАЛЬНЫХ МОМЕНТОВ ДЛЯ АНАЛИЗА ОДНОРОДНОСТЕЙ ДИСКРЕТНЫХ СЛУЧАЙНЫХ ВЕКТОРОВ
}
\vzmsauthor{Близняков}{Н.\,М.}
\vzmsinfo{Воронеж, ВГУ; {\it bliznyakov@vsu.su}}
\vzmsauthor{Вахтель}{В.\,М.}
\vzmsinfo{Воронеж, ВГУ}
\vzmsauthor{Костомаха}{Д.\,Е.}
\vzmsinfo{Воронеж, ВГУ}
\vzmsauthor{Работкин}{В.\,А.}
\vzmsinfo{Воронеж, ВГУ; {\it rabotkin@phys.vsu.su}}

\vzmscaption
{ В радиометрии различных излучений применяется счётный режим
измерений, то есть получение последовательностей М серий, каждая из
$ n $    отсчётов  $k_{j} (\Delta t)\geqslant 0$
, $j=l,n$
  случайных количеств частиц в постоянном интервале времени }$\Delta t_{\,
}${   в течение полного времени измерения  }$T=\Delta t${ .   В
циклических условиях получения М серий, каждая длительностью  }$\theta
=n\Delta t\cdot T${    при~}$M=\frac{T}{\theta }q${ $_{~}${
стационарность и однородность полученных последовательностей серий отсчётов
может нарушаться.}

{ В течение одного
цикла ~ }${{\theta
}}${ ~ фиксируется серия из $n $ последовательных
интервалов~}$\Delta t${ ~ и соответствующих значений
}$K_{i}(\Delta t)${ , которые образуют случайную выборку ~}$\{K_{i}(\Delta t)\}${
 ~ и соответствующий }{ $q-$}{ й случайный вектор
(СВР): ~}$\nu(.)_{q}=(\nu_{0},...,\nu_{l})_{q}${ ~ частот ~}$\nu_{j}${ , с
которыми случайная величина $K_{j}(\Delta t)$ принимает значение
 $j $ и }$n = \sum\limits_{j=0}^l \nu_{jq},~\nu_{jq}=0,1,...,n$.
То есть СВР -- это реализация флуктуаций возможных комбинаций $\nu_{jq}$, например с полиномиальной
вероятностью $P(\nu_{jq}(.))$.

{ Вследствие стохастических механизмов, определяющих генерацию
отсчётов ~}$K_{j}(\Delta t)${ ~ и статистических флуктуаций процессов
формирования СВР, параметры которых~}$\theta,n,\Delta t${
$_{~}${ и ~}{ $I$}{~ - средняя интенсивность отсчётов заданы,
анализ характеристик совокупностей СВР при }$M=1${ ~ и
малой статистике традиционно сложен, учитывая, что количество возможных
различных СВР~}$Q=1${ .}

{ В данной работе предложен метод обработки и анализа характеристик
флуктуаций больших совокупностей СВР на основе: (\ref{eq1}) анализа характеристик
распределений целочисленного идентификатора -- функционала СВР в виде
скалярного произведения ~}$ID_{q}=\sum\limits_{i}^l a_{i}\nu_{j4}${ ~СВР - }$\nu_{q}(.)${ ~ и заданного неслучайного вектора; }$a=(a_{0},a_{1},...,a_{l})${ ; (\ref{eq2}) анализа проекций фазовых траекторий функций дробного порядка ~}$l<S<5${ ~ комплексных
центральных~моментов ~}$\mu(S_{l},\nu_{q}(.))${ ~ СВР: \\
}$\mu=(S,\nu_{q},a)=\frac{1}{n-1} \sum\limits_{i=1}^l (k_{i,\bar{q},\bar{k_{q}}})^{S}=Re(\mu(S,\nu_{q},a))+iIm(\mu(S,\nu_{q},a)),i^{2}=-1${

{ Таким образом, каждому СВР - }$\nu(.)_{q}${ ~ можно
поставить в соответствие функционал ~}$ID(\nu_{q},\nu)_{q}${ ~ в виде
скалярного произведения СВР и определённого вектора }$a=(a_{0},a_{1},...,a_{l})${ ~ с натуральными компонентами }$a_{j}${. ~Можно доказать, что для однозначности соответствия необходимо, чтобы
компоненты вектора  }{ $a$}{ ~не имели общих делителей. Основная часть
доказательства приведена ниже.}

{ Из однозначности соответствия ~}$ID(.)_{q}${ ~ и
СВР}{ $_{q}${  следует однозначность
соответствия упорядоченной последовательности}
\[
ID_{l}(\nu(.)_{l},a)\leq ID_{2}(\nu(.)_{2},a)\leq...\leq ID_{M}(\nu(.)_{M},a)
\]
{ дискретному распределению частот функционала ~ }$0\leq M_{j}(ID_{j}(.),a)\leq M${ ~ в виде последовательностей
 сгруппированных пиков. СВР,
образующие каждый пик ~}{ $m$}{ , имеют однородные характеристики,
удовлетворяющие определённым условиям. При }$a_{0}<...<a_{l}${
компоненты }$\nu_{i,q}${  СВР, образующего пик }{
$m$}{ , удовлетворяют условиям }
\[
\nu_{0,q}=(m)+\nu_{1,\varphi}(m)=const(m),
\]
\[
\nu_{l,q}(m)=const(l,m), \nu_{l-1,q}(m)=const(l-1,m).
\]
{ При }$a_{0}>a_{1}>...>a_{l}${  распределение }$M(ID(\nu,a))${  также имеет вид последовательности пиков, образованных
СВР, удовлетворяющих условиям их однородности }
$ \nu_{l,q}+\nu_{l-1,q}=const(m), ~
 \nu_{0}(m)=const(m),   $
{  }
 $ \nu_{1}(m)=const(m) $

{ Исходные, то есть неупорядоченные последовательности }$ID_{q}(.),  q=l,M${,
полученные с различными векторами  }{ $a$}{  позволяют выявлять тренды в
последовательностях  }$k_{i}(\Delta t),  i=l,N${ , анализируя средневыборочные
значения }$\bar{k_{q}}=\sum\limits_{l}^n k_{iq}/n${ , что эквивалентно сглаживанию
высокочастотных флуктуаций }$k_{j}(.)${ .}

{ Отметим, что }$ID_{l}(.)${  и соответствующие СВР
каждого типа  }{ $q$}{ образуются независимо и случайно и поэтому
они распределены равномерно в исходной последовательности }$ID_{1}(.),...,ID_{M}${ ,  а интервалы
}$ID${  между
идентичными СВР согласуются с геометрическим распределением, а в асимптотике
-- с экспоненциальным со средним значением  }
$\overline{ID}_{q}=1/P_{q}$
{  и
среднеквадратичным разбросом  }$G(ID)=\overline{ID}=1/P${ , где }$P_{q}(.)${
полиномиальная вероятность реализации СВР. Поэтому
последовательности идентичных СВР можно анализировать по их совокупностям
}{ \textit{ID}}{   на основе модели Пуассона.}

{ Степень однородности СВР пика неидентичных, но подобных по
соотношениюих компонент, например}
\[
\nu_{0}+\nu_{1}=const, \nu_{l}=const(l,m), \nu_{l-1}=const(l-1)
\]
{ можно оценить по матрице метрики проекций функции дробных
порядков  }$l<S<5${  центральных моментов. Проекции }$\mu(S,\nu,a)${  СВР на плоскость}
\[
Re(\mu(S),\nu), Im(\mu(S),\nu)
\]
{ характеризует подобие СВР в целом, а проекции функции  }$\mu(S,\nu,a)${
на поверхность }$Im(\mu(S),\nu)$, {
характеризует неоднородность с учётом выборочных средних значений }$\bar{k_{q}}${ .}

{ Лемма. Пусть }{ $n,k$}{  -- некоторые фиксированные
натуральные числа, }$n\leq k${  и }$(a_{1},a_{2},...,a_{n})${
-- некоторое решение уравнения}
\begin{equation}
\label{eq1}
x_{1}+x_{2}+...+x_{n}=k,
\end{equation}
{ где }$a_{1},a_{2},...,a_{n}${  -- целые неотрицательные числа.}

{ Пусть }$m_{1},m_{2},...,m_{n}${   -- натуральные числа,
удовлетворяющие условиям}
\begin{equation}
\label{eq2}
km_{1}<m_{2}, km_{1}+km_{2}<m_{3},...,km_{1}+...+km_{n-1}<m_{n}.
\end{equation}
{ Обозначим}
\begin{equation}
\label{eq3}
m_{1}a_{1}+m_{2}a_{2}+...+m_{n}a_{n}=l.
\end{equation}
{ Тогда система уравнений}
\begin{equation}
\label{eq4}
X(\omega) =
 \begin{cases}
   x_{1}+x_{2}+...+x_{n}=k\\
   m_{1}x_{1}+m_{2}x_{2}+...+m_{n}x_{n}=l ,
 \end{cases}
\end{equation}
имеет единственное целочисленное неотрицательно решение $(a_{1},a_{2},...,a_{n})$ .

{ Доказательство. Предложим противное. Пусть $(b_{1},b_{2},...,b_{n})$   -- целочисленное неотрицательное решение уравнения (\ref{eq1}),
отличное от }$(a_{1},a_{2},...,a_{n})${ . Пусть p -- наибольшее из чисел
}$1,2,...,n${ , для которого }$a_{i}\neq b_{i}${ ,
}$i=1,2,...,n${ , т.е. }$a_{p}\neq b_{p}${ , }$a_{p+1}=b_{p+1},...,a_{n}=b_{n}${ . Тогда}
\begin{equation}
\label{eq5}
m_{p+1}a_{p+1}+...+m_{n}a_{n}=m_{p+1}b_{p+1}+...+m_{n}b_{n}
\end{equation}
{ Возможны два случая:}

{ 1) }$a_{p}<b_{p}${ .}

{ Учитывая неравенства (\ref{eq2}), равенство }$b_{p}-a_{p}\geq 1${  и
неравенства }$a_{i}\leq k${ , }$b_{i}\leq k${ ,
}$b_{i}+k\geq a_{i}, i=1,2,...,n$

{ Имеем}
\[
m_{1}b_{1}+...+m_{p-1}b_{p-1}+m_{p}b_{p}=
\]
\[
(m_{1}b_{1}+...+m_{p-1}b_{p-1})+
m_{p}(b_{p}-a_{p})+m_{p}a_{p} \geq
\]
\[
\geq(m_{1}b_{1}+...+m_{p-1}b_{p-1})+m_{p}+m_{p}a_{p}>
\]
\[
>m_{1}b_{1}+...+m_{p-1}b_{p-1}+(m_{1}k+...+m_{p-1}k)+m_{p}a_{p}=
\]
\[
=m_{1}(b_{1}+k)+...+m_{p-1}(b_{p-1}+k)+m_{p}a_{p}\geq
\]
\[
\geq m_{1}a_{1}+...+m_{p-1}a_{p-1}+m_{p}a_{p}.
\]
{ Таким образом, справедливо неравенство}
\begin{equation}
\label{eq6}
m_{1}b_{1}+...+m_{p-1}b_{p-1}+m_{p}b_{p}>m_{1}a_{1}+...+m_{p-1}a_{p-1}+m_{p}a_{p}.
\end{equation}
{ Учитывая равенство (\ref{eq5}) и неравенство (\ref{eq6}) получаем}
\[
(m_{1}b_{1}+...+m_{p-1}b_{p-1}+m_{p}b_{p})+(m_{p+1}b_{p+1}+...+m_{n}b_{n})>
\]
\[
(m_{1}a_{1}+...+m_{p-1}a_{p-1}+m_{p}a_{p})+(m_{p+1}a_{p+1}+...+m_{n}a_{n})=l
\]
{ Откуда и следует, что } $ (b_{1},...b_{n})${   не является
решением системы (\ref{eq4}).}

{ 2) }$a_{p}> b_{p}${ .}

{ Рассмотрим два подслучая:}

{ 2.1) }$b_{p}=0${ .}

{ Учитывая неравенства (\ref{eq2}) и неравенства ~\\ }$a_{p}\geq b_{p}+1\geq 1${
  имеем}
\[
(m_{1}b_{1}+...+m_{p-1}b_{p-1})+m_{p}b_{p}=(m_{1}b_{1}+...m_{n-1}b_{n-1})<
\]
\[
<(m_{1}b_{1}+...+m_{p-1}b_{p-1})+(m_{1}a_{1}+...+m_{p-1}a_{p-1})<
\]
\[
<(m_{1}a_{1}+. .+m_{p-1}a_{p-1})+m_{p}\leq m_{1}a_{1}+. .+m_{p-1}a_{p-1}+m_{p}a_{p}.
\]
{ Таким образом, справедливо неравенство}
\begin{equation}
\label{eq7}
m_{1}b_{1}+. .+m_{p-1}b_{p-1}+m_{p}b_{p}<m_{1}a_{1}+. .+m_{p-1}a_{p-1}+m_{p}a_{p}.
\end{equation}
{ Учитывая равенство (\ref{eq5}) и неравенство (\ref{eq7}) получаем}
\[
(m_{1}b_{1}+. .+m_{p-1}b_{p-1}+m_{p}b_{p})+(m_{p+1}b_{p+1}+. .+m_{n}b_{n})<
\]
\[
<(m_{1}a_{1}+. .+m_{p-1}a_{p-1}+m_{p}a_{p})+(m_{p+1}a_{p+1}+. .+m_{n}a_{n})=l
\]
{ Откуда и следует, что }$(b_{1},...,b_{n})${    не является
решением системы (\ref{eq4}).}

{ 2.2) }$b_{p}>0${ .}

{ Учитывая неравенства (\ref{eq2}),  неравенство \\ }$a_{p}\geq b_{p}+1${    и
неравенства }$b_{i}\leq k, i=1,2,...,n$
{ имеем}
\[
(m_{1}b_{1}+. .+m_{p-1}b_{p-1})+m_{p}b_{p}\leq (m_{1}k+. .+m_{p-1}k)+m_{p}b_{p}<
\]
\[
<m_{p}+m_{p}b_{p}=m_{p}(b_{p}+1)\leq m_{p}a_{p}\leq
\]
\[
\leq (m_{1}a_{1}+...+m_{p-1}a_{p-1})+m_{p}a_{p}.
\]
{ Таким образом, справедливо неравенство}
\begin{equation}
\label{eq8}
m_{1}b_{1}+. .+m_{p-1}b_{p-1}+m_{p}b_{p}<m_{1}a_{1}+...+m_{p-1}a_{p-1}+m_{p}a_{p}
\end{equation}
{ Учитывая равенство (\ref{eq5}) и неравенство (\ref{eq8}) получаем}
\[
(m_{1}b_{1}+. .+m_{p-1}b_{p-1}+m_{p}b_{p})+(m_{p+1}b_{p+1}+...+m_{n}b_{n})<
\]
\[
<(m_{1}a_{1}+. .+m_{p-1}a_{p-1}+m_{p}a_{p})+(m_{p+1}a_{p+1}+...+m_{n}a_{n})
\]
Откуда и следует, что $(b_{1},...,b_{n})$ не является
решением системы (\ref{eq4}).
