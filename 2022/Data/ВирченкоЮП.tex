\vzmstitle[]{ПЕРКОЛЯЦИОННЫЕ СТРУКТУРЫ}
\vzmsauthor{Вирченко}{Ю.\,П.}
\vzmsinfo{Белгород, БелГУ; {\it virch@bsu.edu.ru}}

\vzmscaption

Посредством выделения общих свойств всех математических моделей, которые называются в статистической физике перколяционными,  предлагается формулировка абстрактного понятия математической структуры  ${\frak P}$, в смысле Бурбаки, которую мы называем {\it перколяционной}. По нашему мнению, вводимое нами понятие объединяет в единый класс все имеющиеся к настоящему времени математические модели, основанные на физическом понятии перколяции.

Перколяционная структура  ${\frak P}$ представляет собой {\it четвёрку} $\langle {\frak S}, {\frak M}, {\frak W}, \Gamma \rangle$ математических структур, в которой ${\frak S}$ "--- метрическое пространство, ${\frak M} \subset {\sf P}({\frak S})$ "--- семейство {\it допустимых} подмножеств $A$ в ${\frak S}$, ${\frak W}$ "--- вероятностное пространство, в котором  ${\frak M}$ является пространством элементарных  событий (случайных множеств), $\Gamma$ "--- отношение связности на ${\frak S}$, определённое для всех множеств из ${\frak M}$. Отношение $\Gamma$ является измеримым по отношению к структуре измеримости, на вероятностном пространстве ${\frak W}$.

Отношение связности $\Gamma$ выделяет в ${\frak M}$ подкласс ${\frak C}$ связных подмножеств. Ввиду измеримости отношения $\Gamma$, в перколяционной структуре для каждого множества $A \in {\frak M}$ определена вероятность $\mathrm{Pr}\{A \in {\frak C}; A \subset B, B \in {\frak M}\}$, то есть вероятность того, что случайное множество $A$ связно. В частности, определены такие вероятности $p({\sf x}_1, ..., {\sf x}_n)$  для каждого конечного набора элементов ${\sf x}_1, ..., {\sf x}_n$ из ${\frak S}$. Их значения представляют вероятности связности всех этих элементов ${\sf x}_1, ..., {\sf x}_n$.

Особенный интерес представляют перколяционные структуры, у которых определяющее их метрическое пространство ${\frak S}$ является некомпактным. В этом случае для каждой точки ${\sf x} \in {\frak S}$ определена вероятность
$${\rm Pr}\{ \exists ({\sf y} \in {\frak S}) ({\sf y} \in A, A \in {\frak M}, {\rm dist}({\sf x}, {\sf y}) \ge r > 0)  \}\equiv P ({\sf x}, r) \eqno (1) $$
связности точки ${\sf x}$ с некоторой точкой ${\sf y}$, удалённой на расстояние не менее чем $r$. На основе же этого функционала вводится вероятность $P({\sf x})$ существования бесконечного пути из точки ${\sf x}$ (ухода их этой точки на бесконечность)
$$P({\sf x}) = \lim_{r \to \infty} P({\sf x}, r)\,. \eqno (2) $$
В том случае, когда эта вероятность отлична от нуля, говорят о наличии {\it перколяции на бесконечность} из заданной точки ${\sf x} \in {\frak S}$.

С точки зрения изучения явления перколяции, наибольший интерес представляют собой перколяционные структуры, которые допускает такое погружение пространства ${\frak S}$  в ${\Bbb R}^m$, $m \in {\Bbb N}$, при котором расстояние ${\rm dist}(\cdot, {\cdot})$ на ${\frak S}$ согласовано с расстоянием в ${\Bbb R}^m$. Такие перколяционные структуры мы называем {\it конечномерными}. Это связано с тем, что в неконечномерных перколяционных структурах, при всяком <<разумно>> заданном распределении вероятностей случайных множеств, всегда имеется перколяция на бесконечность. Кроме того, в конечномерных перколяционных структурах возможно определение {перколяции  через все пространство} ${\Bbb R}^m$. А именно, для каждой пары противоположных граней $S_1$, $S_2$ каждого параллелепипеда $\Lambda \subset {\Bbb R}^m$ можно ввести вероятность
$$P_\Lambda(S_1, S_2) = {\rm Pr}\{\exists ({\sf x}_1 \in S_1, {\sf x}_2 \in S_2) (\{{\sf x}_1, {\sf x}_2\}\subset A, A \cap \Lambda \in {\frak M}) \} \eqno (3)$$
перколяционного перехода с одной грани на другую по внутренней части $\Lambda$. На основе вероятностей $P_\Lambda (S_1, S_2)$  и расширяющейся последовательности  $\langle \Lambda_k \subset {\Bbb R}^m; k \in {\Bbb N}\rangle$  параллелепипедов такой, что $\lim_{k \to \infty}\Lambda_k = {\Bbb R}^m$, вводится вероятность
$$P  = \lim_{k \to \infty} P_{\Lambda_k} (S_1, S_2)\,.  \eqno (4)$$
Если вероятность (4) отлична от нуля говорят о наличии перколяции через ${\Bbb R}^m$ в направлении, перпендикулярном граням $S_1$ и $S_2$ по случайной реализации множества из ${\frak S}$.

Наконец, укажем, что, также как в случае гиббсовских случайных полей, которые изучаются с целью их применения в статистической механике, наибольший интерес представляет не вычисление вероятности перколяции для фиксированного распределения вероятностей  в вероятностном пространстве ${\frak W}$, а её изучение для целого класса распределений вероятностей, параметризованного некоторым набором параметров $\langle\theta_1, ..., \theta_s \rangle$, которые принимают значения в некоторой замкнутой области $\Omega \subset {\Bbb R}^s$. Тогда область $\Omega$ разбивается, вообще говоря, на открытые связные множества $\Sigma_j$, в которых имеется перколяция, а в их дополнении $\Omega \setminus \bigcup_j \Sigma_j$ вероятность перколяции равна нулю. В этом случае говорят о наличии фазового перехода в перколяционной структуре. Поверхность, вообще говоря многочсвязную, являющуюся границей областей $\bigcup_j \Sigma_j$, называют {\it фазовой диаграммой}. Такое положение имеет место как в случае перколяции из  фиксированной точки ${\sf x} \in {\frak S}$, так и в отношении перколяции через все пространство ${\Bbb R}^m$.

Сделаем ещё два замечания в связи с построением конкретных реализаций перколяционных структур. Во"=первых, как и во всяком содержательном вероятностном пространстве, структуры измеримости в вероятностных пространствах ${\frak W}$ перколяционных структур должны быть основаны на счётно"=порождённых $\sigma$-алгебрах. Во"=вторых, при определении содержательного отношения связности $\Gamma$ приходится избегать наличие в классе ${\frak M}$ таких множеств, которые содержат канторовскую компоненту, то есть континуальное замкнутое множество, которое нигде не плотно в ${\frak S}$.
