\vzmstitle[
    % \footnote{Работа выполнена при поддержки гранта}
    ]{
    Напряжения на поверхности сопаряжения упругой и нелинейно-наследственной среды при механических и тепловых нагрузках
}
\vzmsauthor{Бырдин}{А.\,П.}
% \vzmsinfo{Воронеж; {\it tata\_sti@rambler.ru}}
\vzmsauthor{Сидоренко}{А.\,А.}
% \vzmsinfo{Воронеж, ВГУ; {\it artem132rus@gmail.com}}
\vzmscaption

Рассмотрим осесимметрическую деформацию полого\linebreakкругового цилиндра, заключенного в упругую оболочку с толщиной стенки $h \ll b-a$,
на внутренней $r=a$ и внешней $r=b+h$ поверхностях конструкции действуют давления $P_a(t)$ и $P_b(t)$ и заданы температуры $T_a$ и $T_b$.
Предполагаем величину осевой деформации постоянной (длиный цилиндр), механические характеристики системы независящими от температуры, материал цилиндра несжимаемый
\begin{equation}\label{eq:1}
    \begin{gathered}
        \varepsilon_r(r,t)+\varepsilon_\phi(r,t)+\varepsilon^0_z=3\alpha_0 \theta(r), \ Q(r)=\frac{T(r)-T_a}{T_a},
    \end{gathered}
\end{equation}
\noindent с реологией, описываемой интегростепенным полиномом [1]
\begin{equation}\label{eq:2}
    \begin{gathered}
        \frac{\sigma_r(r,t)-\sigma_\phi(r,t)}{2G_0}=G_1(\varepsilon_r-\varepsilon_\phi)+a_3(G_3\varepsilon_r-G_3\varepsilon_\phi),
    \end{gathered}
\end{equation}
\noindent где $\varepsilon_k$, $\sigma_k(k=r,\phi)$ - радиальные и окружные деформации и напряжения,
T(r) - решение уравнения теплопроводности с заданными условиями на границах,
$\alpha_0=\alpha T_a$, $\alpha$ - коэффициент линейного расширения цилиндра, $a_3 \ll 1$,
$G_0$ - нерелаксированный модуль сдвига материала, $G_n(n=1;3)$ - нелинейный и полилинейный интегральные операторы Вольтерра с ядрами ($\delta(t)$ - $\delta$ - функция)
\begin{equation}\label{eq:3}
    \begin{gathered}
        G_1(t)=\delta(t)-g_1(t), G_3(t_1,t_2,t_3)=\Pi^3_{k=1} \delta(t_k)-g_3(t_1,t_2,t_3),
    \end{gathered}
\end{equation}
\noindent где $g_1(t)$, $g_3(t_1,t_2,t_3)$ - регулярные части ядер наследственности.
В указанных условиях нагружения конструкции возникают только радиальные перемещения $u(r,t)$, радиальные, осевые и кольцевые деформации и напряжения.

Уравнение движение системы с граничными условиями (начальные условия – нулевые) имеет вид 
\begin{equation}\label{eq:4}
    \begin{gathered}
        \frac{\partial\sigma_r}{\partial r}-\frac{\sigma_r-\sigma_\phi}{r}=\rho\frac{\partial^2u}{\partial t^2}, \sigma_r(a,t)=-P_a(t), \\
        \sigma_r(b,t)=-P_b(t)-\frac{hE_1}{b^2 (1-\upsilon^2_1)}u(b,t)-h\rho_1{\partial^2u(b,t)}{\partial t^2},
    \end{gathered}
\end{equation}
\noindent где $\rho$, $\rho_1$ - плотности материалов цилиндра и оболочки,
$E_1$, $v_1$ - модуль упругости и коэффициент Пуассона материала оболочки. В безразмерных переменных
\begin{equation*}
    \begin{gathered}
        r'=\frac{r}{a}, u'=\frac{u}{a}, t'=\beta_0t, \beta^2_0=\frac{2G_0}{\rho a^2}, P'_a, \\
        b=\frac{P_a,b}{2G_0}, \sigma'_r,\phi=\sigma_r,\frac{\phi}{2G_0}
    \end{gathered}
\end{equation*}
(штрифи отбросим) из условия (\ref{eq:1}) получим 
$$u(r,t)=\frac{Y(t)}{r^2}-\frac{\varepsilon^0_z}{2}+\frac{3\alpha}{r^2}I(r), I(r)=\int^{r}_{1} \theta(r)rdr,$$
\noindent где функция $Y(t)$ определяется из уравнений (\ref{eq:4}) и (\ref{eq:2}). Из уравнения (\ref{eq:4}) с граничными условиями,
получается нелинейное интегродифференциальное уравнение для $Y(t)$ с нулевыми начальными условиями, которое решаем методом возмущений по параметру $a_3$.
$$Y(t)=Y_0(t)+a_3Y_1(t)+\ldots$$

Функции $Y_n(t)$ удовлетворяют уравнениям вида
\begin{equation}\label{eq:5}
    \begin{gathered}
        (\frac{d^2}{dt^2}+A_1 G_1)Y_n(t)=F_n(t), (n=0,1,...),
    \end{gathered}
\end{equation}
$$F_0(t)=\frac{P_a(t)-P_b(t)}{\ln r_0}-\frac{3\alpha_0I(r_0)}{r^2_0\ln r_0}G_10(t), G_n0(t)=G_n \cdot 1,$$
\begin{equation*}
    \begin{gathered}
        F_1(t)= \\
        =\bigg[\frac{\rho_1}{\rho\ln r_0}\bigg(\frac{d^2}{dt^2}+B\bigg)+(A_3 G_3 - A_2 G_{32} + B_1 G_{31})\bigg] \cdot \\
        \cdot Y_0+B_0 G_{30}(t),
    \end{gathered}
\end{equation*}
\begin{equation*}
    \begin{gathered}
        A_n=\frac{1}{n\ln r_0}\bigg(1-\frac{1}{r^2n_0}\bigg) (n=1,3), \\
        B=\frac{\beta^2_1}{\beta^2_0\ln r_0(1-v_1)}, \beta^2_1=\frac{E}{\rho b^2},
    \end{gathered}
\end{equation*}

\noindent где постоянные $A_2$ и $B_n$ $(n=0,1)$ зависят от температуры $\theta(r)$,
геометрических и реологических параметров материалов конструкции $G_3k$ - операторы вида
\begin{equation*}
    \begin{gathered}
        G_{3kf}(t)= \\
        =\int\!\!\int\!\!\int^{t}_{0}G_3(t-t_1,t-t_2,t-t_3)\Pi^k_{m=1}Y_0(t_m)dt_m\Pi^3_{l=k+1}f(t_l)dt_l.
    \end{gathered}
\end{equation*}


Пусть $K$ - резольвентный оператор для оператора в (\ref{eq:5}). Если ядро оператора $G_1$ описывает модель Кельвина, то ядро $K(t)$ выражается через экспоненты с отрицательными показателями,
но если ядро $G_1(t)$ является слабосингулярной функцией, то ядро резольвенты $K(t)$ представляет собой функциональный ряд.
Из линейности $K$ и структуры функции $F_0(t)$ следует, что вклады в деформации материала, порождаемые $Y_0(t)$ от действия давления и перепада температур на поверхностях разделяются.
Решение уравнения для $Y_1(t)$ получим в предположении сепарабельности ядра оператора $G_3$.

Из граничного условия в (\ref{eq:4}), используя построенное решение для радиального перемещения Y(t), получим выражение для радиального напряжения на поверхности сопряжения цилиндра и оболочки. В первом приближении по параметру нелинейности имеем
\begin{equation}\label{eq:6}
    \begin{gathered}
        \sigma_r(r_0,t)\approx P_0(t)-a_3\bigg(\frac{\rho_1}{\rho}+B\ln r_0-A_1G_1\bigg)KF_0(t).
    \end{gathered}
\end{equation}
При отсутствии разности температур на поверхностях\linebreak консструкции второй член в выражении $F_0(t)$ исчезает и выражение $\sigma_r$ совпадает с контактным напряжением в работе [2].


\litlist

1. {\it Работнов Ю.Н.} Элементы наследственной механики твердых тел / Ю.Н. Работнов. – М.: Наука, – 1977. – 384 с.

2. {\it Бырдин А.П.} Метод рядов Вольтерра в динамических задачах  нелинейной наследственной упругости /  А.П. Бырдин, М.И. Розовский // Изв. АН Арм. ССР – 1985. Т. 38. №5. – С. 49 – 56.
