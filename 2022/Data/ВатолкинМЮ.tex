


\vzmstitle[
	% \footnote{Работа выполнена при моральной поддержке Урюпинского районного военкомата}
]{
	СЛУЧАЙ ТОЧНЫХ ОЦЕНОК ДЛЯ КОЭФФИЦИЕНТОВ СТЕПЕННЫХ РЯДОВ, ЯВЛЯЮЩИХСЯ 
	ПРЕДСТАВЛЕНИЯМИ СОБСТВЕННЫХ ФУНКЦИЙ ОДНОЙ КВАЗИДИФФЕРЕНЦИАЛЬНОЙ КРАЕВОЙ 
	ЗАДАЧИ ВТОРОГО ПОРЯДКА
}
\vzmsauthor{Ватолкин}{М.\,Ю.}
\vzmsinfo{Ижевск, ИжГТУ имени М.Т.~Калашникова; {\it vmyu6886@gmail.com}}

\vzmscaption

Пусть $ I\subseteq {\mathbb R} $ --- открытый интервал, 
$ {\cal P}=(p_{ik})_{0}^2$
--- нижняя треугольная  матрица, $ p_{ik}:I\to {\mathbb R},$ такая, что 
$ p_{00}(\cdot) $ и
$ p_{22}(\cdot) $ измеримы, почти всюду конечны и почти всюду отличны от нуля, а
$ \frac {1}{p_{11}(\cdot)},\
\frac {p_{10}(\cdot)}{p_{11}(\cdot)},
\ \frac {p_{20}(\cdot)}{p_{22}(\cdot)},
$
$ \frac {p_{21}(\cdot)}{p_{22}(\cdot)}$
локально\,\,суммируемы\,\,в\,\,$I$.
Определим\,\,квазипроизводные\,$_{\cal P}^{\;0}x,$\,$_{\cal P}^{\;1}x,$\,$_{\cal P}^{\;2}x$
$\mbox{функции}\,x$ $:I\to{\mathbb R}$\,
равенствами:\quad
$
 _{\cal P}^{\;0}x\doteq p_{00}x,\quad
$
$ 
 _{\cal P}^{\;1}x\doteq p_{11} \frac {d(_{\cal P}^{\;0}x)}{dt}
+p_{10}(_{\cal P}^{\;0}x),\quad\quad
$
\quad
\quad
\quad
$
  _{\cal P}^{\;2}x\doteq
p_{22} \frac {d(_{\cal P}^{\;1}x)}{dt}
+p_{21}(_{\cal P}^{\;1}x)
+p_{20}(_{\cal P}^{\;0}x).
$

Линейным \;однородным \;квазидифференциальным\linebreakназывается
уравнение [1] (в [1] изучается уравнение произвольного порядка) 
\vspace{-0,85mm}
$$ 
 (_{\cal P}^{\;2}x)(t)=0, \quad t \in I.
 \eqno(1)
$$

Его решением называется всякая функция 
$ x:I\to {\mathbb R}, $
имеющая локально абсолютно непрерывные нулевую\linebreakи первую квазипроизводные
и удовлетворяющая (1)
почти всюду в $ I $.
\vspace{-0,7mm}

Уравнение (1) называется неосцилляционным на промежутке $J\subset I$ 
(здесь 
$ J=[a,b], \ -\infty<a<b< +\infty $), если 
нулевая квазипроизводная любого его нетривиального решения 
имеет на $J$ не более одного нуля [1].
\vspace{-0,7mm}

Рассмотрим 
краевую задачу на собственные 
значения
\vspace{-1,7mm}
$$
(_{\cal P}^{\;2}x)(t)= 
  -\lambda \ ( _{\cal P}^{\;0}x)(t) \  \  \ (t\in J=[a,b]),
\eqno(2)
$$
$$
x(a)=x(b)=0.
\eqno(3)
$$

Последовательность решений
$\{ x_{k}(\cdot) \}_{0}^{\infty}$
построим так:\linebreak
$x_{0}(\cdot)$ есть решение задачи
$(_{\cal P}^{\;2}x)(t)=0 \ (t\in J)$, \
$_{\cal P}^{\;0}x(a)=0$,
$\ _{\cal P}^{\;1}x(a)=1;
$
\\
$x_{k}(\cdot)$ находятся рекуррентно как решения задач
\linebreak
$(_{\cal P}^{\;2}x_{k})(t)=(_{\cal P}^{\;0}x_{k-1})(t) \ (t\in J), \quad$
$_{\cal P}^{\;0}x_{k}(a)=0,$ $\quad _{\cal P}^{\;1}x_{k}(a)=0$ \\$(k=1,2,\dots).$

Решение 
$ u(t,\lambda) $ 
уравнения (2), удовлетворяющее первому \;из \;условий (3),
представимо в виде ряда [2],
$$
u(t,\lambda)=x_{0}(t)-
\lambda x_{1}(t)+ \lambda^{2} x_{2}(t)- \lambda^{3} x_{3}(t)
+ \, \dots \ \ .
\eqno(4)
$$

Собственные значения задачи (2),(3) представляют собой корни уравнения
$
\Phi(\lambda)=0,
$
где $\Phi(\cdot)$ --- сумма ряда (4) при $t=b$.
Функция
$
 u(t,\lambda^{*})= \sum_{k=0}^{\infty} (-1)^{k} (\lambda^{*})^{k} x_{k}(t)
 \ (t\in J)
$
есть собственная функция задачи (2),(3), отвечающая собственному значению
$\lambda^{*}$.
\vspace{-1,7mm}
\paragraph{Теорема~1.} {\it
Пусть уравнение (1) неосциляционно на $J$ и 
$
C(t,s) 
$
есть функция Коши уравнения (1),
вещественные константы $M_{1},$  $M_{2}$  и функция $\varphi(\cdot)$ таковы, что выполняются следующие неравенства:\;
$1 \leqslant \varphi(t)$ 
при всех значениях \,$t$\;$(a\leqslant t \leqslant b),$
\vspace{-3,7mm}
$$
\;\;\;\;_{\cal P}^{\;0}C(t,s) \leqslant 
M_{1}
{\int_{s}^{t}}\,
\dfrac{\varphi(s)}{p_{22}(s)}\,ds  
$$
\mbox{ \it{при всех значениях}} \;$s$ \mbox{ \it{и}}  \;\,$t$ \;\mbox{\it{таких, что}}
$
\;a\leqslant s\leqslant t \leqslant b,
$
$$
\;\;\;\;_{\cal P}^{\;0}C(t,a)\leqslant
M_{2}
{\int_{a}^{t}}\,
\dfrac{\varphi(s)}{p_{22}(s)}\,ds
$$
\mbox{\it{при всех значениях}} \; $t$
$
\quad(a\leqslant t \leqslant b).\quad
$
\mbox{ \it{Пусть \quad далее,}}\\
$
 M \doteq max \{ M_{1}, M_{2} \}.
$
Тогда (при $t\in J$) 
справедливы точные (см. примеры 1 и 2 ниже) оценки
\vspace{-1,7mm}
$$
\begin{array}{c}
0 \leqslant \, _{\cal P}^{\;0}x_{k}(t)\; \leqslant
\dfrac{M^{k+1}\displaystyle{\Bigg({\int_{a}^{t}}\,\dfrac{\varphi(s)}{p_{22}(s)}\,ds
\Bigg)}  
^{2k+1}} {(2k+1)!}
\big (
k=0,1,\dots \bigr).
\end{array}
\eqno(5)
$$
}
\paragraph{Доказательство.}
Докажем эту теорему методом математической индукции.
При $k=0$ оценка (5) верна, т.к.\linebreak
$_{\cal P}^{\;0}x_{0}(t) \,= \,_{\cal P}^{\;0}C(t,a),$
следовательно,
$$_{\cal P}^{\;0}x_{0}(t) = _{\cal P}^{\;0}C(t,a)\leqslant M_{2}
{\int_{a}^{t}}\,
\dfrac{\varphi(s)}{p_{22}(s)}\,ds  
\leqslant
M
{\int_{a}^{t}}\,
\dfrac{\varphi(s)}{p_{22}(s)}\,ds.  
$$
Пусть оценка (5) имеет место для некоторого 
$k {\in{\mathbb N}} $. 
\\
Покажем её справедливость для $k+1.$
$$
_{\cal P}^{\;0}x_{k+1}(t)=\displaystyle{\int_{a}^{t}}
\dfrac{_{\cal P}^{\;0}C(t,s)_{\cal P}^{\;0}x_{k}(s)}
{p_{22}(s)}\,ds
\leqslant
$$
$$
\leqslant
\dfrac{M^{k+1}}{(2k+1)!}
\displaystyle{\int_{a}^{t}}\,
\Bigg(\dfrac{M_{1}\cdot\varphi(s)}{p_{22}(s)}\Bigg)
\cdot
{\Bigg(\int_{s}^{t}}\,
\dfrac{\varphi(\tau)}{p_{22}(\tau)}\,d\tau\Bigg)\,\cdot
$$
$$
\cdot\,
\displaystyle{\Bigg({\int_{a}^{s}\,
\dfrac{\varphi(\tau_1)}{p_{22}(\tau_1)}\,d\tau_1\Bigg)}  
^{2k+1}}
ds
\leqslant
\dfrac{M^{k+2}}{(2k+1)!}
\displaystyle{\int_{a}^{t}}\,
{\Bigg(\int_{s}^{t}}\,
\dfrac{\varphi(\tau)}{p_{22}(\tau)}\,d\tau\Bigg)\cdot
$$
$$
\cdot
\displaystyle{\Bigg({\int_{a}^{s}\,
\dfrac{\varphi(\tau_1)}{p_{22}(\tau_1)}\,d\tau_1\Bigg)}  
^{2k+1}}
d
\displaystyle{\Bigg({\int_{a}^{s}\,
\dfrac{\varphi(\tau_2)}{p_{22}(\tau_2)}\,d\tau_2\Bigg)}}
\small{=}
$$

$$
=
\dfrac{M^{k+2}}{(2k+2)!}
\displaystyle{\int_{a}^{t}}\,
{\Bigg(\int_{s}^{t}}\,
\dfrac{\varphi(\tau)}{p_{22}(\tau)}\,d\tau\Bigg)\,
d
\displaystyle{\Bigg({\int_{a}^{s}\,
\dfrac{\varphi(\tau_1)}{p_{22}(\tau_1)}\,d\tau_1\Bigg)}
^{2k+2}}
=
$$

$$
=
\dfrac{M^{k+2}}{(2k+2)!}
\displaystyle{\int_{a}^{t}}\,
\Bigg(
\dfrac{\varphi(s)}{p_{22}(s)}\Bigg)\,
\displaystyle{\Bigg({\int_{a}^{s}\,
\dfrac{\varphi(\tau_1)}{p_{22}(\tau_1)}\,d\tau_1\Bigg)}
^{2k+2}}
ds
=
$$

$$
=
\dfrac{M^{k+2}}{(2k+3)!}
\displaystyle{\int_{a}^{t}}
d
\displaystyle{\Bigg({\int_{a}^{s}\,
\dfrac{\varphi(\tau_1)}{p_{22}(\tau_1)}\,d\tau_1\Bigg)}
^{2k+3}}=
$$

$$
=\;\,
\dfrac{M^{k+2}}{(2k+3)!}
{\displaystyle{\Bigg({\int_{a}^{t}\,
\dfrac{\varphi(s)}{p_{22}(s)}\,ds\Bigg)}
^{2k+3}}}
=\;\,
$$
$$
=\;\,
\dfrac{M^{(k+1)+1}}{(2(k+1)+1)!}
{\displaystyle{\Bigg({\int_{a}^{t}\,
\dfrac{\varphi(s)}{p_{22}(s)}\,ds\Bigg)}
^{2(k+1)+1}}}.
$$
Таким образом, получаем следующее неравество
$$
0 \leqslant \, _{\cal P}^{\;0}x_{k}(t)\; \leqslant
\dfrac{M^{(k+1)+1} \cdot \displaystyle{\Bigg({\int_{a}^{t}}\,
\dfrac{\varphi(s)}{p_{22}(s)}\,ds\Bigg)}
^{2(k+1)+1}_{}} {(2(k+1)+1)!}.
$$
По индукции (5) имеет место для всех $k {\in{\mathbb N}} $.
\;Теорема доказана. 
\paragraph{Пример~1.}
Рассмотрим задачу вида (2),(3):
$$
\dfrac{1} {\cos t} 
\bigg(\dfrac{1} {\cos t}\, {x}'\bigg)'=-\lambda x\; (t\in[0,1])
\eqno(6)
$$
$\bigl($уравнение (6) получено из уравнения:\linebreak
$
x''+ \tg t \cdot x' = -\lambda \cos^2 t \bigr),
$
$$
x(0)=x(1)=0.
\eqno(7)
$$
Для задачи (6),(7) обозначения теоремы 1 примут следующий вид:
$$a=0, \; b=1, \; p_{00}(t)=1,$$
$$p_{11}(t)=
\dfrac{1}{\cos t},
\; 
 p_{10}(t)=p_{21}(t)=0,\;
p_{22}(t)=
\dfrac{1}{\cos t}, 
\;
p_{20}(t)=0;$$
$$_{\cal P}^{\;0}C(t,s)=C(t,s)
=\int_s^t\dfrac{1}{\frac{1}{\cos \tau}}\,d\tau=
\int_s^t \cos \tau \,d \tau=
\sin t - \sin s,
$$

$$
M_{1}=M_{2}=1,
\;
\varphi (t) \equiv 1,
$$
$$\;_{\cal P}^{\;0}x_{k}(t)=x_{k}(t)=\dfrac{\;(\sin t)^{2k+1}}{(2k+1)!} 
\; 
\bigl( \,  k=0,1,2,\dots\bigr).$$
Правые части оценок (5) функциями 
$_{\cal P}^{\;0}x_{k}(t)$
достигаются. Представление (4) для задачи (6),(7) примет вид:
$$u(t,\lambda)=
\dfrac {\sin (\sqrt 
{\lambda}\cdot\sin t)}{\sqrt 
{\lambda}}.
$$
Корнями уравнения 
$\Phi(\lambda)=0,$ где 
$\Phi(\lambda)=u(1,\lambda),$
и собственными значениями задачи (6),(7) являются
$\lambda_{k}=
\bigg(\dfrac {\pi k} {\sin 1}\bigg)^2.
$
Функции
$$
 u(t,\lambda_{k})= \sum_{m=0}^{\infty} (-1)^{m} \,\lambda_{k}^m
\cdot x_{m}(t)
=
\dfrac { 
\sin 1 \cdot \sin \bigg(\dfrac {\pi k \cdot\sin t}{\sin 1} \bigg)
}
{\pi k}
$$
\\
являются собственными функциями задачи (6),(7), 
соответствующими собственным значениям $\lambda_{k}$  задачи  
(6),(7).

Обобщим пример 1, а, именно, рассмотрим следующий пример.

\paragraph{Пример~2.}
Рассмотрим задачу вида (2),(3):
$$
p(t) 
\bigg(p(t)\, {x}'\bigg)'=-\lambda x\; (t\in[0,1])
\eqno(8)
$$
$\bigl($функция $p(t)$ измерима, почти всюду конечна, неотрицательна и суммирруема на $[0,1]$$\bigl),$
$$
x(0)=x(1)=0.
\eqno(9)
$$
Для задачи (8),(9) обозначения теоремы 1 примут следующий вид:
$$a=0, \; b=1, \; p_{00}(t)=1,$$
$$p_{11}(t)=
p(t),
\; 
 p_{10}(t)=p_{21}(t)=0,
 $$
 $$
p_{22}(t)=
p(t), 
\;
p_{20}(t)=0;
$$
$$_{\cal P}^{\;0}C(t,s)=C(t,s)
=\int_s^t\dfrac{d\tau}{\ p(\tau)},
$$
$$
M_{1}=M_{2}=1,
\;
\varphi (t) \equiv 1,
$$
$$\;_{\cal P}^{\;0}x_{k}(t)=x_{k}(t)=\frac{\;\;\bigg(\displaystyle{\int_0^t}\dfrac{d\tau}{\ p(\tau)}\bigg)^{2k+1}}{(2k+1)!} 
\; 
\bigl( \,  k=0,1,2,\dots\bigr).$$
Правые части оценок (5) функциями 
$_{\cal P}^{\;0}x_{k}(t)$
 достигаются.
Представление (4) для задачи (8),(9) примет вид:
$$u(t,\lambda)=
\dfrac {\sin\bigg(\sqrt 
{\lambda}\cdot \bigg(\displaystyle{\int_0^t}\dfrac{d\tau}{\ p(\tau)}\bigg)\bigg)}{\sqrt
{\lambda}}
\,.$$
Корнями уравнения 
$\Phi(\lambda)=0, \; \mbox{где}\;\;
\Phi(\lambda)=u(1,\lambda),$
и собственными значениями задачи (8),(9) являются
$$\lambda_{k}=
\Bigg(\dfrac {\pi k} {\int_0^1\frac{d\tau}{\ p(\tau)}}\Bigg)^2.
$$
Функции
\vspace{-1,5mm}
$$
 u(t,\lambda_{k})= \sum_{m=0}^{\infty} (-1)^{m} \,\lambda_{k}^m
\cdot x_{m}(t)
=
$$
$$
=
\dfrac { 
\bigg(\displaystyle{\int_0^1}\dfrac{d\tau}{\ p(\tau)} \bigg)\cdot \sin \Bigg(\dfrac {\pi k \cdot\ \int_0^t\frac{d\tau}{\ p(\tau)}}{\int_0^1\frac{d\tau}{\ p(\tau)}} \Bigg)
}
{\pi k}
$$
являются собственными функциями задачи (8),(9), 
соответствующими собственным значениям $\lambda_{k}$ задачи  
(8),(9).

% Оформление списка литературы
\litlist

1. {\it Дерр В.Я.} Неосцилляция решений линейного\linebreakквазидифференциального 
уравнения~/ В.Я.~Дерр~// Изв. института математики и информатики УдГУ.~--- 1999.~--- №~1(16).~---
С.~3--105.

2. {\it Ватолкин М.Ю.} О представлении решений 
квазидифференциального уравнения~/ {\it М.Ю.~Ватолкин, В.Я.~Дерр~}// Изв. вузов.~--- 1995.~--- №~10(401).~---
С.~27--34.

