\vzmstitle[
	\footnote{Работа выполнена при финансовой поддержке РФФИ в рамках проектов 20-31-90086, 20-08-01154.}
]{Математическое моделирование
собственных вибраций механической системы с осцилляторами}
\vzmsauthor{Коростелева}{Д.\,М.}
\vzmsinfo{Казань, КГЭУ; {\it diana.korosteleva.kpfu@mail.ru}}

\vzmscaption

Пусть ось балки длиной $l$
занимает в равновесном горизонтальном положении отрезок $\overline{\Omega}=[0,l]$
оси $Ox$, $\Omega=(0,l)$. Обозначим через $\rho=\rho(x)$ и $E=E(x)$ линейную плотность и
модуль упругости материала балки в точке $x$,
через $S=S(x)$ и $J=J(x)$ -- площадь поперечного сечения балки и момент
инерции сечения в точке $x$ относительно своей горизонтальной оси.
Пусть торцы балки $x=0$ и $x=l$ заделаны жёстко.
Предположим,
что в точках балки с координатами
$x^{(ij)}\in(0,l)$ упруго присоединены грузы (осцилляторы) с массами $M_{ij}$  и
коэффициентами жёсткости подвески $K_{ij}$,
$\sigma_{ij}=K_{ij}/M_{ij}$,
$\sigma_{i}=\sigma_{ij}$,
$j=1,2,\ldots,r_i$, $r_i\geqslant 1$,
$i=1,2,\ldots,m$, $m\geqslant 1$,
$0<\sigma_1<\sigma_2<\ldots<\sigma_m<\infty$.

Обозначим через $w(x,t)$ отклонение от положения равновесия точки $x$ оси
балки в момент времени $t$, через
$\eta_{ij}(t)$ -- отклонение от положения равновесия груза массой
$M_{ij}$ в момент времени $t$, $j=1,2,\ldots,r_i$, $i=1,2,\ldots,m$.
Собственные вибрации системы балка"=осцилляторы определяются функциями $w(x,t)$ и $\eta_{ij}(t)$ вида
\begin{equation*}
w(x,t)=u(x)v(t), x\in\Omega,\
\eta_{ij}(t)=c_{ij}u(x^{(ij)})v(t), t>0,
\eqno{(1)}
%\tag{1}
\end{equation*}
где
$v(t)=a_{0}\,{\rm cos}\sqrt{\lambda}t+b_{0}\,{\rm sin}\sqrt{\lambda}t$,
$t>0$;
$a_{0}$, $b_{0}$, $c_{ij}$, $\lambda$ --
вещественные постоянные,
$j=1,2,\ldots,r_i$,
$i=1,2,\ldots,m$.

Для функций (1) выполняются уравнение колебания\,бал\-ки
\begin{equation*}
(E(x)J(x)w_{xx}(x,t))_{xx}+\rho(x)S(x)w_{tt}(x,t)=f(x,t),
x\in\Omega,
\eqno{(2)}
%\tag{2}
\end{equation*}
и уравнения колебаний осцилляторов
\begin{equation*}
M_i(\eta_{ij}(t))_{tt}+K_{ij}(\eta_{ij}(t)-w(x^{(ij)},t))=0,
\eqno{(3)}
%\tag{3}
\end{equation*}
$j=1,2,\ldots,r_i$, $i=1,2,\ldots,m$, $t>0$,
$w_{x}(x,t)=\partial w(x,t)/\partial x$, $w_{t}(x,t)=\partial
w(x,t)/\partial t$, $(\psi(t))_t=d\psi(t)/dt$. Действие
присоединённых осцилляторов на балку
задаётся функцией:
\begin{equation*}
f(x,t)=
\sum_{i=1}^{m}
\sum_{j=1}^{r_i}
K_{ij}(\eta_{ij}(t)-w(x^{(ij)},t))
\delta(x-x^{(ij)}),
\eqno{(4)}
%\tag{4}
\end{equation*}
где $\delta(x)$ -- дельта"=функция Дирака.
К уравнениям (1)--(4) добавляются
условия закрепления на торцах балки:
\begin{equation*}
w(0,t)=w_{x}(0,t)=w(l,t)=w_{x}(l,t)=0,\quad t>0.
\eqno{(5)}
%\tag{5}
\end{equation*}
Формулировки подобных задач содержатся в работах [1--5].

Учитывая вид функций (1) в уравнениях
колебаний осцилляторов (3),
выводим
$c_{ij}=\sigma_{ij}/(\sigma_{ij}-\lambda)$,
$\sigma_{ij}=K_{ij}/M_{ij}$,
$j=1,2,\ldots,r_i$,
$i=1,2,\ldots,m$.
Применяя (1) и (4),
из соотношений (2) и (5)
сформулируем следующую нелинейную задачу на собственные значения:
найти числа $\lambda$ и ненулевые функции $u(x)$, $x\in\Omega$,
для которых выполняется уравнение
\begin{equation*}
(EJu^{\prime\prime})^{\prime\prime}+\sum_{i=1}^{m}
\frac{\lambda}{\lambda-\sigma_{i}}
\sum_{j=1}^{r_i}
K_{ij}\delta(x-x^{(ij)})u=
\lambda\,\rho S\,u,
x\in\Omega,
\eqno{(6)}
%\tag{6}
\end{equation*}
и граничные условия
\begin{equation*}
u(0)=u^{\prime}(0)=u(l)=u^{\prime}(l)=0.
\eqno{(7)}
%\tag{7}
\end{equation*}

Задача (6), (7)
имеет возрастающую последовательность положительных простых собственных
значений с предельной точкой на бесконечности.
Последовательности собственных значений отвечает полная ортонормированная
система, составленная из собственных функций.

Дифференциальная задача на собственные\,зна\-чения
(6), (7)
приближается
сеточной схемой метода конечных элементов с эрмитовыми
конечными элементами произвольного порядка на регулярной неравномерной сетке.
Исследуется скорость сходимости приближённых собственных значений
и собственных функций.

\litlist

1.
{\it Тихонов А. Н., Самарский А. А.}
{Уравнения математической физики.}~--
М.: Наука, 1977.~-- 736~с.

2.
{\it Стрелков С. П.}
{Введение\,в\,теорию\,колебаний.}~--\,Санкт-Петербург: Издательство <<Лань>>, 2005.~-- 440~с.

3.
{\it Андреев Л. В., Дышко А. Л., Павленко И. Д.}
{Динамика пластин и оболочек с сосредоточенными массами.}~--
М.: Машиностроение, 1988.~-- 200 с.

4.
{\it Solov'ev S. I.}
{Eigenvibrations of a bar with elastically attached load}
//Differ. Equations.~--
2017.~-- V.~53.~-- No~3.~-- P.~409--423.

5.
{\it Samsonov A. A., Korosteleva D. M., Solov'ev S. I.}
{Inves\-tigation of the eigenvalue problem on eigenvibration of a loaded string}
//J. Phys.: Conf. Ser.~--
2019.~-- V.~1158.~-- No~4.~-- Art.~042010.~-- P.~1--5.
