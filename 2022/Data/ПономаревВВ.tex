\vzmstitle[
	\footnote{Работа выполнена за счёт гранта РНФ, проект 17-11-01303}
]{
	Связь между кольцом $Ad^{\ast}$-инвариантных полиномов и инвариантами Жордана-Кронекера нильпотентных алгебр Ли рамерности меньшей или равной семи
}
\vzmsauthor{Пономарёв}{В.\,В.}
\vzmsinfo{Москва, МГУ; {\it boba1997@yandex.ru}}


\vzmscaption



Пусть $\mathfrak{g}$ - алгебра Ли. На двойственном пространстве к ней $\mathfrak{g}^{\ast}$ естественным образом определяются скобка Ли-Пуассона $\mathcal{A}_{x}$ и скобка Пуассона с постоянными коэффициентами $\mathcal{A}_{a}$. При помощи теоремы Жордана-Кронекера мы можем классифицировать пары кососимметричных билинейных форм, приводя их одновременно к каноническому блочно-диагональному виду. Для любой пары точек $x,a \in\mathfrak{g}^{\ast}$ мы можем определить пучок билинейных форм $\mathcal{A}_{x} - \lambda \mathcal{A}_{a}$ в точке $x$. Размеры блоков в разложении Жордана-Кро\-не\-ке\-ра форм $\mathcal{A}_{x}$ и $\mathcal{A}_{a}$ называются алгебраическим типом пучка. Алгебраический тип почти всех пучков одной алгебры Ли одинаков и называется инвариантом Жордана-Кронекера этой алгебры Ли $\left[ 3 \right]$. А.Ю. Грознова в своей дипломной работе $\left[ 1 \right]$ вычислила инварианты Жордана-Кронекера нильпотентных алгебр Ли размерностей не больше семи.

С другой стороны, статья А. Оомса $\left[ 2 \right] $ посвящена изучению свойств колец инвариантов коприсоединённого представления тех же самых алгебр Ли. Вычисления Грозновой привели к предположению о том, что существование пучков кронекерова типа одинакового ранга, но с разными инвариантами Жордана-Кронекера для кронекеровых нильпотентных алгебр Ли может быть эквивалентно несвободной порождённости колец инвариантов коприсоединённого представления. Так возникла задача по поиску таких пучков для всех кронекеровых алгебр Ли размерностей не больше семи с несвободно порождённым кольцом $Ad^{\ast}$-инвариантных полиномов.

На докладе будет представлен алгоритм поиска таких пучков, а также продемонстрированы результаты его работы для всех подходящих под условия задачи алгебр Ли.

\litlist


1. {\it Groznova A. Yu.}
Calculation of Jordan-Kronecker in\-va\-riants for Lie algebras of small dimension // diploma work. - 2018. - Lomonosov Moscow State University.\\
\\
2. {\it Ooms A.} 
The Poisson center and polynomial, maximal Poisson commutative subalgebras, especially for nilpotent Lie algebras of dimension at most seven // Journal of Algebra. - 2012. - №. 365. - C. 83--113.\\
\\
3. {\it A.V.Bolsinov, P.Zhang.} 
Jordan-Kronecker invariants of finite-dimensional Lie algebras // Transformation Groups. - 2016. -   Vol.21. - No. 1. -  С. 51 - 86.
