\vzmstitle[
	\footnote{Работа выполнена при финансовой поддержке РНФ (проект 20--11--20085).}
]{
	Асимметричные произведения гармонических функций и обратные задачи для волнового уравнения в непереопределенной постановке
}
\vzmsauthor{Кокурин}{М.\,Ю.}
\vzmsinfo{Йошкар--Ола, МарГУ; {\it kokurinm@yandex.ru}}

\vzmscaption
Пусть ${\mathcal H}_1$, ${\mathcal H}_2$ --- семейства гармонических функций из $C^2({\overline D})$, $D$--ограниченная область в $\mathbb{R}^3$.
Рассматривается вопрос о том, в каких случаях семейство $\pi({\mathcal H}_1,{\mathcal H}_2)=\{ u_1 u_2: u_j \in {\mathcal H}_j, j=1,2 \}$
образует полную систему в $L_2(D)$.

{\bf Теорема 1.} {\it Пусть $L$ --- прямая в $\mathbb{R}^3$, $Y$ есть открытое множество, принадлежащее неограниченной компоненте $L\backslash {\overline D}$,
$$
{\cal H}_1={\cal H}(D), \,\, {\cal H}_2=\Bigl\{ {{1}\over{\vert x-y\vert}}: y\in Y \Bigr\}.
$$
Тогда семейство ${\cal H}_1 \cdot {\cal H}_2$ полно в $L_{2}(D)$.
}


Волновое поле $u(x,t)=u_y(x,t)$, возбуждаемое в момент $t=0$ источником в точке $y\in Y$, определяется решением задачи
$$
{{1}\over{c^2(x)}} u_{tt}(x,t)=\Delta
u(x,t)-\delta(x-y) g(t), \,\, x\in \mathbb{R}^3, \,\, t\geq 0
$$
с нулевыми начальными условиями. Пусть $c(x)\equiv c_0$ вне априори заданной ограниченной области $D \subset \mathbb{R}^3$ с известной $c_0$, $c(x)$ при $x\in D$ неизвестны, функция $c$ кусочно непрерывна. Считаем, что ${\mathbb R}^3 \backslash {\overline D}$ связно.
В обратной задаче волнового зондирования по измерениям усреднённого рассеянного поля $\int_0^{\infty} t^2 u_y(x;t)dt=h(x,y)$ для $x\in X$, $y\in Y$, $(X\cup Y)\cap {\overline D}=\emptyset$, требуется определить $c(x)$, $x\in D$. Здесь $X$ и $Y$ --- многообразия, на которых размещены детекторы рассеянного поля и его источники. С использованием Теоремы 1 доказывается

{\bf Теорема 2.} {\it Пусть $\Pi$, ${L}$ -- произвольные плоскость и прямая в $\mathbb{R}^3$, $\Pi\cap {\overline D}=\emptyset$; $X$ -- область в $\Pi$ и $Y$ есть открытое множество, принадлежащее неограниченной компоненте $L\backslash {\overline D}$, ${\overline X} \cap {\overline Y}=\emptyset$. Тогда функция $c(x)$, $x\in D$, однозначно определяется данными наблюдения $\{ h(x,y): x\in X, y\in Y \}$.
}

Аналог Теоремы 2 установлен для обратных задач волнового зондирования в бесфазной постановке.
