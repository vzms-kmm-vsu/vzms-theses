\vzmstitle
{
	СЛАБАЯ РАЗРЕШИМОСТЬ НАЧАЛЬНО-КРАЕВОЙ ЗАДАЧИ ДЛЯ ПАРАБОЛИЧЕСКОГО УРАВНЕНИЯ В КЛАССЕ СУММИРУЕМЫХ НА ГРАФЕ ФУНКЦИЙ
}
\vzmsauthor{Василенко}{П.\,М.}
\vzmsinfo{Воронеж, ВГУ; {\it}}

\vzmscaption

\paragraph{1. Основные понятия и вспомогательные предложения.}
Используются понятия и сохраняются обозначения, принятые в [1]:
$\partial\Gamma$ и $J(\Gamma)$ "--- множества граничных и внутренних узлов графа $\Gamma$, соответственно; $\Gamma_{0}$ "--- объединение всех рёбер, не содержащих концевых точек;  $\Gamma_{t}=\Gamma_{0}\times(0,t)$, $\partial \Gamma_{t}=\partial \Gamma\times(0,t)$ ($t\in[0,T]$). Ребра $\gamma$ графа $\Gamma$ ориентированы и параметризуются отрезком $[0,1]$.

Введём необходимые пространства: $L_{p}(\Gamma)$ ($p=1,2$) "--- банахово пространство измеримых на $\Gamma_0$ функций, суммируемых с $p$ степенью (аналогично определяются пространства $L_{p}(\Gamma_T)$); $L_{2,1}(\Gamma_{T})$~--- пространство функций из $L_1(\Gamma_{T})$ с нормой  $\|u\|_{L_{2,1}(\Gamma_{T})}=\int\limits_0^T(\int\limits_{\Gamma}u^2(x,t)dx)^{1/2}dt$; $W_{\,2}^{1}(\Gamma)$~--- пространство функций из $L_{2}(\Gamma)$, имеющих обобщённую производную первого порядка также из $L_{2}(\Gamma)$; $W_{\,2}^{1,0}(\Gamma_{T})$~--- пространство функций из $L_{2}(\Gamma_{T})$, имеющих обобщённую производную первого порядка по $x$, принадлежащую $L_{2}(\Gamma_{T})$ (аналогично вводится пространство $W^{1}(\Gamma_{T})$.
Обозначим через $V_{2}(\Gamma_{T})$ множество всех функций  $u(x,t)\in W_{2}^{1,0}(\Gamma_{T})$ с конечной нормой
\begin{equation}\label{eq1}
{{\begin{array}{*{20}c}
\|u\|_{2,\Gamma_{T}}\equiv \mathop{\max}\limits_{0\leqslant t\leqslant T}\left\|u(\cdot,t)\right\|_{L_2(\Gamma)}+\left\|u_x\right\|_{L_2(\Gamma_{T})}
\end{array} }}
\end{equation}
и сильно непрерывных по $t$ в норме $L_{2}(\Gamma)$; $L_{2,1}(\Gamma_{T})$ "--- пространство функций $u\in L_1(\Gamma_{T})$ с нормой  $\|u\|_{L_{2,1}(\Gamma_{T})}=\int\limits_0^T(\int\limits_{\Gamma}u^2(x,t)dx)^{1/2}dt$.

Рассмотрим билинейную форму
$\ell(\mu,\nu)=\int\limits_{\Gamma}\left(a(x)\frac{d \mu(x)}{d x}\frac{d \nu(x)}{d x}+b(x)\mu(x)\nu(x)\right)dx$ с  фиксированными измеримыми и ограниченными на $\Gamma_0$ функциями $a(x)$, $b(x)$, суммируемыми с квадратом: $0<a_\ast\leqslant a(x)\leqslant a^\ast$, $|b(x)|\leqslant \beta$, $x\in \Gamma_0$. Из работы  [1] следует, что пространство $W^1_{\,2}(\Gamma)$ содержит множество $\Omega_a(\Gamma)$ функций $u(x)\in C(\Gamma)$ ($C(\Gamma)$ "--- пространство непрерывных на $\Gamma$ функций), удовлетворяющих соотношениям
$\sum\limits_{\gamma_j\in R(\xi)}a(1)_{\gamma_j}\frac{du (1)_{\gamma_j}}{dx}
 =\sum\limits_{\gamma_j\in r(\xi)}a(0)_{\gamma_j}\frac{du (0)_{\gamma_j}}{dx}$
во всех узлах $\xi\in J(\Gamma)$ ($R(\xi)$ и $r(\xi)$ "--- множества рёбер, соответственно ориентированных <<к узлу $\xi$>> и <<от узла $\xi$>>, $u(\cdot)_\gamma$ "--- сужение функции $u(\cdot)$ на ребро $\gamma$). Замыкание в норме $W^1_{\,2}(\Gamma)$ множества $\Omega_a(\Gamma)$  обозначим через $W^1(a,\Gamma)$ (если допустить, что функции $u(x)$ из $\Omega_a(\Gamma)$ удовлетворяют ещё и краевому условию $u(x)|_{\partial\Gamma}=0$, то получим пространство $W^1_{\,0}(a,\Gamma)$).  Пусть далее $\Omega_a(\Gamma_{T})$ "--- множество функций $u(x,t)\in V_{2}(\Gamma_{T})$, чьи следы определены на сечениях области $\Gamma_{T}$ плоскостью $t=t_0$ ($t_0\in [0,T]$) как функции класса $W^1(a,\Gamma)$ и удовлетворяют соотношениям
\begin{equation}\label{eq3}
{{\begin{array}{*{20}c}
\sum\limits_{\gamma_{j}\in R(\xi)}a(1)_{\gamma_{j}}\frac{\partial u(1,t)_{\gamma_{j}}}{\partial x}
=\sum\limits_{\gamma_{j}\in r(\xi)}a(0)_{\gamma_{j}}\frac{\partial u(0,t)_{\gamma_{j}}}{\partial x}
\end{array} }}
\end{equation}
для всех узлов $\xi\in J(\Gamma)$. Замыкание множества $\Omega_a(\Gamma_{T})$ по норме (1)  обозначим через $V^{1,0}(a,\Gamma_{T})$; ясно, что  $V^{1,0}(a,\Gamma_{T})\subset W_{\,2}^{1,0}(\Gamma_{T})$. Другим подпространством  пространства $ W_{\,2}^{1,0}(\Gamma_{T})$  является $W^{1,0}(a,\Gamma_{T})$ "--- замыкание в норме $W_{\,2}^{1,0}(\Gamma_{T})$ множества гладких функций, удовлетворяющих соотношениям (2) для всех узлов $\xi\in J(\Gamma)$ и для любого $t\in [0,T]$ (аналогично вводится пространство $W^{1}(a,\Gamma_{T})$). Отличием элементов пространства $V^{1,0}(a,\Gamma_{T})$ от элементов $W^{1,0}(a,\Gamma_{T})$ является отсутствие у последних непрерывности по переменной $t$, соотношение (2) имеет место почти всюду на $(0,T)$.


\paragraph{2. Начально"=краевая задача, однозначная слабая разрешимость.}
В пространстве $V^{1,0}(a,\Gamma_{T})$ рассмотрим уравнение
\begin{equation}\label{eq5}
{{\begin{array}{*{20}c}
 \frac{\partial y(x,t)}{\partial t}-\frac{\partial }{\partial x }\left(a(x)\frac{\partial y(x,t)}{\partial x}\right)+b(x)y(x,t)=f(x,t),
 \end{array} }}
\end{equation}
представляющее собой систему дифференциальных уравнений с распределёнными параметрами на каждом ребре $\gamma$ графа $\Gamma$. Состояние системы (3) в области $\overline{\Gamma}_{T}$ определяется  слабым решением $y(x,t)$ уравнения (3), удовлетворяющим начальному
\begin{equation}\label{eq6}
{{\begin{array}{*{20}c}
y\mid_{t=0}=\varphi(x),\quad x\in\Gamma,
 \end{array} }}
\end{equation}
и краевому
\begin{equation}\label{eq8}
{{\begin{array}{*{20}c}
a(x)\frac{\partial y}{\partial x}\mid_{x\in\partial\Gamma_T}=\phi(x,t)
 \end{array} }}
\end{equation}
условиям; $f(x,t)\in L_{2,1}(\Gamma_{T})$, $\varphi(x)\in L_2(\Gamma)$, $\phi(x,t)\in L_{2}(\Gamma_{T})$.


\paragraph{Определение 1.}  \emph{Слабым решением начально"=краевой задачи} (3)--(5) \emph{называется функция $y(x,t)\in V^{1,0}(a,\Gamma_{T})$, удовлетворяющая интегральному тождеству}
\begin{equation}\label{eq8}
{{\begin{array}{*{20}c}
\int\limits_{\Gamma}y(x,t)\eta(x,t) dx-\int\limits_{\Gamma_{t}}y(x,t)\frac{\partial \eta(x,t)}{\partial t}dxdt+\ell_{t}(y,\eta)=
\\
=\int\limits_{\Gamma}\varphi(x)\eta(x,0)dx+
\int\limits_{\partial\Gamma_{t}}\phi(x,t)\eta(x,t) dxdt+\\
+\int\limits_{\Gamma_{t}}f(x,t)\eta(x,t) dxdt
 \end{array} }}
\end{equation}
\emph{при любом $t\in[0,T]$ и для любой функции $\eta(x,t)\in W^{1}(a,\Gamma_{T})$; $\ell_t(y,\eta)$ "--- билинейная форма, определённая соотношением}
$$
{{\begin{array}{*{20}c}
\ell_t(y,\eta)=\int\limits_{\Gamma_{t}}\left(a(x)\frac{\partial y(x,t)}{\partial x}\frac{\partial \eta(x,t)}{\partial x}+b(x)y(x,t)\eta(x,t)\right)dxdt.
 \end{array} }}
$$



\paragraph{Теорема 1.} \emph{Начально"=краевая задача} (3)--(5) \emph{имеет по крайней мере одно слабое  решение в пространстве $V^{1,0}(a,\Gamma_{T})$.}

Доказательство теоремы предваряется доказательством разрешимости задачи (3)--(5) в пространстве $W^{1,0}(a,\Gamma_{T})$. При этом используются метод Фаэдо-Галеркина и специальный базис "--- система обобщённых собственных функций $\{u_n(x)\}_{n\geqslant 1}$ краевой задачи
$-\frac{d}{dx}\left(a(x)\frac{du(x)}{dx}\right)+b(x)u(x)=\lambda u(x)$, $\frac{du(x)}{dx}=0$,
в пространстве $W^1_{\,2}(a,\Gamma)$. Эта задача обладает множеством собственных чисел $\lambda_n$, каждому из которых соответствует по крайней мере одна обобщённая собственная функция $u_n(x)\in W^1_{\,2}(a,\Gamma)$, удовлетворяющее тождеству $\ell(u,\eta)=\lambda(u,\eta)$
при любой функции $\eta(x)\in W^1_{\,0}(a,\Gamma)$ (здесь и всюду ниже через $\left(\cdot,\cdot\right)$ обозначено скалярное произведение в $L_{\,2}(\Gamma)$, $L_{\,2}(\partial\Gamma_T)$ или $L_{\,2}(\Gamma_T)$). Собственные значения $\lambda_n$  имеют конечную кратность, их можно занумеровать в порядке возрастания модулей с учётом кратностей: $\{\lambda_n\}_{n\geqslant 1}$. Система обобщённых собственных функций $\{u_n(x)\}_{n\geqslant 1}$ плотна в $W_{\,2}^{1}(a,\Gamma)$ и ортонормирована в $L_{\,2}(\Gamma)$ [2].
Не умоляя общности, считаем, что краевое условие (5) однородное: $\phi(x,t)=0$.





\paragraph{Определение 2.} \emph{Слабым решением класса} $W^{1,0}(\Gamma_{T})$ \emph{начально"=краевой задачи} (3)--(5) ($\phi(x,t)=0$) \emph{называется функция $y(x,t)\in W^{1,0}(a,\Gamma_{T})$, удовлетворяющая интегральному тождеству}
\begin{equation}\label{eq8}
{{\begin{array}{*{20}c}
-\int\limits_{\Gamma_{T}}y(x,t)\frac{\partial \eta(x,t)}{\partial t}dxdt+\ell_T(y,\eta)=
\int\limits_{\Gamma}{\varphi}(x)\eta(x,0)dx+\\
+\int\limits_{\Gamma_{T}}{f}(x,t)\eta(x,t) dxdt
 \end{array} }}
\end{equation}
\emph{для любой $\eta(x,t)\in W^{1}(a,\Gamma_{T})$, равной нулю при $t=T$.}




\paragraph{Теорема 2.} \emph{Начально"=краевая задача} (3)--(5) ($\phi(x,t)=0$) \emph{имеет по крайней мере одно слабое решение в пространстве $W^{1,0}(a,\Gamma_{T})$.}

При доказательстве теоремы 2 строятся приближенные решения $y^N(x,t)$ задачи (3)--(5) ($\phi(x,t)=0$) в виде $y^N(x,t)=\sum\limits_{i=1}^Nc_i^N(t)u_i(x)$ ($c_i^N(t)$ "--- абсолютно непрерывные на $[0,T]$ функции: $c'_i(t)\in L_2(0,T)$), где $c_i^N(t)$ определяются  из системы
\begin{equation}\label{eq9}
{{\begin{array}{*{20}c}
\left(\frac{\partial y^N}{\partial t},u_i\right)+\\
+\int\limits_{\Gamma}\left(a(x)\frac{\partial y^N(x,t)}{\partial x}\frac{d u_i(x)}{d x}+b(x)y^N(x,t)u_i(x)\right)dx=
\left({f},u_i\right),\\i=\overline{1,N},
 \end{array} }}
\end{equation}
и равенств
\begin{equation}\label{eq10}
{{\begin{array}{*{20}c}
c_i^N(0)=\left({\varphi},u_i\right),\,\,\,i=\overline{1,N}.
 \end{array} }}
\end{equation}
Соотношения (8), (9) суть задача Коши на интервале $[0,T)$  для системы $N$ линейных дифференциальных уравнений относительно $c_i^N(t)$ ($i=\overline{1,N}$). Так как  $N$-матрица $\|(u_i,u_j)\|$  неособенная, то система (8), (9) имеет единственное решение $c_i^N(t)$ ($i=\overline{1,N}$). Дальнейшее построено на получении оценок норм $y^N(x,t)$, не зависящих от $N$, и почти дословно повторяют приведённые в [1] рассуждения.

Для доказательства теоремы 1 достаточно установить, что при каждом фиксированном $t\in [0,T]$ след принадлежащего пространству $W^{1,0}(a,\Gamma_{T})$ слабого решения задачи (3)--(5) ($\phi(x,t)=0$) суть элемент $W_{\,2}^{1}(a,\Gamma)$ и непрерывно зависит от $t$ в норме $W_{\,2}^{1}(\Gamma)$, а значит, и в норме $L_2(\Gamma)$.


\litlist

1. \it Провоторов В. В. \rm Оптимальное управление параболической системой с распределёнными параметрами на графе // Вестн. С.-Петерб. ун-та. Сер. 10. Прикладная математика. Информатика. Процессы управления. 2014. Вып. 3. С. 154-163.

2. \it Провоторов В. В. \rm Разложение по собственным функциям задачи Штурма-Лиувилля на графе"=пучке // Известия высших учебных заведений. Математика. 2008. № 3. С. 50-62.
