\selectlanguage{english}

\vzmstitle[
	\footnote{The study was supported by the Russian Foundation of Basic Researches (project
No. 20-01-00051)}
]{
	On the weak solvability of high-order viscoelastic fluid dynamics models
}
\vzmsauthor{Orlov}{V.\,P.}
\vzmsinfo{Voronezh, VSU; {\it orlov\_vp@mail.ru}}

\vzmscaption


We consider the motion of an high-order Oldroyd incompres\-si\-ble fluid with memory along the trajectories of the velocity field of constant density (it
is supposed to be equal to $1$) which occupies a bounded domain $\Omega$ in ${R}^N$, $N=2,3$, $\partial\Omega\in C^2$.

The constitutive law has the following form
\begin{equation*}
\left( 1+ \sum\limits_{i=1}^L \lambda_i
\dfrac{d^i}{d t^i }\right) \sigma =2\left(\nu +
\sum\limits_{i=1}^{L}\varkappa_i\dfrac{d^i}{d
t^i}\right)\mathcal{E}(v),\ \lambda_L>0, \varkappa_M>0,
\end{equation*}
$ \dfrac{d}{dt}=\dfrac{\partial}{\partial t} + \sum\limits_{i=1}^N v_i\dfrac{\partial}{\partial
x_i}$.
Here $\sigma$ and $\mathcal{E}(v)$ are the deviator
of the stress tensor and the strain rate tensor, respectively.
\par We assume that the roots $\alpha_1,\alpha_2,\ldots,\alpha_L$ of the polynomial
$Q(p)= 1 + \sum\limits_{i=1}^L \lambda_i p^i
$
are real, negative and distinct.
%The requirement of realness and negativity is caused by the physical meaning of the problem. The requirement of different roots is imposed solely for simplicity and reduction of calculations.

Let
\begin{equation*}
\begin{split}\mu_0 =\dfrac{\varkappa_L}{\lambda_L} > 0, \quad
C(p)=\sum\limits_{i=1}^{L-1}\left(\varkappa_i -
\mu_0\lambda_{i}\right)p^i + \nu -
\mu_0,\\
\beta_k = \dfrac{C(\alpha_k)}{Q'(\alpha_k)},\quad G(t-s)=\sum\limits_{k=1}^L \beta_k \exp({\alpha_k(t-s)}).
\end{split}\end{equation*}

The corresponding initial-boundary value problem $Z$ has the form
\begin{gather*}\label{d7}
\partial v/\partial t+\sum\limits_{i=1}^N v_i\,\partial v/\partial x_i-\mu_0\Delta (v)(t)-\\
{\rm Div}\int\limits_{0}^tG(t-s)\,\mathcal{E}(v)(s,
z(\tau,t,x))\,ds+\nabla\,p= f, \quad (t,x)\in Q_T;\\
\label{d8} z(\tau; t, x)=x+\int_t^\tau v(s, z(s; t, x))\,ds, \quad 0\leqslant t, \tau\leqslant T, \quad x\in\overline{\Omega};\\
\label{d9} \mathrm{div}\,v(t, x)=0, \quad (t, x)\in Q_T=[0, T]\times \partial\Omega;\\
\label{d10} v(0, x)=v^0(x),\ x\in \Omega, \quad v(t, x)=0, \quad (t, x)\in Q_T.
\end{gather*}
Here $v(t,x)=(v_1(t,x),\ldots,v_N(t,x))$~ is the velocity field, $p$ is the pressure, $f(t,x)$ is the density of external forces.
%${\rm Div}\,A(t,x)$ is the vector, coordinates of which are divergences with respect to $x$ of the rows of matrix $A(t,x)$; $z(s; t, x))$ is a regular Lagran\-gian flow ( $RLF$) generated by $v$.

Let $V=\{v\in W_2^1(\Omega)^N:v|_\Gamma=0,{\rm div}\, v=0\}$ is a Hilbert space with a scalar product
$(v, u) _V=\sum_{i, j=1}^N\int_\Omega\mathcal{E}_{ij}(u)\cdot\mathcal{E}_{ij}(v)\, dx$.
 The
space $H$ is defined as the closure of $V$ in the norm of the space $L_2(\Omega)^N,$ while $V^{-1}$ is conjugate to $V$ space. The
sign $ \langle g, u\rangle$ denotes the action of the functional $g\in V^{-1}$ on the element $u\in V$.


Next, $ (\cdot,\cdot)$ denotes the scalar product in suitable Hilbert spaces, $
W_1(0,T)\equiv \{v:\ v\in L_2(0,T;V)\cap L_{\infty}(0,T;H), \ v'\in L_1(0,T;V^{-1})\}.
$

{\bf Definition 1.}{\it
The weak solution to problem $Z$ is the function $v\in W_1(0,T)$ that satisfies the identity
\begin{equation*}\label{tozh1}
\begin{split}
{d}(v, \varphi)/{dt}-\sum_{i=1}^N(v_iv, {\partial \varphi}/{\partial x_i})+\mu_0(\mathcal{E}(v), \mathcal{E}(\varphi))\, + \\
(\int\limits_{0}^tG(t-s)\,\mathcal{E}(v)(s, z(s; t, x))\,ds,
\mathcal{E}(\varphi))=\langle f,\varphi\rangle % \quad v(0, x)=v^0(x),
\end{split}\end{equation*}
for any $ \varphi\in V$ and a.e. $t\in[0, T]$ and initial condition from problem $Z$.
}
Here $z$ is the $RLF$ generated by $v$.

%{\bf Remark 1. } The inner product in (\ref{tozh1}) is defined as %$(\mathcal{E}(g),\mathcal{E}(r))$ for $g,r\in V$
%\begin{equation}\label{int}
%(\mathcal{E}(g),\mathcal{E}(r))=\int_{\Omega}\mathcal{E}(g):\mathcal{E}(r)\,dx= \sum_{i,j=1}^N\int_{\Omega}\mathcal{E}_{ij}(g)\mathcal{E}_{ij}(r)\,dx,\ \
%g,r\in V.\end{equation}


%The following result is the main one.

{\bf Theorem 1.}
{\it
Let $f\in L_2 (0,T;V^{-1}), v^0\in H.$ Then problem $Z$ has a weak solution $v.$}
%%%%%%
%%%%%%%%

To prove Theorem 1 we construct the sequence of approxima\-ti\-ve problems $Z_n$ with regularized convective term in the momen\-tum equation and velocity field in the Cauchy problem in $Z$. Appro\-pri\-ate apriori estimates admit to obtain result of Theorem 1 via the passage to the limit.

This is a joint result with V.G. Zvyagin.

\litlist

1. {\it Oskolkov A. P.} Initial-boundary value problems for equations of motion of Kelvin–Voight fluids and Oldroyd fluids //
Proc. Steklov Inst. Math. - 1989. -V. 179. - P. 137–-182.

2. {\it Zvyagin V. G., Orlov V. P.} Solvability of one non-Newto\-ni\-an fluid dynamics model with
Memory //Nonlinear Analysis: TMA. - 2018. V. 172. P. 73--98.

3. {\it Zvyagin V. G., Orlov V. P.} Weak solvability of fractional Voigt model
of viscoelasticity //Discrete and Continuous Dyna\-mi\-cal Systems, Series A. - 2018. - V. 38. – №. 12. - P. 6327--6350.
% doi:10.3934/dcds.2018270.

4. {\it Zvyagin V. G., Orlov V. P.} On one problem of viscoelastic fluid dynamics with memory on an infinite time interval // Discrete and
Continuous Dynamical Systems, Series B. - 2018. - V. 23. – №. 9. - P. 3855--3877. % doi:10.3934/dcdsb.2018114.

5. {\it DiPerna R. J., Lions P. L.} Ordinary differential equations, transport theory and Sobolev spaces //Invent. Math. - 1989. V. 98. - P. - 511-–547.


% Этот комментарий тут не просто так.
% Иначе скрипты принимают этот файл за не-юникод, и пытаются конвертировать!

