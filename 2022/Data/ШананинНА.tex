\vzmstitle[
]{
	О продолжении решений уравнений с аналитическими коэффициентами
}
\vzmsauthor{Шананин}{Н.\,А.}
\vzmsinfo{Москва, ГУУ; {\it nashananin@inbox.ru}}

\vzmscaption

Пусть
$$
P=\sum_{|\alpha|\le m}a_{\alpha}(x)D^{\alpha},\,
x\in \Omega\subset{\mathbb R}^n,\,D_k=\frac{1}{i}\frac{\partial}{\partial x_k},\,i^2=-1,\eqno (1)
$$
- линейный дифференциальный оператор порядка $m\ge 1$  с вещественно аналитическими, комплекснозначными коэффициентами, определённый в открытом множестве $\Omega$. Согласно теореме единственности Хольмгрена (теорема 8.6.5 [1]) обобщённая функция, удовлетворяющая уравнению $Pu=0$ и обращающаяся
в ноль по одну сторону нехарактеристической $C^1$-поверхности, равна нулю в некоторой этой окрестности поверхности. Если главный символ оператора $P$ удовлетворяет приведённым ниже условиям, то ростки обобщённых решений уравнения $Pu=f$ однозначно продолжаются вдоль интегральных кривых индуцированной главным символом дифференциальной системы.





Пусть $x\in\Omega$ и  ${\mathcal K}_{x}(P)$ --\,ядро симметрической $m$-линейной формы
$$
{\mathcal F}_x(\eta^1,\eta^2,\dots,\eta^m)=\frac{1}{m!}
\sum_{j_1,j_2,\ldots,j_m=1}^n\frac{\partial^m p_m}{\partial\xi_{j_1}\ldots\partial\xi_{j_m}}(x, \xi)\eta^1_{j_1}\ldots\eta^m_{j_m},
$$
порождённой старшим символом $p_m(x, \xi)=\sum\limits_{|\alpha|=m}a_{\alpha}(x)\xi^{\alpha}$ оператора $P$. Если размерность ядра ${\mathcal K}_{x}(P)$ не зависит от выбора $x\in\Omega$, то отображение $x\to  {\mathcal K}_{x}(P)$ порождает в касательном расслоении  $T\Omega$ дифференциальную систему
$$
{\mathcal L}(P)=
\{(x,\tau)\in T\Omega\,|\,x\in\Omega\mbox{ и }\xi(\tau)=0\,\, \forall \xi\in {\mathcal K}_{x}(P)\}
$$
Предположим, что старший символ $p_m(x, \xi)$ удовлетворяет следующим условиям:\\
\begin{minipage}[t]{100mm}
\hspace{0.325cm}
\begin{minipage}[t]{96mm}
(1) $\bigcup\limits_{x\in\Omega}(x,{\mathcal K}_{x}(P)\setminus\{0\})={\rm Char}(P)$;\\
(2) коразмерность ядра ${\mathcal K}_{x}(P)$ не зависит от выбора точки $x\in\Omega$ и равна $k$.\\
\end{minipage}
\end{minipage}

Гладкую кривую $\Gamma\subset\Omega$ называют интегральной для дифференциальной системы ${\mathcal L}(P)$, если
$T_x\Gamma\in{\mathcal L}_{x}(P)$
для всех $x\in\Gamma$.

Говорят, что ростки обобщённых  функций $u^1(x)$ и $u^2(x)\in {\mathcal D}'(\Omega)$ равны в точке $x^0\in \Omega$ и пишут  $u_{x^0}^1\cong u_{x^0}^2$, если существует открытая окрестность $V\subset \Omega$ этой точки, в которой $u^1(x)=u^2(x)$, то есть для любой основной функции $\varphi(x)\in {\mathcal D}(\Omega)$ с носителем ${\rm supp}\,\varphi(x)\subset V$ выполняется равенство
$\langle u^1,\varphi  \rangle= \langle u^2,\varphi  \rangle.$
По определению из равенства ростков функций $u_{x^0}^1\cong u_{x^0}^2$ следует равенство ростков образов $(Pu^1)_{x^0}\cong (Pu^2)_{x^0}$ при отображении  $P: {\mathcal D}'(\Omega) \to {\mathcal D}'(\Omega)$.



\textbf{Теорема~1.} {\it Предположим, что оператор $P$ вида {\rm (1)} имеет аналитические коэффициенты и удовлетворяет условиям {\rm (1)} и {\rm (2)} и
пусть кривая $\Gamma$ является интегральной кривой индуцированной дифференциальной системы ${\mathcal L}(P)$.
Тогда из равенства ростков
$u_{x^0}^1\cong u_{x^0}^2$  обобщённых функций
$u^1(x)$ и $u^2(x)\in {\mathcal D}'(\Omega)$
 в  точке $x^0\in \Gamma$ и равенства ростков
$Pu^1_x\cong Pu^2_x,\,\forall\,x\in \Gamma,$ следует равенство ростков  $u^1_x\cong u^2_x,\,\forall\,x\in \Gamma$.
}

Если дифференциальная система ${\mathcal L}(P)$ является инволютивной, то теореме Фробениуса через каждую точку $x^0\in \Omega$ проходит максимальное интегральное многообразие ${\mathcal M}_{x^0}$. При таком предположении и из теоремы~1 вытекает, что росток решения уравнения $Pu=f$ в точке $x^0$ однозначно определяет ростки решения этого уравнения во всех точках слоя ${\mathcal M}_{x^0}$.


% Оформление списка литературы
\litlist

1. {\it 1. Хермандер Л.} Анализ линейных дифференциальных операторов с частными производными.~(т. 1)/Л.~Хермандер.~--- М.~: Мир, 1986.~--- 464~с.
