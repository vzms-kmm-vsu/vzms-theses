
%%%%\usepackage[russian]{babel}
%%%%\usepackage[utf8]{inputenc}
%%%%\usepackage{graphicx}
%%%%\usepackage{amsmath}
%%%%\usepackage{amsfonts}
%%%%\usepackage{amssymb}
%%%%\pagestyle{empty} \textwidth=108mm \textheight=165mm
%%%%\usepackage{calc,ifthen}
%%%%
%%%%\input{style.tex}



\vzmstitle[
	\footnote{Работа выполнена при поддержке РФФИ, проект \No~19-01-00332-а}
]{
	Обобщение теоремы Майкла
}
\vzmsauthor{Царьков}{И.\,Г.}
\vzmsinfo{Москва, МГУ, механико"=математический факультет,
Центр фундаментальной и прикладной математики; {\it tsar@mech.math.msu.su}}


\vzmscaption





%%%
%%%\tezis{ОБОБЩЕНИЕ ТЕОРЕМЫ МАЙКЛА}  % Заголовок прописными буквами
%%%{Работа выполнена при финансовой
%%%поддержке РФФИ (проект \No\ 19-01-00332-a).}% Если нет, то поле оставить пустым {}
%%%{И.Г. Царьков} {\me{Царьков И.Г.}}{ (Москва)} {tsar@mech.math.msu.su}
%%%
%%%

\textbf{Определение~1.} {\it
Функция $\varrho:X\times X\rightarrow \mathbb{R}_+$ называется несиммметричной полуметрикой на множестве $X,$
если выполняются следующие свойства:

$1.$ $\varrho(x,x)= 0$ для всех $x\in X;$


$2.$ $\varrho(x,z)\leqslant \varrho(x,y)+ \varrho(y,z)$ для всех $x,y,z\in X.$

Пару $\mathcal{X}=(X,\varrho)$ в этом случае будем называть несимметричным полуметрическим пространством.
Функцию $$\sigma(x,y):=\max\{\varrho(x,y), \varrho(y,x)\}$$ будем называть полуметрикой симметри\-за\-ции $($мы будем рассматривать случай конечной полуметрики симметри\-за\-ции$)$. Пространство $\mathcal{X}$ называется полным, если оно полно относительно полуметрики  $\sigma.$
}


\textbf{Определение~2.} {\it   Множество $A$ в   полуметрическом $($сим\-мет\-ричном или несимметричном$)$ пространстве $(X,\nu) $
называется бесконечно связным, если для всех $n\in \mathbb{N}$ и
единичного шара $B\subset \mathbb{R}^n$ и произвольного
непрерывного отображения $\varphi:\partial B\rightarrow A$
существует непрерывное продолжение $\tilde{\varphi}: B\rightarrow
A.$
  Множество $M\subset X $ называется
$\mathaccent'27{B}$-бесконечно связным $({B}$-бесконечно связным$),$ если пересечение множества
$M$ с любым открытым $($замкнутым$)$ шаром либо пусто, либо бесконечно связно.  Множество $M\subset X $ называется
$\mathaccent'27{B}$-стягиваемым $({B}$-стягиваемым$),$ если пересечение множества
$M$ с любым открытым $($замкнутым$)$ шаром либо пусто, либо стягиваемо.
}



Для произвольного множества $M$   некоторого    несимметричного   полуметрического пространства $(X,\nu) $ через $\varrho(y,M):=\inf\limits_{z\in M}\nu(y,z)$ ($y\in X,$ $M\subset X$) обозначим правую метрическую функцию, аналогично определяется   $\varrho^-(y,M)=\inf\limits_{z\in M}\nu(z,y)$, т.е. левая метрическая функция.
  Через $P_Mx$ и $P_M^-x$ обозначим множество всех ближайших точек из $M$ для $x\in X$, т.е. соответственно множество $$\{y\in M\mid \varrho(x,y)=\varrho(x,M)\} \mbox{ и }\{y\in M\mid \varrho(y,x)=\varrho^-(x,M))\}.$$ Эти отображения будем называть соответственно правая и левая метрическая проекция.
%%%%
%%%%\textbf{Определение~3.} {\it
%%%%Пусть $(X,q)  $ -- полуметрическое
%%%%пространство, отображение $\chi:X\rightarrow
%%%%\overline{\mathbb{R}}$ называется полунепрерывным снизу на этом
%%%%полуметрическом пространстве, если для любых точки $x\in X$ и
%%%%последовательности $\{x_n\}\subset X:$ $q(x,x_n)\rightarrow 0$
%%%%$(n\rightarrow\infty)$  верно неравенство
%%%%$\mathop{\underline{\lim}}\limits_{n\rightarrow\infty}\chi(x_n)\geqslant
%%%%\chi(x)$.}

\textbf{Определение~3.} {\it
 $(X,q)$, $(Y,\nu) $ -- полуметрические
пространства, $M\subset Y$. Отображение $F:X\rightarrow 2^M$
назовём устойчивым снизу, если $F(x)\neq \emptyset$ для всех $x\in
X,$  и для любых $x_0\in X$  и $\varepsilon>0$ найдётся такое
число $\delta>0,$ что
$\varrho(y,F(x))-\varrho(y,F(x_0))\leqslant\varepsilon$ для всех
$y\in Y$ и $x\in X:$ $q(x,x_0)\leqslant\delta$.}

\textbf{Определение~4.} {\it
Множество $X$ будем называть полулинейным пространством $($или конусом$)$ над полем $\mathbb{R},$ если на этом множестве определены операции сложения элементов $($векторов$)$ и умножения их на неотрицательные числа, и выполняются следующие свойства для произвольных $x,y,z\in X$ и $\alpha,\beta\in \mathbb{R}_+$:
$1.$ $x+y=y+x;$
$2.$ $x+(y+z)=(x+y)+z;$
$3.$ существует и единственен нулевой элемент $\theta\in X,$ для которого $x+\theta=x;$
$4.$ $\alpha(x+y)=\alpha x+\alpha y;$
$5.$ $(\alpha+\beta)x=\alpha x+\beta y;$
$6.$ $\alpha(\beta x)=(\alpha\beta)x;$
$7.$ $0\cdot x=\theta;$ $1\cdot x=x.$ }


\textbf{Определение~5.} {\it
 Пару $\mathcal{X}=(X,\varrho)$ будем называть полулинейным несимметричным полуметрическим пространством, если несимметричная полуметрика $\varrho$  определена на полулинейном пространстве $X$ и операции сложения элементов $($векторов$)$ и умножения их на неотрицательные числа непрерывны, т.е. для любых последовательностей $\{\alpha_n\},\{\beta_n\}\subset \mathbb{R}_+$ и $\{x_n\},\{y_n\}\subset X,$ сходящихся соответственно к $\alpha,\beta\in \mathbb{R}_+$ и $x,y\in X,$ последовательность  $\sigma(\alpha_n x_n+\beta_n y_n,\alpha x +\beta  y ) $ сходится к $0$, где $\sigma$ -- полуметрика симметризации.
Множество $M\subset X$ называется выпуклым, если $[a,b]\subset M$ для всех $a,b\in M.$

Несимметричная метрика $\varrho$ называется обобщённо выпуклой (выпуклой) на выпуклом множестве $M\subset X,$ если $\varrho(z,\alpha x+(1-\alpha)y)\leqslant  \max\{\varrho(z,x),\varrho(z,y)\}$ $\big(\varrho(z,\alpha x+(1-\alpha)y)\leqslant \alpha\varrho(z,x)+(1-\alpha)\varrho(z,y)\big)$ для всех $\alpha\in[0,1],$ $z\in X$ и $x,y \in M.$ В частности, из выпуклости $\varrho$ вытекает её обобщённая выпуклость.}
%%
%%Отметим, что если несимметричная полуметрика обобщенно выпукла на выпуклом множестве $M,$ то   непустое пересечение этого множества с произвольным открытым шаром является стягиваемым по себе в точку множеством, и, следовательно, $M$ --   $\mathaccent'27{B}_\varrho$-бесконечно связно $($т.е. непустое пересечение с любым открытом шаром бесконечно связно$)$.
%%


\textbf{Теорема~1.} {\it    Пусть  $(X,\upsilon)$ --
 метрическое пространство,
$(Y,\varrho)$  -- полное полулинейное
полуметрическое пространство, в котором метрика $\varrho$ обобщённо выпукла и для полуметрики выполняются равенства: $\varrho(x,y )=\varrho(y,x)$ и
$\varrho(x,(1-\alpha)x+\alpha y )= \alpha \varrho(x,y) $ для всех $x,y\in X$ и $\alpha\in [0,1]$;
а многозначное отображение $\Phi:X\rightarrow 2^Y$  устойчиво
снизу  и имеет замкнутые $\mathaccent'27{B} $-бесконечно связные в $ (Y,\|\cdot\|) $ образы.
 Тогда существует отображение $\varphi\in C(X,  Y)$:
$\varphi(x)\in \Phi(x)$ для всех $x\in  X$.

}





\textbf{Следствие~1.} {\it   Пусть
 $(Y,\varrho)$  -- полное полулинейное
полуметрическое пространство, в котором метрика $\varrho$ обобщённо выпукла и для полуметрики выполняются равенства: $\varrho(x,y )=\varrho(y,x)$ и
$\varrho(x,(1-\alpha)x+\alpha y )= \alpha \varrho(x,y) $ для всех $x,y\in X$ и $\alpha\in [0,1]$. Пусть $M\subset Y$ обладает
многозначной выборкой $\Psi:X\rightarrow 2^M$ из метрической проекцией $P_M$ $(P_M^-)$,
 устойчивой
снизу, образы которой замкнуты и $\mathaccent'27{B} $-бесконечно связные в $Y$.
 Тогда существует отображение $\varphi\in C(X,  M)$:
$\varphi(x)\in \Psi(x)\subset P_M(x)\ (P_M^-(x))$ для всех $x\in  X$.
}



%%%
%%% %%%%  ОФОРМЛЕНИЕ СПИСКА ЛИТЕРАТУРЫ %%%
%%%%\smallskip \centerline{\bf Литература}\nopagebreak
 %%%\liter
%%%
%%%



%%%
%%%2.   Царьков~И.\,Г.    Некоторые приложения геометрической теории приближения // ВИНИТИ РАН. Итоги науки и техники. Современные проблемы математики. Тематические обзоры. Комплексный анализ и смежные вопросы. ---   2017.
%%%
%%%3. Царьков~И.\,Г.  Особые множества и их связь с особенностями решений уравнения эйконала // Труды НИИСИ РАН. --- 2016. --- Т. 6, \No~2. --- С. 126--128.

