


\vzmstitle{
	О характеристике плоских множеств с целочисленными расстояниями с ребром 1, 2 или 3
}
\vzmsauthor{Лушина}{Е.\,А.}
\vzmsinfo{Воронеж, ВГУ; {\it ekaterinalushina18@gmail.com}}


\vzmscaption

Рассматривается множество точек $M$ в евклидовом пространстве $\mathbb{R}^{2}$ такое, что расстояние между любыми двумя точками из $M$ есть целое число. Всякое бесконечное множество $M$ содержится в некоторой прямой $L \subset \mathbb{R}^2$ [1, 2].

\paragraph{Определение 1.}
Мощностью множества $M$ (обозначается $\#M$), состоящего из конечного
числа элементов, называется количество элементов этого множества.

Для заданного $n \in \mathbb{N}$ обозначим через $\mathfrak{M_n}$ множество таких множеств $M$, что  $\#M=n$ и $M \not \subset L$ для любой прямой $L \subset \mathbb{R}^{2}$.

\paragraph{Определение 2.}
Ребром множества $M$ называют любой отрезок $AB$, где $A, B \in M$.

\paragraph{Определение 3.}
Характеристикой множества $M \in \mathfrak{M}_{n}$ называется свободное от квадратов число $p$ такое,
что для любых $A, B, C \in M$ площадь треугольника $ABC$ соизмерима с $\sqrt{p}$.
Пишут: $\operatorname{char}{M}={p}$.


Про множества $M \in \mathfrak{M}_{n}$ известно следующее:
\begin{itemize}
	\item
		Для любого $n \in \mathbb{N}, n \geq 3$ верно $\mathfrak{M}_{n} \neq \varnothing$ [3, теорема 2].

	\item
		Для любой мощности $n \in \mathbb{N}, n \geq 3$ и любого свободного от квадратов числа $p$ существует множество $M \in \mathfrak{M}_{n}$ такое, что $\operatorname{char}{M}={p}$ [3, теорема 5].
	\item
		Для любой мощности $n \in \mathbb{N}, n \geq 3$ и любого $k \in \mathbb{N}$ существует несжимаемое \foreignlanguage{english}{(prime)} множество $M \in \mathfrak{M}_{n}$, содержащее ребро ${k}$ [4].
	\item
		Не существует множества $M \in \mathfrak{M}_{n}$ характеристики 1 с ребром 1 [3, утверждение 6].
\end{itemize}

Возникает вопрос о существовании множества $M\in\mathfrak{M}_n$
с заданными характеристикой и минимальным ребром. При рассмотрении множеств $M \in \mathfrak{M}_{3}$ удобно ввести систему координат (см., напр., [3]):
 $A(-m/2;0)$, $B(m/2;0)$,
$C(a_i/2m;$ $b_i \sqrt{p}/2m)$, где $a_i, b_i \in \mathbb{Z}$, $m$ "--- длина любого ребра $M$.
Сформулируем основные результаты работ [5-7].



\paragraph{Утверждение 1.} [5, утверждение 3]
Всякое множество $M \in \mathfrak{M}_{3}$ с минимальным ребром 1 имеет характеристику вида
$p=4k+3$, $k\in\mathbb{N}_{0}$, где $ \mathbb{N}_{0}=\mathbb{N} \cup 0$.

В частности, ранее было доказано [5, 6], что не существует множества  $M \in \mathfrak{M}_{3}$ характеристики 2, 5 или 6 с ребром 1.



\paragraph{Утверждение 2.} [5, утверждение 4]
Для всякой характеристики $p>1$ существует бесконечное семейство множеств $\{M_j\}\subset\mathfrak{M}_3$ с ребром 2
таких, что $\operatorname{char}M_j = p$.

Для рёбер $|AB| \geq 2$ третья точка может лежать не только на серединном перпендикуляре к ребру $AB$, но и на гиперболах специального вида [2].

\paragraph{Утверждение 3.} [7, утверждение 1]
Всякое множество $M=\{A, B, C\} \in \mathfrak{M}_{3}$ с минимальным ребром $|AB|=2$, где $|AC|-|BC|=1$, имеет характеристику $p=8k+7$, $k\in\mathbb{N}_{0}$.

\paragraph{Следствие 1.}
Не существует множества $M \in \mathfrak{M}_{3}$,  $\operatorname{char} M = 1$ с ребром 2.

\paragraph{Утверждение 4.} [7, утверждение 2]
Всякое множество $M=\{A, B, C\} \in \mathfrak{M}_{3}$ с минимальным ребром $|AB|=3$, где $|AC|-|BC|=0$, или $|AC|-|BC|=2$, имеет характеристику $p=4k+3$, $k\in\mathbb{N}_{0}$.

\paragraph{Замечание 1.}
В случае множества $M=\{A, B, C\} \in \mathfrak{M}_{3}$ с ребром $|AB|=3$ и $|AC|-|BC|=1$, характеристика может принимать значения, отличные от $4k+3$, $k\in\mathbb{N}_{0}$.

Полученные результаты планируется применить к улучшению оценок, построенных в [7].

\litlist

1. {\it Erdös P.}
Integral distances // Bull. Amer. Math. Soc. — 1945. — Vol. 51, N 12 . — P. 996.

2. {\it Anning N. H., Erdös P. }
Integral distances // Bull. Amer. Math. Soc. – 1945. – Т. 51. – №. 8. – С. 598-600.

3. {\it Авдеев Н. Н., Семенов Е. М.}
Множества точек с целочисленными расстояниями на плоскости и в евклидовом пространстве //Математический форум (Итоги науки. Юг России). – 2018. – Т. 12. – С. 217-236.

4. \foreignlanguage{english}{{\it Zvolinsky R. E.}
Facher integral point sets with particular distances of arbitrary cardinality} //Актуальные проблемы прикладной математики, информатики и механики. – 2021. – С. 668-674.

5. {\it Лушина Е. А., Авдеев Н. Н.}
О характеристике плоских множеств с целочисленными
расстояниями с рёбрами 1 и 2 // Вестник факультета ПММ. – 2021. – Вып. 15. – С.
117–123.

6. {\it Лушина Е. А.}
О существовании специальных множеств с целочисленными расстояниями с ребром 1 //71-я Международная студенческая научно"=техническая конференция. – 2021. – С. 595-598.

7. {\it Лушина Е. А.}
О характеристике плоских множеств с целочисленными расстояниями с малым ребром. "--- в печати.


