\vzmstitle[]{
	Специальные случаи  частично проективных кватернионных многообразий Штифеля
}
\vzmsauthor{Горденин}{Л.\,В.}
\vzmsinfo{Москва; {\it db0zmail@gmail.com}}

\vzmscaption

Многообразие Штифеля $V_{n,k}(\mathbb{F})$, где $\mathbb{F} = \mathbb{R}, \mathbb{C}, \mathbb{H}$ "--- это многообразие, состоящее из ортонормированных $k$-реперов в $\mathbb{F}^n$. Рассматривается случай $\mathbb{F} = \mathbb{H}$; кватернионные многообразия Штифеля обозначаем $H_{n,k} = V_{n,k}(\mathbb{H})$. Их кольца когомологий вычислены в [1]:
$$H^*(H_{n,k}; \mathbb{Z}) = \Lambda(z_{n-k+1}, \ldots, z_n),$$
где $\deg z_j = 4j - 1$. Многообразие Штифеля допускает действие скаляров, по модулю равных единице, состоящее в умножении всех векторов репера на скаляр; соответствующее факторпространство называется \emph{проективным многообразием Штифеля $PV_{n,k}(\mathbb{F})$}; обозначаем $PH_{n,k}=PV_{n,k}(\mathbb{H})$. В [2] показано, что
$$H^*(PH_{n,k}, \mathbb{Z}_2) = \mathbb{Z}_2[x]/(x^N) \otimes \Lambda(y_{n-k+1},..., y_{N-1}, y_{N+1}, ..., y_n),$$
где $N = \min \{j | n - k + 1 \leqslant j \leqslant n, \binom{n}{j} \not\equiv 0 \mod 2\}$, $\deg y_j = 4j - 1$, $\deg x = 4$.

Пусть $G$ "--- конечная группа, допускающая свободное действие на $S^3$; такими группами являются: конечные циклические группы $\mathbb{Z}_m$, обобщённые группы кватернионов $Q_{4m}$, двойная тетраэдральная группа $P_{24}$, двойная октаэдральная группа $P_{48}$, двойная икосаэдральная группа $P_{120}$, две серии подгрупп $SO(4)$: $P'_{8\cdot3^s}$ и $B_{2^s(2m+1)}$, а также все прямые произведения любой из перечисленных групп с циклической группой взаимно простого порядка. Тогда $G$ свободно и дискретно действует левыми сдвигами на слоях расслоения $H_{n,k} \to PH_{n,k}$. Факторпространство $H^G_{n,k} = H_{n,k}/G$ называем \emph{частично проективным кватернионным многообразием Штифеля}. В [2] вычислены их кольца когомологий с коэффициентами $\mathbb{Z}_p$ в зависимости от $G$ и $p$ для всех случаев $G$, кроме случая прямого произведения с циклической группой взаимно простого порядка. Этот случай рассматривается в докладе.


\litlist

\selectlanguage{english}

1. \emph{Borel A.} Sur la cohomologie des espaces fibres principaux et des espaces homogenes
des groupes de Lie compacts // Ann. Math. 57 (1953), 115–207.

\selectlanguage{russian}

2. \emph{Жубанов Г.Е., Попеленский Ф.Ю.} О кольцах когомологий частично проективных кватернионных многообразий Штифеля // Матем. сборник, 2022 (в печати).
