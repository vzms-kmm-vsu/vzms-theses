\documentclass{vzmsthesis}

\begin{document}

\vzmstitle[
	\footnote{Работа выполнена за счёт гранта РФФИ, проект 19-01-00016.}
]{
	Тензорные инварианты динамических систем с диссипацией с двумя степенями свободы
}
\vzmsauthor{Шамолин}{М.\,В.}
\vzmsinfo{Москва, МГУ им. М. В. Ломоносова; {\it shamolin@rambler.ru,~shamolin@imec.msu.ru}}
%\vzmsauthor{Семенов}{Е.\,М.}
%\vzmsinfo{Воронеж, ВГУ; {\it nadezhka\_ssm@geophys.vsu.ru}}

\vzmscaption



Как известно [1, 2, 3], наличие достаточного количества не только первых интегралов (скалярных инвариантов), но и других тензорных инвариантов позволяет полностью проинтегрировать систему дифференциальных уравнений. Так, например, наличие инвариантной формы фазового объема позволяет понизить порядок рассматриваемой системы. Для консервативных систем этот факт естественен. А вот для систем, обладающих притягивающими или отталкивающими предельными множествами, не только некоторые первые интегралы, но и коэффициенты имеющихся инвариантных дифференциальных форм должны, вообще говоря, состоять из трансцендентных (в смысле комплексного анализа) функций [4, 5, 6]. 

Так, например, задача о движении пространственного маятника на сферическом шарнире в потоке набегающей среды приводит к системе на касательном расслоении к двумерной сфере, при этом метрика специального вида на ней индуцирована дополнительной группой симметрий [7]. Динамические системы, описывающие движение такого маятника, обладают знакопеременной диссипацией, и полный список первых интегралов состоит из трансцендентных функций, выражающихся через конечную комбинацию элементарных функций. Известны также задачи о движении точки по двумерным поверхностям вращения, плоскости Лобачевского и т.д. Полученные результаты особенно важны в смысле присутствия в системе именно неконсервативного поля сил [5].

В работе предъявлены тензорные инварианты (дифференциальные формы) для однородных динамических систем на касательных расслоениях к гладким двумерным многообразиям. Показана связь наличия данных инвариантов и полным набором первых интегралов, необходимых для интегрирования геодезических, потенциальных и диссипативных систем. При этом вводимые силовые поля делают рассматриваемые системы диссипативными с диссипацией разного знака и обобщают ранее рассмотренные. 


\litlist

1. {\it Poincar\'{e} H.} Calcul des probabilit\'{e}s, Gauthier--Villars, Paris, 1912, 340 pp. 

2. {\it Колмогоров А.Н.} О динамических системах с интегральным инвариантом на торе 
// Доклады АН СССР. -- 1953. -- Т.~93. ---№~5. --- С.~763--766.

3. {\it Козлов В.В.} Тензорные инварианты и интегрирование дифференциальных
уравнений // Успехи матем. наук. --- 2019. --- Т.~74. ---№~1(445). --- С.~117--148.

4. {\it Шамолин М.В.} Об интегрируемости в трансцендентных функциях // Успехи матем. наук. 1998. Т. 53. Вып. 3. С. 209–210. 

5. {\it Шамолин М.В.} Новые случаи однородных интегрируемых систем с диссипацией на касательном расслоении двумерного многообразия // Доклады РАН. Математика, информатика, процессы управления, 2020. Т. 494. № 1. С. 105–111. 

6. {\it Шамолин М.В.} Новые случаи интегрируемых систем нечетного порядка с диссипацией // Доклады РАН. Математика, информатика, процессы управления, 2020. Т. 491. № 1. С. 95–101. 

7. {\it Шамолин М.В.} Новый случай интегрируемости в пространственной динамике твердого тела, взаимодействующего со средой, при учете линейного демпфирования // Доклады РАН, 2012. Т. 442. № 4. С. 479–481. 


%1. {\it Semenov E. M., Sukochev F. A.}
 %Invariant Banach limits and applications //Journal of Functional Analysis. – 2010. – Т. 259. – №. 6. – С. 1517-1541.

\end{document}
