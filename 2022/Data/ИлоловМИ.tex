\documentclass{vzmsthesis}

\begin{document}

\vzmstitle[

]{
	Дробное дифференциальное уравнение Балакришнана
}
\vzmsauthor{Илолов}{М.\,И.}
\vzmsinfo{Душанбе, Таджикистан, ЦИРННТ НАНТ; {\it ilolov.mamadsho@gmail.com}}
\vzmsauthor{Лашкарбеков}{С.\,М.}
\vzmsinfo{Душанбе, Таджикистан, ЦИРННТ НАНТ; {\it lashkarbekov89@bk.ru}}
\vzmsauthor{Рахматов}{Дж.\,Ш.}
\vzmsinfo{Душанбе, Таджикистан, ЦИРННТ НАНТ; {\it jamesd007@rambler.ru}}

\vzmscaption


Теория стохастических эволюционных уравнений и ее приложений получили существенное развитие за последние десятилетия. Это связано с одной стороны с бесконечномерным анализом полугрупп и эволюционных операторов решений, а с другой с тем, что их конечномерные реализации часто возникают в качестве математических моделей в физике, технике,
химии, математической биологии, финансовой математике и других областях науки. Обобщение теории Ито-Стратоновича-Скорохода на бесконечномерном случае берет свой начало в работах [1,2]. В рамках этой теории, в частности, было исследовано линейное дифференциальное уравнение Ито с мультипликативным шумом [3-6]. В работах [7-10] был
предложен иной подход к анализу стохастических уравнений на основе производной Нельсона-Гликлиха. В работе [10] установлено, что производная Нельсона-Гликлиха от винеровского процесса хорошо согласуется с предсказаниями теории броуновского движения Эйнштейна-Смолуховского, поэтому соответствующий стохастический процесс был назван "белым шумом".

В настоящей работе изучается класс дифференциальных уравнений дробного порядка возмущенных абсолютным случайным процессом или белым шумом типа Балакришнана. Такой тип белого шума впервые введен в монографии [11]. Свойства белого шума Балакришнана подробным образом описаны в [12].

Пусть $H$ - вещественное сепарабельное гильбертово пространство. Через $W=L_{2}((0,T),H)$ обозначим сепарабельное гильбертово пространство, а через $\mu$ стандартную гауссову меру на $W$. Так определенное функциональное пространство $W$ называется белым шумом. Каждый элемент $\omega\in W$ называется реализацией белого шума. Отметим, что всякая функция $\omega$ определена лишь с точностью до меры нуль на отрезке $[0, T]$, и поэтому значение $\omega$ в точке $t$ не определено. С другой стороны, каждый элемент $h\in W$ определяет гауссову случайную величину $[\omega, h]$ с нулевым математическим ожиданием и дисперсией равной $[h, h]$. В этом случае, естественно предположить, что каждый физический процесс является случайной величиной. Пусть $\{\varphi_{k}\}$ - ортонормированный базис ространства $W$. Тогда гауссовые случайные величины

$$\zeta_{k}(\omega)=[\omega, \varphi_{k}]$$
имеют нулевые математические ожидания а их дисперсии равны единице. Кроме того, в отличие от классического случая,

$$\sum\limits_{1}^{\infty}(\zeta_{k}(\omega))^{2}=\omega^{2}<\infty.$$

Далее, введем непрерывную по $t$ при фиксированном $\omega$ функцию

$$W(t,\omega)=\int\limits_{0}^{t}\omega(s)ds.$$

При $t>s$ разность $[W(t,\omega)-W(s,\omega),h]$ является гауссовой случайной величиной с нулевым математическим ожиданием и дисперсией равной $(t-s)\|h\|^{2}$.

Тем не менее, функция $W(t,\omega,h)$ не может быть реализацией винеровского процесса, ввиду того, что она имеет ограниченную вариацию на каждом конечном интеграле при фиксированном $\omega$.

В случае, когда $H=L_{2}(D)$, где $D$ - открытое и ограниченное подмножество евклидова пространства $\mathbb{R}^{n}$, элементами пространства $W$ являются измеримые по совокупности переменных функций $\omega(t,x)$ такие, что

$$\int\limits_{0}^{T}\int\limits_{D}|(t,x)|^{2}d|x|dt=\|\omega\|^{2}<\infty.$$

Пусть $x_{1},x_{2}$ - пара различных точек множества $D$,a $U_{1}$ и $U_{2}$ их непересекающиеся окрестности. Тогда для любой функции $g(t)$ величины

$$\int\limits_{0}^{T}\int\limits_{D}g(t)h_{1}(x)\omega(t,x)d|x|dt, \int\limits_{0}^{T}\int\limits_{D}g(t)h_{2}(x)2\omega(t,x)d|x|dt$$
независимы. Это свойство сохраняется при стягивании выбранных окрестностей к соответствующим точкам, т.е. процесс пространственно не зависимые. Имеет место аналогичное свойство временно независимости. Мы имеем математическую модель пространственно-временной корреляции, определяемая дельта-функцией. Существуют другие модели предоставляемые теорией маргиналов и винеровскими процессами.

Рассматривается задача Коши вида

$$^{c}D_{t}^{\alpha}u(t)+Au(t)=f(u(t))+B\omega(t),u(0)=u_{0},\eqno{(1.1)}$$

где ${^c}D_{t}^{\alpha}$  - дробная производная Капуто порядка ${\alpha, 0<\alpha<1}$, $A$ - почти секториальный оператор в сепарабельном гильбертовом пространстве $H,f:H\rightarrow H$ - нелинейное заданное отображение, $\omega(t)$ - абсолютный случайный процесс (белый шум в смысле Балакришнана) в другом сепарабельном гильбертовом пространстве $H_{n}, B$ - линейный оператор, определенный в H со значениями в пространстве линейных операторов из $H_{n}$ в $H$.
При анализе разрешимости задачи Коши (1.1) стандартным требованием является порождение оператором А резольвентных семейств операторов $\{S_{\alpha}(t)\}_{t\geq0}$ и $\{Z_{\alpha}(t)\}_{t\geq0}$. Это условие гарантирует корректность задачи Коши для детерминированного, невозмущенного однородного уравнения

$$^{c}D_{t}^{\alpha}u(t)+A(t)u(t)=0$$

Далее нужно потребовать от нелинейного отображения $f(\cdot)$ условия типа Липщица. Условия, накладываемые на оператор $В$, тесно связаны со свойствами абсолютного случайного процесса (белого шума). 









\litlist

1.	{\it Ichikawa A.} Stability of semilinear stochastic evolution equations. // J. Math. Anal. Appl. – 1982.– N90.– pp. 12-44.
 
2.	{\it Ichikawa A.} Semilinear stochastic evolution equations // Stochastics. – 1984.– N12.– pp. 1-39.

3.	{\it Da Prato G., Zabczyk J.} Stochastic equations in infinite dimensions. // Cambridge: Camb. Univ. Press. – 2014.

4.	{\it Gawarecki L., Mandrekar V.} Stochastic Differential equations in infinite dimensions with applications to stochastic partial differential equations. // Berlin; Heidelberg: Springer - Verl. – 2011.

5.	{\it Melnikova I.V.} Stochastic Cauchy problems in infinite dimensions. Generalized and regularized solutions. // Boca Raton; London; New York: CRS Press – 2016.

6.	{Filinkov A., Sorensen J.}  Differential equations in spaces of abstract stochastic distributions. // Stoch. Stoch. Rep.– 2003.– v. 72.– N3-4.– pp. 129-173.

7.	{\it Gliklich Yu. E.} Global and Stochastic Analysis with Appercations to Mathematical Physics. // London; Dordrecht; Heidelberg; New York: Springer/ – 2011.

8.	{\it Nelson E.} Dynamical Theory of Brownian motion. // Princeton: Princeton University Press/ – 1967.
9.	{\it Kovach M., Larsson S.} Introduction to Stochastic Partial Differential Equations. New Directions on the Mathematical and Computer Sciences. // National Universities Commission, Auja, Nigeria, 2007. -V.4, Publications of the ICMCS, Lagos (2007), pp. 159232.

10.	{\it Sviridynk G.A, Zamyshlyaeva A.A., Zagrebina S.A.} Multipoint initial-final problem for one class of Sobolev type models of higher order with additive "white noise”. // Vestnik YUUrGU. Seriya "Matematicheskoe modelirovanie i proramirovanie". - V.11.– N3. – (2018).– pp. 103-117.

11.	{\it Balakrishnan A.V.} Applied Functional Analysis. // Springer - Verlag, Applications of Mathematics. – 1976.

12. {\it Ilolov, M., Kuchakshoev, K.S., Rahmatov, J.S.} Fractional stochastic evolution equations: Whitenoise model // Communications on Stochastic Analysis. - 2020. - 14(3-4). - pp. 55-69.


\end{document}
