\vzmstitle{Нули функционалов и параметрическая версия теоремы Майкла о сечениях}
\vzmsauthor{Фоменко}{Т.\,Н.}
\vzmsinfo{Москва, МГУ; {\it tn-fomenko@yandex.ru}}

\vzmscaption

Доклад посвящен некоторым результатам о существовании нулей поисковых функционалов и их применению к проблеме существования непрерывных сечений многозначных отображений.

В 2009--2013 автором было введено понятие  $(\alpha,\beta)$-по\-ис\-ко\-во\-го функционала и представлено несколько версий каскадного поиска нулей таких функционалов в метрическом пространстве [1,2]. В дальнейшем эти результаты были развиты и распространены на квазиметрические пространства [3]. Для метрических пространств была получена теорема о сохранении существования нулей у семейства многозначных функционалов при изменении числового параметра [4]. Перечисленные результаты дали ряд следствий о существовании и аппроксимации неподвижных точек и совпадений однозначных и многозначных отображений метрических и квазиметрических пространств, а также о сохранении существования неподвижных точек и совпадений параметрических семейств  отображений при изменении числового параметра, обобщающих некоторые известные теоремы.

При обсуждении этих результатов возник вопрос о применении их к задаче о существовании непрерывных сечений многозначных отображений и о связи с известной теоремой Майкла о сечениях. Как известно, в  1956 году Эрнест Майкл (Ernest Michael)  доказал знаменитую теорему о существовании непрерывно\-го однозначного сечения у многозначного отображения метрических прост\-ранств (см. [5]).

Эта задача  является важной и востребованной для приложений. Имеется много работ разных авторов по этой тематике. В частности, многозначные отображения возникают как обратные к однозначным. Например, пусть $X$ и $Y$ --- топологические пространства, $f: X\to Y$ --- сюръективное отображение, и многозначное отображение $F: Y\rightrightarrows X$  обратно к $f$. Если исходное отображение $f$ непрерывно, и многозначное отображение $F$ имеет непрерывное однозначное сечение $\varphi: Y\to X$ (то есть для любого $y\in Y$ верно, что $\varphi(y)\in F(y)$), то очевидно, что $Y$ гомеоморфно подпространству $\varphi(Y)\subseteq X$, а в случае сюръективности сечения $\varphi$ отображение $f$ задает гомеоморфизм между $X$ и $Y$. Как известно, задача о существовании гомеоморфизма является важнейшей задачей топологии.

Напомним следующее определение.

\paragraph{Определение. [6]} Пусть $X,Y$ --- топологические пространства. Многозначное отображение $F: X\rightrightarrows Y$ (с непустыми образами) называется {\it полунепрерывным снизу}, если для любого открытого подмножества  $U\subseteq Y$ его полный (расширенный) прообраз $F^{-1}_{-}(U):=\{x\in X | F(x)\cap U\ne \emptyset\}$ открыт в $X$.

Достаточные условия и полезный критерий полунепрерывности снизу многозначного отображения имеются в  [7].

В докладе представлена теорема, обобщающая теорему Майкла. Кроме того, получена следующая параметрическая версия теоремы Майкла.

\paragraph{Теорема.}
{\it Пусть $X$ паракомпактно, $(Y,||\cdot ||)$ банахово пространство, $F_{t}: X\rightrightarrows Y, t\in [0;1],$ семейство многозначных полунепрерывных снизу отображений с непустыми замкнутыми выпуклыми образами. Пусть $W$  открыто в полном метрическом пространстве $(C(X,Y),\mu)$ непрерывных отображений ($\mu(f,g):=\mathop{\sup}\limits_{x\in X}||f(x)-g(x)||$, $f,g\in C(X,Y)$), $M_{W}:=\{(f,t)\in W\times [0;1] | f \mbox{ непрерывное сечение } F_{t}\}\ne \emptyset$, и на границе $\partial W$ нет непрерывных сечений у любого отображения $F_{t}, t\in [0;1]$. Пусть также для некоторой возрастающей непрерывной функции $\theta: [0;1]\to \mathbb R$ верно, что
$$
H(F_{t}(x),F_{t'}(x))\le|\theta(t)-\theta(t')|,
$$
для всех $t$ таких, что существует $(f,t)\in M_{W}$, и для любых $t'\in [0;1]$. Здесь $H(A,B)$--- метрика Хаусдорфа. Тогда, если есть непрерывное сечение $f_{0}\in W$ у отображения $F_{0}$, то есть и непрерывное сечение $f_{1}\in W$ у отображения $F_{1}$.}

1.
{\it Fomenko T.N.} Cascade search principle and its applica\-ti\-ons to the
coincidence problem of n one-valued or multi-valued mappings. //Topology and its Applications, 157, 760--773 (2010)

2.
{\it Фоменко Т.Н.} Каскадный поиск прообразов и совпадений: глобальная и
локальная версии. //Математические заметки,  93:1, 127--143 (2013).

3.
{\it Фоменко Т.Н.} Существование нулей многозначных фу\-нкционалов, совпадения и неподвижные точки в $f$-ква\-зи\-мет\-ри\-чес\-ком пространстве. //Математические заметки, \\110:4, 598--609 (2021).

4.
{\it Захарян~Ю.Н., Фоменко ~Т.Н.}  Сохранение существования нулей у семейства многозначных функционалов и некоторые следствия. //Математические заметки, МИАН (Мос\-ква), 108:6, 837--850 (2020).

5.
{\it Michael E. } Continuous selections I. //Ann. of Math. 63, 361--381 (1956).

6.
{\it Борисович~Ю.Г., Гельман~Б.Д., Мышкис~А.Д., Обухов\-ский~В.В.}
Введение в теорию многозначных отображений и дифференциальных включений. //КомКнига, Москва, 2005.

7.
{\it Гельман~Б.Д.} Непрерывные аппроксимации многозначных отображений и неподвижные точки. //Математические заметки,  78:2, 212--222 (2005).
