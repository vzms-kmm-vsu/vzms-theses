\vzmstitle[
	\footnote{Работа выполнена в рамках реализации государственного задания Министерства науки и высшего образования РФ по проекту «Нелинейные сингулярные интегро"=дифференциальные уравнения и краевые задачи» в соответствии с Соглашением № 075-03-2021-071 от 29.12.2020 г.}
]{
	Краевые задачи для нелинейных сингулярных интегро"=дифференциальных уравнений с ядром Коши
}
\vzmsauthor{Асхабов}{С.\,Н.}
\vzmsinfo{Грозный, ЧГПУ, ЧГУ; {\it askhabov@yandex.ru}}

\vzmscaption

В данной работе изучаются краевые задачи для различных классов нелинейных сингулярных интегро"=дифференциальных уравнений, содержащих операторы вида
$$(Bu)(x)=-\frac{b(x)}{\pi}\int\limits_{-1}^1\frac{[b(s)\cdot u(s)]'ds}{s-x},
$$
где интеграл понимается в смысле главного значения по Коши. Методом максимальных монотонных (по Браудеру-Минти) операторов [1], при достаточно легко обозримых ограничениях на нелинейность доказаны глобальные теоремы о существовании и единственности решения рассматриваемых задач в вещественных весовых пространствах Лебега.

Пусть $1<p<\infty$ и $q=p/(p-1)$. Обозначим через $L_p(\varrho)$ множество всех  измеримых по Лебегу на отрезке $[-1, 1]$ функций таких, что $\displaystyle{\int\limits_{-1}^1 \varrho(x)\cdot |u(x)|^pdx<\infty}$, где вес $\varrho(x)=(1-x^2)^{-1/2}$. Известно, что сопряжённым с $L_p(\varrho)$ является пространство  $L_{q}(\sigma)$, где $\sigma(x)=(1-x^2)^{({q}-1)/2}$ [2, 3]. Множество всех неотрицательных функций из $L_p(\varrho)$ обозначим через $L_p^+(\varrho)$.

В работе [4] доказано, что при $b(x)=1$ оператор $B$ с областью определения
$$
D(B)=\{u\in L_p(\varrho): u\in CA[-1,1]\,, \  u'\in L_{q}(\sigma) \  \hbox{\rm и} \ u(\pm 1)=0\},
$$
где $CA[-1,1]$ есть множество всех абсолютно непрерывных на отрезке $[-1,1]$ функций, является максимальным монотонным оператором, действующим из $D(B)\subset L_p(\varrho)$ в $L_{q}(\sigma)$.

Пусть функция $F(x,u)$, порождающая нелинейность в рассматриваемых уравнениях, определена при $x\in [-1, 1]$, $u\in \mathbb R$ и удовлетворяет известным условиям Каратеодори: она измерима по $x$ при каждом $u\in \mathbb R$ и непрерывна по $u$ для почти всех $x\in [-1, 1]$.

При доказательстве следующих теорем важную роль играет неравенство [2]:
$$
\int\limits_1^1\left(-\frac{b(x)}{\pi}\int\limits_{-1}^1\frac{[b(s)\cdot u(s)]'ds}{s-x}\right)\cdot u(x)\,dx\geq \int\limits_1^1\frac{[b(x)\cdot u(x)]^2}{\sqrt{1-x^2}}dx,
$$
обобщающее известное неравенство Шлайфа [4] и справедливое для любого $u\in D(B)$.

\textbf{Теорема~1.} {\it Пусть  $p\geq 2$,  $\varrho(x)=(1-x^2)^{-1/2}$, $b(x)\in C^1[-1,1]$ и $f(x)\in L_p(\varrho)$. Если для почти всех $x\in [-1, 1]$ и всех $u\in \mathbb R$  выполняются условия: \\
1)  $|F(x,u)|\leq a(x)+d_1\cdot \varrho(x)\,|u|^{p-1}$, где $a(x)\in L_q^+(\sigma)$, $d_1>0$; \\
2) $F(x,u)$  не убывает по $u$  почти при каждом $x\in [-1, 1]$; \\
3) $F(x,u)\cdot u\geq d_2\cdot \varrho(x)\,|u|^p-D(x)$,  где $D(x)\in L_1^+(-1,1)$, $d_2>0$, \\
то при любых значениях параметра $\lambda>0$ краевая задача
$$
\lambda\cdot F(x, u(x))-\frac{b(x)}{\pi}\int\limits_{-1}^1\frac{[b(s)\cdot u(s)]'ds}{s-x}=f(x),\eqno(1)
$$
$$
u(-1)=u(1)=0,\eqno(2)
$$
имеет решение $u(x)\in L_p(\varrho)$ c $u'(x)\in L_{q}(\sigma)$. Это решение единственно, если $b(x)\neq 0$ почти всюду на отрезке $[-1,1]$ или если в условии 2) нелинейность $F(x,u)$ строго возрастает по $u$.}


\textbf{Теорема~2.} {\it Пусть $p\geq 2$, $\varrho(x)=(1-x^2)^{-1/2}$, $b(x)\in C^1[-1,1]$, функция $f(x)\in L_p(\varrho)$  определена в точках  $\pm 1$   и  $f'(x)\in L_{q}(\sigma)$. Если  для почти всех $x\in [-1, 1]$ и всех $u\in \mathbb R$ выполняются условия 1)---3) теоремы~1, причём в условии 1) $a(\pm 1)=0$, а в условии 2) нелинейность $F(x,u)$ строго возрастает по $u$, то при любых значениях параметра $\lambda>0$ краевая задача
$$
u(x)+\lambda\cdot F\left(x, -\frac{b(x)}{\pi}\int\limits_{-1}^1 \frac{[b(s)\cdot u(s)]'\,ds}{s-x}\right)=f(x),\eqno(3)
$$
$$
u(-1)=f(-1), \quad u(1)=f(1),\eqno(4)
$$
имеет единственное решение $u(x)\in L_p(\varrho)$ c \ $u'(x)\in L_{q}(\sigma)$.}

Аналогичные теоремы справедливы и для других классов нелинейных сингулярных интегро"=дифференциальных уравнений с ядром Коши [2--4], в том числе и для уравнений типа Гаммерштейна, а также для соответствующих нелинейных уравнений с ядром Гильберта в вещественных пространствах Лебега $L_p(-\pi,\pi)$, состоящих из $2\pi$- периодических функций [5].

В заключение отметим, что следуя работе [6] методом максимальных монотонных операторов приведённые результаты можно обобщить на случай краевых задач для соответствующих систем уравнений нелинейных сингулярных интегро"=дифференциальных уравнений с ядрами Коши и Гильберта.

% Оформление списка литературы
\litlist

1. {\it Гаевский Х.} Нелинейные операторные уравнения и операторные дифференциальные уравнения~/ Х.~Гаевский, К.~Грегер, К.~Захариас~--- М.~: Мир, 1978.~--- 336~с.

2. {\it Асхабов С.Н.} Нелинейные сингулярные интегро"=дифференциальные уравнения с произвольным параметром~/ С.Н.~Асхабов~// Матем. заметки.~--- 2018.~--- Т.~103, №~1.~--- С.~20--26.

3. {\it Askhabov S.N.} Nonlinear singular integral equations in Lebesgue spaces~/ S.N.~Askhabov~// J. Math. Sciences.~--- 2011.~--- V.~173, №~2.~--- P.~155--171.

4. {\it Wolfersdorf L.V.} Monotonicity methods for nonlinear singular integral and integro-differential equations~/ L.v.~Wolfersdorf~// Zeitschrift
fur Angewandte Mathematik und Mechanik.~--- 1983.~--- V.~63, №~6.~--- P.~249--259.

5. {\it Асхабов С.Н.} Сингулярные интегро"=дифференциальные уравнения с ядром Гильберта и монотонной нелинейностью~/ С.Н.~Асхабов~// Владикавказский матем. журнал~--- 2017.~--- Т.~19, №~3.~--- С.~11--20.

6. {\it Асхабов С.Н.} Применение метода монотонных операторов к некоторым классам нелинейных сингулярных интегральных уравнений и их системам в $L_{p,n}(\rho)$~/ С.Н.~Асхабов~// Деп. в ВИНИТИ 12.02.1981, №~684-81.~--- С.~1--28.
