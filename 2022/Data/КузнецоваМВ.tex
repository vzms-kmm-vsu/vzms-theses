\vzmstitle{
Примеры моделирования вырожденных особенностей интегрируемых гамильтоновых систем биллиардными книжками
}
\vzmsauthor{Кузнецова}{А.\,А.}
\vzmsinfo{Москва, МГУ им. М.В. Ломоносова; {\it anastasiakuznecova0143@gmail.com}}

\vzmscaption


Известна интегрируемость биллиарда в плоской области, ограниченной дугами софокусных квадрик семейства $(b-\lambda)x^2+(a-\lambda)y^2=(b-\lambda)(a-\lambda)$ для $0<b<a$. В.В. Ведюшкина ввела новый важный класс интегрируемых биллиардных книжек, получающихся из плоских биллиардов их склейками вдоль граничных ребер, с указанием перестановок, диктующих правила перехода биллиардного шара с одного листа биллиарда на другой [1-2]. Такие столы-комплексы можно понимать и как плоские многослойные биллиарды. Слоения Лиувилля систем на них классифицируются инвариантами Фоменко-Цишанга, т.е. графами-молекулами с особенностями-атомами в вершинах и числовыми метками, задающими склейки граничных 2-торов этих особенностей друг с другом. Подробнее теория топологической классификации интегрируемых гамильтоновых систем, развитая А.Т.Фоменко и его соавторами, описана в [3].

В программной работе [4] А.Т.Фоменко сформулировал гипотезу о моделировании произвольных интегрируемых систем подходящими биллиардами. В.В. Ведюшкиной и И.С. Харчевой удалось доказать, что произвольные невырожденные особенности слоения (боттовские 3-атомы) и любая база слоения (граф с вершинами-атомами без меток) реализуются алгоритмически задаваемыми биллиардными книжками. Каждый боттовский 3-атом (типичная бифуркация регулярных торов Лиувилля) является произведением 2-базы на окружность, возможно, с факторизацией по $\mathbb{Z}_2$.

Оказывается, гипотеза Фоменко справедлива и для некоторых гамильтоновых систем, чей интеграл не является боттовским на уровне энергии $Q^3$, т.е. имеет вырожденные особенности. В нашей работе приведем примеры реализации биллиардными книжками бифуркаций слоений Лиувилля, 2-база которых содержит неморсовские мультиседла.

Рассмотрим элементарный плоский биллиард $A'_{0}$, ограниченный гиперболой, эллипсом, фокальной прямой и вертикальной осью $O_{Y}$.
Рассмотрим множество биллиардных книжек, склеенных ровно из трех копий $A'_{0}$. Левая и правая граница каждого листа свободна (т.е. перестановка тождественная), а на фокальном отрезке и на дуге эллипса рассмотрим все возможные варианты перестановок из трех элементов. Все книжки такого вида перечислены на рис. 1:
\vspace*{-0.4cm}
\begin{figure}[h!]
	\begin{center}
	\includegraphics[height=2.5cm]{Kuznetsova_table.jpg}\\
	Рис. 1
	\end{center}
\end{figure}
\vspace*{-1.1cm}\\

Столбцы таблицы соответствуют перестановкам на фокальном отрезке, а строки --- перестановкам на дуге эллипса. В ячейке указан седловой 3-атом, содержащийся в слоении Лиувилля полученной книжки: серые ячейки соответствуют слоениям несвязных биллиардов, остальные столы-комплексы связны. Точками в таблице отмечены столы, слоения Лиувилля биллиарда на которых содержат неботтовские 3-атомы, а серым цветом --- отсутствие 3-атомов, отличных от эллиптических атомов $A$.\\

 \vspace*{-0.6cm}
\paragraph{Теорема~1.}
{\it
	Рассмотрим биллиардные книжки, склеенные из трех листов $A_0'$. В прообразе малой окрестности значения интеграла $\lambda=b$ такие столы моделируют тривиальное слоение — для двух столов, боттовский 3-атом $B$ — для 12 столов, и неботтовские мультиседловые 3-атомы для остальных 12 столов (примеры таких пар стола и 3-атома приведены на рис. 2).
}
\vspace*{-0.5cm}
\begin{figure}[h!]
\begin{minipage}[h!]{0.49\linewidth}
\center{\includegraphics[width=0.8\linewidth]{pc1.jpg}}
\end{minipage}
\begin{minipage}[h!]{0.49\linewidth}
\center{\includegraphics[width=1\linewidth]{pc2.jpg}}
\end{minipage}
\begin{center}
Рис.2
\end{center}
\end{figure}\\

\vspace*{-0.8cm}
Исследование выполнено при поддержке гранта РНФ 21-11-00355 в МГУ имени М.В. Ломоносова.

\litlist


1. {\it Ведюшкина В. В.}
 Интегрируемые биллиарды на клеточных комплексах и интегрируемые гамильтоновы системы. Докт. дисс., МГУ, Москва, 2020.

2. {\it Фоменко А.Т., Ведюшкина В. В.}
 Биллиарды и интегрируемость в геометрии и физике. Новый взгляд и новые возможности, Вестн. Моск. ун-та. Сер.1, 2019, 3, 15–25

3.  {\it Болсинов А.В., Фоменко А.Т.}
 Интегрируемые гамильтоновы системы. Геометрия, топология, классификация, Т.1,2. Ижевск: НИЦ "Регулярная и хаотическая динамика", 1999.

4. {\it Ведюшкина В. В., Харчева И. С.}
 Биллиардные книжки реализуют все базы слоений Лиувилля интегрируемых гамильтоновых систем, Матем. сб., 212:8 (2021),  89-150

5. {\it Ведюшкина В. В., Фоменко А.Т., Харчева И. С.}
 Моделирование невырожденных бифуркаций замыканий решений интегрируемых систем с двумя степенями свободы интегрируемыми топологическими биллиардами, Докл. РАН, 479:6 (2018), 607-610

 6. {\it Ведюшкина В.В., Харчева И. С.}
 Биллиардные книжки моделируют все трехмерные бифуркации интегрируемых гамильтоновых систем, Матем. сб., 209:12 (2018), 17–56
\end{document}
