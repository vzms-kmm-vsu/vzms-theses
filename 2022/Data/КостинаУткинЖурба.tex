\vzmstitle[
    \footnote{
		Работа выполнена при финансовой поддержке Министерства науки и высшего образования РФ в рамках государственного задания в сфере науки (номер темы {FZGF}-2020-0009)
	}
]
{Оптимизация модели вибропогружателя с учётом неоднородности среды}

\vzmsauthor{Журба}{А.\,В.}
\vzmsinfo{Воронеж, ВГУ; {\it av.zhurba.93@gmail.com}}
\vzmsauthor{Костин}{Д.\,В.}
\vzmsinfo{Воронеж, ВГПУ, ВГУ; {\it dvk605@mail.ru}}
\vzmsauthor{Костина}{Т.\,И.}
\vzmsinfo{Воронеж; ВГПУ, ВГТУ {\it tata\_sti@rambler.ru}}
\vzmsauthor{Уткин}{А.\,А.}
\vzmsinfo{Воронеж, ВГУ; {\it artem132rus@gmail.com}}
\vzmscaption

% Аннотация на русском
\textit{Комплексные математическая модель функционирования импульсного погружателя состоят из моделей работы самого импульсного погружателя, модели взаимодействия свайного элемента с грунтом в виде силы трения боковой поверхности и лобового сопротивления,
которые имеют феноменологический характер. Сам процесс работы погружающего агрегата описан с использованием ряд Максвелла-Фейера и его оптимальность в смысле коэффициента асимметрии строго доказана.
В тоже время при использовании оптимальных соотношений при проектировании в обязательном порядке закладываются допуски, которые неизбежны при производстве элементов. Эти несовершенства нарушают форму оптимального импульса.\linebreak
Возникает задача исследования зависимости импульса от отклонений в параметрах и оценки допустимых значений этих отклонений. Для этого применялась программная реализация математической модели процесса функционирования импульсного погружателя,
которая легла в основу численного эксперимента. В статье приводятся характерные результаты эксперимента и их анализ.}


\subsection{Введение}

В работе [1] представлена задача оптимизации устройства погружения свай в грунт. Характерным отличием данного агрегата было наличие несколько пар дебалансов, которые позволяет создать положительный импульс действующий на свайный элемент.
Благодаря этому принципу устройство получило название импульсный погружатель. Математическая модель работы импульсного погружателя описана в работе [1], там же поставлена задача оптимизации импульса по коэффициенту асимметрии,
представляющего собой отношение максиального к минимальному значению вынуждающей силы на периоде. Данная задача оптимизации была решена и описана в работах [3].

Построенная математическая модель с оптимальными\linebreakпараметрами позволила начать исследования самого процесса погружения, где помимо погружателя в комплексную математическую модель вошли параметры характеризующие свайный элемент и грунт.
Феноменологическая математическая модель бокового и лобового сопротивления были выбраны из таблицы сопротивления грунта. На основе полученной модели был разработан программный комплекс,
позволяющий проводить численный эксперимент погружения свайных элементов при различных управляющих параметрах и параметрах грунта.
Были получены результаты, проведено сравнение их с результатами натурных экспериментов.

Результатом данной работы также является алгоритм программного обеспечения численного эксперимента, которое позволяет проводить оптимизацию конструкции под заданные типы грунтов.


\subsection{Математическая модель работы импульсного погружателя}

Математическая модель работы импульсного погружателя молотом, заключается в работе без ущерба близким строениям. Этот способ погружения основан на эффекте резкого снижения сопротивления погружению свайного элемента при сообщении последнему вибрации.
Такие агрегаты называются вибропогружателями и предназначены для погружения в песчаные и глинистые грунты различных свайных элементов.

\subsubsection{Вынуждающая сила вибропогружателя}

Конструктивная схема вибропогружателя изображена на\linebreakрисунке \ref{fig:scheme_porg}.
\begin{figure}[h]
    \centering
    \includegraphics[width=0.3\linewidth]{scheme_porg_2.png}
    \caption{Схема вибрационного погружателя}
    \label{fig:scheme_porg}
\end{figure}

При вращении валов (1) с дебалансами (2) на их ось крепления действует центробежная сила и вибрационный погружатель получает вибрирующее движение,
которое сообщается через наголовник (3) свайному элементу (4). Симметрично расположенные дисбалансы синхронно вращаются в разные стороны для уравновешивания радиальных нагрузок и компенсированная горизонтальных сил.

Математическая модель вынуждающей силы вибропогружателя может быть представлена в виде:
\begin{equation}\label{eq:centrifugal}
    \begin{gathered}
        F_{\textrm{вс}} = 2mn\omega^2 R\cos(\omega t),
    \end{gathered}
\end{equation}
\noindent где $F_{\textrm{вс}}$ "--- вынуждающая сила вибропогружателя, $n$ "--- количество пар дисбалансов, $m$ "--- масса дисбаланса, $\omega$ "---угловая скорость вращения дисбалансов (на рисунке из две),
$R$ "--- радиус смещения центра масс дисбаланса относительно оси вращения, $t$ "--- время.

Модифицированной конструкции вибропогружателя, \linebreakпредложенной в работе [1], поставила перед математиками задачу оптимизации.
Добавление в конструкцию вибропогружателя дисбалансов, вращающихся на удвоенных скоростях по отношению к основным валам, привело к отличию абсолютных
значений максимума и минимума вынуждающей силы, создаваемой погружателем.

\subsection{Вынуждающая сила импульсного погружателя}
Математическая модель формирования вынуждающей силы импульсным погружателем, который был предложен Ермоленко В.Н. и Насоновым И.В. в работе [1], отличается от классического вибропогружателя разными радиусами пар дисбалансов.
\begin{equation}\label{eq:short_harmonic_sum}
    \begin{gathered}
        F_{\textrm{вс}} = \sum\limits_{k = 0}^N m_k \omega_k^2 R_k \cos(\omega_k t),
    \end{gathered}
\end{equation}
\noindent где $m_k$ – масса $k$-ой пары дисбалансов, $\omega_k$ – угловая скорость вращения $k$-ой пары дисбалансов, $R_k$ – радиус дисбалансов.

Различие радиусов делает различным и угловую скорость вращения пар дисбалансов $\omega_k$. При этом выполняется соотношение
\begin{equation}\label{eq:angle_speed}
    \begin{gathered}
        \omega_k = k \omega_0, / k=1...N.
    \end{gathered}
\end{equation}
В импульсном погружателе кроме эффекта снижения трения с помощью вибрации появляется ключевое свойство, заключающееся в появлении асимметрии между полезной и вредной вынуждающей силой.
Полезной будем называть вынуждающую силу в тот момент времени, когда установка погружает сваю в грунт ($F_{\textrm{вс(t)}}>0$).
Отрицательной вынуждающую силу будем называть в тот момент времени, когда она направлена в противоположную погружению сторону ($F_{\textrm{вс(t)}}<0$).
В случае классического вибропогружателя полезная и отрицательная вынуждающие силы равны по амплитуде, что наглядно видно из графика функции (\ref{eq:centrifugal}) изображённого на рисунке \ref{fig:vibr_grap}.
Для импульсного погружателя эти амплитуды различны рисунок \ref{fig:impulse_grap}.
\begin{figure}[h]
    \centering
    \includegraphics[width=0.5\linewidth]{impulse_1.png}
    \caption{График вынуждающей силы, создаваемой вибрационным погружателем}
    \label{fig:vibr_grap}
\end{figure}
\begin{figure}[h]
    \centering
    \includegraphics[width=0.5\linewidth]{impulse_7.png}
    \caption{График вынуждающей силы, создаваемой импульсным погружателем}
    \label{fig:impulse_grap}
\end{figure}
Максимум полезной силы есть максимум функции $F_{\textrm{вс}}(t)$, минимальное значение функции $F_{\textrm{вс}}(t)$ есть наибольшее абсолютное значение отрицательной вынуждающей силы.
Абсолютное отношение максимального значения функции к минимальному значению называется коэффициентом асимметрии $K_n$.
\begin{equation}\label{eq:min_max}
    \begin{gathered}
        K_n = \left|\frac{\max \limits_{-\pi<t<\pi} F_{\textrm{вс}}(t)}{\min \limits_{-\pi<t<\pi} F_{\textrm{вс}}(t)}\right|,
    \end{gathered}
\end{equation}
\noindent где $n$ "--- число пар дисбалансов в импульсном погружателе, $t$ "--- время работы в течении одного периода функции $F_{\textrm{вс}}(t)$.

Этот эффект асимметрии позволил при меньшем весе установки, создавать больший положительный импульс. Таким образом, критерием
оптимизации стал параметр – коэффициент асимметрии, который равен отношению максимального к минимальному значению функции
вынуждающей силы. Проблема, как выбирать радиусы дисбалансов для получения наилучшего коэффициента асимметрии, была решена в работах [2],
[3], [5], где даётся определение оптимального импульса и доказывается теорема о выборе параметров оптимального импульсного погружателя.

При построении математической модели погружения\linebreakсваи, описываемой в разделе 4 настоящей работы, использовалась модель оптимального
импульсного погружателя, заданная с точностью до константы формулой:
\begin{equation}\label{eq:constant_accr}
    \begin{gathered}
        f_n(t,\lambda) = \sum\limits_{k = 1}^n (n-k+1) \cos(k \omega t), \ t \in [-\pi, \pi]
    \end{gathered}
\end{equation}
\noindent где $\omega$ "--- скорость вращения первой пары дебалансов. Функция (\ref{eq:constant_accr}) называется импульсом Максвелла–Фейера.
Период $t \in [-\pi, \pi]$ соответствует полному обороту наибольшего по радиусу дебаланса. Данный отрезок выбран для
удобства математического формализации задачи, на практике и численном эксперименте длинная периода зависит от скорости вращения $\omega$,
которая изменяет при управлении работой погружателя.

\subsubsection{Математическая модель погружения свай}

Дальнейшим развитием этой темы стало применение модели импульсного погружателя к математической модели процесса погружения
сваи или шпунта в грунт с учётом реологических свойств среды. Эта модель представляет собой дифференциальное уравнение второго порядка с
начальными условиями [4]. Она учитывает лобовое и боковое сопротивление грунта движению сваи, а также включает в себя параметр управления работой
импульсного погружателя – частота вращения дисбалансов.

В приложениях к строительной тематике, в частности к установкам свайного фундамента, для моделирования процессов погружения свай, используется уравнение (\ref{eq:model_eq}):
\begin{equation}\label{eq:model_eq}
    \begin{gathered}
        R = F_{\textrm{вс}} + F_{\textrm{тяж}} - F_{\textrm{бс}} - F_{\textrm{лс}},
    \end{gathered}
\end{equation}
\noindent где $R$ – равнодействующая сила, $F_{\textrm{вс}}$ – вдавливающая сила, создаваемая погружателем, $F_{\textrm{тяж}}$ – сила тяжести, $F_{\textrm{бс}}$ – сила бокового сопротивления, $F_{\textrm{лс}}$ – сила лобового сопротивления.

Решение данного уравнения позволяет определить время и глубину погружения в зависимости от типа погружающей установки, габаритов сваи и типа грунта.

Этап, когда погружающая установка тянет сваю вверх, преодолевая силу тяжести и сопротивление грунта по боковой поверхности в настоящей работе
рассматриваться не будет. Так как, в этом случае возможно разрушение сваи, поскольку бетонная свая хорошо переносит сильное сжатие, но разрушается
при попытке растяжения. На практике этого не допускают с помощью управления угловой скорости вращения дисбалансов, не позволяя устройству
работать на больших оборотах в начале погружения.

Если полезная сила погружающей установки и сила тяжести не смогут превысить сопротивление грунта, то погружение остановится. После этого
этапа происходит увеличение скорости оборотов валов дисбалансов, до тех пор, пока вынуждающей силы не станет достаточно, чтобы продолжить
погружение сваи.

В дальнейшем будем пользоваться уравнением (\ref{eq:model_eq}) для численного моделирование процесса.

Через $m$ обозначить массу всей установки, через $x(t)$ "--- глубину погружения сваи, а через $t$ "--- время погружения сваи. Тогда

$R=ma=m\ddot{x}$, где $a$ "--- ускорение,
$F_{\textrm{тяж}}=mg$, где $g$ "--- ускорение свободного падения,
$F_{\textrm{лс}}=S_{\textrm{пс}}h_i(x(t),\varepsilon)$, где $S_{\textrm{пс}}$ "--- площадь поперечного сечения сваи, $h_i(x(t),\varepsilon)$ "--- удельное лобовое сопротивление,
$\varepsilon$ "--- коэффициент условий работы грунта под нижним концом сваи,
$F_{\textrm{бс}}=Px(t)f_i(\psi)$ "--- сила бокового сопротивления, представляющая собой произведение периметра сваи $P$, глубины погружения $x(t)$
и удельной силы бокового сопротивления $f_i(\psi)$, зависящей от типа грунта.

Будем считать, что в момент времени $t=0$ глубина погружения равна $0$ и свая неподвижна. Исходя из этого получаем следующее дифференциальное уравнение второго порядка:
\begin{equation}\label{eq:diff_2}
    \begin{gathered}
        m\ddot{x} = F_{\textrm{вс}} + mg + S_{\textrm{пс}} \cdot h_i(x(t),\varepsilon) + Px(t)f_i(\psi)
    \end{gathered}
\end{equation}
с начальными условиями:
\begin{equation}\label{eq:diff_start}
    \begin{gathered}
        x(0) = \dot{x}(0) = 0.
    \end{gathered}
\end{equation}

Решение задачи (\ref{eq:diff_2})-(\ref{eq:diff_start}) позволяет определить время и глубину погружения в зависимости от характеристик погружающей установки, размеров и веса сваи, а также типа грунта.

Заменим $\ddot{x}$ в уравнении (\ref{eq:diff_2}) разностной аппроксимацией:
\begin{equation}\label{eq:rzn_aprok}
    \begin{gathered}
        m \frac{x_{i+1}\!-\!2x_i\!+\!x_{i-1}}{h}\! = \!F_{\textrm{вс}}\!+\!mg\!+\!S_{\textrm{пс}}h_i(x(t),\varepsilon)\!+\!Px(t)f_i(\psi), \\
        x_{i+1} - 2x_i + x_{i-1}\! = \!\frac{h^2}{m}(F_{\textrm{вс}}\!+\!mg\!+\!S_{\textrm{пс}}h_i(x(t),\varepsilon)\!+\!Px(t)f_i(\psi)), \\
        x_{i+1}\! = \!2x_i\!-\!x_{i-1}\!+\!\frac{h^2}{m}(F_{\textrm{вс}}\!+\!mg\!+\!S_{\textrm{пс}}h_i(x(t),\varepsilon)\!+\!Px(t)f_i(\psi)).
    \end{gathered}
\end{equation}
Полученное рекуррентное равенство позволяет численно\linebreakрассчитать текущее значение функции $x$, при условии $x_0 = x_1 = 0$.
Для программной реализации и численного расчёта глубины и времени погружения сваи был выбран языке программирования Python. Программа рассчитывает глубину погружения сваи в дискретные моменты времени, используя равенства (\ref{eq:rzn_aprok}).


\litlist

1. {\it В.Н.Ермоленко, В.А. Костин, Д.В. Костин, Ю.И. Сапронов}
Оптимизация полигармонического импульса / Вестн. Юж.-Урал. гос. ун-та. Сер.: Мат. моделирование и программирование. Челябинск 2012, 27 (286), вып. 13.~---С.~35-44.

2. {\it В.А. Костин, Д.В. Костин, Ю.И. Сапронов}
Многочлены Максвелла-Фейера и оптимизация\linebreakполигармонических импульсов / ДАН. 2012. -- Т.~445, 3.~---С.~271-273.

3. {\it Костин, Д.В.}
Бифуркации резонансных колебаний и оптимизация тригонометрического импульса по коэффициенту несимметрии / Математический сборник. 2016. Т. 207, 12.~---С.~90-109.

4. {\it Костина Т. И., Журба А. В., Мызников А. С., Бабошин С.Д.}
Программная реализация математической модели работы импульсного погружателя / Современные методы теории функций и смежные проблемы. 2021 С.~168-169.
