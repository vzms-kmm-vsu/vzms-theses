\vzmstitle[
	\footnote{Работа выполнена при поддержке МОН Украины (проект {0119U100337}) и НАН Украины (проект {0119U101608}).}
]{
Новые алгоритмы экстраградиентного типа для вариационных неравенств
}
\vzmsauthor{Семёнов}{В.\,В.}
\vzmsinfo{Киев, КНУТШ; {\it semenov.volodya@gmail.com}}
%\vzmsauthor{Петров}{П.\,П.}
%\vzmsinfo{Урюпинск, УрГУ; {\it tractorist\_petr@example.org}}

\vzmscaption

Множество актуальных  прикладных проблем можно записать в форме вариационных неравенств. Особенно популярны эти постановки в оптимизации, теории игр и теории управления. Отметим, что часто негладкие задачи оптимизации могут эффективно решаться, если
их переформулировать в виде седловых задач, а к последним применить алгоритмы решения вариационных неравенств. С появлением
генерирующих состязательных нейронных сетей
\foreignlanguage{english}{(generative adversarial networks, GANs)}
устойчивый интерес к алгоритмам решения вариационных неравенств возник и в среде специалистов в области машинного обучения.

В докладе будет сделан обзор результатов работ [1--6], в которых предложены новые методы решения вариационных неравенств
$$
\mbox{найти} \  x \in C :  \ \left\langle Ax, y-x \right\rangle \geq 0\quad \forall y \in C,
$$
где $C$ "--- замкнутое выпуклое подмножество равномерно выпуклого и равномерно гладкого банахова пространства $E$, $A: E \to E^\ast$ "--- нелинейный монотонный или псевдомонотонный оператор.

Основное внимание будет уделено доказательству сходимости и
формулировке новых вопросов. В частности, будут представлены новые результаты о характере сходимости для следующих алгоритмов.

	\textbf{Алгоритм 1 (Экстраполяция из прошлого).} {\it Начиная с элементов $y_0$, $x_1 \in E$ строим последовательность $(x_n)$ при помощи итерационной схемы
$$
\begin{array}{l}
y_{n}  = \Pi _{C} J^{-1} \left( Jx_{n} -\lambda_n Ay_{n-1} \right), \\
x_{n+1} = \Pi _{C} J^{-1} \left( Jx_{n} -\lambda_n Ay_{n} \right),
\end{array}
$$
где $\lambda_n >0$.}


\textbf{Алгоритм 2 (Операторная экстраполяция).} {\it Начиная с элементов $x_0$, $x_1 \in E$ строим последовательность $(x_n)$ при помощи итерационной схемы
$$
x_{n+1} =\Pi _{C} J^{-1} \left(Jx_{n} -\lambda _{n} Ax_{n} -\lambda _{n-1} \left(Ax_{n} -Ax_{n-1} \right)\right) ,
$$
где $\lambda_n >0$.}


Здесь $\Pi_{C} x$ "--- оператор обобщённого проектирования Альбера
$$
\Pi_{C} x = {\rm argmin}_{y \in C} \left(  \left\| y\right\| ^{2} -2\left\langle Jx, y\right\rangle +\left\| x\right\| ^{2} \right),
$$
$J: E \to E^\ast$ "--- нормализованное дуальное отображение.

Алгоритм 1 является модификацией известного алгоритма Попова для задач в банаховых пространствах с использованием обобщённой проекции Альбера вместо метрической. Алгоритм 2 "--- модификация нового \foreignlanguage{english}{<<forward-reflected-backward algorithm>>} [2] для вариационных неравенств в банаховых пространствах.

Привлекательными свойствами  второго алгоритма является вычисление на итерационном шаге одного значения оператора $A$ и одной проекции на допустимое множество

Для вариационных неравенств с монотонными и ли\-пши\-цевыми операторами, действующими в  $2$-равномерно выпуклом и равномерно гладком банаховом пространстве, доказаны теоремы слабой сходимости и $O(\frac{1}{\varepsilon})$-оценки сложности в терминах функции зазора.

% Оформление списка литературы

\litlist

1. {\it Vedel Y., Semenov V.} Adaptive extraproximal algorithm for the equilibrium problem in Hadamard spaces // Optimi\-za\-tion and Applications. OPTIMA 2020. Lecture Notes in Compu\-ter Science, vol 12422. Cham: Springer, 2020. -- P. 287-300.

2. {\it Malitsky Y., Tam M.K.} A forward-backward spli\-tting method for monotone inclusions without cocoercivity // SIAM Journal on Optimization. -- 2020. -- Vol.~30. -- No.~2. -- P.~1451-1472.

\selectlanguage{russian}

3. {\it Семенов В.В.,  Денисов С.В.,  Сирык Д.С.,  Харьков О.С.} Сходимость метода экстраполяции из прошлого и метода операторной экстраполяции // Проблемы управления и информатики. -- 2021. -- № 3. -- С. 57-71.

4. {\it Семенов В.В.,  Денисов С.В.} Адаптивный метод операторной экстраполяции для вариационных неравенств в банаховых пространствах // Проблемы управления и информатики. -- 2021. -- № 5. -- С. 82-92.


\selectlanguage{english}

5. {\it Vedel  Y.I., Denisov  S.V., Semenov V.V.} An Adaptive Algorithm for the Variational Inequality Over the Set of So\-lu\-tions of the Equilibrium Problem // Cybernetics and Systems Analysis. -- 2021. -- Vol.~57. -- Issue~1. -- P.~91-100.

6. {\it	Denisov S.V., Semenov V.V., Stetsyuk P.I.} Bregman ex\-tra\-gradient me\-thod with monotone rule of step adjustment // Cybernetics and Systems Analysis.  -- 2019. -- Vol. 55. -- Issue 3. -- P. 377-383.
