\documentclass{vzmsthesis}

\begin{document}

\vzmstitle[
%	\footnote{Работа выполнена за счёт гранта РНФ, проект 16-11-10125}
]{	О существовании и единственности положительного решения краевой задачи для одного нелинейного функционально - дифференциального уравнения дробного порядка
}
\vzmsauthor{Абдурагимов}{Г.\,Э.}
\vzmsinfo{Махачкала, ДГУ; {\it gusen\_e@mail.ru}}
%\vzmsauthor{Семенов}{Е.\,М.}
%\vzmsinfo{Воронеж, ВГУ; {\it nadezhka\_ssm@geophys.vsu.ru}}

\vzmscaption

Обозначим через $C$~--- пространство $C[0,1]$, через $\mathbb{L}_p \  (1<p<\infty)$~--- пространство $\mathbb{L}_p(0,1)$ и через $\tilde{K}$~--- конус неотрицательных функций пространства $C$.

Рассмотрим краевую задачу
\begin{align}
&D_{0+}^\alpha x(t)+f \left (t,\left(Tx \right)(t) \right)=0,~~~~ 0<t<1,\label{1}\\
&x(0)=0,\   x(1)=0,\label{2}
\end{align}
где  $\alpha\in (1,2]$~--- действительное число, $D_{0+}^\alpha$~--- дробная производная Римана-Лиувилля [1, c.469], $T\colon C \to \mathbb{L}_p \  (1<p<\infty)$~--- линейный положительный непрерывный оператор, функция $f(t,u)$ неотрицательна, не убывает по второму аргументу, удовлетворяет условию Каратеодори и $f(\cdot, 0)\equiv0$

Предположим, что функция $f(t,u)$ удовлетворяет условию
\begin{align*}
f(t,u)\leq a(t)+bu^{p/q},
\end{align*}
где $b>0,~~a(t)\in \mathbb{L}_q, ~~1<q<\infty$.

\paragraph{Теорема~1.}
{\it
Предположим, что $T\colon C \to \mathbb{L}_p \  $~--- положительный на конусе $\tilde{K}$ оператор, существуют неотрицательные функции $\upsilon$ и $\omega$ такие, что $\upsilon\leq \omega$ и почти всюду на $[0,1]$ выполнены  условия
      \begin{itemize}
         \item[$(1)$] $-D_{0+}^\alpha \upsilon(t)\leq f\left (t,\left(T\upsilon \right)(t) \right)$;
         \item[$(2)$] $-D_{0+}^\alpha \omega(t)\geq f\left (t,\left(T\omega \right)(t) \right)$;
         \item[$(3)$] $\upsilon(0) \leq 0 \leq \omega(0)$;
         \item[$(4)$] $\upsilon(1) \leq 0 \leq \omega(1)$.
      \end{itemize}
      Тогда краевая задача~{\rm \eqref{1}}--{\rm \eqref{2}} имеет по крайней мере одно положительное решение на конусном отрезке $\langle \upsilon, \omega \rangle$.}


\paragraph{Теорема~2.}
{\it
Предположим, что при $u>0$ и любом $\tau\in (0,1)$
\begin{gather*}
f(t,\tau u)>\tau f(t,u),~~~t\in(0,1).
\end{gather*}
Тогда при выполнении условий теоремы~1 краевая задача {\rm \eqref{1}}--{\rm \eqref{2}} имеет единственное положительное решение на конусном отрезке $\langle \upsilon, \omega \rangle$.}


\litlist

1. {\it Zhanbing B., Haishen L.}
 Positive solutions for boundary value problem on nonlinear fractional differential equation // J. Math. Anal. Appl.  – 2005. – Т. 311. – №. 2. – С. 495-505.

\end{document}
