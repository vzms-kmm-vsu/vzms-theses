\vzmstitle[
	\footnote{Работа выполнена при финансовой поддержке РНФ (проект 20--11--20085)}
]{Оценка точности метода Тихонова в применении к некорректным экстремальным задачам}
\vzmsauthor{Кокурин}{М.\,М.}
\vzmsinfo{Йошкар--Ола, МарГУ; {\it kokurin@nextmail.ru}}

\vzmscaption

Пусть $J:\,H\to\mathbb{R}$ "--- нелинейный функционал на вещественном гильбертовом пространстве $H$. Требуется решить задачу оптимизации
$$
J(x)\to\min\limits_{x\in H}, \eqno{(1)}
$$
т.е. найти точку $x^*\in H$, доставляющую глобальный минимум функционалу $J$. Существование решения $x^*$ ниже предполагается. Задача оптимизации (1) в общем случае является некорректной [1]. Потребуем, чтобы функционал $J$ был дважды непрерывно дифференцируем по Фреше и его вторая производная Фреше удовлетворяла условию Липшица
$$
\|J^{\prime\prime}(x)-J^{\prime\prime}(y)\|_{L(H)}\leq\Lambda\|x-y\|_H,\quad x,y\in H \eqno{(2)}
$$
с некоторой константой $\Lambda>0$. Пусть также выполнено условие истокопредставимости
$$
\exists w\in H:\,x^*-\xi=J^{\prime\prime}(x^*)w \eqno{(3)}
$$
с некоторым известным элементом $\xi\in H$. Наконец, потребуем, чтобы
$$
\Lambda\|w\|_H<1. \eqno{(4)}
$$

Пусть вместо точного функционала $J$ известно его приближение $J_\delta:\,H\to\mathbb{R}$, такое что
$$
|J_\delta(x)-J(x)|\leq\delta(1+\|x\|_H^2),\quad x\in H \eqno{(5)}
$$
с известным уровнем погрешности $\delta>0$. Рассмотрим функционал
$$
T_{\alpha,\delta}(x)=J_\delta(x)+\alpha\|x-\xi\|_H^2.
$$
Метод Тихонова в применении к задаче (1) заключается в нахождении точки $x_{\alpha,\varepsilon,\delta}\in H$, такой что
$$
\inf\limits_{x\in H}T_{\alpha,\delta}(x)\leq T_{\alpha,\delta}(x_{\alpha,\varepsilon,\delta})\leq \inf\limits_{x\in H}T_{\alpha,\delta}(x)+\varepsilon \eqno{(6)}
$$
с параметрами $\alpha,\,\varepsilon>0$, выбранными подходящим образом в зависимости от $\delta$. Такая точка $x_{\alpha,\varepsilon,\delta}$ всегда существует.

\paragraph{Теорема~1.} {\it Пусть выполнены условия (2)--(5) и параметры $\alpha$, $\varepsilon$ в (6) выбираются по правилу $\alpha(\delta)=C_1\delta^{1/3}$, $\varepsilon(\delta)=C_2\delta$ с некоторыми $C_1,\,C_2>0$. Тогда для точности приближений $x_{\alpha(\delta),\varepsilon(\delta),\delta}$, доставляемых методом Тихонова в применении к задаче (1), справедлива оценка
$$
\|x_{\alpha(\delta),\varepsilon(\delta),\delta}-x^*\|_H\leq C_3\delta^{1/3},\quad\delta\in(0,\delta_0] \eqno{(7)}
$$
с некоторыми константами $C_3,\,\delta_0>0$, вообще говоря, зависящими от $\xi$ и самого функционала $J$.}

Данная теорема усиливает полученную в [2] оценку точности $O(\delta^{1/4})$ для метода Тихонова в применении к некорректным задачам оптимизации. В то же время, оценка (7) хуже оценки точности $O(\delta^{1/2})$, справедливой для данного метода при дополнительном условии выпуклости функционала $J$, не используемом в теореме 1.

% Оформление списка литературы

\litlist

1. {\it Васильев Ф. П.} Методы решения экстремальных задач //М.: Наука, 1981. 400 с.

2. {\it Кокурин М. Ю.} Оценки скорости сходимости в схеме Тихонова для решения некорректных невыпуклых экстремальных задач //Журнал вычислительной математики и математической физики. – 2017. – Т. 57. – №. 7. – С. 1103-1112.
