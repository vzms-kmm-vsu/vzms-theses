\documentclass{vzmsthesis}

\begin{document}

\vzmstitle[
	\footnote{}
]{
	Об одном классе операторов в пространстве Крейна 
}
\vzmsauthor{Сухочева}{Л.\,И.}
\vzmsinfo{Воронеж, ВГУ; {\it l.suhocheva@ yandex.ru}}


\vzmscaption
\sloppy
 Пусть $\mathfrak{H}$ -- пространство Крейна с индефинитной метрикой $[x, y]$ \,\,[1],  $U$-$[ \,, \,]$-унитарный оператор  в $\mathfrak{H}( U:\mathfrak{H}\rightarrow\mathfrak{H},\,\,[Ux, Ux] = [x, x],\,\,x\in \mathfrak{H})$, обладающий свойством: 
 \begin{equation}
Re[\xi(I - U)x, x]\geq 0,\,\,\text{для некоторого}\,\, \xi\in \mathbb{C},\,\,\xi\neq\bar{\xi}
\end{equation}          
        
				Рассматриваются спектральные свойства такого оператора. Оказывается, что спектр  $U$ лежит на единичной окружности. Только собственные векторы, отвечающие собственным значениям $\lambda = 1$,\,$\lambda=\xi/\bar{\xi}$ могут иметь присоединенные и длина соответствующих жордановых цепочек не больше 2. При $\lambda\neq 1, \lambda\neq\xi/\bar{\xi}$
 $$
\mathfrak{L}_{\lambda}(U) = Ker (U-\lambda I),$$$$ \,\, \text{где}\,\,\mathfrak{L}_{\lambda}(U) - \text{корневое подпространство оператора}\,\,U.
$$
 Получено необходимое и достаточное условие полноты системы корневых векторов оператора $U$: при условии, что спектр $U$ имеет не более счетного множества точек сгущения, система корневых векторов оператора $U$  полна в $\mathfrak{H}$ тогда и только тогда, когда ядра операторов $(U - I)^2$,\,$(U - \xi/\bar{\xi})^2$  невырождены. При этом невырожденность указанных ядер является следствием условия:
$$
Ker(U - \lambda I)\cap\bar{\it{R}}(U - \lambda I)\in \it{R}(U - \lambda I)\,\,\,\,\lambda = 1,\,\lambda = \xi/\bar{\xi}
$$             
    
		Известно, что обратное преобразование Кэли-Неймана
$$
K^{-1}_{\lambda}(U) = (\lambda U - \bar{\lambda}I)(U - I)^{-1}  = A\,\,\,\text{при}\,\, \lambda\neq \bar{\lambda}, \, 1\notin\sigma_p(U)
$$        
				
$[ \,, \,]$-унитарному оператору $U$ однозначным образом ставит в соответствие самосопряженный оператор $A$, действующий в пространстве Крейна.  Оказывается, что самосопряженный оператор $A = K^{-1}_{\lambda}(U)$  в пространстве Крейна $\mathfrak{H}$ будет $[\, , \,]$ -неотрицательным тогда и только тогда, когда унитарный оператор  $U$ удовлетворяет условию (1). В этом случае, как следствие, можно получить критерий  полноты  системы корневых векторов  $[ , ]$-неотрицательного  $[\, ,\, ]$-самосопряженного оператора в терминах невырожденности  $Ker A^2$ в
пространстве Крейна.



\litlist

1. {\it Азизов Т.Я., Иохвидов И.С.}
 Основы теории линейных операторов в пространствах с индефинитной метрикой. М.: Наука, 1986. – 352 с.

\end{document}
