\vzmstitle[
]{ОБ ИСПОЛЬЗОВАНИИ МЕТОДА ПАРАМЕТРИЧЕСКИХ ОБОБЩЕННЫХ СТЕПЕНЕЙ ДЛЯ ПОСТРОЕНИЯ РЕШЕНИЙ ОДНОГО КЛАССА ДИФФЕРЕНЦИАЛЬНЫХ УРАВНЕНИЙ}
\vzmsauthor{Гладышев}{Ю.\,А.}
\vzmsinfo{Калуга, КГУ}
\vzmsauthor{Лошкарева}{Е.\,А.}
\vzmsinfo{Калуга, КГУ; {\it losh-elena@yandex.ru}}

\vzmscaption

Настоящее сообщение является непосредственным продолжением работы [1]. Большой класс линейных дифференциальных уравнений в частных производных имеет или может быть приведён к виду

$${D_1}u + {D_2}u = 0,\eqno(1)$$
где ${D_1},{D_2}$ произведение операторов

$$\begin{array}{l}
{D_1} = {D_{1\alpha }}...{D_{1i}}...{D_{11}},i = \overline {1,\alpha } ,\\
{D_2} = {D_{2\beta }}...{D_{2j}}...{D_{21}},j = \overline {1,\beta } ,
\end{array}$$
и операторы ${D_{1i}},{D_{2j}}$ определены как

$${D_{1i}} = {a_{1i}}\left( {{x_1}} \right)\frac{\partial }{{\partial {x_1}}},{D_{2j}} = {a_{2j}}\left( {{x_2}} \right)\frac{\partial }{{\partial {x_2}}}.\eqno(2)$$
Здесь ${a_{1i}}\left( {{x_1}} \right),{a_{2j}}\left( {{x_2}} \right)$ непрерывные функции
соответствующих независимых переменных ${x_1},{x_2}$ в заданной области.

Кроме сравнительно простого случая

$$\frac{{{\partial ^\alpha }u}}{{\partial {x_1}^\alpha }} + \frac{{{\partial ^\beta }u}}{{\partial {x_2}^\beta }} = 0,\eqno(3)$$
рассмотренного ранее в [1] в изучаемый класс входят параболические, эллиптические и гиперболические уравнения при % MathType!MTEF!2!1!+-
$\left( {\alpha = 2,\beta = 1,\alpha = \beta = 2} \right),\left( {\alpha = 4,2} \right)$ имеющие практическое применение в теории переноса, теории упругости и механике жидкости и газа.

Применение метода параметрических обобщённых степеней (ПОС) требует выполнение ряда
условий. Все операторы ${D_{1i}},{D_{2j}}$ должны иметь не пустые ядра

$${D_{1i}}{C_{1i}} = 0,{D_{2j}}{C_{2j}} = 0.\eqno(4)$$
Из вида операторов (2) следует, что ${C_{1i}},{C_{2j}}$ константы.

Операторы первого набора должны коммутировать с любым оператором второго

$${D_{1i}}{D_{2j}} = {D_{2j}}{D_{1i}}.\eqno(5)$$
Кроме этого требуем, чтобы все операторы ${D_{1i}},{D_{2j}}$ имели правые обратные ${I_{1i}},{I_{2j}}$

$${D_{1i}}{I_{1i}} = 1,{D_{2j}}{I_{2j}} = 0.\eqno(6)$$

Очевидно, что эти требования выполнены.

Как следствие (2) правые обратные для ${D_1},{D_2}$ определены как
${I_{1i}},{I_{2j}}$
$${I_1} = {I_{11}}...{I_{1\alpha }},{I_2} = {I_{21}}...{I_{2\beta }}.\eqno(7)$$

В отличии от матричного варианта ОС, где требуется построение матричных операторов и доказательства их коммутации, в параметрическом варианте ОС структура решения значительно проще. Основным является нахождение обобщённой константы (ОК) согласно общей теории [1]
${I_{1i}},{I_{2j}}$
$$C = \left( {\sum\limits_{i = 0}^{\alpha - 1} {{I_{10}}...{I_{1i}}{C_{1i + 1}}} } \right)\left( {\sum\limits_{j = 0}^{\beta - 1} {{I_{20}}...{I_{2j}}{C_{2j}}} } \right),{I_{10}} = {I_{20}} = 1.\eqno(8)$$

Все ${C_{1i}},{C_{2j}}$ определены в (4). Для ПОС в общем виде запишем ${I_{1i}},{I_{2j}}$
$$X_1^{\left( p \right)}X_2^{\left( q \right)}C = p!q!I_1^pI_2^qC,\eqno(9)$$
где ${I_1},{I_2},C$ определены в (7), (8). Эти, так называемые бинарные ОС, по построению удовлетворяют соотношениям
$${D_1}X_1^{\left( p \right)}X_2^{\left( q \right)}C = pX_1^{\left( {p - 1} \right)}X_2^{\left( q \right)}C,{D_2}X_1^{\left( p \right)}X_2^{\left( q \right)}C = qX_1^{\left( p \right)}X_2^{\left( {q - 1} \right)}C.\eqno(10)$$

Частное решение (1) запишем
$${u_n}\left( {{x_1},{x_2}} \right) = {\left( {{X_1} - {X_2}} \right)^n}C = \sum\limits_{i = 0}^n {{{\left( { - 1} \right)}^i}C_n^iX_1^{n - i}X_2^i} C.\eqno(11)$$

В этом можно убедиться прямой подстановкой правой части (11) в (1), имея в виду, что запись ${\left( {{X_1} - {X_2}} \right)^n}C$ имеет чисто символическое значение.

Используя линейность уравнения (1), можно утверждать, что многочленное решение (1) можно записать
$$u\left( {{x_1},{x_2}} \right) = \sum\limits_{i = 0}^n {{{\bar Z}^i}} {C_i},\eqno(12)$$
где введено обозначение
$${\bar Z^n}{C_n} = {\left( {{X_1} - {X_2}} \right)^n}{C_n}.\eqno(13)$$

При определённой метрике можно говорить о рядах (12).

Положительной стороной метода ПОС является возможность его обобщения на любое число независимых переменных. Если рассмотреть уравнение типа (1) в случае трёх независимых переменных, запишем
$${D_1}u + {D_2}u + {D_3}u = 0,\eqno(14)$$
где ${D_1},{D_2}$ определены по (2), а
$${D_3} = {D_{3\gamma }}...{D_{3k}}...{D_{31}}\eqno(15)$$
при
$${D_{3k}} = {a_{3k}}\left( {{x_3}} \right)\frac{\partial }{{\partial {x_3}}},k = \overline {l,\gamma } .\eqno(16)$$

Решение (14) типа ОК содержит три сомножителя
$$
	C \!=\!
	\left( {\sum\limits_{i = 0}^{\alpha - 1} {{I_{10}}...{I_{1i}}{C_{1\left( {i + 1} \right)}}} } \!\!\right)\!\!
	\left( {\sum\limits_{j = 0}^{\beta - 1} {{I_{20}}...{I_{2j}}{C_{2j}}} } \!\!\right)\!\!
	\left( {\sum\limits_{k = 0}^{\gamma - 1} {{I_{30}}...{I_{3k}}{C_{3\left( {k + 1} \right)}}} } \!\!\right)\!\!.
	\eqno(17)
$$

Очевидно, что ПОС в этом случае имеет вид
$$X_1^{\left( p \right)}X_2^{\left( q \right)}X_3^{\left( r \right)}C = p!q!r!I_1^pI_2^qI_3^rC.\eqno(18)$$

В символической форме частное решение (14) запишем
$${u_n} = {\left( {{X_1} + {X_2} - 2{X_3}} \right)^n}C = \sum\limits_{i = 0}^n {{{\left( { - 2} \right)}^i}C_n^i{{\left( {{X_1} + {X_2}} \right)}^{n - i}}X_2^i} C.\eqno(19)$$

Если ввести операторы
$$\begin{array}{l}
{D_z}\left( 3 \right) = \frac{1}{4}\left( {{D_1} + {D_2} + {D_3}} \right) = \frac{1}{4}{\Delta _e}\left( 3 \right),\\
{{\bar D}_z}\left( 3 \right) = \frac{1}{4}\left( {{D_1} + {D_2} - {D_3}} \right) = \frac{1}{4}{\Delta _g}\left( 3 \right)
\end{array}\eqno(20)$$
и обозначение
$${\bar Z^n}\left( 3 \right)C = {\left( {{X_1} + {X_2} - 2{X_3}} \right)^n}C,\eqno(21)$$
то непосредственно можно проверить правила
$${D_z}{\bar Z^n}C = 0,{\bar D_z}{\bar Z^n}C = n{\bar Z^{n - 1}}C.\eqno(22)$$

Можно говорить о многочленах и рядах от ${x_1},{x_2},{x_3}$, представленных в символической форме.

В сообщении показано, что метод параметрических ОС позволяет построить решение большого класса дифференциальных уравнений высокого порядка с нелинейными коэффициентами и может быть распространён на уравнения с любым числом переменных. Приведён случай трёхмерного пространства.


%%%% ОФОРМЛЕНИЕ СПИСКА ЛИТЕРАТУРЫ %%%
\litlist

1. {\it Гладышев Ю.А.} Метод обобщённых степеней Берса и его приложения в математической физике~/
Ю.А.~Гладышев.~--- Калуга.~: КГПУ, 2011.~--- 201~с.
