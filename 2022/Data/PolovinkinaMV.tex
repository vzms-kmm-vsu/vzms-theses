\vzmstitle[
    \footnote{}
]{
    К вопросу об устойчивости стационарного решения в одной модели
    распространения эпидемии ВИЧ с учетом миграций
}
\vzmsauthor{Шашкин}{А.\,И.}
\vzmsinfo{Воронеж, ВГУ; {\it dean@amm.vsu.ru}}
\vzmsauthor{Половинкина}{М.\,В.}
\vzmsinfo{Воронеж, ВГУИТ; {\it polovinkina-marina@yandex.ru}}
\vzmsauthor{Половинкин}{И.\,П.}
\vzmsinfo{Воронеж, ВГУ, БелГУ; {\it polovinkin@yandex.ru}}

\vzmscaption

Пусть $\Omega\subset\mathbb{R}^n$ --- ограниченная область с
кусочно гладкой границей. Рассматривается начально-краевая задача
$$
\frac{\partial S}{\partial t}=\mu \rho -
\left(\beta_1-\frac{\beta_2 I}{m+I}\right)SI-c\beta bJS-\mu
S+\vartheta_1\Delta S,\eqno(1)
$$
$$
\frac{\partial I}{\partial t}= \left(\beta_1-\frac{\beta_2
I}{m+I}\right)SI+c\beta bJS-(\mu+k_1)I+\delta J+\vartheta_2\Delta
I,\eqno(2)
$$
$$
\frac{\partial J}{\partial t}=k_1
I-(\mu+k_2+\delta)J+\vartheta_3\Delta J,\ x=(x_1,x_2)\in\Omega,\
t>0,\eqno(3)
$$
$$
\left.\left(\mu_1 S+\eta_1\frac{\partial S}{\partial
\overrightarrow{\nu}}\right)\right|_{\partial\Omega}=B_1,\
%$$
%$$
\left.\left(\mu_2 I+\eta_2\frac{\partial
I}{\partial\overrightarrow{\nu}}\right)\right|_{\partial\Omega}=B_2,\
$$
$$
\left.\left(\mu_3 J+\eta_3\frac{\partial
J}{\partial\overrightarrow{\nu}}\right)\right|_{\partial\Omega}=B_3,
\ t\geqslant 0;\eqno(4)
$$
$$
S\mid_{t=0}=S_0,\  I\mid_{t=0}=I_0,\ J\mid_{t=0}=J_0,\
x\in\overline{\Omega}.\eqno(5)
$$
Здесь искомыми функциями являются $S=S(x_1,x_2,t)$ --- численность
группы индивидуумов, восприимчивых к инфекции, $I=I(x_1,x_2,t)$
--- численность группы бессимптомных носителей инфекции,
$J=J(x_1,x_2,t)$ --- численность симптоматической группы.
Параметры, входящие в систему (1)-(5), имеют следующий смысл: $k$
--- общая численность населения;   $\mu$ --- уровень смертности
населения; $c$ --- уровень контактности;   $\beta$ --- вероятность
передачи заболевания при каждом контакте с инфицированным в
бессимптомной стадии;  $b$ --- вероятность передачи заболевания
при каждом контакте с инфицированным в симптоматической стадии;
$k_1$ --- скорость перехода от бессимптомной стадии к
симптоматической; $k_2$ --- скорость перехода от симптоматической
стадии к полномасштабному СПИДу; $\delta$ --- скорость лечения от
симптоматической до бессимптомной стадии; $d$ --- уровень
смертности от СПИДа; $m > 0$ --- константа полунасыщения, которая
отражает влияние освещения в средствах массовой информации на
передачу контактов;  $\beta$ ---  частота контактов до оповещения
в средствах массовой информации; $\beta_1\geqslant\beta_2>0$;
отношение $\beta_2I/(m+I)$ отражает эффект снижения частоты
контактов, когда инфицированные предупреждаются через средства
массовой информации. Функция $I/(m+I)$ представляет собой
непрерывную ограниченную функцию, которая учитывает насыщенность
болезнями или психологические эффекты [1]. Мы включаем эффект СМИ
только на этапе $I$,  потому что он относится к инфицированным
людям без симптомов: роль средств массовой информации здесь
заключается в том, чтобы предупредить этих людей о возможности
заражения и повысить их осведомленность. Модель (1)--(3) является
результатом нескольких этапов модификации классической модели
Кермака --- Маккендрика [2] распространения эпидемии [3--6]. Наш
вклад в эту модификацию состоит в добавлении последних
"диффузионных"\ слагаемых, получаемых действием оператора
$\vartheta_\kappa\Delta$ на искомые функции $S, I, J$,
$\kappa=1,2,3$, учитывающих миграционные процессы, которые мы
полагаем подчиняющимися закону Фурье. Здесь
$\Delta=\partial^2/\partial x_1^2+\partial^2/\partial x_2^2$ ---
оператор Лаплласа в $\mathbb{R}^2$. В краевых условиях (4)
полагается, что $\mu_\kappa^2+\eta_\kappa^2>0$, $\mu_\kappa
\eta_\kappa\geqslant 0$, $\kappa=1,2,3$.

Пусть  $w(x)=(w_1(x_1,x_2),w_2(x_1,x_2),w_3(x_1,x_2))$ ---
регулярное стационарное решение системы (1)--(3), удовлетворяющее
краевым условиям (4). Пусть  $H$ --- диаметр области $\Omega$.
Далее положим
$$
A_{11}=- c\beta\left(\beta_1-\frac{\beta_2 }{m+I}\right)I-c\beta b
J-\mu - \frac{\vartheta_1}{H^2},
$$
$$
A_{22}=c\beta\beta_1S-\frac{c\beta\beta_2mS}{(m+I)^2}-\mu-k_1-\frac{\vartheta_2}{H^2},
$$
$$
A_{33}=-\mu-k_2-\delta-\frac{\vartheta_3}{H^2},
$$
$$
A_{12}=A_{21}=\frac{1}{2}\left(-c\beta\beta_1S+\frac{c\beta\beta_2mS}{(m+I)^2}+c\beta\beta_1I-
\frac{c\beta\beta_2I}{m+I}+c\beta b J\right),
$$
$$
A_{13}=A_{31}=-\frac{1}{2}c\beta b S,\
A_{23}=A_{32}=\frac{1}{2}(c\beta b S+\delta+k_1).
$$
Доказано, что если при $S=w_1$, $I=w_2$, $J=w_3$ квадратичная
форма
$$
Q(w_1,w_2,w_3;z_1,z_2,z_3)=\sum\limits_{\kappa=1}^3
\sum\limits_{\iota=1}^3\, A_{\kappa\iota}z_\kappa z_\iota \eqno(7)
$$
отрицательно определена, то стационарное решение
$w=(w_1(x_1,x_2),w_2(x_1,x_2),w_3(x_1,x_2))$  системы (1)--(3),
удовлетворяющее краевым условиям (4), устойчиво по отношению к
малым отклонениям
$w=(w_1(x_1,x_2,t),w_2(x_1,x_2,t),w_3(x_1,x_2,t))$. Заметим, что
если диаметр области достаточно мал, а $w_\kappa=const$,   то
квадратичная форма (7) будет отрицательно определенной, так что
постоянное стационарное решение в малой области заведомо будет
устойчивым в малой области. Следует также отметить, что всякое
постоянное (стационарное) решение системы (1)-(3) без учета
миграций, то есть при  $\vartheta_1=\vartheta_2=\vartheta_3=0$,
когда она является системой ОДУ, будет и стационарным решением
этой системы (УЧП) при условии учета миграции, то есть при
$\vartheta_1^2+\vartheta_2^2+\vartheta_3^2>0$, конечно, при
соответствующих краевых условиях. При этом может оказаться, что
это постоянное решение является неустойчивым в модели без учета
миграций, но устойчивым в малой области, если миграционные
процессы будут учтены. Это можно продемонстрировать на простом
примере. Постоянная точка (стационарное состояние) $(k,0,0)$
является решением системы (1)--(3) при любом наборе коэффициентов
диффузии. В случае $\vartheta_1=\vartheta_2=\vartheta_3=0$ это
решение устойчиво при
$$
\mu<k_1-\frac{(k_1+\delta)^2}{4(k_2+\mu)}
$$
и только при этом условии. Если же
$\vartheta_1^2+\vartheta_2^2+\vartheta_3^2>0$, решение $(k,0,0)$
при достаточно малом значении $H$ будет устойчиво и без этого
требования. Методы исследования, примененные для доказательства
основного результата этой работы, применялись ранее в работах
[7--11]. В работе [12] техника исследования скорректирована в
связи с тем, что уравнение содержит оператор Бесселя (см. в связи
с этим [13-17]).


\litlist

%1. {\it Semenov E. M., Sukochev F. A.}
% Invariant Banach limits and applications //Journal of Functional Analysis. - 2010. - Т. 259. - №. 6. - С. 1517-1541.
1. {\it  Capasso V.} A generalization of the Kermack - McKendrick
deterministic epidemic model / V. Capasso, G. Serio // Math.
Biosci. - 42 (1978). - P. 43-62.
http://dx.doi.org/10.1016/0025-5564(78)90006-8 .

2. {\it  Kermack W.O.,  McKendrick A.G.} A Contribution to the
Mathematical Theory of  // Proceedings of the Royal Society of
London. Series A, Containing Papers of a Mathematical and Physical
Character. --- 1927. --- Vol. 115, No. 772 (Aug. 1). --- pp.
700--721.

3. {\it   Cai L.} Stability analysis of an HIV/AIDS epidemic model
with treatment / L. Cai, X. Li, M. Ghoshc, B. Guod // J. Comput.
Appl. Math. --- 229 (2009), l. --- P. 313--323. DOI:
10.1016/j.cam.2008.10.067 .

4.  {\it Tchuenche J.M.} The impact of media coverage on the
transmission dynamics of human influenza / J.M. Tchuenche, N.
Dube, C.P Bhunu, R.J. Smith, C.T. Bauch // BMC Public Health. ---
2011. -- Feb 25;11 Suppl 1(Suppl 1):S5. DOI:
10.1186/1471-2458-11-S1-S5 .

5. {\it Zhao H.} Global Hopf bifurcation analysis of an
susceptible-infective-removed epidemic model incorporating media
coverage with time delay / H. Zhao, M. Zhao // J. Bio. Dyn. 11(1)
(2017). --- P. 8--24. DOI: 10.1080/17513758.2016.1229050 .

6. {\it Sanaa Moussa Salman} Memory and media coverage effect on
an HIV/AIDS epidemic model with treatment/ Sanaa Moussa Salman //
Journal of Computational and Applied Mathematics. --- Volume 385.
--- 2021. 113203, ISSN 0377-0427,
https://doi.org/10.1016/j.cam.2020.113203.
(https://www.sciencedirect.com/science/article/pii/S0377042720304945).

7. {\it  Мешков В.З.} Об устойчивости стационарного решения
уравнения Хотеллинга / В.З.  Мешков, И.П. Половинкин, М.Е. Семенов
// Обозрение прикладной и промышленной математики. --- 2002. --- Т. 9,
вып. 1. --- С. 226--227.

8. {\it  Gogoleva T.N.} On stability of a stationary solution to
the Hotelling migration equation / T.N. Gogoleva, I.N. Shchepina,
M.V. Polovinkina, S.A. Rabeeakh // J. Phys.: Conf. Ser. --- 2019.
--- 1203 012041. DOI: 10.1088/1742-6596/1203/1/012041 .

9. {\it Половинкин И.П.} К вопросу об устойчивости стационарного
решения в миграционной модели / И.П. Половинкин, М. В.
Половинкина, С. А. Рабееах // Актуальные проблемы прикладной
математики, информатики и механики. --- 2019. --- С. 889--892.

10. {\it Половинкина М.В.} Об изменении характера устойчивости
тривиального решения при переходе от модели с сосредоточенными
параметрами к модели с распределенными параметрами / М.В.
Половинкина, И.П. Половинкин // Прикладная математика $\&$ Физика.
- 2020. -  Т. 52, № 4. - С. 255-261. DOI:
10.18413/2687-0959-2020-52-4-255-261
http://pmph.bsu.edu.ru/index.php/journal/article/view/40к письму .


11. {\it Debbouche A.} On the stability of stationary solutions in
diffusion models of oncological processes / A. Debbouche, M. V.
Polovinkina, I. P. Polovinkin, C. A. Valentim, S. A. David, // The
European Physical Journal Plus. (2021). --- 136(1). --- P. 1--18.
https://doi.org/10.1140/epjp/s13360-020-01070-8 .

12. {\it  Polovinkina M. V.} Stability of stationary solutions for
the glioma growth equations with radial or axial symmetries / M.V.
Polovinkina, A. Debbouche, I.P. Polovinkin, S.A. David //
Mathematical Methods in the Applied Sciences 44(15):12021-12034.
DOI: 10.1002/mma.7194 .

13. {\it Катрахов В. В.} Метод преобразования и краевые задачи для
сингулярных эллиптических уравнений / В.В. Катрахов, С.М. Ситник
// Совр. мат. Фундам. напр. - 2018. - 64, № 2. --- С. 211--426.
https://doi.org/10.22363/2413-3639-2018-64-2-211-426 .

14. {\it Киприянов И. А.} Преобразование Фурье --- Бесселя и
теоремы вложения для весовых классов / И.А. Киприянов // Тр. Мат.
ин-та им. В. А. Стеклова. --- 1967. --- 89. --- С. 130--213.

15. {\it Киприянов И. А.} Сингулярные эллиптические краевые задачи
/ И.А. Киприянов --- М.: Наука, 1997. --- 208 с/. --- ISBN
5-02-014799-0.

16.0 {\it Ляхов Л. Н.} Весовые сферические функции и потенциалы
Рисса, порожденные обобщенным сдвигом / Л. Н. Ляхов --- Воронеж:
ВГТА, 1997. --- 143 с. --- ISBN 5-89448-037-X.

17. {\it Ситник С. М.} Метод операторов преобразования для
дифференциальных уравнений с операторами Бесселя / С. М. Ситник,
Э. Л. Шишкина. --- Москва: Физматлит, 2019. --- 221 с. --- ISBN
978-5-9221-1819-4.



\end{document}



\paragraph{Теорема~1.}
{\it
    Не существует такого $\gamma < 1$,
    что для любого $x\in A$ выполнена мультипликативная оценка
    $$
        \alpha(Cx) \leqslant \gamma \cdot \alpha(x)
    ,
    $$
или, что то же самое, не существует такого $p\in \mathbb{N}$,
    что для любого $x\in A$ выполнено неравенство
    $
        \alpha(Cx) \leqslant (1-2^{-p+1})\cdot \alpha(x).
    $
}

Для доказательства теоремы~1 потребуются вспомогательные
построения.
\begin{equation}\label{AvdSem_summa_drobey}
    \sum_{i=0}^{p-1} \frac{i \cdot 2^i}{p} = \frac{2^p(p-2) + 2}{p}
\end{equation}
Введём вспомогательный оператор $S:l_\infty \to l_\infty$:
\begin{equation*}\label{operator_S}
    (Sy)_k = y_{i+2}, \mbox{ где } 2^i < k \leqslant 2^i+1
\end{equation*}
Нам потребуются следующие свойства оператора $S$.
\begin{equation}\label{AvdSem_alpha_S}
    \alpha(Sx) = \varlimsup_{k\to\infty} |x_{k+1} - x_{k}|
\end{equation}
\begin{equation}\label{AvdSem_summa_S_less}
    \sum_{k=2}^{2^p} (Sy)_k =
    \sum_{i=0}^{p-1} 2^i y_{i+2}
\end{equation}
Здесь и далее $(Tx)_n = x_{n+1}$.
\begin{equation}\label{AvdSem_summa_S}
    \sum_{k=2^i+1}^{2^{i+j+1}} (Sx)_k =
    2^i\sum_{k=2}^{2^{j+1}} (ST^ix)_k
\end{equation}
Введём вспомогательную функцию
%\vspace{-2.28em}
\begin{equation*}\label{def_k_b}
    k_b(x) = (2b)^{-1} \left|
        \sum\nolimits_{k=1}^{b}x_k - \sum\nolimits_{k=b+1}^{2b}x_k
    \right|
\end{equation*}
Тогда
\begin{equation}\label{AvdSem_alpha_greater_k_b}
    \alpha (Cx) \geqslant \varlimsup_{i\to \infty} k_i(x)
\end{equation}

\paragraph{Схема доказательства теоремы~1.}
Зафиксируем $p$ и построим $y\in l_\infty$:
\begin{equation*}\label{y_construction}
    y = \left\{
        0, 0, \frac{1}{p}, \frac{2}{p}, %\frac{3}{p},
        ...,
        \frac{p-1}{p}, 1, \frac{p-1}{p},
        ...,
        \frac{1}{p},
        0, ..., 0,
        \frac{1}{p}, ...
    \right\}
\end{equation*}
так, что
\begin{equation}\label{AvdSem_T_y}
    T^{5p}y = y
\end{equation}
Положим $x = Sy$, тогда с учётом (\ref{AvdSem_alpha_S}) $
    \alpha (x) = \alpha (Sy) = \frac{1}{p}
$.

Оценим $\alpha(Cx)$, принимая во внимание
(\ref{AvdSem_summa_drobey}) и (\ref{AvdSem_summa_S_less}) ---
(\ref{AvdSem_alpha_greater_k_b}):
\begin{multline*}
    \alpha (Cx) \mathop{\geqslant}^{(\ref{AvdSem_alpha_greater_k_b})}
    \varlimsup_{b\to \infty} k_b(x) \geqslant
    \!\!\!
    \varlimsup_{
        i\to \infty,~
        b=2^i~
    }\frac{1}{2^{i+1}}\left|
        \sum_{k=1}^{2^i}(Sy)_k -
        \!\!\!\!\!
        \sum_{k=2^i+1}^{2^{i+1}}(Sy)_k
    \right|
    \!
    \geqslant
    \\ \geqslant
    \varlimsup_{
        m\to \infty,~
        i=5pm+p~
    }\left|
        \frac{1}{2^{5pm+p+1}}\sum_{k=1}^{2^{5pm+p}}(Sy)_k - \frac{y_{5pm+p+2}}{2}
    \right| =
    \\=
    \varlimsup_{m\to \infty}\left|
        \frac{1}{2^{5pm+p+1}}\sum_{k=1}^{2^{5pm}}(Sy)_k
        +
        \frac{1}{2^{5pm+p+1}}\sum_{k=2^{5pm}+1}^{2^{5pm+p}}(Sy)_k
        - \frac{1}{2}
    \right|
    \mathop{=}^{(\ref{AvdSem_summa_S})}
    \\=
    \varlimsup_{m\to \infty}\left|
        \frac{1}{2^{5pm+p+1}}\sum_{k=1}^{2^{5pm}}(Sy)_k
        +
        \frac{2^{5pm}}{2^{5pm+p+1}} \sum_{k=2}^{2^p}(ST^{5pm}y)_k
        - \frac{1}{2}
    \right|
    \mathop{=}^{(\ref{AvdSem_T_y})}
    \\=
    \varlimsup_{m\to \infty}\left|
        \frac{1}{2^{5pm+p+1}}\sum_{k=1}^{2^{5pm}}(Sy)_k
        +
        \frac{1}{2^{p+1}} \sum_{k=2}^{2^p}(Sy)_k
        - \frac{1}{2}
    \right|
    \mathop{=}^{(\ref{AvdSem_summa_S_less})}
    \\=
    \varlimsup_{m\to \infty}\left|
        \frac{1}{2^{5pm+p+1}}\sum_{k=1}^{2^{5pm}}(Sy)_k
        +
        \frac{1}{2^{p+1}} \sum_{i=0}^{p-1}2^i \cdot \frac{i}{p}
        - \frac{1}{2}
    \right|
    \mathop{=}^{(\ref{AvdSem_summa_drobey})}
%\end{multline*}
%\begin{multline*}
    \\=
    \varlimsup_{m\to \infty}\left|
        \frac{1}{2^{5pm+p+1}}\sum_{k=1}^{2^{5pm-2p}}(Sy)_k
        -\frac{1}{p} + \frac{1}{p 2^p}
    \right| \geqslant
    \\ \geqslant
    \varlimsup_{m\to \infty} \left(
        \frac{1}{p} (1-2^{-p})
        - \frac{1}{2^{3p+1}}
    \right) >
    \frac{1}{p} (1-2^{-p+1})
\end{multline*}


Таким образом, $
    \alpha(Cx) >
    (1-2^{-p+1}) \cdot \alpha(x)
$.
