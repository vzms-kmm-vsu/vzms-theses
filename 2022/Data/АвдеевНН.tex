\vzmstitle[
	\footnote{Работа выполнена в Воронежском госуниверситете при поддержке РНФ, грант 19-11-00197.}
]{
	Обобщения предела и мультипликативные свойства носителя последовательности
}
\vzmsauthor{Авдеев}{Н.\,Н.}
\vzmsinfo{Воронеж, ВГУ; {\it nickkolok@mail.ru; avdeev@math.vsu.ru}}

\vzmscaption


Обозначим $\ell_\infty$ пространство ограниченных последовательностей с обычными полуупорядоченностью и нормой
\begin{equation*}
	\|x\| = \sup_{k\in\mathbb{N}} |x_k|
	,
\end{equation*}
где $\mathbb{N}$ "--- множество натуральных чисел.
Естественным обобщением предела с пространства сходящихся последовательностей $c$ на $\ell_\infty$
является понятие банахова предела.


\paragraph{Определение.}
	Линейный функционал $B\in \ell_\infty^*$ называется банаховым пределом
	(пишем: $B \in \mathfrak{B}$),
	если
	\begin{enumerate}
		\item
			$B\geq0$, т.~е. $Bx \geq 0$ для $x \geq 0$,
		\item
			$B\mathbb{I}=1$, где $\mathbb{I} =(1,1,\ldots)$,
		\item
			$B(x_1,x_2,\ldots)=B(x_2,x_3,\ldots)$.
	\end{enumerate}

Существование банаховых пределов было анонсировано С. Мазуром в~1929~г. и доказано С.~Банахом~\cite{banach1993theorie}.
%
Сачестон установил~\cite{sucheston1967banach}, что
для любых $x\in \ell_\infty$ и $B\in\mathfrak{B}$
\begin{equation}\label{Sucheston}
	q(x) \leqslant Bx \leqslant p(x)
	,
	\quad\mbox{~~где}
\end{equation}
\begin{equation*}
	q(x) = \lim_{n\to\infty} \inf_{m\in\mathbb{N}}  \frac{1}{n} \sum_{k=m+1}^{m+n} x_k
	~~~~\mbox{и}~~~~
	p(x) = \lim_{n\to\infty} \sup_{m\in\mathbb{N}}  \frac{1}{n} \sum_{k=m+1}^{m+n} x_k
\end{equation*}
суть нижний и верхний функционалы Сачестона соотв.
Неравенства \eqref{Sucheston} точны:
для данного $x$ для любого $r\in[q(x); p(x)]$ найдётся банахов предел
$B\in\mathfrak{B}$ такой, что $Bx = r$.

Множество таких $x\in\ell_\infty$, что $p(x)=q(x)$,
образует~\cite{lorentz1948contribution} пространство почти сходящихся последовательностей $ac$.
На каждом $x\in ac$ все $B\in \mathfrak{B}$ принимают одинаковые значения.




Дальнейшим ослаблением понятия сходимости является сходимость по Чезаро (сходимость в среднем).
Говорят, что последовательность $\{x_n\}\in\ell_\infty$ сходится по Чезаро к $t$, если
\begin{equation*}
	\lim_{n\to\infty}\frac1{n}\sum_{i=1}^n x_i = t
	.
\end{equation*}
Для обобщений верхнего и нижнего пределов имеем:
\begin{multline}
	\label{eq:generalization_of_limits}
	\liminf_{n\to\infty} x_n \leq q(x) \leq \liminf_{n\to\infty}\frac1{n}\sum_{i=1}^n x_i
	\leq
	\\ \leq
	\limsup_{n\to\infty}\frac1{n}\sum_{i=1}^n x_i
	\leq p(x)
	\leq \limsup_{n\to\infty} x_n
	.
\end{multline}


Каждый $x\in \{0;1\}^\mathbb{N}$ можно отождествить с подмножеством
$\operatorname{supp} x \subset \mathbb{N}$.
Обозначим
\begin{equation*}
	\mathscr{M}A = \{ka: k\in\mathbb{N}, a\in A\}
	,
\end{equation*}
через $\chi A$ "--- характеристическую функцию множества $A$.

Возникает вопрос о взаимосвязи структуры множества $A$
и значений, которые принимают обобщения верхнего и нижнего пределов~\eqref{eq:generalization_of_limits}
на последовательности $\chi \mathscr{M}\!A$.
Известно~\cite{davenport1951sequences}, что для любого
$A=\{a_1,a_2,...\}\subset\mathbb{N}$
выполнено
\begin{equation*}
	\liminf_{n\to\infty}\frac1{n}\sum_{i=1}^n (\chi\mathscr{M}A)_i =
	\lim_{j\to\infty}\lim_{n\to\infty}\frac1{n}\sum_{i=1}^n (\chi\mathscr{M}\{a_1,a_2,...,a_j\})_i
	.
\end{equation*}
В~\cite[\S 7]{besicovitch1935density} построено такое множество $A\subset\mathbb{N}$, что
\begin{equation}
	\liminf_{n\to\infty}\frac1{n}\sum_{i=1}^n (\chi\mathscr{M}A)_i \neq
	\limsup_{n\to\infty}\frac1{n}\sum_{i=1}^n (\chi\mathscr{M}A)_i
	.
\end{equation}
В~\cite{avdeev2021vmzprimes} получены аналогичные результаты для функционалов Сачестона.
\begin{theorem}
	Пусть $A\subset \mathbb{N}\setminus\{1\}$.
	Тогда следующие условия эквивалентны:
	\begin{enumerate}%[label=(\roman*)]
		\item
			Для любого $n\in\mathbb{N}$ найдётся набор попарно взаимно простых чисел
			$
				\{a_{n,1}, a_{n,2}, ..., a_{n,n}  \} \subset A
				.
			$
		\item
			В $A$ существует бесконечное подмножество попарно взаимно простых чисел.
		\item
			$p(\chi\mathscr{M}A)=1$.
	\end{enumerate}
\end{theorem}





\vspace{-1.5em}
\begin{thebibliography}{99}
\vspace{-0.7em}
{}
\bibitem{banach1993theorie}
\emph{Banach} \emph{S.} Théorie des opérations
linéaires. — Sceaux : Éditions Jacques Gabay, 1993. —
С. iv+128. — ISBN 2-87647-148-5. — Reprint of the
1932 original.
{}
\bibitem{sucheston1967banach}
\emph{Sucheston} \emph{L.} Banach limits // Amer.
Math. Monthly. — 1967. — Т. 74. — С. 308—311. —
ISSN 0002-9890.
% — DOI: \href
%{https://doi.org/10.2307/2316038} {\nolinkurl
%{10.2307/2316038}}.
{}
\bibitem{lorentz1948contribution}
\emph{Lorentz} \emph{G. G.} A contribution to the
theory of divergent sequences // Acta Math. —
1948. — Т. 80, No 1. — С. 167—190.
%— ISSN 0001-5962. —
%DOI: \href {https://doi.org/10.1007/BF02393648}
%{\nolinkurl {10.1007/BF02393648}}.{}
\bibitem{davenport1951sequences}
\emph{Davenport} \emph{H.},
\emph{Erdös} \emph{P.} On sequences of positive
integers // J. Indian Math. Soc., New Series. — 1951. —
Т. 15. — С. 19—24. — DOI: \href
{https://doi.org/10.18311/JIMS/1951/17063}
{\nolinkurl {10.18311/JIMS/1951/17063}}.
{}
\bibitem{besicovitch1935density}
\emph{Besicovitch} \emph{A.} On the density of
certain sequences of integers // Math.
Annalen. — 1935. — Т. 110, No 1. — С. 336—341.
 — DOI:
\href {https://doi.org/10.1007/BF01448032}
{\nolinkurl {10.1007/BF01448032}}.
{}
\bibitem{avdeev2021vmzprimes}
\emph{Авдеев} \emph{Н. Н.} Почти сходящиеся
последовательности из 0 и 1 и простые числа //
Владикавказский математический журнал. — 2021. —
Т. 23, No 4. — С. 5—14.
% — DOI: \href
%{https://doi.org/10.46698/p9825-1385-3019-c}
%{\nolinkurl {10.46698/p9825-1385-3019-c}}.
\end{thebibliography}
