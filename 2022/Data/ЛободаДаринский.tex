\vzmstitle[
	\footnote{Работа выполнена при финансовой поддержке РФФИ (проект \No~20-01-00497) и Московского Центра
		фундаментальной и прикладной математики (МГУ им. М.В. Ломоносова).}
]{
	О больших семействах аффинно однородных поверхностей в $\mathbb{R}^4$.
}
\vzmsauthor{Лобода}{А.\,В.}
\vzmsinfo{Воронеж, ВГТУ; {\it lobvgasu@yandex.ru}}
\vzmsauthor{Даринский}{Б.\,М.}
\vzmsinfo{Воронеж, ВГУ {\it darinskii@mail.ru}}

\vzmscaption

В связи с задачей описания аффинно однородных гиперповерхностей пространства $\mathbb{R}^4$ (см., например, [1]) в [2] был введён класс $\mathcal{A}_E$ 3-мерных \textbf{абелевых} подалгебр Ли алгебры $gl(4,\mathbb{R})$, \textbf{содержащих единичную матрицу}.

Ниже мы рассматриваем любую квадратную матрицу $ \|a_{ij}\| $ 4-го порядка как линейное векторное поле в $\mathbb{R}^4 $ вида $E_k = \sum_{i,j =1 }^4 {a_{ij} x_j} \frac{\partial}{\partial x_i}$, а подалгебры $gl(4,\mathbb{R})$ "--- как алгебры линейных векторных полей. Семейство алгебр мы называем \textit{нетривиальным}, если любая алгебра из этого семейства имеет 3-мерные орбиты в $\mathbb{R}^4$, отличные от цилиндров.


\paragraph{Теорема 1.} {\it В множестве $\mathcal{A}_E$ имеется пять \textbf{нетривиальных} семейств алгебр Ли, каждое из которых описывается не менее чем двумя вещественными параметрами. Базисы этих семейств имеют следующий вид:
	\begin{equation*}
		\begin{gathered}
			S_1 : \begin{bmatrix}
				1 & 0 & 0 & 0 \\
				0 & 0 & 0 & 0 \\
				0 & 0 & \lambda & 0 \\
				0 & 0 & 0 & 0
			\end{bmatrix},
			\ \begin{bmatrix}
				0 & 0 & 0 & 0 \\
				0 & 1 & 0 & 0 \\
				0 & 0 & \mu & 0 \\
				0 & 0 & 0 & 0
			\end{bmatrix},
			\ \begin{bmatrix}
				1 & 0 & 0 & 0 \\
				0 & 1 & 0 & 0 \\
				0 & 0 & 1 & 0 \\
				0 & 0 & 0 & 1
			\end{bmatrix},
		\end{gathered}
		\eqno{(1)}
\end{equation*}
\begin{equation*}
	\begin{gathered}
		S_2 \ : \ \begin{bmatrix}
			1 & 0 & 0 & 0 \\
			0 & \lambda & 0 & 0 \\
			0 & 0 & 0 & 0 \\
			0 & 0 & 0 & 0
		\end{bmatrix},
		\ \begin{bmatrix}
			0 & 0 & 0 & 0 \\
			0 & b & 0 & 0 \\
			0 & 0 & 0 & 1 \\
			0 & 0 & -1 & 0
		\end{bmatrix},
		\ \begin{bmatrix}
			1 & 0 & 0 & 0 \\
			0 & 1 & 0 & 0 \\
			0 & 0 & 1 & 0 \\
			0 & 0 & 0 & 1
		\end{bmatrix},
	\end{gathered}
	\eqno{(2)}
\end{equation*}
\begin{equation*}
	\begin{gathered}
		S_3 \ : \ \begin{bmatrix}
			0 & 1 & 0 & 0 \\
			0 & 0 & 0 & 0 \\
			0 & 0 & 0 & 1 \\
			0 & 0 & 0 & 0
		\end{bmatrix},
		\ \begin{bmatrix}
			\nu_1 & 0 & \nu_2 & 1 \\
			0 & \nu_1 & 0 & \nu_2 \\
			1 & 0 & 0 & 0 \\
			0 & 1 & 0 & 0
		\end{bmatrix},
		\ \begin{bmatrix}
			1 & 0 & 0 & 0 \\
			0 & 1 & 0 & 0 \\
			0 & 0 & 1 & 0 \\
			0 & 0 & 0 & 1
		\end{bmatrix},
	\end{gathered}
	\eqno{(3)}
\end{equation*}
\begin{equation*}
	\begin{gathered}
		S_4 \ : \ \begin{bmatrix}
			\lambda & 0 & 0 & 0 \\
			0 & \lambda & 0 & 0 \\
			0 & 0 & 0 & 1 \\
			0 & 0 & -1 & 0
		\end{bmatrix},
		\ \begin{bmatrix}
			b & 1 & 0 & 0 \\
			-1 & b & 0 & 0 \\
			0 & 0 & 0 & 0 \\
			0 & 0 & 0 & 0
		\end{bmatrix},
		\ \begin{bmatrix}
			1 & 0 & 0 & 0 \\
			0 & 1 & 0 & 0 \\
			0 & 0 & 1 & 0 \\
			0 & 0 & 0 & 1
		\end{bmatrix},
	\end{gathered}
	\eqno{(4)}
\end{equation*}
\begin{equation*}
	\begin{gathered}
		S_5 \ : \ \begin{bmatrix}
			0 & 1 & 0 & 0 \\
			-1 & 0 & 0 & 0 \\
			0 & 0 & 0 & 1 \\
			0 & 0 & -1 & 0
		\end{bmatrix},
		\begin{bmatrix}
			\mu_1 & 1 & 1 & 0 \\
			-1 & \mu_1 & 0 & 1 \\
			\mu_2 & a & 0 & 0 \\
			-a & \mu_2 & 0 & 0
		\end{bmatrix},
		\begin{bmatrix}
			1 & 0 & 0 & 0 \\
			0 & 1 & 0 & 0 \\
			0 & 0 & 1 & 0 \\
			0 & 0 & 0 & 1
		\end{bmatrix},
	\end{gathered}
	\eqno{(5)}
\end{equation*}}

\paragraph{Предложение~1.} {\it 3-мерные орбиты всех алгебр из теоремы 1 имеют в общих точках нулевую гауссову кривизну. Вторая квадратичная форма этих гиперповерхностей в любой из таких точек имеет сигнатуру $(+,-,0)$.}

Ниже дано описание орбит алгебр (1)-(4).

\paragraph{Теорема~2.} {\it Орбитами алгебр (1)-(4) являются следующие аффинно однородные поверхности пространства $\mathbb{R}^4$:
	\begin{equation*}
		x_4 = x_1^{\lambda}\, x_2^{\mu}\, x_3^{1-(\lambda + \mu)} \ (\lambda, \mu \in \mathbb{R}),
		\eqno{(6)}
	\end{equation*}
	\begin{equation*}
		x_2 = x_1^{\lambda}\,|x_3 + i x_4|^{(1-\lambda)} e^{b \arg(x_3 + i x_4)},
		\eqno{(7)}
	\end{equation*}
	\begin{equation*}
		x_1 x_4 - x_2 x_3 = (x_2^2 - \nu_1 x_2 x_4 - \nu_2 x_4^2)\, \ln\left(\frac{x_2 - a_2 x_4}{x_2 - a_1 x_4}\right),
		\eqno{(8)}
	\end{equation*}
где $\nu_1 = a_1 + a_2, \ \nu_2 = - a_1 a_2,\ a_1 > a_2$;
	\begin{equation*}
		\ln{\frac{|x_1+ix_2|}{|x_4+ix_3|}} + b \arg(x_1+ix_2) - \lambda \arg(x_4 + i x_3) = 0.
		\eqno{(9)}
	\end{equation*}
}

\paragraph{Замечание~1.} {\it Трубки в пространстве $\mathbb{C}^4 $ над такими поверхностями являются 2-невырожденными голоморфно однородными гиперповерхностями.
}

Отметим, что семейства поверхностей (6) и (7) являются подмножествами 3-пара\-метрических семейств аффинно однородных гиперповерхностей
\begin{equation*}
	x_4 = x_1^{\lambda}\, x_2^{\mu}\, x_3^{\nu}
	\quad \mbox{и} \quad
	x_2 = x_1^{\lambda}\,|x_3 + i x_4|^{\nu} e^{\mu \arg(x_3 + i x_4)},
	\eqno (10)
\end{equation*}
обобщающих известные однородные гиперповерхности в $\mathbb{R}^3$.

Наиболее интересными являются орбиты семейства (5).
\paragraph{Предложение~2.} {\it При $ b_{10} \neq 1 $ орбиты любой алгебры из (5), проходящие через точку $Q(0,1,0,1)$, описываются интегралами обыкновенных дифференциальных уравнений
	\begin{equation*}
		w'(r) =
		{\frac { \left( -b_{{9}}{r}^{2}-2\,b_{{10}}r+b_{{9}} \right) {w}^{3}+
				\left( r-b_{{1}} \right) {w}^{2}-w}{ \left( 1+{r}^{2} \right)
				\left( b_{{9}}r+b_{{10}} \right) {w}^{2}- \left( 1+{r}^{2} \right) w+
				r}}.
		\eqno (11)
	\end{equation*}
}

При $b_1 = b_9 = b_{10} = 0$ интегрирование уравнения (11) и алгебры (5) приводят к однородным гиперповерхностям
\begin{equation*}
	{\frac { \left( x_{{1}}x_{{4}}-x_{{2}}x_{{3}} \right) ^{2}}{ \left( x_3^{2}+x_4^{2} \right)^{2}}}-
	\frac {2\,(x_2 x_4 + x_1 x_3) - (x_1 x_4 - x_2 x_3)}{x_3^2+x_4^2}=C.
	\eqno (12)
\end{equation*}
В целом же представляется естественной гипотеза о существовании у семейства (5) орбит, которые \textbf{невозможно} описать элементарными функциями.


\litlist

1. {\it Можей Н. П.} Однородные подмногообразия в четырёхмерной аффинной и проективной геометрии // Изв. вузов. Матем.~-- 2000.~-- №~7.~-- С.~41-52.

2. {\it Лобода А. В., Даринский Б. М.} Об орбитах в $\mathbb{R}^4$ абелевой 3-мерной алгебры Ли // Уфимская осенняя математическая школа -- 2021 : Материалы международной научной конференции.~-- Уфа: Аэтерна, 2021.~-- Т.~1.~-- С.~239-241.
