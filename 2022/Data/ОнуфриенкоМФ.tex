\documentclass{vzmsthesis}

\begin{document}

\vzmstitle[
\footnote{Автор является стипендиатом Фонда развития теоретической физики и математики <<БАЗИС>>.}
%Работа выполнена за счёт гранта РНФ, проект 16-11-10125}
]{
	Бифуркации критических точек гладких инвариантных функций: классификация и приложение к интегрируемым системам
}
\vzmsauthor{Онуфриенко}{М.\,В.}
\vzmsinfo{Москва, МГУ; Москва, МЦФиПМ; {\it mary.onufrienko@gmail.com}}


\vzmscaption

Фиксируем $s\in\mathbb N$ и рассмотрим действие группы $G=\mathbb{Z}_s$ на плоскости $\mathbb{R}^2$ вида 
$z\longrightarrow e^{2\pi i/{s}}z,$ где $z=x+iy\in\mathbb C\approx\mathbb{R}^2.$
Рассмотрим морсовские функции $F_{s,0}=F_{s,0}^{\pm,\pm}(z)=\pm|z|^2=\pm (x^2+y^2)$ при $s\ge1$, и $F_{s,0}=F_{s,0}^{+,-}(z)=x^2-y^2$ при $s=1,2$. Рассмотрим два семейства $G$--инвариантных ростков $F_{s,k}=F_{s,k}(z,a,\lambda)$, $k=1,2$, от переменных $z=(x,y)$ в нуле:
\begin{equation*} \label{On_sistem1}
F_{s,1}
%=F_{s,1}(z,a,\lambda)
=\left\{
\begin{array}{ll}
%\pm x^2+y^3+\lambda y, & s=1, \\
%\pm x^2\pm y^4 +\lambda y^2, & s=2, \\
\pm x^2\pm y^{s+2} +\lambda y^s, & s=1,2, \\
\mathrm{Re}(z^3)+\lambda|z|^2, & s=3, \\
\mathrm{Re}(z^s)\pm a|z|^4+\lambda |z|^2, & s\ge4,
%a^2\neq 1 \mbox{ при }s=4, %\\ & %  a>0 \mbox{ при }s\ge5,
\end{array}
\right.
\end{equation*}
\begin{equation*} \label{On_sistem2}
F_{s,2}
%=F_{s,2}(z,a,\lambda)
=\left\{
\begin{array}{ll}
%\pm x^2\pm y^4-\lambda_2 y^2+\lambda_1 y, & s=1, \\
%\pm x^2\pm y^6 +\lambda_2 y^4+\lambda_1 y^2, & s=2, \\
\pm x^2\pm y^{2s+2} +\lambda_2 y^{2s}+\lambda_1 y^s, & s=1,2, \\
\mathrm{Re}(z^4)\pm
(1+\lambda_2)
|z|^4 \pm a|z|^6+
%R
\lambda_1 |z|^2
, & s=4, \\ %\ a>0, \\
\mathrm{Re}(z^5)\pm a|z|^6+
%R
\lambda_2|z|^4+\lambda_1 |z|^2
, & s=5, \\ %\ a>0, \\
\mathrm{Re}(z^s)\pm a_1 |z|^6+ a_2 |z|^8+
%R
\lambda_2|z|^4+\lambda_1 |z|^2
, & s\ge6.
%, \\ %,\ a_1^2\ne1,\\ &  a_1a_2\ne0.
%\mathrm{Re}(z^s)\pm a_1 |z|^6+a_2 |z|^2\mathrm{Re}(z^s)+R
%%\lambda_2|z|^4+\lambda_1 |z|^2
%, & s\ge7.
\end{array}
\right.
\end{equation*}
Здесь $\lambda=(\lambda_i)\in\mathbb{R}^k$ --- параметры, $a=(a_i)\in\mathbb{R}^m$ --- модули, $m\in\{0,1,2\}$ --- модальность;
%$R=\lambda_2|z|^4+\lambda_1 |z|^2$;
$a^2\ne 1$ при $(s,k)=(4,1)$; $a>0$ при $s\ne k+3$;
%$s\ge5$, $k=1$; остальных $(s,k)\ne(4,1),(5,2)$;
%$(a_1^2-1)a_1a_2\ne0$ при $(s,k)=(6,2)$.
$a_1^2\ne1$ при $(s,k)=(6,2)$;
$a_1>0$ иначе.
%при $(s,k)\ne(6,2)$.
%$s\ge7$, $k=2$.
%остальных $(s,k)=(s,2)$.

\smallskip
\paragraph{Теорема~1.}
{\it
Рассмотрим классы {\em правой} $G$--эк\-ви\-ва\-лент\-ности $G$--ин\-ва\-ри\-ант\-ных ростков функций $F_{s,k}(z,\hat a,0)$ двух переменных в нуле, $k=0,1,2$.
Эти особенности имеют $G$--кораз\-мер\-ность $k$, $G$--кратность Милнора $k+m+1$, $G$--версальную деформацию $F_{s,k}(z,a,\lambda)+\lambda_0$.
 % и образуют полный список $G-$ин\-ва\-ри\-ант\-ных особенностей $G-$ко\-раз\-мер\-ности $k\le 2$. 
Дополнение к их объединению в множестве $\mathfrak{n}^2_G
%:=\mathfrak{m}^2\cap \mathcal{E}_G
$
$G$--инва\-ри\-ант\-ных ростков в нуле, имеющих критическую точку $0$ с критическим значением $0$, имеет коразмерность $>2$ в $\mathfrak{n}_G^2$.}  

Теорема 1 не следует из классификации [1].
 % особенностей $G$--кратности Милнора $\le 5$. 
Доказательство теоремы 1 основано на следующих леммах.

\smallskip
\paragraph{Лемма~1.}
{\it Алгебра $G$--инвариантных многочленов от $x,y$ с вещественными коэффициентами мультипликативно\\ порождена однородными многочленами $|z|^2$, $\mathrm{Re}(z^s)$, $\mathrm{Im}(z^s)$.
%где $G=\mathbb Z_s$, 
Она является линейной оболочкой 
%(над $\mathbb{R}$) 
многочленов $\mathrm{Re}(z^k \bar{z}^l)$, $\mathrm{Im}(z^k \bar{z}^l)$, где $s\mid(k - l)$. 
%Многочлен $f(z,\bar z)=\sum_{k,l} a_{kl} z^k \bar{z}^l$ вещественный $\Longleftrightarrow$ $a_{kl}=\bar{a}_{lk}$. Кроме того, 
В частности, любой $G$--инва\-ри\-ант\-ный росток $f(z)$ в нуле имеет ряд Тейлора вида 
%Алгебра $G-$инвариантных многочленов от $z$, $\bar{z}$ с комплексными коэффициентами является линейной оболочкой (над $\mathbb{C}$) мономов $z^k \bar{z}^l$, где $s\mid(k - l)$. Многочлен $f(z,\bar z)=\sum_{k,l} a_{kl} z^k \bar{z}^l$ вещественный $\Longleftrightarrow$ $a_{kl}=\bar{a}_{lk}$. Кроме того, любая гладкая $G$--инвариантная функция $f(z)$ имеет вид 
\medskip\\ $
%\phantom{ooo} 
f(z)=\mathrm{Re} \displaystyle 
\sum_{p,q\ge0} c_{pq}|z|^{2p}z^{qs}
=c_0+c_1|z|^2
%+\smallskip\\
%\phantom{ooooooo} 
+\mathrm{Re}(c_2 z^s+c_3 |z|^2 z^s)+\smallskip\\
\phantom{oooooooooooi f(z)=c_0+c_1|z|^2} 
+c_4|z|^4+c_5|z|^6+c_6|z|^8+\ldots 
$
\hfill $(1)$
}

\medskip
\paragraph{Лемма~2.}
{\it Пусть $f(z)$ --- $G$--инвариантный росток в нуле с рядом Тейлора вида {\rm(1)}, где
%$G=\mathbb Z_s$, 
$s\ge3$, $c_0 = 0$.
%$f\in\{c_0 = 0\}$. 
Росток $f(z)$ право--$G$--эквивалентен ростку $F_{s,k}(z,a,0)$, $k=0,1,2$, тогда и только тогда, когда либо
% выполнено одно из следующих условий:
\\
{\rm(a)} $k=0$,
%$f(z)$ право--$G$--эквивалентен $F_{s,0}^{\pm,\pm} (z)$ $\Longleftrightarrow$ 
$c_1 \neq 0$, либо
%$f\in\{c_1 \neq 0\}$;
\\
{\rm(b)} $k=1$,
%$f(z)$ право--$G$--эквивалентен $F_{s,1}(z,a,0)$ $\Longleftrightarrow$ 
%$c_1 = 0$ и $c_2^2+c_3^2 > 0$ при $s=3$; 
$c_1 = 0$, $c_2 \ne 0$ и $a=|c_4|/|c_2|^{4/s}$, либо
%%, $a^2\neq 1$ при $s=4$; 
%%$c_1 = 0$, $c_2^2+c_3^2 > 0$ и $a=|c_4|>0$ 
%при $s\ge 4$;
%%$f\in\{c_1 = 0,c_2^2+c_3^2 > 0\}$ при $s=3$, $f\in\{c_1 = 0,c_2^2+c_3^2 > 0, c_4^2\neq 1\}$ при $s=4$, $f\in\{c_1 = 0,c_2^2+c_3^2 > 0, c_4 >0\}$ при $s\ge 5$;
\\
{\rm(c)} $k=2$,
%$f(z)$ право--$G$--эквивалентен $F_{s,2}(z,a,0)$ $\Longleftrightarrow$ 
$c_1 = |c_2|^2-c_4^2 = 0$, $c_4\ne 0$, $a=|c_5-c_4\mathrm{Re}(\frac{c_3}{c_2 })|/|c_2|^{3/2}$ при $s=4$; 
$c_1 = c_4 = 0$, $c_2 \neq 0$, $a=|c_5|/|c_2|^{6/5}$ при $s=5$; 
$c_1 = c_4 = 0$, $c_2 \neq 0$, $a_1=\frac{|c_5|}{|c_2|^{6/s}}$, $a_2=(c_6-c_5\mathrm{Re}(\frac{6c_3}{sc_2}))/|c_2|^{8/s}$
%$(c_5^2-1) c_5 c_6\neq 0$ 
при $s\ge6$.}
%$f\in\{c_1 = 0,c_2^2+c_3^2 > 0, c_4>0\}$ при $s=4$, $f\in\{c_1 = 0,c_2^2+c_3^2 > 0, c_5 >0\}$ при $s= 5$; $f\in\{c_1 = 0,c_2^2+c_3^2 > 0, c_5^2\neq 1, c_5 c_6\neq 0\}$ при $s= 6$.}

\medskip
Опишем приложение теоремы 1 к классификации структурно устойчивых особенностей интегрируемых систем с 2 и 3 степенями свободы. 
Напомним, что {\em интегрируемая система $(M,\omega,\mathcal{F})$ с $n$ степенями свободы}
%на $2n$-мерном симплектическом многообразии $(M^{2n},\omega)$ 
задается гладким отображением $\mathcal{F}=(f_1,\dots,f_n)
%:\ M \longrightarrow
: M\to {\mathbb R}^n,$ где $\{f_i,f_j\}=0$, $\dim M=2n$. 
%Возникает {\em  лагранжево слоение с особенностями} на $M$, слои которого --- это связные компоненты множеств $\mathcal{F}^{-1}(c)$. Пусть $M$ компактно, поля $X_{f_j}=sgrad f_j$ касаются $\partial M$, и $\mathcal{F}$ имеет <<хорошее>> поведение около $\partial M$. Отображение $\mathcal{F}$ порождает { гамильтоново $\mathbb{R}^n-$действие} на $M$. 
%
Рассмотрим 
%свободное 
действие группы $G
%=\mathbb{Z}_s
$ на полнотории $V=D^2\times S^1
%\subset\mathbb{R}^2\times S^1
$ вида 
$
(z,\varphi_1)\mapsto (e^{2\pi i {\frac \ell s}}z,\varphi_1+\frac{2\pi}{s}),
$
где 
%$\varphi_1\in S^1,
%%\eqno(2)
%$
$0\le\ell<s$, $(\ell,s)=1$. Рассмотрим цилиндр $W=D^{n-1}\times(S^1)^{n-2}$ с координатами $\lambda=(\lambda_i)_{i=1}^{n-1}$, $\varphi'=(\varphi_j)_{j=2}^{n-1}$.
%$$
%(\lambda,\varphi')=(\lambda_1,\ldots,\lambda_{n-1},\varphi_2,\dots,\varphi_{n-1}).
%$$

%Оказывается, 
Из теоремы 1 можно вывести, что {\em локальные особенности} (т.е.\ $\mathbb{R}^n-$орбиты) ранга $n-1$ типичных интегрируемых систем с $n\le3$ степенями свободы имеют окрестности, гладко эквивалентные {\em  стандартной модели} $(M_{\frac\ell s},\omega_{\frac\ell s},\mathcal{F}_{\frac\ell s,k,a})$ вида 
$$
M_{\frac\ell s}=(V/G)\times W, \quad 
\omega_{\frac\ell s} = \mathrm dx\wedge\mathrm dy + \sum_{j=1}^{n-1}\mathrm d\lambda_j\wedge\mathrm  d\varphi_j,
$$ 
$$
\mathcal{F}_{\frac\ell s,k,a}:M_{\frac\ell s}
%(V/\Gamma)\times W 
\to \mathbb R^n,\ \ \mathcal{F}_{\frac\ell s,k,a}(z,\varphi_1,\lambda,\varphi')=(\lambda,{F_{s,k}}(z,a(\lambda),\lambda')),
$$
где $0\le k<n$, $\lambda'=(\lambda_1,\dots,\lambda_k)$,
$a(\lambda)$ и $a_1(\lambda)$ --- гладкие функции при $(s,k)\in\{(4,1),(6,2)\}$,
$a(\lambda)\equiv1$ для остальных пар $(s,k)$; 
$a(\lambda)=(a_1(\lambda),1)$ при $(s,k)=(6,2)$.

%Заметим, что переменные действия $ \lambda_1,\dots,\lambda_{n-1}$ порождают {локально-свободное гамильтоново <<линейное>> действие $(n-1)$-мерного тора} на многообразии $M_{\ell/s}=(V/\Gamma)\times W\approx V\times W$
%%$(V/\Gamma)\times W = ((D^2\times S^1)/\Gamma)\times D^{n-1}\times(S^1)^{n-2}$ 
%сдвигами вдоль угловых координат $\varphi_1,\dots,\varphi_{n-1}$. 




%\paragraph{Теорема~2.}
%{\it Рассмотрим класс интегрируемых гамильтоновых систем (ИГС) на $M$, для которых функции $f_1,\dots,f_{n-1}$ порождают локально-свободное гамильтоново действие $(n-1)-$мерного тора на $M$. 
%Если $n
%%=\frac12\dim M
%\in\{2,3\}$, то $\mathbb{R}^n-$орбиты, окрестности которых {гладко эквивалентны стандартным}, 
%%являются ``типичными'', т.е.\ 
%обладают свойствами {<<типичности>>}: (i) равномерно структурно устойчивы (относительно малых возмущений в указанном пространстве ИГС), и даже равномерно гладко структурно устойчивы в случае $(s,k)\not\in\{(4,1),\ (6,2)\}$; (ii) множество ИГС, имеющих только такие локальные особенности, имеет полную меру в указанном пространстве ИГС (т.е.\ дополнение к этому множеству имеет меру 0, и даже положительную коразмерность).
%
%В частности, любая структурно устойчивая $\mathbb{R}^n-$орбита имеет окрестность, {топологически эквивалентную стандартной} модели $(M_{\ell/s},\omega_{\ell/s},\mathcal{F}_{\ell/s,k,a})$. Стандартные модели с разными $(\pm\ell/s\ mod\ \mathbb{Z},k)$ топологически не эквивалентны.}

%Теорема 2 
При $n = 2$, $k = 1$ описанные выше особенности --- это параболические орбиты с резонансом $\ell/s$ [2]. В вещественно-аналитическом случае они типичны и структурно устойчивы [2], а также гладко структурно устойчивы, если порядок резонанса $s\ne4$ [3]. При $n = 3$, $k = 2$ получаем типичные бифуркации параболических особенностей с резонансами.

\litlist

1. {\it Wassermann G.} Classification of singularities with com\-pact Abelian symmetry //Banach Center Publications. – 1988. – Т. 20. – №. 1. – С. 475-498.

2. {\it Калашников В. В.} Типичные интегрируемые гамильтоновы системы на четырехмерном симплектическом многообразии //Изв. РАН. Сер. матем. - 1998. - Т. 62. - №. 2. - С. 49-74.
%{\it Kalashnikov V. V.} Typical integrable Hamiltonian systems on a four-dimensional symplectic manifold // Izvestiya: Mathematics. – 1998. – Т. 62. – №. 2. – С. 261-285.

3. {\it Kudryavtseva E. A.} Hidden toric symmetry and structu\-ral stability of singularities in integrable systems //European J. Math. - 2021. - 
\href{https://doi.org/10.1007/s40879-021-00501-9}{doi.org/10.1007/s40879-021-00501-9},
%(published 25 October 2021), 
63 pp. 
%\href{https://arxiv.org/abs/2008.01067}{arXiv:2008.01067}


\end{document}
