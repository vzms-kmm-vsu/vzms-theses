\vzmstitle[
]{
	Развитие метода регуляризации для сингулярно возмущённой задачи Коши с <<простой>> рациональной точкой поворота для параболического уравнения
}
\vzmsauthor{Елисеев}{А.\,Г.}
\vzmsinfo{Москва, НИУ <<МЭИ>>; {\it eliseevag@mpei.ru}}
\vzmsauthor{Ратникова}{Т.\,А.}
\vzmsinfo{Москва, НИУ <<МЭИ>>; {\it ratnikovata@mpei.ru}}
\vzmsauthor{Шапошникова}{Д.\,А.}
\vzmsinfo{Москва, НИУ <<МЭИ>>; {\it shaposhnikovda@mpei.ru}}

\vzmscaption


Рассмотрим задачу Коши
$$
\left\{ \begin{array}{l}
\smallskip
\varepsilon\left(\dfrac{\partial u}{\partial t}-\left( \dfrac{\partial}{\partial x}\left( k(x)\dfrac{\partial u}{\partial x}'\,\right)\right)\right)+t^{m/n}a(t)u=h(x,t),\\
u(x,0)=f(x), \quad -\infty<x<+\infty.\\
\end{array} \right.
\eqno(1)
$$
и пусть выполнены условия:
$$
\begin{array}{l}
1) \ h(x,t)\in \mathbb{C}^\infty(R\times[0,T]), \ \text{функция} \ h(x,t) \ \text{и все её }\\
\hspace{0.5cm} \text{производные ограничены на} \ \{R\times[0,T]\};\\
2) \ k(x)\in \mathbb{C}^\infty(R), \ \exists k_0,M>0 \ \ 0<k_0\leqslant k(x)<M(x^2+1),\\
\hspace{0.5cm} |k'(x)|<M \sqrt{x^2+1};\\
3) \ f(x)\in \mathbb{C}^\infty(R), \ \text{функция} \ f(x) \ \text{и все её производные}\\
\hspace{0.5cm} \text{ограничены на} \ R;\\
4) \ a(t)\in \mathbb{C}^\infty([0,T]),\ a(t)\neq 0,\ {\rm Re\,} (a(t))\geq 0;\\
5) \ m,n\in N, \ \ p=m+n-1, \ \ m/n=r \text{ --- дробное}. \hspace{0.5cm}\\
\end{array}
\eqno(2)
$$

Сингулярности задачи (1) имеют вид:
$$
e^{-\varphi(t)/\varepsilon}, \ \ \sigma_i(t,\varepsilon)=e^{-\varphi(t)/\varepsilon} \int\limits_0^t e^{\varphi(s)/\varepsilon} s^{\frac{i+1}{n}-1}ds,
$$
где $\displaystyle{\varphi(t)=\int\limits_{0}^{t} s^{m/n}a(s)ds}$, $i=\overline{0,(p-1)}$, $p=m+n-1$.

Согласно методу регуляризации, введём расширенную функцию
$\widetilde{u}(x,t,\tau,\sigma)$, где $\sigma=\{\sigma_0,\sigma_1,\ldots,\sigma_{p-1}\}$, такую, что сужение
$$
\widetilde{u}(x,t,\tau,\sigma)\bigl|_{\tau=\varphi(t)/\varepsilon, \ \ \sigma_i=\sigma_i(t,\varepsilon)}= u(x,t)
$$
даёт решение задачи (1).
В результате получим задачу для расширенной функции $\widetilde{u}(x,t,\tau,\sigma,\varepsilon)$:
$$
\left\{ \begin{array}{l}
\displaystyle{-q(t) \frac{\partial\widetilde{u}}{\partial\tau} +\sum\limits_{i=0}^{p-1} -q(t)(\sigma_i \frac{\partial\widetilde{u}}{\partial\sigma_i}) +q(t)\widetilde{u}=} \\
\hspace{0.5cm} \displaystyle{= -\varepsilon\left( \dot{\widetilde{u}}+\sum\limits_{i=0}^{p-1} t^{\frac{i+1}{n}-1} \frac{\partial\widetilde{u}}{\partial\sigma_i} - \frac{\partial}{\partial x} \left( k(x) \frac{\partial\widetilde{u}}{\partial x}\right)\right)+h(x,t),}\\
\widetilde{u}(x,0,0,0,\varepsilon)=f(x).
\end{array} \right.
\eqno(3)
$$

В дальнейшем волну будем опускать. Для решения задачи (3) введём пространство безрезонансных решений $E$, элементы которого имеют вид
$$
u=X(x,t)e^\tau+ \sum\limits_{i=0}^{p-1} Z^i(x,t)\sigma_i +W(x,t),
$$
где $X(x,t),Z^i(x,t),W(x,t)\in C^\infty(R\times (0,T]) \cap \mathbb{C}(R\times[0,T])$.

Решение задачи (3) будем искать в виде ряда по $\varepsilon$:
$$
u=\sum\limits_{k=-1}^{\infty} \varepsilon^k u_k(x,t,\tau,\sigma),
\eqno(4)
$$
где $\displaystyle{u_k(x,t,\tau,\sigma)=X_k(x,t)e^{\tau}+\sum\limits_{i=0}^{p-1} Z^i(x,t)\sigma_i +W_k(x,t)}$.

Подставляя (4) в (3), получим серию итерационных задач, которые запишем покомпонентно:
$$
\left\{ \begin{array}{l}
\smallskip
\displaystyle{\frac{\partial X_k(x,t)}{\partial t}=\frac{\partial}{\partial x}\left( k(x)\frac{\partial X_{k}(x,t)}{\partial x}\right),}\\
\displaystyle{\frac{\partial Z_k^i(x,t)}{\partial t}=\frac{\partial}{\partial x}\left( k(x)\frac{\partial Z^i_{k}(x,t)}{\partial x}\right), \ \ i=\overline{0,(p-1)},}\\
\displaystyle{t^{m/n}a(t)W_k(x,t)=-\frac{\partial W_{k-1}(x,t)}{\partial t} }+\\
\displaystyle{+\frac{\partial}{\partial x}\left(k(x)\frac{\partial W_{k-1}(x,t)}{\partial x}\right)-
\sum\limits_{i=0}^{p-1} t^{\frac{i+1}{n}-1} Z^i_{k-1}(x,t)+\delta_0^k h(x,t),}\\
X_k(x,0)+W_k(x,0)=\delta_k^0 f(x), \ \ k=\overline{-1,\infty}; \\
\text{если индекс} \ (k-1)\le -2, \ \text{то слагаемые}\\
\ \ \text{по определению равны 0}.\\
\end{array} \right.
\eqno(5)
$$

Для решения итерационных задач используется теорема о разрешимости. Рассмотрим систему (5) при $k=-1$:
$$
\varepsilon^{-1}:\ \ \left\{ \begin{array}{l}
\displaystyle{\frac{\partial X_{-1}(x,t)}{\partial t}=\frac{\partial}{\partial x}\left( k(x)\frac{\partial X_{-1}(x,t)}{\partial x}\right),}\\
\displaystyle{\frac{\partial Z_{-1}^i(x,t)}{\partial t}=\frac{\partial}{\partial x}\left( k(x)\frac{\partial Z^i_{-1}(x,t)}{\partial x}\right), \ \ i=\overline{0,(p-1)},}\\
t^{m/n}a(t)W_{-1}(x,t)\equiv 0,\\
\displaystyle{X_{-1}(x,0)+W_{-1}(x,0)=0}.\\
\end{array} \right.
\eqno(6)
$$
Из начальных условий при $k=-1$ системы (6) следует, что
$$
\left\{ \begin{array}{l}
X_{-1}(x,t)\equiv 0,\\
Z_{-1}^i(x,t) \text{ --- произвольное решение уравнения,}\\
\ \ \ i=\overline{0,(p-1)},\\
W_{-1}(x,t)\equiv 0.
\end{array} \right.
\eqno(7)
$$
Функции $Z_{-1}^i(x)$ найдём из условия разрешимости системы при~$\varepsilon^0$:
$$
\varepsilon^0: \ \ \left\{ \begin{array}{l}
\smallskip
\displaystyle{\frac{\partial X_0(x,t)}{\partial t}=\frac{\partial}{\partial x}( k(x)\frac{\partial X_{0}(x,t)}{\partial x})},\\
\displaystyle{\frac{\partial_t Z_0^i(x)}{\partial t}=\frac{\partial}{\partial x}\left( k(x)\frac{\partial Z^i_{-1}(x,t)}{\partial x}\right), \ \ i=\overline{0,(p-1)},}\\
\displaystyle{t^{m/n}a(t)W_0(x,t)=h(x,t)-\sum\limits_{i=0}^{p-1} t^{\frac{i+1}{n}-1} Z^i_{-1}(x,t),}\\
X_0(x,0)+W_0(x,0)=f(x).\\
\end{array} \right.
\eqno(8)
$$

Разложим $h(x,t)$ по формуле Маклорена в точке $t=0$ по $t$:
$$
\begin{array}{c}
\displaystyle{h(x,t)=h(x,0)+t\frac{\partial h(x,0)}{\partial t}+\ldots+ \frac{1}{\left[\frac{m}n\right]!}t^{\left[\frac{m}n\right]}\, \frac{\partial^{\left[\frac{m}n\right]}h(x,0)}{\partial t^{\left[\frac{m}n\right]}} +}\\
\displaystyle{+t^{\left[\frac{m}n\right]+1} h_0(x,t).}
\end{array}
$$
Из теоремы разрешимости уравнения для $W_0(x,t)$ следует,
что если $i=n(j+1)-1$, $j=\overline{0,\left[\frac{m}n\right]}$, то положим
$$
Z_{-1}^{n(j+1)-1}(x,0)=\frac{1}{j!}\, \frac{\partial^j h(x,0)}{\partial t^j},
$$
здесь $\left[\frac{m}n\right]$ "--- целая часть.
Если $i\ne n(j+1)-1$, $i=\overline{0,(p-1)}$, то положим
$$
Z_{-1}^i(x,0)=0.
$$

Таким образом, функции $Z_{-1}(x,t)$ являются решениями задач Коши
$$
\left\{\begin{array}{l}
\displaystyle{\frac{\partial Z_{-1}^{^{n(j+1)-1}}(x,t)}{\partial t}=\frac{\partial}{\partial x}\left( k(x)\frac{\partial Z^{^{n(j+1)-1}}_{-1}(x,t)}{\partial x}\right), \ \ j=\overline{0,\left[\frac{m}n\right]},}\\
\displaystyle{Z_{-1}^{n(j+1)-1}(x,0)=\frac{1}{j!}\, \frac{\partial^j h(x,0)}{\partial t^j}}.
\end{array}\right.
$$
Остальные функции $Z^i_{-1}(x,t)$ с индексами  $i\ne n(j+1)-1$, \linebreak $i=\overline{0,(p-1)}$ будут тождественно равны $Z_{-1}^i(x,t)\equiv 0,$  так как $Z_{-1}^i(x,0)=0.$

В результате после сужения на регуляризирующие функции получим решение на <<$-1$>> шаге
$$
u_{-1}(x,t,\varepsilon)=\sum\limits_{j=0}^{\left[\frac{m}n\right]} Z_{-1}^{n(j+1)-1}(x,t)\sigma_{n(j+1)-1}(t,\varepsilon).
\eqno(9)
$$

Система (8) имеет решения:

a) $X_0(x,t)$ "--- решение задачи Коши
$$
\left\{\begin{array}{l}
\displaystyle{\frac{\partial X_{0}(x,t)}{\partial t}=\frac{\partial}{\partial x}\left( k(x)\frac{\partial X_{0}(x,t)}{\partial x}\right)},\\
\displaystyle{X_0(x,0)=f(x);}
\end{array}\right.
$$

б) $Z^i_0(x,t)$ "--- на данном этапе произвольное решение уравнения теплопроводности
$$
\frac{\partial Z^i_{0}(x,t)}{\partial t} =\frac{\partial}{\partial x}\left( k(x)\frac{\partial Z^i_{0}(x,t)}{\partial x}\right), \ \ i=\overline{0,p-1};
$$

в) $\displaystyle{W_0(x,t)=\frac{\!h(x,t)-\!\sum\limits_{j=0}^{\left[\frac{m}n\right]} t^j Z_{-1}^{n(j+1)-1}(x,t)\!}{t^{m/n}a(t)}=t^{1-\left\{\frac{m}{n}\right\}}\widetilde{h}_0(x,t),}$
где $\widetilde{h}_0(x,t)$ "--- гладкая функция.

Для определения начальных условий $Z_0^i(x,0)$ рассмотрим итерационную
систему (5) на шаге <<1>>:
$$
\left\{ \begin{array}{l}
\smallskip
\displaystyle{\frac{\partial X_{1}(x,t)}{\partial t}=\frac{\partial}{\partial x}\left( k(x)\frac{\partial X_{1}(x,t)}{\partial x}\right),}\\
\displaystyle{\frac{\partial Z^i_{1}(x,t)}{\partial t}=\frac{\partial}{\partial x}\left( k(x)\frac{\partial Z^i_{1}(x,t)}{\partial x}\right), \ \ i=\overline{0,(p-1)},}\\
\displaystyle{t^{m/n}a(t)W_1(x,t)=-\frac{\partial W_{0}(x,t)}{\partial t}+\frac{\partial}{\partial x}\left( k(x)\frac{\partial W_0(x,t)}{\partial x}\right)}-\\
\ \ -\displaystyle{\sum\limits_{i=0}^{p-1} t^{\frac{i+1}{n}-1} Z^i_{0}(x,t),}\\
X_1(x,0)+W_1(x,0)=0.\\
\end{array} \right.
\eqno(10)
$$
Для определения $Z_0^i(x,0)$ подчиним уравнение относительно $W_1(x,t)$ условию разрешимости. Для этого разложим $\displaystyle{-\frac{\partial W_0(x,t)}{\partial t}+}$ \linebreak $\displaystyle{+\frac{\partial}{\partial x}\left( k(x)\frac{\partial W_0(x,t)}{\partial x}\right)} $ по формуле Маклорена в точке $t=0$. Предварительно вычислим
$$
\begin{array}{c}
\displaystyle{\frac{\partial W_0(x,t)}{\partial t}-\frac{\partial}{\partial x}\left( k(x)\frac{\partial W_0(x,t)}{\partial x}\right) =}\\
\displaystyle{=\left( 1-\left\{\frac{m}{n}\right\}\right) t^{-\{\frac{m}{n}\}} \widetilde{h}_0(x,t)+t^{1-\{\frac{m}{n}\}} \frac{\partial \widetilde{h}_0(x,t)}{\partial t}-}\\
\displaystyle{-t^{1-\{\frac{m}{n}\}}\frac{\partial}{\partial x}\left( k(x)\frac{\partial \widetilde{h}_0(x,t)}{\partial x}\right)
=-t^{-\left\{\frac{m}{n}\right\}} h_1(x,t).}
\end{array}
$$
Тогда получим
\begin{multline*}
-\frac{\partial W_0(x,t)}{\partial t}+\frac{\partial}{\partial x}\left( k(x)\frac{\partial W_0(x,t)}{\partial x}\right)=\\
=t^{-\left\{\frac{m}{n}\right\}} h_1(x,0)+ t^{1-\left\{\frac{m}{n}\right\}} \frac{\partial h_1(x,0)}{\partial t}+\ldots+ \\
\displaystyle{+\frac{t^{k-\left\{\frac{m}{n}\right\}}}{k!}\,
\frac{\partial^k h_1(x,0)}{\partial t^k}+ t^{k+1-\left\{\frac{m}{n}\right\}} \widetilde{h}_1(x,t)},
\end{multline*}
где $k=\left[ \frac{m}{n}+\left\{\frac{m}{n}\right\}\right],$ $\left\{\frac{m}{n}\right\}=\frac{s}{n}$, $1\le s\le n-1$. Если $\frac{m}{n}+\left\{\frac{m}{n}\right\}$ "--- целое, то $k=\frac{m}{n}+\left\{\frac{m}{n}\right\}-1$.

Для удовлетворения условиям разрешимости возможны следующие
случаи:

а) если $j-\left\{\frac{m}{n}\right\}=\frac{i+1}{n}-1$, т.е. $i=n(j+1)-n\{\frac{m}{n}\}-1$, $j=\overline{0,k}$, положим
\begin{equation*}
Z_0^i(x,0)=\frac{1}{j!}\, \frac{\partial^j h_1(x,0)}{\partial t^j},
\end{equation*}
отсюда $Z_0^i(x,t)$ являются решениями задач Коши
$$
\left\{ \begin{array}{l}
\displaystyle{\frac{\partial Z^i_{0}(x,t)}{\partial t}=\frac{\partial}{\partial x}\left( k(x)\frac{\partial Z^i_{0}(x,t)}{\partial x}\right)}\\
\displaystyle{Z_0^i(x,0)=\frac{1}{j!}\, \frac{\partial^j h_1(x,0)}{\partial t^j};}
\end{array}\right.
\eqno(11)
$$

б) если $i\ne n(j+1)-n\{\frac{m}{n}\}-1$, $j=\overline{0,k}$, положим
$Z_0^i(x,0)=0.$

Следовательно в этом случае $Z_0^i(x,t)\equiv 0.$

На данном шаге мы определили слагаемое $u_0(x,t)$, а следовательно после
сужения на регуляризирующие функции  главный член асимптотики имеет
вид
$$
\begin{array}{c}
\displaystyle{u_0(x,t)=f(x)e^{-\varphi(t)/\varepsilon}+}\\
\displaystyle{+\sum\limits_{j=0}^{k} Z_0^{n(j+1)-n\{\frac{m}{n}\}-1}(x,t) \sigma_{n(j+1)-n\{\frac{m}{n}\}-1}(t,\varepsilon)+ }\\
\displaystyle{+\frac{h(x,t)-\sum\limits_{j=0}^{\left[\frac{m}{n}\right]}t^j Z_{-1}^{n(j+1)-1}(x,t)}{t^{m/n}a(t)}.}
\end{array}
\eqno(12)
$$
Главный член асимптотики запишется в виде суммы
$$
\begin{array}{c}
\displaystyle{u_{\text{гл}}(x,t)=\frac1{\varepsilon}\sum\limits_{j=0}^{\left[\frac{m}n\right]} Z_{-1}^{n(j+1)-1}(x,t)\sigma_{n(j+1)-1}(t,\varepsilon)+}\\ \displaystyle{+\sum\limits_{j=0}^{k} Z_0^{n(j+1)-n\{\frac{m}{n}\}-1}\sigma_{n(j+1)-n\{\frac{m}{n}\}-1}(t,\varepsilon)} + f(x)e^{-\varphi(t)/\varepsilon}+ \\
+ \displaystyle{t^{1-\left\{\frac{m}{n}\right\}} \widetilde{h}_0(x,t).}
\end{array}
\eqno(13)
$$

Решение на <<1>> шаге примет вид
$$
\left[ \begin{array}{l}
\smallskip
\displaystyle{W_1(x,t)=\frac{t^{-\left\{\frac{m}{n}\right\}} h_1(x,t)-\sum\limits_{i=0}^{p-1} t^{\frac{i+1}{n}-1} Z^i_{0}(x,t)}{t^{m/n}a(t)}=}\\
\ \ \ =t^{\{1-\{2\{\frac{m}{n}\}\}\}}h_1(x,t),\\
Z_1^i(x,t) \text{ --- общее решение уравнения теплопроводности},\\
\displaystyle{\frac{\partial Z^i_{1}(x,t)}{\partial t}=\frac{\partial}{\partial x}\left( k(x)\frac{\partial Z^i_{1}(x,t)}{\partial x}\right), \ \ i=\overline{0,(p-1)}}.
\end{array} \right.
\eqno(14)
$$
Отсюда следует, что если:

1) $2\left\{\frac{m}{n}\right\}<>1$, то $W_1(x,0)=0$ и $X_1(x,0)=0$, следовательно $X_1(x,t)\equiv 0$;

\smallskip
2) $2\left\{\frac{m}{n}\right\}=1$, то $W_1(x,0)=h_1(x,0)$ и $X_1(x,0)=-h_1(x,0)$, следовательно $X_1(x,t)$ "--- решение задачи Коши
$$
\left\{ \begin{array}{l}
\smallskip
\displaystyle{\frac{\partial X_{1}(x,t)}{\partial t}=\frac{\partial}{\partial x}\left( k(x)\frac{\partial X_{1}(x,t)}{\partial x}\right),}\\
X_1(x,0)=-h_1(x,0).
\end{array}\right.
$$

Произвольные функции $Z_1^{i}(x,t)$ находятся из условия разрешимости системы (5) при $k=2$.

По данной схеме можно определить любой член асимптотического регуляризованного ряда.

Пусть решены $(N+1)$ итерационных задач. Тогда решение задачи Коши после сужения на регуляризирующие функции можно представить в виде
$$
u(x,t,\varepsilon)=\sum\limits_{k=-1}^{N} u_k(x,t,\varepsilon) \varepsilon^k+\varepsilon^{N+1} R_N(x,t,\varepsilon),
\eqno(15)
$$
где $\displaystyle{u_k(x,t,\varepsilon)=X_k(x,t)e^{-\frac{\varphi(t)}{\varepsilon}}+\sum\limits_{i=0}^{p-1} Z^i(x,t)\sigma_i(t,\varepsilon) +W_k(x,t)}.$

Подставив (15) в (1) и учитывая, что $u_k(x,t,\varepsilon)$ являются решениями итерационных задач, получим задачу Коши для определения остатка $R_N(x,t,\varepsilon)$:
$$
\left\{ \begin{array}{l}
\displaystyle{L(R)=\varepsilon\left(\frac{\partial R_N(x,t)}{\partial t}- \frac{\partial}{\partial x} \left( k(x)\frac{\partial R_N(x,t)}{\partial x}\right)\right) +}\\
\ \ \ +t^{m/n} a(t)R_N=H(x,t,\varepsilon),\\
R_N(x,0,\varepsilon)=0, \ \ -\infty<x<+\infty,\\
\end{array} \right.
\eqno(16)
$$
где
\begin{multline*}
H(x,t)=-\frac{\partial W_N(x,t)}{\partial t} +\frac{\partial}{\partial x} \left( k(x)\frac{\partial W_{N}(x,t)}{\partial x}\right) -
\\-
\sum\limits_{i=0}^{p-1} t^{\frac{i+1}{n}-1} Z_N^i(x,t)=-q(t)W_{N+1}(x,t),
\end{multline*}

\textbf{Теорема} (оценка остаточного члена). {\it Пусть выполнены условия:}

{\it 1) условия $1)\div 5)$ задачи Коши $(1)$};

{\it 2) $\exists M_1>0$ $\forall(x,t)\in(-\infty,+\infty)\times[0,T]$ $\forall\varepsilon\in(0,\varepsilon_0]$ \\
$|H(x,t,\varepsilon)|\le M_1$;}

{\it 3) $\exists M>0$ $0<k_0\leq k(x)<M(x^2+1),$ $|k'(x)|<M \sqrt{x^2+1}$.}

\noindent {\it Тогда $\exists \mathbb{C}>0$ $\forall(x,t)\in(-\infty,+\infty)\times[0,T]$ $\forall\varepsilon\in(0,\varepsilon_0]$ $|R_n(x,t,\varepsilon)|\le \mathbb{C}.$ }

% Оформление списка литературы
\litlist


1. {\it Тихонов А.Н.} О зависимости решений дифференциальных уравнений от малого параметра~/ А.Н.~Тихонов~// Математический сборник.~--- 1948.~--- Т.~22(64), №~2.~--- С.~193--204.

2. {\it Маслов В.П.} Теория возмущений и асимптотические методы~/ В.П.~Маслов.~--- М.~: МГУ, 1965.~--- 554~с.

3. {\it Васильева А.Б.} Асимптотические разложения решений сингулярно возмущённых уравнений~/ А.Б.~Васильева, В.Ф.~Бутузов.~--- М.~: Наука, 1973.~--- 272~с.

4. {\it Ломов С.А.} Введение в общую теорию сингулярных возмущений~/ С.А.~Ломов. "--- М.~: Наука, 1981.~--- 400~с.

5. {\it Елисеев А.Г.} Теория сингулярных возмущений в случае спектральных особенностей предельного оператора~/ А.Г.~Елисеев, \linebreak С.А.~Ломов~// Математический сборник.~--- 1986.~--- Т.~131, №~4.~--- С.~544--557.

6. {\it Елисеев А.Г.} Об аналитических решениях по параметру сингулярно возмущённого уравнения при наличии простейшей точки поворота у предельного оператора~/ А.Г.~Елисеев~// Вестник МЭИ.~--- 1995.~--- №~6.~--- С.~41--47.

7. {\it Ильин А.М.} Линейные уравнения второго порядка параболического типа~/ А.М.~Ильин, А.С.~Калашников, О.А.~Олейник~// УМН.~--- 1962.~--- Т.~17, вып.~3(105).--- С.~3--116.

8. {\it Ратникова Т.А.} \foreignlanguage{english}{Singularly Perturbed Cauchy Problem for a Parabolic Equation with a Rational <<Simple>> Turning Point}~/ Т.А.~Ратникова~// \foreignlanguage{english}{Axioms}.~--- 2020.~--- №~9, 138.~--- http://doi.org/10.3390/ axioms9040138.

