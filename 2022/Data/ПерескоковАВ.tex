\vzmstitle[
	\footnote{Результаты получены в рамках выполнения государственного задания Минобрнауки России
(проект {FSWF}-2020-0022) }
]{
Об	асимптотике спектра оператора типа Хартри с
экранированным кулоновским потенциалом самодействия
вблизи верхних границ спектральных кластеров
}
\vzmsauthor{Перескоков}{А.\,В.}
\vzmsinfo{ Москва, НИУ МЭИ, НИУ ВШЭ; {\it pereskokov62@mail.ru }}

\vzmscaption

Рассмотрим задачу на собственные значения для нелинейного
оператора типа Хартри в  пространстве $L^2(\mathbb{R}^3)$
\begin{equation}
( -{\Delta}_q-\frac {1}{\mid q \mid}+ \varepsilon \int_{\mathbb{R}^3}  e^{-\varkappa \mid q-q' \mid}\frac {\mid \psi(q') \mid^2}
{\mid q-q' \mid}dq')\psi=\lambda\psi,
\end{equation}
\begin{equation}
\|\psi\|_{L^2(\mathbb{R}^3)}=1,
\end{equation}
где  ${\Delta}_q $ --- оператор Лапласа, $\varepsilon > 0 $ ~--- малый параметр. Здесь потенциал
самодействия является экранированным кулоновским потенциалом ( или потенциалом Юкавы  )
$$
V(q)=\frac {e^{-\varkappa \mid q \mid}}{ \mid q \mid},
$$
где $\varkappa > 0 $  --- константа.

Уравнения самосогласованного поля Хартри во внешнем поле, а также уравнения типа Хартри, в которых потенциал
самодействия отличен от кулоновского, возникают в ряде моделей квантовой теории и нелинейной оптики.
Изучению асимптотических решений таких уравнений,  локализованных вблизи маломерных
инвариантных подмногообразий в фазовом пространстве, посвящено большое число работ начиная с работ В.~П.~Маслова [1]  и И.~В.~Сименога [2].
Отметим также работу [3], где были найдены асимптотические собственные значения
оператора Хартри  с кулоновским взаимодействием в пространстве $L^2(\mathbb{R}^3)$
вблизи верхних границ спектральных кластеров. Эти кластеры
образуются около уровней энергии  невозмущенного оператора.  Соответствующие асимптотические собственные функции
локализованы вблизи окружности   $ \Gamma $ в $\mathbb{R}^3$, на которой кулоновский потенциал самодействия имеет
логарифмическую особенность.  В данной работе будут
построены асимптотические собственные функции задачи  (1), (2) при  $\varkappa > 0 $, которые также локализованы вблизи
окружности $ \Gamma $.

 При $\varepsilon = 0 $ собственные значения
 $\lambda=\lambda_ n (\varepsilon) $   задачи (1), (2) имеют вид
 $$
\lambda_ n (0)=-\frac {1}{4n^2}, \quad  n=1,2,\dots ,
$$
где $ n $ --- главное квантовое число. Пусть теперь $\varepsilon $ не равно нулю.
 Рассмотрим случай, когда число $n$ велико.
Для определенности будем считать, что $\lambda$ имеет порядок
 $\varepsilon$. Тогда $n$  имеет порядок    $\varepsilon^{-1/2}$.

Пусть $p=n- |m| -1$, где $m$ --- магнитное квантовое число. В данной работе  для каждого фиксированного
$ p=0,1,2,\dots $  найдены асимптотические собственные значения
$$
\lambda _ {n,i}^{(p)} (\varepsilon)= -\frac {1}{4n^2}+\frac{\varepsilon  \alpha_{i}^{(p)} }
{32 \pi \varkappa^2 n^5}+O\left(\frac {\varepsilon }{n^{6}}\right), \quad  n \to \infty ,
$$
вблизи верхних границ спектральных кластеров.
Здесь $i$ принимает конечное число значений  $i=0, \dots, I_{p}$. В частности, при  $p=0$ существует одно число
$
\alpha_{0}^{(0)}=4,
$
при $p=1$ --- два числа
$
\alpha_{0}^{(1)}=3, \quad \alpha_{1}^{(1)}=2,
$
при $p=2$ --- шесть чисел
$
\alpha_{0}^{(2)}=2+4/7, ~ \alpha_{1}^{(2)}=2+1/4, ~ \alpha_{2}^{(2)}=2, ~ \alpha_{3}^{(2)}=2-4/13,
~ \alpha_{4}^{(2)}=2-7/20,~ \alpha_{5}^{(2)}=2-4/9.
$

Наконец,  вблизи окружности $ \Gamma $, где локализовано решение (1), (2),
главный член его асимптотического разложения $ \gamma = \gamma _i^{(p)}$ является
 решением  задачи о двумерном осцилляторе: [4]
$$
{\bf L}\gamma (\tau,s)=0, \quad  \| \gamma \|_{L^2(\mathbb{R}^2)}=1.
$$
Здесь оператор ${\bf L}$ имеет вид
\begin{equation*}
 {\bf L}= -\frac{1}{2}\left(\frac {\partial ^{2}} {\partial s^{2} } + \frac {\partial ^{2}  } {\partial \tau ^{2} }\right)
 +\frac {s^2+\tau^2}{2}-(p+1).
 \label{9}
\end{equation*}

Отметим, что ранее в работе [5] для уравнения Хартри в случае ньютоновского взаимодействия с экранировкой
 была построена квазиклассическая серия собственных значений и сферически-симметричных
собственных функций.


\litlist

1. {\it Маслов В.П.} Комплексный метод ВКБ в нелинейных уравнениях. //М.: Наука, 1977. 384 с.

2. {\it Сименог И. В.} Об асимптотике решения стационарного нелинейного уравнения Хартри //Теоретическая
и математическая физика. – 1977. – Т. 30. – №. 3. – С. 408-414.

3. {\it Перескоков А. В.} Асимптотика спектра оператора Хар\-три вблизи верхних границ
спектральных кластеров. Асимптотические решения, локализованные вблизи окружности //Теоретическая
и математическая физика. – 2015. – Т. 183. – №. 1. – С. 78-89.

4. {\it Перескоков А. В.} Асимптотика спектра оператора типа Хартри с
экранированным кулоновским потенциалом самодействия
вблизи верхних границ спектральных кластеров  //Теоретическая
и математическая физика. – 2021. – Т. 209. – №. 3. – С. 543-560.

5. {\it Карасев М. В., Маслов В. П.} Квазиклассические солитонные решения уравнения Хартри. Ньютоновское
взаимодействие с экранировкой //Теоретическая
и математическая физика. – 1979. – Т. 40. – №. 2. – С. 235-244.

