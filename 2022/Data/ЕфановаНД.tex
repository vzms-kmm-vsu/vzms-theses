\vzmstitle[
]{
	Расчёт локализованных состояний в модели решётки Дирака
}
\vzmsauthor{Даринский}{Б.\,М.}
\vzmsinfo{Воронеж, ВГУ; {\it darinskii@mail.ru}}
\vzmsauthor{Сайко}{Д.\,С.}
\vzmsinfo{Воронеж, ВГТУ; {\it dmsajko@mail.ru}}
\vzmsauthor{Ефанова}{Н.\,Д.}
\vzmsinfo{Воронеж, ВГУ; {\it efanowanatalia@gmail.com}}
\vzmscaption

В докладе изложены результаты исследований условий возникновения локализованных собственных функций оператора Гамильтона
для полубесконечной решетки Дирака, а также этой решетки с измененным периодом в одной из ячеек. Рассматривается дифференциальное уравнение 
\begin{equation} \label{Darinskiy_eq1}
 H\Psi(x)=E\Psi(x),
\end{equation}
где оператор $H=-\frac{1}{2}\frac{d^2}{dx^2}+V(x)$, функция $V(x)=\sum\limits_nV\delta(x-n)$, $n=0,1,2,3...$ для $x\geq0$ и 
функция $V(x)=U$, где $U>0$ для $x<0$, $\Psi(x)\in C^2$ - вещественная функция,
$\delta(x)$ - функция Дирака. Периодическая компонента $u(k,x)$ решения $\Psi=e^{-kx}u(k,x)$ этого уравнения  для $E=\frac{q^2}{2}>0$ записывается в виде:
\begin{equation} \label{Darinskiy_eq2}
\begin{array}{c}
u(k,x)=(A\cos(qx)+B\sin(qx))e^{kx}\;\text{для}\;x>0,\\
\Psi(x)=Ce^{\varkappa x}\;\text{для}\;x<0.
\end{array}
\end{equation} 
Оно соответствует локализованной функции вблизи края периодической структуры.
\par Из граничных условий при $x=1$ получаются соотношения [1]:
\begin{gather}
B=\frac{e^{-k}-\cos(q)}{\sin(q)}A,\\
\ch(k)=\cos(q)+V\frac{\sin(q)}{q}. \label{Darinskiy_eq4}
\end{gather}
Из граничных условий при $n=0$ находим:
\begin{equation}\label{Darinskiy_eq5}
e^{-k}=\cos(q)+\left(\varkappa+2V\right)\frac{\sin(q)}{q}.
\end{equation}
После умножения \eqref{Darinskiy_eq4} на -2 и последующего сложения с \eqref{Darinskiy_eq5} получается 
\begin{equation} \label{Darinskiy_eq6}
 e^{k}=\cos(q)-V\frac{\sin(q)}{q}.
\end{equation}
Далее, перемножая правые и левые части \eqref{Darinskiy_eq5} и \eqref{Darinskiy_eq6}, получим
\begin{equation} \label{Darinskiy_eq7}
V=\frac{U}{q\ctg(q)-\sqrt{2U-q^2}}.
\end{equation}
Формула \eqref{Darinskiy_eq7}  представляет собой неявную зависимость собственного значения $E$ от параметров
системы $U$ и $V$. Она показывает, что для малых $q$ параметр $V<0$.При больших величинах $q$ и $U>\pi$ величина $V$
может иметь положительные и отрицательные значения, при этом зависмость $q(U,V)$ становится
неоднозначной. Поэтому в этих условиях появляются  дополнительные локализованные состояния.
\par Для отрицательных значений $E$ решение уравнения \eqref{Darinskiy_eq1} записывается в виде \eqref{Darinskiy_eq4}, в котором тригонометрические функции заменяются гиперболическими. 
В результате получаем
\begin{multline} \label{Darinskiy_eq8}
\ch(k)=\ch(q)+V\frac{\sh(q)}{q},\;V=\frac{U}{q\cth(q)-\sqrt{2U+q^2}}.
\end{multline}
Формула \eqref{Darinskiy_eq8} показывает, что отрицательные значения также приводят к появлению локализованных функций.
\par Далее предложена модель межкристаллитной границы в твердых телах как малое изменение расстояния между соседними потенциальными ямами. Учитывая то обстоятельство, что полученная таким образом зависимость $U(x)$
инвариантна относительно преобразования инверсии, уравнение \eqref{Darinskiy_eq1} дополнено двумя вариантами граничных условий           
$\Psi(a)=0$, $\frac{d\Psi(x)}{dx}=0$ $(a\neq1)$. Первое условие соответствует
нечетной функции, второе — четной. Неявная зависимость энергетичекого параметра $q$ от парметров системы в случае нечетной функции дается формулой
\begin{equation*}
V=\frac{q}{2(\ctg(q)+\ctg(qa))\sin^2(qa)},
\end{equation*}
в случае четной функции
\begin{equation*}
V=\frac{q}{2(\ctg(q)+\tg(qa))\cos^2(qa)}.
\end{equation*}
\litlist

1. {\it Девисон С., Левин Дж.} Поверхностные (таммовские) состояния. //М.: Мир, 1987. 232 с.
