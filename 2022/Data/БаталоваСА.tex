\vzmstitle
{
	РАЗРЕШИМОСТЬ ДИФФЕРЕНЦИАЛЬНО-РАЗНОСТНОЙ СИСТЕМЫ ПАРАБОЛИЧЕСКОГО ТИПА С РАСПРЕДЕЛЕННЫМИ ПАРАМЕТРАМИ НА ГРАФЕ
}
\vzmsauthor{Баталова}{С.\,А.}
\vzmsinfo{Воронеж, ВГУ; {\it s.sonya.batalova@gmail.com}}

\vzmscaption

\paragraph{1. Обозначения, понятия и основные утверждения.} Используются понятия и  обозначения, принятые в [1, 2] и для аппроксимаций на сетке  $\omega_t$:

$\Gamma$ "--- ограниченный ориентированный геометрический граф с одинаковыми рёбрами $\gamma$, параметризованными отрезком $[0,1]$;

$\partial\Gamma$ и $J(\Gamma)$ "--- множества граничных и внутренних узлов графа, соответственно;

$\Gamma_0$ "--- объединение всех рёбер $\gamma_0$, не содержащих концевых точек;

$\Gamma_{T}=\Gamma_0\times(0,T)$, $\partial \Gamma_{T}=\partial \Gamma\times(0,T)$, $T<\infty$ "--- произвольная фиксированная постоянная.

Необходимые пространства и множества:

$L_{p}(\Gamma)$ ($p=1,2$) "--- банахово пространство измеримых на $\Gamma_0$ функций, суммируемых с $p$-й степенью (аналогично определяются пространства $L_{p}(\Gamma_T)$);


$W_{\,2}^{1}(\Gamma)$~--- пространство функций из $L_{2}(\Gamma)$, имеющих обобщённую производную первого порядка также из $L_{2}(\Gamma)$;

Рассмотрим билинейную форму
\[
{{\begin{array}{*{20}c}
\ell(\mu,\nu)=\int\limits_{\Gamma}\left(a(x)\frac{d \mu(x)}{d x}\frac{d \nu(x)}{d x}+b(x)\mu(x)\nu(x)\right)dx.
 \end{array} }}
\]
где $a(x)$, $b(x)$ "--- фиксированные измеримые и ограниченные на $\Gamma_0$ функции, суммируемые с квадратом:
\begin{equation}\label{eq1}
{{\begin{array}{*{20}c}
0<a_\ast\leqslant a(x)\leqslant a^\ast,\,\,|b(x)|\leqslant \beta,\,\,x\in \Gamma_0.
 \end{array} }}
\end{equation}

Имеет место следующее утверждение [2].

\paragraph{Лемма.} \emph{Пусть функция $u(x)\in W^1_{\,2}(\Gamma)$ такова, что $\ell(u,\nu)-\int\limits_{\Gamma}f(x)\eta(x)dx=0$ для любой $\eta(x)\in W^1_{\,2}(\Gamma)$} ($f(x)\in L_2(\Gamma)$ "--- \emph{фиксированная функция}). \emph{Тогда для любого ребра $\gamma\subset\Gamma$ сужение $a(x)_\gamma\frac{du(x)_\gamma}{dx}$ непрерывно в концевых точках ребра $\gamma$.}

Обозначим через $\Omega_a(\Gamma)$ множество функций $u(x)$, удовлетворяющих условиям леммы и соотношениям
\[
{{\begin{array}{*{20}c}
\sum\limits_{\gamma\in R(\xi)}a(1)_\gamma\frac{du (1)_\gamma}{dx}
 =\sum\limits_{\gamma\in r(\xi)}a(0)_\gamma\frac{du (0)_\gamma}{dx}
 \end{array} }}
\]
во всех узлах $\xi\in J(\Gamma)$ (здесь $R(\xi)$ и $r(\xi)$~--- множества рёбер $\gamma$, соответственно ориентированных <<к узлу $\xi$>> и <<от узла $\xi$>>). Замыкание в норме $W^1_{\,2}(\Gamma)$ множества $\Omega_a(\Gamma)$  обозначим через $W^1(a,\Gamma)$. При этом, если допустить, что функции $u(x)$ из $\Omega_a(\Gamma)$ удовлетворяют ещё и краевому условию $u(x)|_{\partial\Gamma}=0$, то получим пространство $W^1_{\,0}(a,\Gamma)$.


В работах [1, 2] при тех же условиях, что и выше, рассматривались начально"=краевые задачи, где были установлены условия, гарантирующие их однозначную слабую разрешимость в классе суммируемых функций, используя метод Фаэдо-Галеркина со специальным базисом "--- множеством обобщённых собственных функций класса $W_{\,0}^{1}(a,\Gamma)$ одномерного эллиптического оператора, порождённого дифференциальным выражением ${\Lambda}u=-\frac{d }{d x }\left(a(x)\frac{d u(x)}{d x}\right)+b(x)u(x)$. А именно (см. работу [3]), при условиях (1) рассматривалась спектральная задача ${\Lambda}\phi=\lambda \phi$, $\phi|_{\partial\Gamma}=0$, в классе $W_{\,0}^{1}(a,\Gamma)$ "--- задача определения множества чисел $\lambda$, каждому из которых соответствует по крайней мере одно нетривиальное решение $\phi(x)\in W_{\,0}^{1}(a,\Gamma)$, удовлетворяющее тождеству $\ell(\phi,\nu)=\lambda (\phi,\nu)$  при любой функции $\nu(x)\in W_{\,0}^{1}(a,\Gamma)$. Для этой задачи установлены следующие свойства.

1. \emph{Собственные значения  вещественные и имеют конечную кратность, их можно занумеровать в порядке возрастания модулей с учётом кратностей: $\{\lambda_k\}_{k\geqslant 1}$; соответственно нумеруются и обобщённые собственные функции: $\{\phi_k(x)\}_{k\geqslant 1}$.}

2. \emph{Собственные значения $\lambda_k$ положительны, за исключением, может быть, конечного числа первых; если $b(x)\geqslant0$, тогда собственные значения неотрицательны.}

3. \emph{Система обобщённых собственных функций $\{\phi_k(x)\}_{k\geqslant 1}$ образует ортогональный базис в пространстве $W_{\,0}^{1}(a,\Gamma)$ и пространстве $L_2(\Gamma)$ (везде ниже $\|\phi_k\|_{L_2(\Gamma)}=1$). }

4. \emph{Для задачи } ${\Lambda}\phi=\lambda \phi+g$, $g\in L_2(\Gamma)$, \emph{имеет место альтернатива Фредгольма}.



В представленной работе используется метод Галеркина с базисом $\{\phi_k(x)\}_{k\geqslant 1}$  при доказательстве теоремы существования слабого решения дифференциально"=разностной системы (см. ниже утверждение теоремы 3), а также указывается метод отыскания приближенного решения и пути анализа устойчивости решения.

\

\paragraph{2. Дифференциально"=разностная параболическая система.}
В пространстве $W^1_{\,0}(a,\Gamma)$  рассмотрим дифференциально"=разностное уравнение
\begin{equation}\label{eq2}
{{\begin{array}{*{20}c}
\frac{1}{\tau}(u(k)-u(k-1))-\frac{d }{d x }\left(a(x)\frac{d u(k)}{d x}\right)+b(x)u(k)=f(k),
 \end{array} }}
\end{equation}
где $f(k) =f(x,k)\in L_2(\Gamma)$, $k=1,2,...,M$.


Функции  $u(k)$ ($k=1,2,...,M$) определим как слабые решения эллиптических уравнений (2), удовлетворяющие условиям
\begin{equation}\label{eq3}
{{\begin{array}{*{20}c}
u(0)=\varphi(x)\in L_2(\Gamma), \quad u(k)\mid_{x\in\partial\Gamma}=0.
 \end{array} }}
\end{equation}
Таким образом, для фиксированного $k$, $k=1,2,...,M$, соотношения (2), (3)~--- краевая задача для  эллиптического уравнения (2) относительно $u(k)$.

\paragraph{Определение.} \emph{Слабым решением дифференциально"=разностного уравнения} (2) с условиями (3)  \emph{называются функции}  $u(k) \in W^1_{\,0}(a,\Gamma)$, $(k=1,2,...,M)$, \emph{удовлетворяющие интегральному тождеству}
\[
{{\begin{array}{*{20}c}
		\int\limits_{\Gamma}u(k)_t\,\nu(x)dx+\ell(u(k),\nu(x)=
		\int\limits_{\Gamma}f(k)\nu(x) dx,
		\end{array} }}
\]
\emph{для любой функции $\nu(x)\in W^1_{\,0}(a,\Gamma)$; равенство $u(0)=\varphi(x)$ понимается почти всюду, $u(k)_t \equiv u(x,k)_t=\frac{1}{\tau}[u(k)-u(k-1)]$.}


\paragraph{Теорема 1.} \emph{При выполнении условий }(1) \emph{функции $u(k)$  $(k=1,2,...,M)$ при достаточно малых $\tau$ ($\tau<\tau_0$) однозначно определяются как элементы пространства $W^1_{\,\,0}(a,\Gamma)$.}

\paragraph{Доказательство.}
Пусть $k=1$. Исходя из свойств  2 и 4 краевой задачи ${\Lambda}\phi=\lambda \phi$, $\phi|_{\partial\Gamma}=0$, и соотношения
\[
{{\begin{array}{*{20}c}
{\Lambda}u(1)=-\frac{1}{\tau}u(1)+f_\tau(1)+\frac{1}{\tau}u(0),\quad u(0)=\varphi(x)\in L_2(\Gamma),
\end{array} }}
\]
следует утверждение теоремы для функции $u(1)$. Это\linebreak
же утверждение, учитывая соотношение
\[
{{\begin{array}{*{20}c}
{\Lambda}u(k)=-\frac{1}{\tau}u(k)+f_\tau(k)+\frac{1}{\tau}u(k-1),\\
\quad u(0)=\varphi(x)\in L_2(\Gamma),
\end{array} }}
\]
остаётся справедливым и при $k=2,3,...,M$. Ниже, при получении априорной оценки для $u(k)$ будет указана граница $\tau_0$ для изменения $\tau$. Лемма доказана.


Для слабого решения $u(k)$, $k=1,2,...,M$ дифференциально"=разностной системы (2), (3) имеет место априорная оценка, не зависящая от $\tau$.

\paragraph{Теорема 2.} \emph{Пусть выполнены условия} (1) \emph{и $\varphi(x)\in L_2(\Gamma)$. При  $\tau\leqslant \tau_0<\frac{1}{2\varrho}$ и любом $k=1,2,...,M$ для функций $u(k)$ справедливы оценки}
\begin{equation}\label{eq4}
{{\begin{array}{*{20}c}
\|u(k)\|_{2,\Gamma}\leqslant e^{2\varrho T}\left(\|\varphi\|_{2,\Gamma}+2\|f(k)\|_{2,1,\Gamma}\right).
\end{array} }}
\end{equation}
\begin{equation}\label{eq5}
{{\begin{array}{*{20}c}
\|u(m)\|^2_{2,\Gamma}+2a_*\tau\sum\limits_{k=1}^m\|\frac{du(k)}{dx}\|^2 +\tau^2\sum\limits_{k=1}^m\|u(k)_t\|^2_{2,\Gamma}\leqslant\\
\leqslant C(\|\varphi\|^2_{2,\Gamma} + \|f(m)\|^2_{2,1,\Gamma}),
\end{array} }}
\end{equation}
\emph{где постоянная зависит только от $a_*$, $\beta$ и $T$; здесь через $\|\,\cdot\,\|_{2,\Gamma}$ обозначена норма в пространстве} $L_2(\Gamma)$, $\|f(m)\|_{2,1,\Gamma}=\tau\sum\limits_{k=1}^m \|f(k)\|_{2,\Gamma}$.

Априорные оценки (4) и (5) дают возможность установить следующее утверждение.

\paragraph{Теорема 3.} \emph{В условиях теоремы} 1 \emph{существует слабое решение дифференциально"=разностной системы } (2), (3) \emph{в пространстве $W^1_{\,\,0}(a,\Gamma)$ при достаточно малом $\tau$}.

\litlist

1. \emph{Хоанг В. Н.}  Дифференциально"=разностная краевая задача для параболической системы с распределёнными параметрами на графе / В. Н. Хоанг // Процессы управления и устойчивость. -- 2020. -- Т. 7. -- № 1. -- С.127--132.

2. \emph{Провоторов В. В.} Синтез оптимального граничного управления параболической системы с запаздыванием и распределёнными параметрами на графе / В. В. Провоторов, Е. Н. Провоторова // Вестник Санкт--Петербургского университета. Прикладная математика. Информатика. Процессы управления. -- 2017. -- Т. 13.-- № 2. -- С. 209--224.

\selectlanguage{english}

3. \emph{Volkova A. S., Provotorov V. V.}  Generalized solutions and generalized eigenfunctions of boundary-value problems on a geometric graph // Russian Mathematics, 2014, vol. 58, no. 3,  pp.~1--13.
