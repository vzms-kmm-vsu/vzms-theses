\vzmstitle[
	\footnote{
		Работа выполнена
		при финансовой поддержке Минобрнауки РФ (проект № 1.73117311.2017/БЧ).
	}
	Работа поддержана РНФ, проект 20-71-00155.
]{
	Локальная гипотеза Фоменко: комбинация базы слоения и метки $n$
}
\vzmsauthor{Ведюшкина}{В.\,В.}
\vzmsinfo{Москва, МГУ; {\it arinir@yandex.ru}}

\vzmscaption

Недавно В.В.Ведюшкиной был введён [1] класс биллиардных книжек "--- кусочно"=плоских столов"=комплексов с перестановками на рёбрах (комплекс склеен из плоских \textit{софокусных} столов, т.е. ограниченных квадриками с общими фокусами). Это существенно расширило класс интегрируемых биллиардов и разнообразило топологию слоений Лиувилля на их фазовых пространствах (его разбиение на совместные уровни энергии и интеграла) при сохранении тесной связи с классом софокусных столов. Последнее следует из доказанной А.А.Глуцюком [2] гипотезы Биркгофа о полиномиально интегрируемых плоских биллиардах: опуская детали, такие биллиарды должны иметь  софокусный или круговой стол.

А.Т.Фоменко сформулировал в [3] фундаментальную гипотезу о биллиардах:  класс интегрируемых биллиардов окажется ``не уже'' класса интегрируемых систем относительно послойной гомеоморфности слоений Лиувилля. Классифицирующим инвариантом Фоменко--Цишанга является граф"=молекула с числовыми метками, вершины которого оснащены типами особенностей"=атомов, подробнее см. книгу [4].

Ряд положений гипотезы уже доказан. В работах В.В. Ведюшкиной и И.С. Харчевой алгоритмически задаваемыми биллиардами были промоделированы произвольные боттовские 3-атомы [1], т.е. невырожденные особенности ранга 1, и молекулы [5], т.е. базы слоения Лиувилля "--- графы с вершинами"=атомами. В.В.Ведюшкиной и В.А.Кибкало доказано [6], что в биллиардах возникают и произвольные значения числовых меток $r, \varepsilon, n$. Они, напомним, задают диффеоморфизмы склейки граничных торов 3-атомов друг с другом.

Вопрос о реализации их комбинаций пока открыт и является частью локальной версии гипотезы А.Т.Фоменко [6]. Рассмотрим связный подграф графа"=молекулы, каждой вершине которого соответствуют седловые атомы без звёздочек $V_i$, причём на всех внутренних рёбрах $V_i \cfrac{r = \infty}{}V_j$, а на внешних $V_i \cfrac{r \ne \infty}{}A$. Такое слоение есть прямое произведение окружности на слоение гамильтоновой системы с 1 ст.св. на двумерной поверхности с морсовскими сёдлами.

Такой объект в теории интегрируемых систем называется \textit{семьёй} и оснащается целочисленной меткой $n$. Ранее эту метку удавалось реализовать биллиардами лишь на \textit{некоторых} семьях"=подграфах специального вида. %Для замкнутых многообразий эта метка соответствует классу Эйлера.

Покажем, что биллиардами реализуются все такие слоения и любым значением метки $n$ (т.е. она не препятствие к реализации). Рассмотрим стол"=книжку с  $\mathbb{B}_1$ с перестановками $\sigma, \rho$ (рис. 1а) и его инвариант Фоменко--Цишанга (1б). Здесь $m = 3$ "--- количество областей $a_i$ и $c_i$.

Теперь построим стол $\mathbb{B}(A_0)$ (рис. 1в), вклеив плоскую область $A_0$ (рис. 1д). Инвариант ФЦ биллиарда на столе $\mathbb{B}(A_0)$ указан на рис. 1г. Слоение имеет семью из 3-атома $B$ с меткий $n = m$. Объединим теперь результат с алгоритмом Ведюшкиной и Харчевой [4]  реализации баз"=молекул $W$.

\paragraph{Теорема~1.}
{\it
	Биллиардными книжками $\mathbb{B}(W)$ реализуется произвольное значение $m$ метки $n$ на произвольной семье $W$ из седловых атомов типа прямого произведения.
}


\begin{figure}[h]
			\center{\includegraphics[width=110mm]{book_n_molec}}
			\label{elementarybilliards}
		\end{figure}


\litlist

1. {\it Ведюшкина В. В., Харчева И. С.}
Биллиардные книжки моделируют все трёхмерные бифуркации интегрируемых гамильтоновых систем // Матем. сб. – 2018. – Т. 209. – №. 23. – С. 17-56.

2. {\it Glutsyuk A.\,A.} On Two-Dimensional Polynomially Integrable Billiards
on Surfaces of Constant Curvature // Dokl. Math. – 2018. – Т. 98. – №. 1. – С. 382-385.

3. {\it Фоменко А. Т., Ведюшкина В. В.} Биллиарды и интегрируемость в геометрии и физике. Новый взгляд и новые возможности // Вестн. МГУ, Сер.1. – 2019. – №. 3. – С. 15-25.

4. {\it Болсинов А. В., Фоменко А. Т.}
Интегрируемые гамильтоновы системы. Геометрия, топология, классификация, Т.1,2.
Ижевск: НИЦ <<Регулярная и хаотическая динамика>>, 1999.

5. {\it Ведюшкина В. В., Харчева И. С.}
Биллиардные книжки реализуют все базы слоений Лиувилля интегрируемых гамильтоновых систем // Матем. сб. – 2021. – Т. 212. – №. 8. – С. 89-150.

6. {\it Ведюшкина В. В.,  Кибкало В. А., Фоменко А. Т.} Топологическое моделирование интегрируемых систем биллиардами: реализация числовых инвариантов // Докл. РАН. – 2020. – Т. 493. – С. 9-12.
