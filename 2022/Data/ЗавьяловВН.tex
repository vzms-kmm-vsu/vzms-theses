\vzmstitle[
	\footnote{Работа выполнена за счёт гранта РНФ, проект № 20-71-00155}
]{
	Моделирование квадратично интегрируемых геодезических потоков биллиардами с проскальзыванием
}
\vzmsauthor{Завьялов}{В.\,Н.}
\vzmsinfo{Москва, МГУ; {\it vnzavyalov@mail.ru}}

\vzmscaption
Согласно теореме В.В.Козлова, интегрируемые геодезические потоки на двумерных компактных римановых многообразиях без границы (многообразие и метрика $\mathbb{R}$-аналитические) могут существовать лишь на торе, сфере, бутылке Клейна и проективной плоскости. Топология их слоений Лиувилля в случае, если дополнительный интеграл имеет степень не выше двух, была изучена в терминах инвариантов Фоменко-Цишанга в работах В.С. Матвеева, Е.Н. Селивановой, В.В. Калашникова (мл.), Нгуен Тьен Зунга, Л.С.Поляковой, обзор см. [2].

Недавние результаты В.В.Ведюшкиной и А.Т.Фоменко показывают, что рассмотрение класса интегрируемых биллиардов на столах-комплексах (биллиардные книжки и топологические биллиарды) позволяет промоделировать слоения Лиувилля многих интегрируемых систем динамики. В частности, им удалось реализовать биллиардами слоения Лиувилля геодезических потоков на двумерных \textbf{ориентируемых} поверхностях, имеющих линейный или квадратичный интеграл. [3]

В докладе будет представлено развитие этой работы на случай \textbf{неориентируемых} многообразий (будут рассмотрены потоки на проективной плоскости $\mathbb{R}P^2$ или бутылке Клейна $Kl^2$). Для этого понадобилось использовать новый класс биллиардов, введенный А.Т. Фоменко — биллиард с проскальзыванием [1]. Рассмотрим $F$ --- изометрию плоского эллипса, переводящую точку x в диаметрально противоположную ей точку y. Пусть материальная точка движется равномерно и прямолинейно внутри эллипса и попадает на границу. Продолжим ее из точки $y=F(x)$ по лучу, выходящему из нее под углом $\alpha$. При этом направление траектории ``по'' или ``против'' часовой стрелки сохранится. Иными словами, ее продолжение выходит из новой точки под тем же углом, ``проскальзывая'' вдоль границы. На основании этого такой класс систем был назван ``биллиардами с проскальзывание''.

Добавление проскальзывания к системам топологических биллиардов позволило реализововать произвольные линейно интегрируемые геодезические потоки. Этот результат получен докладчиком совместно с В.В. Ведюшкиной.

{\bf Теорема.} {\it Любой линейный по импульсам геодезический поток на двумерном неориентируемом многообразии (бутылке Клейна или проективной плоскости) лиувиллево эквивалентен подходящему биллиарду с проскальзыванием. При этом линейные интегралы геодезических потоков сводятся к одному каноническому линейному интегралу на биллиарде}.

Пусть дан комплекс, ограниченный двумя софокусными эллипсами из семейства $(b-\lambda)x^2+(a-\lambda)y^2=(a-\lambda)(b-\lambda)$  и имеющими параметры $0$ и $\tilde{a}$, где $0 < \tilde{a} < b < a <\infty$. Склеив два таких кольца по внутреннему эллипсу, рассмотрим биллиард на данной области, введя на внешних эллипсах проскальзывание.

 $\linebreak$
{\bf Теорема. [1]} {\it Инварианты Фоменко-Цишанга биллиардов с проскальзыванием на угол $\pi$ внутри эллипса и внутри полученного комплекса изображены на рисунке 2. Биллиардные системы в эллипсе и внутри комплекса кусочно-гладко лиувиллево эквивалентны геодезическим потокам на проективной плоскости и бутылке Клейна, соответственно, имеющим квадратичный дополнительный интеграл}(о них см. \cite{2}).

\begin{figure}[h!]
\center{
\includegraphics[width=30mm]{RP2.png}
\includegraphics[width=30mm]{inv.png}
\includegraphics[width=40mm]{invKl.png}}

Рис.1: Звенья траектории биллиарда в эллипсе с проскальзыванием на угол $\pi$ до и после удара о границу(слева). Меченая молекула биллиарда в эллипсе с проскальзыванием (посередине)  и в столе-комплексе, склееном из эллиптических колец с проскальзыванием (справа).
\end{figure}


\litlist
1. {\it A.T. Fomenko, V.V. Vedyushkina and V.N. Zav'yalov}
Liouville foliations of topological billiards with slipping. Russ. J. Math. Phys. Vol. 28, No. 1, 2021, pp. 37–55.

2. {\it Болсинов А.В., Фоменко А.Т}
Интегрируемые гамильтоновы системы. Геометрия, топология, классификация. Т.1,2, Ижевск: НИЦ “Регулярная и хаотическая динамика”,  1999.

3. {\it Vedyushkina (Fokicheva) V.V., Fomenko A.T.}
Integrable geodesic flows on orientable two-dimensional surfaces and topological billiards. Izv. Math. 83(6), 1137--1173 (2019).
