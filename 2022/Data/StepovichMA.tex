\documentclass{vzmsthesis}
\inputencoding{cp1251}%It's really a good idea to remove it and switch to UTF-8!

\begin{document}

\vzmstitle[
	\footnote{The research was carried out with the partial financial support of the Russian Foundation for Basic Research (project No. 19--03--00271), as well as the Russian Foundation for Basic Research and the Government of the Kaluga Region (project No. 18--41--400001)}
]{
	ON SOME MODELS OF HEAT AND MASS TRANSFER
}
\vzmsauthor{Stepovich$^*$}{M.\,A.} 
\vzmsauthor{Kalmanovich$^{**}$}{V.\,V.}
\vzmsinfo{Kaluga, Tsiolkovsky KSU; {\it $^*$m.stepovich@rambler.ru, $^{**}$v572264@yandex.ru}}
\vzmsauthor{Turtin}{D.\,V.}
\vzmsinfo{Ivanovo, Plekhanov RUE, Ivanovo Branch; {\it turtin@mail.ru}}

\vzmscaption

ПРОВЕРКА русского языка.

Some possibilities of quantitative description of heat and mass transfer processes caused by kilovolt (1--50~keV) electrons in semiconductor targets are considered by means of ma\-the\-ma\-ti\-cal modeling. The processes of heat and mass transfer cau\-sed by the interaction of wide electron beams with homogeneous and multilayer planar semiconductor structures are considered. The use of wide electron beams makes it possible to reduce these problems to one--dimensional ones and to describe the con\-si\-de\-red mathematical models~[1,~2] by ordinary differential equations. The objects of study are mathematical models des\-cri\-bing the diffusion processes of nonequilibrium minority cha\-rge carriers generated by a wide beam of kilovolt electrons in the following semiconductor targets: 1)~in homogeneous semi--in\-fi\-nite targets, 2)~in homogeneous targets of finite thickness, 3)~in multilayer planar structures of finite thickness, including number and with an arbitrary number of layers. As a result, estimates were ob\-tai\-ned that confirm the insignificant effect of un\-cer\-tain\-ties in the initial data on the results of the experiment. It is shown that for the considered problems of heat and mass transfer in homogeneous and multilayer targets, the existence and uni\-que\-ness of the solution of the corresponding differential equations takes place, i.e. the considered mathematical models are correct.

When simulating the processes of interaction of char\-ged particles with matter, the form of the function in the right--hand side of the differential equations was not specified. Because of this, the results obtained are valid for any external source: a wide flux of charged particles or a wide flux of quanta of electromagnetic radiation.

\bigskip
% \litlist
\centerline{\bf References}
\medskip

1.~{\it Kalmanovich~V.~V., Seregina~E.~V., Stepovich~M.~A.} Mathematical Modeling of Heat and Mass Transfer Phenomena Caused by Interaction between Electron Beams and Planar Semiconductor Multilayers //~Bulletin of the Russian Academy of Sciences: Physics.~--- 2020.~--- V.~84.~--- No.~7.~--- P.~844--850.

2.~{\it Stepovich~M.~A., Seregina~E.~V., Kalmanovich~V.~V., Filippov~M.~N.} On some problems of mathematical modeling of diffusion of non--equilibrium minority charge carriers generated by kilovolt electrons in semiconductors //~Journal of Physics: Conf. Series.~--- 2021.~--- V.~1740.~--- Art. No.~012035.

\end{document}
