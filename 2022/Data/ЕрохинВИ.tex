\vzmstitle[
	%\footnote{Работа выполнена за счёт гранта РНФ, проект 16-11-10125}
]{
	Обратная задача нахождения кеплеровых элементов орбиты космического объекта по наблюдениям
}
\vzmsauthor{Ерохин}{В.\,И.}
%\vzmsinfo{Воронеж, ВГУ; {\it nickkolok@mail.ru}}
\vzmsauthor{Кадочников}{А.\,П.}
\vzmsauthor{Сотников}{С.\,В.}
\vzmsinfo{Санкт-Петербург, ВКА им. А.Ф. Можайского; {\it vka@mil.ru}}

\vzmscaption


Движение космического объекта (КО) в поле тяготения Земли в рамках кеплеровской модели невозмущённого движения описывается следующими уравнениями [1]:

\vskip-7mm
\begin{gather}
\label{eq1}
	x = r \cdot \left( {\cos \left( u \right) \cdot \cos \left( \Omega \right) - \sin \left( u \right) \cdot \sin \left( \Omega \right) \cdot \cos \left( i \right)} \right), \hfill \\
\label{eq2}
	y = r \cdot \left( {\cos \left( u \right) \cdot \sin \left( \Omega \right) + \sin \left( u \right) \cdot \cos \left( \Omega \right) \cdot \cos \left( i \right)} \right), \hfill \\
\label{eq3}
	z = r \cdot \sin \left( u \right) \cdot \sin \left( i \right),
\end{gather}
\vskip-5mm
\noindent
где
\vskip-7mm
\begin{gather}
\label{eq4}
	u = \omega + \vartheta , \hfill \\
\label{eq5}
	r = a \cdot \left( {1 - e \cdot \cos \left( E \right)} \right), \hfill \\
\label{eq6}
	\vartheta = 2 \cdot \operatorname{arctg} \left( {\sqrt {\frac{{1 + e}}{{1 - e}}} \cdot \operatorname{tg}\left( {\frac{E}{2}} \right)} \right),\hfill \\
\label{eq7}
	E\;\left| {\;M = E - e \cdot \sin \left( E \right),} \right. \hfill \\
\label{eq8}
	M = \sqrt {\frac{\mu }{{{a^3}}} \cdot \left( {t - \tau } \right)} .
\end{gather}
\vskip-3mm
В формулах \eqref{eq1}--\eqref{eq8}
$x$, $y$, $z$ "--- прямоугольные координаты КО в абсолютной геоцентрической экваториальной системе координат (АГЭСК),
$r$ "--- геоцентрическое расстояние до КО,
$u$ "--- аргумент широты,
$\vartheta $ "--- истинная аномалия,
$E$ "--- эксцентрическая аномалия,
$M$ "--- средняя аномалия,
$\mu = 398600.44\;\text{км}^3 \cdot \text{с}^ {- 2}$ "--- гравитационная постоянная Земли,
$t$ "--- некоторый момент времени. Модель \eqref{eq1}--\eqref{eq8} содержит 6 констант "--- так называемые \textit{кеплеровы} элементы орбиты:
$a$ "--- большая полуось,
$e$ "--- эксцентриситет,
$i$ "--- наклонение,
$\Omega$ "--- долгота восходящего узла,
$\omega$ "--- аргумент перигея,
$\tau$ "--- время прохождения перигея.

В докладе будет рассмотрена задача определения указанных констант по данным наблюдений (обратная задача) с помощью нелинейного метода наименьших квадратов [2], формализованного в виде задачи безусловной минимизации
\vskip-7mm
\begin{equation}
\label{eq9}
\begin{gathered}
	\Phi \left( q \right) = \hfill \\
	= \frac{1}{2}\sum\limits_{j = 1}^P {\sum\limits_{i = 1}^{{Q_j}} {{{\left( {\Delta {x_{i,j}}\left( q \right)} \right)}^2}
\!+\!
{{\left( {\Delta {y_{i,j}}\left( q \right)} \right)}^2}
\!+\!
{{\left( {\Delta {z_{i,j}}\left( q \right)} \right)}^2} \to \min } ,} \hfill \\
\end{gathered}
\end{equation}
\vskip-5mm
\noindent
где
$q = {\left[ {a,e,i,\Omega ,\omega ,\tau } \right]^{\top}}$,
$\Delta {x_{i,j}}\left( q \right) = {x_{i,j}} \!-\! x\left( {q,{t_{i,j}}} \right)$,
$\Delta {y_{i,j}}\left( q \right) = {y_{i,j}} \!-\! y\left( {q,{t_{i,j}}} \right)$,
$\Delta {z_{i,j}}\left( q \right) = {z_{i,j}} \!-\! z\left( {q,{t_{i,j}}} \right)$;
$x\left( {q,{t_{i,j}}} \right)$,
$y\left( {q,{t_{i,j}}} \right)$,
$z\left( {q,{t_{i,j}}} \right)$ "---
рассчитаные в соответствии с \eqref{eq1}--\eqref{eq8} прямоугольные АГЭСК"=координаты КО в моменты времени
${t_{i,j}}$,
${\left( {x,y,z} \right)_{i,j}}$ "---
наблюдаемые прямоугольные АГЭСК"=координаты контролируемого КО,
полученные средством наблюдения (СН) с номером $j$ в моменты времени
${t_{i,j}}$ (единичные замеры),
$j=1,...,P$,
$i=1,...,Q_j$,
$Q_j$ "--- количество единичных замеров, выполненных СН с номером $j$,
$P$ "--- количество СН, наблюдающих КО.

Для задачи \eqref{eq9} будут рассмотрены различные методы решения, использующие аналитические представления соответствующих частных производных функции
$\Phi \left( q \right)$
 -- метод Гаусса-Ньютона, Левенберга-Марквардта, метод Ньютона и квазиньютоновские методы, и приведены результаты вычислительных экспериментов.

\litlist

1. {\it Эльясберг П.Е.}
% Invariant Banach limits and applications //Journal of Functional Analysis. – 2010. – Т. 259. – №. 6. – С. 1517-1541.
 Введение в теорию полёта искусственных спутников Земли. – М.: Наука, 1965. – 540 с.

2. {\it Дэннис Дж., Шнабель Р.} Численные методы безусловной оптимизации и решения нелинейных уравнений: Пер. с англ. М.: Мир, 1988. – 440 с.
