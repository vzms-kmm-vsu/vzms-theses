\vzmstitle[
	\footnote{This research is funded by the Science Committee of the Ministry of Education and Science of the Republic of Kazakhstan (Grants No. AP09258892, 2021-2023).}
]{
	Об одной обратной задаче для линеаризованной двумерной системы Навье-Стокса
}
\vzmsauthor{Дженалиев}{М.\,Т.}
\vzmsinfo{Алматы, ИМММ; {\it muvasharkhan@gmail.com}}
\vzmsauthor{Рамазанов}{М.\,И.}
\vzmsinfo{Караганда, КарГУ им. Е.А. Букетова; {\it ramamur@mail.ru}}
\vzmsauthor{Ергалиев}{М.\,Г.}
\vzmsinfo{Алматы, ИМММ; {\it ergaliev.madi.g@gmail.com}}

\vzmscaption

В докладе изучаются вопросы разрешимости обратной задачи (ОЗ) для уравнений Навье-Стокса. Доказаны теоремы об однозначной слабой и сильной разрешимости ОЗ. Для оператора Стокса с однородными граничными условиями Дирихле построена система ортогональных собственных функций и соответствующих ей собственные значения, на основе которой проведено численное решение ОЗ с конкретными числовыми данными. Представлены графики, иллюстрирующие результаты вычислений.

\paragraph{Постановка задачи.}
Введём пространства
$$
\mathbf{V}=\{v: \, v(y)\in \left(H_0^1(\Omega)\right)^2=\mathbf{H}_0^1(\Omega),\, \textrm{div}\,{v}=0\},
$$
$$
\mathbf{H}=\left\{\textrm{замыкание }\, \mathbf{V}\, \textrm{ в пространстве }\, \left(L_2(\Omega)\right)^2=\mathbf{L}_2(\Omega)\right\},
$$
$$
\,\,\, \mathbf{H}^2(\Omega)=\left(H^2(\Omega)\right)^2,
$$
имеют место плотные вложения
$
\mathbf{V}\subset\mathbf{H}\equiv\mathbf{H}'\subset\mathbf{V}'.
$

Пусть $Q_{yt}=\{y,t:|y|<1,\ 0<t<T\}$ и $\Omega=\{|y|<1\}$.
Рассмотрим обратную задачу по определению вектор"=функции $w(y,t)=\{w_1(y,t),w_2(y,t)\}$, скалярной функции $P(y,t)$ и вектор"=функции $f_1(y),\, f_2(y)$, где $F_j(y,t)=g_j(t)f_j(y)$, $g_1(t),\ g_2(t)$ "--- заданные функции:
\begin{align} \label{NS-eq1}
{\partial }_tw-\nu \Delta w=F-\nabla P,\quad\left\{y,t\right\}\in Q_{yt}, \end{align}
\begin{equation} \label{NS-div1}
{\rm div}\,w=\partial_{y_1}w_1+\partial_{y_2}w_2=0,\quad\left\{y,t\right\}\in Q_{yt}, \end{equation}
\begin{equation} \label{NS-bc1}
w=0,\quad\left\{y,t\right\}\in {\Sigma }_{yt}-\text{\rm боковая\,\,поверхность\,\,цилиндра}, \end{equation}
\begin{equation} \label{NS-ic1} w=0,\quad\left\{y\right\}\in \Omega-\text{\rm нижнее\, основание\,\,цилиндра},
\end{equation}
с финальным условием переопределения:
\begin{equation} \label{NS-fin1}
w(y,T){\rm =}w_{T}(y)\in\mathbf{H},
\end{equation}
где $w_{T}(y)$ "--- заданная функция.

Аналогичные обратные задачи рассматривались, например, в [1, 2].

\paragraph{К разрешимости обратной задачи.}
Запишем обратную задачу \eqref{NS-eq1}--\eqref{NS-fin1} в дифференциально"=операторной форме [3]:
\begin{align} \label{NS-d-op1}
w'(t)+\nu Aw(t)=g(t)f,\,\,\, w(0)=0,\,\,\, w(T)=w_T,
\end{align}
где в зависимости от класса разрешимости описание оператора $A$ будет дано ниже.

Далее, интегрируя уравнение из \eqref{NS-d-op1} по $t$ от $0$ до $T,$ с учётом условия переопределения мы получаем
\begin{align} \label{NS-unk1}
f&=\frac{1}{B_Tg}\left[w_{T}+\nu A B_Tw\right],\\
\label{NS-load}w'(t)&+\nu A w(t)-\nu M(t)A B_Tw
=M(t)w_{T},\,\,\, w(0)=0,
\end{align}
где
\begin{equation}\label{NS-def1}
B_Tv=\int\limits_0^Tv(t)dt,\,\, M(t)=\{M_1(t), M_2(t)\},\,\, M_j\left(t\right)=\frac{g_j\left(t\right)}{B_Tg}.
\end{equation}

\paragraph{О слабой разрешимости.}
Рассмотрим нагруженное [4] уравнение \eqref{NS-load} с ограниченным самосопряжённым положительно определённым оператором $A$:
\begin{equation} \label{NS-opA1}
Aw=\ - \Delta w: \mathbf{V}\to\mathbf{V}',\,\, \textrm{ т.е. }\, A=A^*\in L(\mathbf{V};\mathbf{V}'),
\end{equation}
где $A^*$ "--- оператор, сопряжённый к $A.$

Согласно свойствам оператора $A$ \eqref{NS-opA1} существуют положительные постоянные $\bar{\lambda}$ и $\bar{\Lambda}$, такие, что спектр оператора $A$ удовлетворяет включению: $P_\sigma\{A\}\subset[\bar{\lambda},\bar{\Lambda}].$

\paragraph{Теорема~1.} {\it
Пусть $w_T\in\mathbf{H}$ и выполнены условия
\begin{equation}\label{NS-cond2}
g(t)\in C([0,T]),\,\,\, \left|B_Tg\right|\geq\varepsilon>0,\,\,\, \bar{M}<\frac{\nu \bar{\Lambda}}{\nu T\bar{\Lambda}-1+e^{-\nu T\bar{\Lambda}}},
\end{equation}
где $\bar{M}=\max\limits_{t\in[0,T]}|M(t)|.$
Тогда задача Коши \eqref{NS-load} и вместе с ним обратная задача \eqref{NS-eq1}--\eqref{NS-fin1} однозначно разрешимы, причём,
$$
w\in \mathbf{W}_0\equiv \left\{v:\, v(t)\in L_2(0,T;\mathbf{V}),\,\, v'(t)\in L_2(0,T;\mathbf{V}')\right\},
$$
$$
\nabla P\in L_2(0,T; \mathbf{V}'),\,\, f\in \mathbf{V}',
$$
и имеют место оценки
\begin{equation} \label{NS-est1}
\|w\|_{\mathbf{W}_0}+\|\nabla P\|_{L_2(0,T; \mathbf{V}^{\,'})}\leq C_1\|w_T\|_{\mathbf{H}},
\end{equation}
\begin{equation} \label{NS-est2}
\|f\|_{\mathbf{V}^{\,'}}\leq C_2\|w_T\|_{\mathbf{H}}.
\end{equation}
}

\paragraph{О сильной разрешимости.}
Рассматрим нагруженное уравнение \eqref{NS-load} с неограниченным самосопряжённым положительно определённым оператором $A$:
\begin{multline} \label{NS-opA2}
Aw=\ - \Delta w: \mathbf{H}\to\mathbf{H},
\\
D(A)\equiv \mathbf{V}\cap \mathbf{H}^2(\Omega),\,\, (Aw,w)\geq\bar{\lambda}\|w\|_{\mathbf{V}}^2,
\end{multline}
где $\bar{\lambda}$ "--- положительное число, $P_\sigma\{A\}\subset[\bar{\lambda},\infty].$

\paragraph{Теорема~2.} {\it
Пусть $w_T\in\mathbf{V}$, и дополнительно к условиям теоремы 1 выполнены условия
\begin{equation}\label{NS-cond1}
g(t)\in C^1([0,T]),\,\,\,\left\{\begin{array}{ll}
						\textrm{либо}\,\,\, (a)&M(0)=0,\,\,\, \bar{M}^{\prime}>0,\\
						\textrm{либо}\,\,\, (b)&M(0)>0,\,\,\, \bar{M}^{\prime}\geq0,
						\end{array}\right.
\end{equation}
где $\bar{M}^{\prime}=\min\limits_{t\in[0,T]}M'(t).$
Тогда обратная задача \eqref{NS-eq1}--\eqref{NS-fin1} однозначно сильно разрешима, причём,
$$
w\!\in\!\mathbf{W}\equiv \left\{v: v(t)\in L_2(0,T;\mathbf{V}\cap\mathbf{H}^2(\Omega)),\, v'(t)\in L_2(0,T;\mathbf{H})\right\}\!\!,
$$
$$
\nabla P\in L_2(0,T; \mathbf{H}),\,\, f\in \mathbf{H},
$$
и имеют место оценки
\begin{equation} \label{NS-est3}
\|w\|_{\mathbf{W}}+\|\nabla P\|_{L_2(0,T; \mathbf{H})}\leq C_1\|w_T\|_{\mathbf{V}},
\end{equation}
\begin{equation} \label{NS-est4}
\|f\|_{\mathbf{H}}\leq C_2\|w_T\|_{\mathbf{V}}.
\end{equation}
}

В доказательствах теорем 1 и 2 используется теория спектральных разложений самосопряжённых операторов в гильбертовых пространствах [5].

% Оформление списка литературы
\litlist

1. {\it Прилепко А.И., Васин И.А.} Некоторые обратные начально"=краевые задачи для нестационарных линеаризованных уравнений Навье-Стокса //Дифференц. уравнения. – 1989. – Т. 25. – №. 1. – С. 106–117.

2. {\it Kozhanov A.I.} Inverse Problems of Finding the Absorption Parameter in the Diffusion Equation //Math. Notes. – 2019. – Т. 106. – №. 3. – С. 378-389.

3. {\it Темам Р.} Уравнения Навье-Стокса. Теория и численный анализ. //М.: Мир, 1981.

4. {\it Нахушев А.М.} Нагруженные уравнения и их приложения. //М.: Наука, 2012.

5. {\it Канторович Л.В., Акилов Г.П.} Функциональный анализ. //М.: Наука, 1984. 752 с.
