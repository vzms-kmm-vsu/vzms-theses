\documentclass{vzmsthesis}

\begin{document}

\vzmstitle[
	\footnote{Работа выполнена за счёт гранта РФФИ, проект 20-01-00497,  и  Московского Центра
		фундаментальной и прикладной математики (МГУ им. М.В. Ломоносова)}
]{
О реализуемости 3-мерных алгебр Ли векторными полями
}
\vzmsauthor{Лобода}{А.\,В.}
\vzmsinfo{Воронеж, ВГУ; {\it lobvgasu@yandex.ru}}
\vzmsauthor{Каверина}{В.\,К.}
\vzmsinfo{Москва, Финансовый унивверситет при правительстве~РФ; {\it vkkaverina@fa.ru}}

\vzmscaption


В пространстве $\mathbb{R}^4 $ рассматриваются представления трех алгебр Ли размерности 3, имеющих следующие
коммутационные соотношения (см. [1]):
\begin{align*}
	&g_{3,2} : & [e_1,e_3] &= e_1, & [e_2,e_3] &= e_1 + e_2;\\
	&g_{3,3} : & [e_1,e_3] &= e_1, & [e_2,e_3] &= e_2;\\
    &g_{3,4} : & [e_1,e_3] &= e_1, & [e_2,e_3] &=  h e_2, \quad -1 \leqslant h < 1, h \neq 0.
\end{align*}
При некотором условии невырожденности алгебры линейных векторных полей, отвечающие алгебрам Ли достаточно большой размерности, имеют орбитами аффинно однородные гиперповерхности в $\mathbb{R}^4$.

К такому классу однородных многообразий относятся, например, цилиндры над однородными поверхностями из $\mathbb{R}^3$.
Наша работа имеет целью изучение не сводимых к цилиндрам аффинно однородных поверхностей, имеющих в точности 3-мерные алгебры симметрий.

В целом семейство таких поверхностей является весьма обширным, как показано в работе [2], связанной с орбитами в $\mathbb{R}^4$ 3-мерной абелевой алгебры. Ситуация с тремя обсуждаемыми здесь алгебрами отличается от абелева случая.

\paragraph{Предложение~1.}
{\it
3-мерная орбита линейного представления в $\mathbb{R}^4$ алгебры $g_{3,2}$ может быть только цилиндрической.
}

Идея доказательства этого утверждения, как и работы [2], связана с фиксацией простейшего (жорданова) вида поля $e_1$ -- одного из базисных векторных полей реализуемой 3-мерной алгебры. Для большинства из
20 типов жордановых форм коммутирование двух базисных полей оказывается невозможным совместить с другими коммутационными соотношениями алгебры $g_{3,2}$. Для тройки алгебр $g_{3,2}$, $g_{3,3}$, $g_{3,4}$ относительно содержательными оказываются лишь следующие виды матрицы $e_1$:
\begin{equation*}
	\begin{bmatrix}
		\lambda & 0 & 0 & 0 \\
		0 & \lambda & 0 & 0 \\
		0 & 0 & \lambda & 1 \\
		0 & 0 & 0 & \lambda
	\end{bmatrix},
	\begin{bmatrix}
		\lambda & 1 & 0 & 0 \\
		0 & \lambda & 0 & 0 \\
		0 & 0 & \lambda & 1 \\
		0 & 0 & 0 & \lambda
	\end{bmatrix}, 
	\begin{bmatrix}
		\lambda & 1 & 0 & 0 \\
		0 & \lambda & 1 & 0 \\
		0 & 0 & \lambda & 0 \\
		0 & 0 & 0 & \lambda
	\end{bmatrix}.
	\eqno{(1)}
\end{equation*}

При этом для алгебры $g_{3,2}$ все виды, кроме первой матрицы из (1), оказываются противоречивыми. Для этой исключительной матрицы из коммутационных соотношений алгебры $g_{3,2}$ легко выводится условие $\lambda = 0$, означающее цилиндричность любой ее орбиты.

Для алгебр $g_{3,3}$ и $g_{3,4}$ совпадение $e_1$ c первой матрицей из (1) аналогично приводит к цилиндричности всех орбит. Однако здесь ситуация оказывается более содержательной.

\paragraph{Теорема~1.}
{\it
При совпадении $e_1$ с второй матрицей из (1) имеется 4 различных семейства алгебр Ли векторных полей в $\mathbb{R}^4$ со структурой $g_{3,3}$ и 5 семейств -- со структурой $g_{3,4}$.	
}

\paragraph{Предложение~2.}
{\it
	Все реализации алгебры Ли $g_{3,3}$ линейными векторными полями в $\mathbb{R}^4$, для которых одному из полей соответствует третья матрица из (1), описываются (с точностью до матричных подобий) базисами следующего вида
	
	\begin{equation*}
		\begin{gathered}
			\begin{bmatrix}
				0 & 1 & 0 & 0 \\
				0 & 0 & 1 & 0 \\
				0 & 0 & 0 & 0 \\
				0 & 0 & 0 & 0
			\end{bmatrix},
			\begin{bmatrix}
				0 & 0 & 0 & b \\
				0 & 0 & 0 & 0 \\
				0 & 0 & 0 & 0 \\
				0 & 0 & 1 & 0
			\end{bmatrix}, \\
			\begin{bmatrix}
				c_{1} & 0 & 0 & 0 \\
				0 & c_{1}+1 & 0 & 0 \\
				0 & 0 & c_{1}+2 & 0 \\
				0 & 0 & 0 & c_{1}+1
			\end{bmatrix}.
		\end{gathered}
		\eqno{(2)}
	\end{equation*}
}

\paragraph{Замечание.}
{\it
Интегрирование алгебр с базисом (2) приводит к однопараметрическому семейству однородных поверхностей
\begin{equation*}
   x_3 x_4 + \varepsilon_1 x_1^2 + \varepsilon_2 x_2^2 = x_3^A \ (A \in \mathbb{R}),
\end{equation*}
имеющемуся в списке работы [3]. Каждая из этих поверхностей имеет размерность алгебры симметрий, большую чем 3. 
}

Интегрирование остальных полученных семейств алгебр Ли и проверка размерностей реальных алгебр симметрий их орбит требует значительных затрат времени и усилий. 


\litlist

1. {\it  Мубаракзянов Г. М.} О разрешимых алгебрах Ли // Изв. вузов. Матем.~-- 1963.~-- №~1.~-- С.~114-123.
 

2. {\it Лобода А. В., Даринский Б. М.} Об орбитах в $\mathbb{R}^4$ абелевой 3-мерной алгебры Ли  // Уфимская осенняя математическая школа -- 2021.~-- Уфа: Аэтерна, 2021.~-- Т.~1.~-- С.~239-241.

3. {\it Можей Н. П.} Однородные подмногообразия в четырехмерной аффинной и проективной геометрии // Изв. вузов. Матем.~-- 2000.~-- №~7.~-- С.~41-52.













\end{document}
