\vzmstitle[
	]{
	Функция Грина для сингулярного оператора Киприянова
}
\vzmsauthor{Булатов}{Ю.\,Н.}
\vzmsinfo{Елец, ЕГУ им. И.А. Бунина; {\it y.bulatov@bk.ru}}


\vzmscaption

% Аннотация на русском
\textit{Дано представление функции Грина классической задачи Дирихле для единичного шара произвольной евклидовой размерности. Исследуется   функции Грина сингулярного оператора Киприянова. Показано, что функция Грина может быть выражена в терминах радиальных функций и выписан ее явный вид.}

% Ключевые слова на русском
\textit{Ключевые слова: функция Грина, сингулярный оператор Бесселя, оператор $\Delta_B$, сферическая симметрия.}

\medskip
Через $\mathbb{R}_n$ обозначим евклидово пространство точек 
$x=(x_1,...,x_n)$. Число $n$ предполагается фиксированным, $n\ge 1$.  Пусть $\Omega$ --- область cимметричная относительно  координатных гиперлоскостей $x_i=0$   
с гладкой границей $\Gamma$ в точках пернесечения с указанными координатными гиперплоскостями $x_i=0$.  Такие области в применены в [1] для работы с сингулярными дифференциальными операторами в  
$\mathbb{R}_n$. 

Рассмотрим сингулярный дифференциальный оператор Киприянова (введен в работе [2])
$$
\Delta_B = {\sum\limits_{j=1}^n B_{\gamma_i}}\,,\quad 
B_{\gamma_i}=\frac{\partial^2}{{\partial x}_i^2}+\frac{\gamma_i}{x_i}\frac{\partial}{\partial x}, \quad
-1<\gamma_i\le0\,,$$
который рассматривается на функциях, четных по каждой из переменных $x_i$, для которых $\gamma_i\neq0$.


Следуя [2], однородный порядка $m$ многочлен $P_{m}^{\gamma}$, четных по каждой из переменных $x_i$, для которых $\gamma_i\neq0$, удовлетворяющий уравнению $\Delta_B {P_{m}^{\gamma}}=0$, будем  называть B-гармоническим.
Весовой сферической функцией называется сужение B-гармонического многочлена на сферу:
$$
Y_{m}^{\gamma}(\Theta) = \frac{P_{m}^{\gamma}(x)}{|x|^m} = P_{m}^{\gamma}\left({\frac{x}{|x|}}\right)\,.
$$
 
Фундаментальное решение оператора $\Delta_B$ имеет вид
$${\cal E}_{\gamma}{(x)}=-\frac{|x|^{2-n-|\gamma|}}
{(n+|\gamma|-2)\,|S_1(n)|_\gamma},$$
$$\quad
|S_1(n)|_\gamma=\int\limits_{S_1(n)=\{x:\,|x|=1\}}
\prod\limits_1^n\Theta_i^{\gamma_i} dS.
$$

Фундаментальное решение оператора $\Delta_B$ с особенностью в произвольной точке на сфере $y\in\mathbb{R}_n,~|y|=r$ получается применением обобщенного сдвига Пуассона, который  при условии, что $\nu{=}n{+}|\gamma|{-}1{>}0$ имеет следующий вид 



$$T_{|x|}^{|y|}f(|x|)=C\int\limits_0^{\pi}
f\left(|x|^2+|y|^2-2|x|\,|y|~\cos\,\alpha\right)
~ \sin^{{\nu}-1}~d\alpha,$$
$$\quad \nu>0.$$
 Основной особенностью этого сдвига является его перестановочность с оператором $\Delta_B$:
$$
\Delta_B T_{|x|}^{|y|} f(|x|)=T_{|x|}^{|y|}\Delta_B f(|x|)\,.
$$
 
Функция
$$
k(|x|,|y|)=-\frac{1}{(2-n-|\gamma|)|S_1(n)|_\gamma
}T_{|x|}^{|y|} {|x|^{2-n-|\gamma|}}
$$
является фундаментальным решением оператора $\Delta_B$ с особенностью на сфере $|x|=r$.

Рассмотрим функцию
$ G(x,y)= T_{|x|}^{|y|}k(|x|)+h(|x|,|y|)$.
Эта функция $B$-гармонична в области ${\Omega_0}\backslash\left\{{y=0}\right\}$ и имеет на сфере
 $|y|=r$ особенности типа 
$$O(|x-y|^{2-n-|\gamma|}),\quad \ \textrm{при} \ \quad |y|\to|x|\quad$$
$$\textrm{если} \ \quad \gamma_i>-1\quad \ \textrm{и} \ \quad |\gamma|\neq0.$$

{\bf Определение.}~ Функцией Грина для области ${\Omega}$ с особенностью на сфере $|x|=r$ называется функция $G(x,y)$, заданная и непрерывная в $\overline{\Omega}\backslash\left\{{y}\right\}$, такая что 
$$G(\xi,y)=0 \quad \textrm{для} \quad \xi\in\textrm{Г},$$
$$G(x,y)\quad B-\textrm{гармонична} \quad \textrm{в} \quad {\Omega}\backslash\left\{{y}\right\}.$$

{\bf Теорема.} ~ {\it Пусть для области $\Omega_0$ функция Грина существует и имеет непрерывные производные в $\overline{\Omega}\backslash\left\{{y}\right\}$. Через ${H_B{(\overline{\Omega})}}$ обозначим линейное подпространство,  состоящее из всех B-гармонических функций в $\Omega_0$. Тогда для любой функции $u(x)\in{H_B{(\overline{\Omega})}}$,  имеет место равенство
$$u(y)=-{\int\limits_\textrm{Г} {u(\xi)\frac{\partial}{\partial {\omega_\xi}}{G(\xi,y)}{\xi_n^\gamma}{{d_\xi}\textrm{Г}}}},$$
где $\frac{\partial}{\partial {\omega_\xi}}$ - производная по внешней нормали к $\textrm{Г}$ в точке $\xi\in \textrm{Г}$. }

Доказательство проводится по обычной схеме с помощью второй формулы Грина для оператора $\Delta_B$, примененного к радиальной функции.


\litlist

1. {\it Киприянов И. А.}
 Сингулярные эллиптические краевые задачи // И.А. Киприянов. – 1997.

2. {\it Ляхов Л. Н., Санина Е. Л.}
 Оператор Киприянова-Бельтрами с отрицательной размерностью операторов Бесселя и сингулярная задача Дирихле для $ B $-гармонического уравнения //Дифференциальные уравнения. – 2020. – Т. 56. – №. 12. – С. 1610-1620.

3. {\it Ляхов Л. Н.}
 О радиальных функциях и классических стационарных уравнениях в евклидовом пространстве дробной размерности //Аналитические методы анализа и дифференциальных уравнений AMADE-2011. Минск. Издательский центр БГУ. – 2012. – С. 115-126.
