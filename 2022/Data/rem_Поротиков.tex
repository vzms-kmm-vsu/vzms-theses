\vzmstitle[]{
	О рекламной
	деятельности и стратегии приемлемого риска
}
\vzmsauthor{Поротиков}{С.\,А.}
\vzmsinfo{Воронеж, ВГУ; {\it porotikov.sa@gmail.com}}
\vzmsauthor{Залыгаева}{М.\,Е.}
\vzmsinfo{Воронеж, ВГУ; {\it zalygaeva@math.vsu.ru}}

\vzmscaption



Данная работа посвящена концепции рекламной
деятельности и стратегии приемлемого риска.
Целями и задачами являются:

\begin{itemize}%[nosep]
  \item
    изучение идей и базовых принципов
    планирования и проведения рекламных кампаний;
  \item
  	оценка рисков рекламодателей в
  	различных условиях;
  \item
    разработка стратегии управления риском;
  \item
    создание математической модели
  	и методического обеспечения рисков;
  \item
    оценка разных типов рисков на примере
    группы компаний «Олимп Здоровья».
\end{itemize}

{Рекламная деятельность связана с необходимостью
принятия решений в условиях неполной информации.
Неопределённость приводит к тому, что последствия
этих решений для рекламодателя оценить затруднительно.
Поэтому планирование, осуществление и контроль рекламной кампании следует
выполнять исходя из требования минимизации
или ограничения возможных неудач и потерь.
Так как обычно затраты на рекламу весьма значительны,
то использование на практике метода проб и
ошибок чревато существенными финансовыми потерями рекламодателя.
Основой управления риском и выработки рациональных,
а в отдельных ситуациях "--- и оптимальных, решений
служит предварительное моделирование рекламной кампании с
оценкой необходимых показателей. Такой подход
гарантирует в вероятностном аспекте ограничение риска рекламодателя.}


{В качестве параметров, обеспечивающих
управление рисками, могут служить планируемый
уровень окупаемости рекламы, вид принятой функции
потерь, статистические данные о показателе
эффективности рекламы, допустимые значения
рисков. Уровень окупаемости рекламы $R$ обычно
включает бюджет, который планируется выделить
на рекламную кампанию, а также дополнительные
затраты.}

{Для оценки рисков рекламодателя и выработки
решений по управлению ими разработано методическое
обеспечение и соответствующий программный
комплекс. Он позволяет моделировать риски рекламодателя
при различных условиях. По результатам моделирования
на объективной основе обеспечиваются оценки
рисков и рекомендации по управлению ими. В качестве
основных исходных данных выступают функция
потерь рекламодателя и статистические данные
о показателе эффективности рекламной кампании.}

Риск рекламодателя в общем случае определяется
через плотность распределения дохода от рекламы $j(y)$ и
функцию потерь $L(R, y)$:
$$
	F_{L}(R)=\int_{-\infty}^{\infty} L(R, y) \varphi(y) d y
	.
$$





{Функция потерь $L(R, y)$ должна
выбираться рекламодателем исходя из целей
рекламы и ожидаемого эффекта. Например, если
цель рекламной кампании заключается в получение
дохода $y$, связанного с соответствующим
расширением продаж рекламируемого товара или
услуги, то функция потерь эквивалентна величине
возможных убытков и является линейной:}
$$
	L(R,y) =
	\begin{cases}
		R - y, &\mbox{~если~} y < R,
		\\
		0, &\mbox{~если~} y \geqslant R
		.
	\end{cases}
$$


{В качестве показателя эффективности был
выбран размер дохода, полученного рекламодателем
в результате осуществления рекламной кампании.
Мерой риска служил размер вероятного среднего
ущерба рекламодателя. Функция потерь была принята
типовой для данного показателя риска. Закон
распределения был выбран нормальным, с номинальной
величиной математического ожидания дохода
1,5 млн. руб. и оценкой среднеквадратического
отклонения 0,5 млн. руб. Принято, что
уровень окупаемости, зависящий в основном от
величины рекламного бюджета, составляет 1,0
 млн. руб. В процессе моделирования эти
значения были базовыми (номинальными), то есть
при варьировании какого"=либо численного параметра
остальные принимались равными соответствующим
номиналам. Такой подход обеспечил установление
зависимостей искомых характеристик эффективности
рекламной кампании от соответствующих факторов.
При этом кроме основного риска рекламодателя $F_L(R)$ оценивались
и другие характеристики, отражающие экономическую
эффективность рекламной деятельности:}


{вероятность превышения расходов, связанных
с осуществлением рекламной кампании, над доходами,
которые рекламодатель получит от её проведения,
оцениваемая по формуле:}
$$
	P(R)=\int_{-\infty}^{R} \varphi(y) d y
	;
$$

{средняя прибыль (величина превышения доходов
от рекламной кампании над расходами, затраченными
на её осуществление), которую получит рекламодатель
в результате рекламной кампании, оцениваемая
по формуле:}
$$
	{F}_{G}({R})=\int_{-\infty}^{\infty} {G}({R}, {y}) {\varphi}({y}) d y
	,
$$
где функция прибыли рекламодателя
$$
	G(R, y) =
	\begin{cases}
		G(R,y) = 0, \mbox{~если~} y < R,
		\\
		G(R,y) = y - R, \mbox{~если~} y \geqslant R
		.
	\end{cases}
$$



{Была также проведена оценка рисков, связанных
с тем, что фактический закон распределения
дохода, полученного в результате рекламной
кампании, является нормальным. В результате
статистических подсчётов получены следующие
данные: }


\begin{figure*}[htbp]
	\includegraphics[width=0.95\linewidth]{rem_Porotikov-fig004.png}
	{Табл. 1. Оценка рисков рекламодателя при изменении уровня окупаемости рекламы}
\end{figure*}


\begin{figure*}[htbp]
	\includegraphics[width=0.95\linewidth]{rem_Porotikov-fig005.png}
	Табл. 2.
	Оценка рисков рекламодателя при изменении
	математического ожидания дохода, полученного
	в результате рекламы
\end{figure*}


\begin{figure*}[htbp]
	\includegraphics[width=0.95\linewidth]{rem_Porotikov-fig006.png}
	Табл. 3.
	Оценка рисков рекламодателя при изменении
	среднеквадратического отклонения дохода, полученного
	в результате рекламы
\end{figure*}



{Исследования показали, что обычно минимальная
величина риска достигается только при крайнем
значении соответствующего аргумента. Например,
ущерб рекламодателя полностью отсутствует
только при нулевом рекламном бюджете. Очевидно,
что в этом случае принцип минимального риска
эквивалентен отказу от рекламы. Поэтому более
универсальной является стратегия приемлемого
риска, когда рекламодатель организует
свою деятельность таким образом, что значения
риска и иных показателей экономической эффективности
рекламы не превысят установленных значений.
При использовании этой стратегии рекламодатель
должен сформировать систему своих рисков и
установить для них предельно допустимые значения.
Например, если рекламодатель ограничивает
свой средний вероятный ущерб, который может
быть нанесён ему в результате рекламной кампании,
величиной 0,5 млн. руб., то он должен выделить
на эту кампанию бюджет размером не более 2 млн. руб.}



{Сведения об авторах:}

{Поротиков Сергей Александрович, студент
математического факультета Воронежского государственного
университета.}


{Залыгаева Марина Евгеньевна, ассистент
математического факультета Воронежского государственного
университета.}



