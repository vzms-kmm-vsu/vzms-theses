\vzmstitle[
	\footnote{Степенные пространства Кёте. Изоморфизмы <<больших>> семейств}
]{
	 Степенные пространства Кёте. Изоморфизмы <<больших>> семейств
}
\vzmsauthor{Шубарин}{М.\,А.}
\vzmsinfo{ЮФУ, институт Математики, Механики И Компьютерных Наук им. И. И. Воровича; {\it mas102@mail.ru}}


\vzmscaption
{\bf 1. }В докладе рассматриваются разные подходы к определению отношения эквивалентности на на различных семействах степенных пространств Кёте. В частности, строится слабая изоморфная классификация семейств обобщённых степенных пространств Кёте,
определяемых парой пространств Кёте и общим абсолютным базисом в этой паре.

Пусть дана пара пространств Кёте  $\overline{X}=[K(A^{(0)}),K(A^{(1)})]$ такая, что
$K(A^{(0)})\supset K(A^{(1)})$ и общий абсолютный базис $f=(f_n)$ в этой паре (в частности, можно
рассматривать базис ортов $e=(e_n)$). В [1] введено семейство $(p_{\overline{X},f})$, состоящее из
всех $(\overline{X},f)$--степенных пространств, которые по определению одновременно $\Delta_{f,f}$-- и
$\Delta\beta _{f,f}$~--~интерполяционно между $K(A^{(0)})$ и $K(A^{(1)})$ (пространства, удовлетворяющие
этому условию в дальнейшем называются обобщёнными степенными пространствами Кёте). В свою очередь,
определения $\Delta_{f,f}$-- и $\Delta\beta _{f,f}$--интерполяционного свойства формулируются в
терминах непрерывных и ограниченных операторов, диагональных относительно базиса $f$.

Там же доказывается ([1], теорема 2), критерий принадлежности пространства Фреше семейству
$(p_{\overline{X},f})$. В частности, для пространств Кёте этот
критерий равносилен следующему утверждению:

\noindent\textbf{Теорема~1.}
Пусть пространство Кёте $K(A)$ такое, что
$$K(A^{(1)})\subset K(A)\subset K(A^{(0)}).$$ Это пространство
тогда и только тогда будет $(\overline{X},e)$--степенным пространством, когда
выполняются следующие условия:
\begin{eqnarray*}
\forall{\varphi \in {\Bbb N}^{\Bbb N}}\exists{\psi \in {\Bbb N}^{\Bbb N}:}
\forall{p\in {\Bbb N}}\exists{C}\forall{n \in {\Bbb N}}\\
\dfrac{a_{p,n}}{a_{\psi(p),n}}\leqslant C
\max\limits_{j=0,1,r\leqslant \varphi(p)}
\dfrac{a^{(j)}_{r,n}}{a^{(j)}_{\varphi (r),n}}, \\
\exists{\varphi \in {\Bbb N}^{\Bbb N}}\exists{\psi \in {\Bbb N\times {\Bbb N}}^{\Bbb N}:}
\forall{p,q\in {\Bbb N}}\exists{C>0:}\forall{n}\\
\dfrac{a_{q,n}}{a_{\varphi (p),n}}\leqslant C
\max\limits_{j=0,1}
\dfrac{a^{(j)}_{\psi(p,q),n}}{a_{p,n}}.
\end{eqnarray*}
\noindent\textbf{Определение~1.}
Говорят, что семейства про\-странств $(p_{\overline{X},f})$ и $(p_{\overline{Y},h})$ слабо изоморфны
(и писать
$(p_{\overline{X},f})\stackrel{\text{сл}}{\cong}(p_{\overline{Y},h})$), если любое пространство  из
$(p_{\overline{X},f})$ (соответственно из $(p_{\overline{Y},h})$) изоморфно подходящему пространству из
$(p_{\overline{Y},h})$ (соответственно из $(p_{\overline{X},f})$).

Изучение слабо изоморфных семейств обобщённых степенных про\-ст\-ранств связано с проблемой однозначной
определяемости: определяет ли однозначно совокупность всех интерполяционных пространств
порождающую их пару.

{\bf 2. }Пусть $f:{\mathbb R}\to{\mathbb R}$ "--- нечётная, непрерывная, возрастающая и логарифмически выпуклая
на $[0,+\infty )$ функция. Логарифмическая выпуклость функции $f$  по определению означает, что функция
$x\mapsto \ln f(\ln(x))$ "--- выпуклая функция. Пусть, кроме того, даны числовая последовательность
$a=(a_{n})_{n=1}^{\infty }$, $a_n\nearrow +\infty $ при $n\uparrow +\infty $,
и число $\delta \in (-\infty ,+\infty ]$.   Пространство $L_{f}(a,\delta ):=K(A)$,  определяемое
матрицей $A=(a_{p,n})$, $a_{p,n}:= exp (f(\delta_p\, a_n))$,
называют обобщённым пространством степенных рядов.
Здесь $(\delta_p)$ "--- произвольная
числовая последовательность такая, что $\delta_p \uparrow \delta  $ при $p\uparrow +\infty $
(известно, что пространство $L_{f}(a,\delta )$ не зависит от выбора этой последовательности). Пространства $L_{f}(a,\delta )$
были введены Драгилевым М. М. (обзоры свойств этих пространств можно найти в [2].
Предполагается, что функция $f$ является функцией быстрого роста, т.е.
$\lim\limits_{t\to +\infty }f(\alpha t)/f(t)=+\infty $ для произвольного $\alpha >1$.

\noindent\textbf{Теорема~2.}
Пусть
$$
\overline{X}=[L_{f}(a,0),L_{f}(a,\infty )], \overline{Y}=[L_{f}(b,0),L_{f}(b,\infty )].
$$
Тогда следующие условия попарно эквивалентны:
\begin{itemize}
\item[$1^{\circ}$] $(p_{\overline{X},e})\stackrel{\text{сл}}{\cong}(p_{\overline{Y},e})$,
\item[$2^{\circ}$] $(p_{\overline{X},e})\cong(p_{\overline{Y},e})$,
\item[$3^{\circ}$] $(p_{\overline{X},e})\stackrel{\text{кд}}{\cong}(p_{\overline{Y},e})$,
\item[$4^{\circ}$] $\exists{C>0:}\forall{n} \dfrac{1}{C}\leqslant \dfrac{a_{n}}{b_{n}}\leqslant C$.
\end{itemize}

{\bf 3. }Пространство Кёте $F(\lambda ,a):=K(A)$ называют степенным пространством второго рода, если
$a_{p,n}=\exp(\min\{\lambda_n,p \}-1/p)a_{n}$, где $a=(a_n)$, $\lambda=(\lambda_n)$ числовые
последовательности такие, что $a_n\uparrow +\infty $, $n\uparrow +\infty $ и $\lambda_n\geqslant 1$ для
произвольного $n\geqslant 1$. Впервые степенные пространства
второго рода были введены Захарютой В. П. [4].
Пусть $\overline{X}=[F(\lambda',a),F(\lambda'',a)]$, $\overline{X}=[F(\mu',b),F(\mu'',b)]$.

\noindent\textbf{Теорема~3.}
Если
$(p_{\overline{X},e})\stackrel{\text{сл}}{\cong}(p_{\overline{Y},e})$, то
\begin{eqnarray}
\begin{aligned}
\forall{A>1}\ \exists{B>1}\ \exists{C>1}\ \forall{\Delta', \Delta ''>1}\ \forall{t>0} \\
\left|\left\{
n:\Delta '\leqslant \lambda_n \leqslant \Delta'', \dfrac{t}{A} \leqslant a_n \leqslant At
\right\}\right| \leqslant  \\ \leqslant
\left|\left\{
n:\dfrac{\Delta'}{C}\leqslant \max\{\mu'_n,\mu''_n \}, \dfrac{t}{B} \leqslant b_n \leqslant Bt
\right\}\right|
\end{aligned} \label{Finv}
\end{eqnarray}
для произвольного $f(\lambda,a) \in (p_{\overline{X},e})$.

Для фиксированных последовательностей $a=(a_n)$ и $\tau  =(\tau_n)\subset (0,1]^{\mathbb N}$ через
$({\cal F}_{\tau ,a})$, в свою очередь, обозначим множество всех
пар вида $\overline{X}=[E_1(a),F_g(\tau  ,a)]$.

\noindent\textbf{Теорема~4.}
Если $[E_1(a),F_g(\tau  ,a)]\in ({\cal F}_{\tau  ,a})$, $[E_1(b),F_h(\tau  ,b)]\in ({\cal F}_{\tau  ,b})$ и семейства
$(p_{\overline{X},e})$, $(p_{\overline{Y},e})$ слабо изоморфны, то
\begin{eqnarray*}
\forall{A}\exists{B}\exists{C}\forall{\Delta ',\Delta ''>1} \forall{t}\\
\left|\left\{
n\,:\,g^{-1}(\Delta '') \leqslant \tau_n \leqslant g^{-1}(\Delta'),\dfrac{t}{A}\leqslant a_n \leqslant At
\right\}\right|\leqslant\\ \leqslant
\left|\left\{
n\,:\,h^{-1}\left(\dfrac{\Delta'}{C}\right)\geqslant \tau_n ,\dfrac{t}{B}\leqslant b_n \leqslant Bt
\right\}\right|.
\end{eqnarray*}

Последовательность $\tau  =(\tau_n)$ будем называть $a$--равномерно распределённой, если
\begin{eqnarray}
\begin{aligned}
&&\forall{\Delta ',\Delta '',0<\Delta '<\Delta ''<1} \forall{A>1} \\
&&\left|\left\{ n\,:\, \Delta '\leqslant \lambda_n \leqslant \Delta '',
\dfrac{t}{A}\leqslant a_n\leqslant At\right\}\right|\sim \\
&&\sim
(\Delta ''-\Delta ')m_a(t,A), t\to+\infty .
\end{aligned}\nonumber
\end{eqnarray}
\noindent\textbf{Следствие~1.}
Пусть последовательности $a=(a_n)$, $\tau=(\tau_n)$ такие, что
\begin{itemize}
\item[I.] $\forall{A>1, B>1}\ \exists{C>1:}\forall{t>0}\  m_a(t,A)\leqslant C m_a(t,B)$,
\item[II.] последовательность $\tau=(\tau_n)$ "--- $a$--равномерно распределённая.
\end{itemize}
Тогда в семействе
пар $({\cal F}_{\lambda ,a})$ существует континуум парно не слабо изоморфных элементов.

{\bf 5. } Для доказательства теорем 2, 3 используются
характеристики, инвариантные на множестве семейств пар обобщённых степенных пространств Кёте.

\noindent\textbf{Теорема~5.}
Если $K(A)\cong K(B)$, то существуют возрастающие функции $R:{\Bbb N}\to {\Bbb R}$ и $\varphi : {\Bbb N}\to {\Bbb N}$
такие, что
\begin{eqnarray*}
\left|\bigcap\limits_{p=1}^{+\infty }
\left\{
n\,:\,\dfrac{a_{\varphi(p),n}}{a_{r_0,n}}\leqslant t_p,
\dfrac{a_{p,n}}{a_{\varphi(r_0),n}}\geqslant \tau_p
\right\}\right|
\leqslant \\ \leqslant\left|\bigcap\limits_{p=1}^{+\infty }
\left\{
n\,:\,\dfrac{b_{p,n}}{b_{\varphi(r)_0,n}}\leqslant R(p)t_p,
\dfrac{b_{\varphi(p),n}}{b_{r_0,n}}\geqslant \dfrac{\tau_p}{R(p)}
\right\}\right|
\end{eqnarray*}
для произвольных $r_0$, $r_1<r_2$,  и $(t_p)$, $(\tau _p)$, $t_p>0$, $\tau _p>0$.

\litlist

1. {\it Шубарин М. А.} Классы пространств, порождаемые интерполяцией
диагональных операторов // Изв. ВУЗов Сев. Кавк. региона, сер. естеств. наук.- 2006.-\No 1.- с. 24--26;

\selectlanguage{english}

2. {Meise R., Vogt D.} Introduction in Functional Analysis // Oxford Texxt in Math., 2,
Calderon  Press, Oxford univ. Press, N.-Y.- 1997;

3. {\it Zahariuta V.} Linear topologic invariants and their appli\-cations
to isomorphic classification of generalized power spaces // Tr. J. Math.- 1996.-{\bf 20}.- p. 237--289.
