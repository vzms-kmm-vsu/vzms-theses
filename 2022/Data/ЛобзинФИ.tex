%\usepackage{amsmath,amsthm,amsfonts,amssymb,amscd}
%\usepackage{longtable}

% \usepackage[pdftex]{graphicx}

%\usepackage{colortbl}

\newtheorem{conjecture}{Гипотеза}
%\newtheorem{theorem}{Теорема}


\vzmstitle{
	Построение многочленов в биинволюции для сингулярных элементах коалгебры Ли
}
\vzmsauthor{Лобзин}{Ф.\,И.}
\vzmsinfo{Москва, МГУ, МЦФиПМ; {\it fiadat@mail.ru}}

\vzmscaption


Пусть $\mathfrak{g}$ --- алгебра Ли, соответственно ${\mathfrak{g}}^*$ --- сопряженное пространство. Рассмотрим на ${\mathfrak{g}^*}$ структуру:
$$
\mathcal{A}_x(x)=(c_{ij}^{k}x_k), \ x\in \mathfrak{g}^*.
$$
Данный тензор определяет скобку Пуассона --- Ли на $C^{\infty}(\mathfrak{g}^*)$: $
\{f,g\}(x)=\mathcal{A}_{x}(df(x),dg(x))
$. Функции $f\in C^{\infty}$, лежащие в ядре скобки Пуассона --- Ли, называются функциями Казимира.
Так же можно рассмотреть похожую структуру, называемую скобкой с замороженным аргументом:
$$
\mathcal{A}_a(x)=(c_{ij}^{k}a_k), \ a, x\in \mathfrak{g}^*, \ \ \{f,g\}_a(x)=\mathcal{A}_{a}(df(x),dg(x))
$$
\\
Алгебра Ли называется вполне интегрируемой, если на ней найдется полный набор функций, находящихся в инволюции относительно скобки Пуассона ---  Ли. Полным считается набор, содержащий в себе n функционально независимых функций, где $n$ определяется формулой:
$$
n = \frac1 2 (dim\ \mathfrak{g}+ind\ \mathfrak{g}).
$$
Наибольший практический интерес представляют наборы, состоящие из многочленов. Во второй половине прошлого века была сформирована гипотеза, касающиеся существования полных наборов в инволюции.
\begin{conjecture}[доказана]{Гипотеза Мищенко --- Фоменко.}

На двойственном пространстве $\mathfrak{g}^*$ любой алгебры Ли $\mathfrak{g}$ существует полный набор полиномов в инволюции относительно $\{\cdot,\cdot\}$.
\end{conjecture}
Для построения таких наборов удобно пользоваться методом сдвига аргумента.
    Отметим, что получившиеся сдвигом наборы будут также в инволюции и относительно скобки с замороженным аргументом, так что интересно рассмотреть естественное обобщение гипотезы 1, предложенное в [1].
    \begin{conjecture}
    {Обобщенная гипотеза Мищенко --- Фоменко.} Для любой алгебры Ли $\mathfrak{g}$, для всех регулярных $a$ из коалгебры на $\mathfrak{g}^*$  существует полный набор полиномов в биинволюции, то есть набор, одновременно находящийся в инволюции относительно $\{\cdot,\cdot\}$  и $\{\cdot,\cdot\}_a$.
    \end{conjecture}
   Полученные при применении метода сдвига аргумента наборы являются полными не для всех алгебр Ли. Также эти наборы функционально независимы не для всех $a$.


     Первая гипотеза была доказана Садэтовым в 2004 году (см.[2]), но полученные его алгоритмом наборы не всегда оказываются в инволюции относительно скобки с замороженным аргументом.
Обобщенная гипотеза Мищенко---Фоменко доказана, например, для полупростых алгебр Ли (см.[1]). Несмотря на то, что обобщенная гипотеза была сформулирована только для регулярных сдвигов, в данной работе эта задача рассмотрена и для сингулярных сдвигов тоже.


Существует критерий, по которому можно определить полноту семейства, построенного сдвигом аргумента:
\begin{theorem}[Критерий Болсинова]
Семейство многочленов, полученных сдвигом на элемент $a$ полно тогда и только тогда, когда существует прямая $x+\lambda a$, которая не пересекает множество сингулярных элементов.
\end{theorem}
В работе рассмотрено обобщение этой теоремы:
\begin{theorem}
Набор порождающих для семейства функций, полученных сдвигом вдоль элемента $a\in \mathfrak{g}^*$, состоит из
$$\frac{dim \  \mathfrak{g}- dim (Ann(a))} 2$$
функций тогда и только тогда, когда
\begin{itemize}
    \item [1)] Существует $x\in \mathfrak{g}^*$ такой, что плоскость, натянутая на $x,a$, без прямой $\lambda a$ состоит только из регулярных элементов,
    \item[2)] $ind(Ann(a))=ind \  \mathfrak{g}$.
\end{itemize}
\end{theorem}
При помощи него был получен метод построения полных биинволютивных наборов для сингулярных элементов алгебры Ли:
\begin{theorem}
Пусть для некоторого сингулярного $a$ из коалгебры Ли верны условия теоремы 2.
Тогда набор многочленов, полученных сдвигом на $a$,  дополняется инволютивным набором для $Ann(a)$ до полного набора в биинволюции.
\end{theorem}
Следствием этой теоремы является, например, тот факт, что на всех полупростых алгебрах Ли существуют полные наборы в биинволюции для всех сингулярных элементов.
\litlist

1. {\it Bolsinov A. V., Zhang P.} Jordan–Kronecker invariants of finite-dimensional Lie algebras //Transformation Groups. – 2016. – Т. 21. – №. 1. – С. 51-86.

2. {\it Sadetov S. T.} A proof of the Mishchenko-Fomenko conjecture //Doklady Mathematics. – Pleiades Publishing, Ltd. (Плеадес Паблишинг, Лтд), 2004. – Т. 70. – №. 1. – С. 635-638.
