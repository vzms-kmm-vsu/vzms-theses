\vzmstitle[
%	\footnote{Работа выполнена за счёт гранта РНФ, проект 16-11-10125}
]{
	Частные интегралы в классе сферически симметричных функций
}
\vzmsauthor{Трусова}{Н.\,И.}
\vzmsinfo{Липецк, ЛГПУ имени П.П. Семенова"=Тян"=Шанского; {\it trusova.nat@gmail.com}}
%\vzmsauthor{Семенов}{Е.\,М.}
%\vzmsinfo{Воронеж, ВГУ; {\it nadezhka\_ssm@geophys.vsu.ru}}

\vzmscaption

Частным интегралом называется выражение
\begin{equation}\label{1}
A=\int_{\Omega}k(x',x''; t)~f(t,x'')~dt,
\end{equation}
где $\Omega\in\mathbb{R}_m,~x=(x',x'')\in\mathbb{R}_{m}\times\mathbb{R}_{n-m}$.\\
Положим
$
f(x)=u\left(\sqrt{x_1^2+\ldots+x_n^2}\right)=
u\left(\sqrt{|x'|^2+|x''|^2}\right)\,.
$
Частный интеграл (1) в сферических координатах
$ x=r\Theta,$ $\Theta=(\Theta_1,\ldots,\Theta_m),~|\Theta|=1$
примет вид
$$
\int_{|t|=\rho<\sqrt{R^2-|x''|^2}}k(x',x''; t)~f(t,x'')~dt=
$$
$$=
\int\limits_{|t|=1} \,dS\,\,\int\limits_{\rho=0}^{\sqrt{R^2-r^2\Theta''^2}} k(x',\,\rho\Theta'';\, |\rho\Theta'|)\, f(|\rho\Theta'|,\, \rho\Theta'')\, \rho^{m-1} d\rho\,.
$$
Как видим, получили ЧИ по $\rho$ с переменным верхним пределом, который более напоминает интеграл Вольтерра.

Рассмотрим частный случай такой конструкции частного интеграла. Пусть интеграл берётся по шару в $\mathbb{R}_m$ от сферичеки симметричных функций
по соответствующей части переменных. Тогда
$$
A=\int_{|t|<R}k(x',x''; |t|)~u(|t|,x'')~dt,
$$
$x=(x',x'')\in\mathbb{R}_m\times\mathbb{R}_{n-m}$. Применим к (1) сферическое преобразование $t=r\Theta,~\Theta=(\Theta_1,\ldots,\Theta_m), |\Theta|=1$. В результате получим одномерный частный интеграл со степенным весом от радиальной функции
$$
\int_{|t|<R}k(x',x''; |t|)~u(|t|,x'')~dt=
$$
$$=
|S_1(m)|\,\int_0^R k(x; r)~u(r,x'')~r^{m-1}~dr,\quad (r,x'')\in\mathbb{R}_1\times\mathbb{R}_{n-1}\,.
$$
Здесь уже видим частный интеграл с весом $r^{m-1}$.
% где $m$ --- натуральное число.

Рассмотрим многоосевую сферическую симметрию\\
$f(t,x'')=f\left(\sqrt{t_{1,1}^2+\ldots+
t_{1,\ell_1}^2},\ldots,\,\sqrt{t_{m,1}^2+\ldots+
t_{m,\ell_m}^2}, x'' \right).
$\\
Тогда (1) примет вид\\
$
\prod\limits_{i=1}^m |S_1(\ell_i)|\int\limits_{r_1=0}^{\rho_1}\ldots \int\limits_{r_m=0}^{\rho_m} k(r,\,x'';\,r )\, u(r_1,\ldots, r_m,\, x'')\, \prod\limits_{i=1}^m r_i^{\ell_m-1} d r\,,
$ \\
где $m=\ell_1+\ldots+\ell_m$, $r_i=|x_i^{(\ell_i)}|$, $r=(r_1,\ldots,r_m)$.
Таким образом, ЧИ с целыми параметрами $\gamma_i=\ell_i-1$, могут интерпретироваться, как ЧИ функций от многоосевой сферической симметрии.
В общем случае для $\gamma_i>-1$ имеем
$$
\int_{0}^{a_1}\ldots \int_{0}^{a_m} k(x,\,x'';\,t)\, u(t,\, x'')\, \prod\limits_{i=1}^m t_i^{\gamma_i} d t.
$$
Здесь\, интегрирование\, происходит\, по\, параллелепипеду\, с рёбрами параллельными координатным осям.

Справедливо следующее утверждение о сложении коэффициентов сферической симметрии частного интегрирования радиальных функций при условии, что все параметры $\gamma_i>-1$ (см. [1]).

\paragraph{Теорема~1.} {\it Пусть
$$
\widehat{f}(t,x'')=f\left(\sqrt{t_{1}^{2}+\ldots+
t_{m}^2}, x''\right)
$$
и
$$
I=\int_{t\in\mathbb{R}_m,~|t|<R} k(x',x'';|t|)\,\, \widehat{f}(t,x'')\,
\prod_{i=1}^m t_i^{\gamma_i}~ dt\,.
$$
Тогда для любых параметров $\gamma_i>-1$ имеет место равенство
$$
I=\int_0^R k(x',x'';r)\,\, f(r,x'')\,r^{m+|\gamma|-1}~ dr\,.
$$}
Доказательство\, следует из возможности применения сферического преобразования координат с параметрами
$\gamma_i>-1$. Действительно, полагая
$t=r\Theta,\, |\Theta|=1$, получим
$$I= \int_{S_1(m)}\prod_{i=1}^m \Theta_i^{\gamma_i}dS\,\,\int_{0}^{R} k\left(x',x''; r\right)
f\left(r, x''\right)\, r^{m+|\gamma|-1} d r=
$$
$$
=|S_1(m)|_\gamma\,\,\int_{0}^{R} k(x',x'';\, r)\, f(r,\, x'')\, r^{m+|\gamma|-1} d r\,.
$$
Площадь единичной сферы определяется через <<площадь нагруженной сферы>>
\begin{equation*}
|S_1(m)|_\gamma=2^n\,|S^+_1(m)|_\gamma=2^n
\int\limits_{S_1^+(m)=\{|\Theta|=1,~\Theta_i>0\} } \prod_{i=1}^m\Theta_i^{\gamma_i}\,dS\,,
\end{equation*}
$\gamma_i>-1\,.$\\
Таким образом, многомерный весовой частно"=интегральный оператор по шару от радиальной функции является весовым одномерным частно"=интегральным оператором с новым степенным весом $r^{m+|\gamma|-1}$\, при условии $m+|\gamma|>0$.


\litlist

1. {\it Ляхов Л.Н.,\, Санина Е.Л.}\,\,
Оператор\, Киприянова--- Бельтрами с отрицательными параметрами операторов Бесселя и сингулярная задача Дирихле для В-гармонического уравнения. Дифференциальные уравнения. - 2020. - Т. 56. № 12. С. 1-11.
