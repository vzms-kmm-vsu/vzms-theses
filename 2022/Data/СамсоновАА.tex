\documentclass{vzmsthesis}

\begin{document}

\vzmstitle[
	\footnote{Работа выполнена при финансовой поддержке РФФИ в рамках проекта
20-08-01154.}
]{Конечно-элементная аппроксимация задачи\\ о
собственных колебаниях\\ пластины с присоединённым грузом}
\vzmsauthor{Самсонов}{А.\,А.}
\vzmsinfo{Казань, КФУ; {\it anton.samsonov.kpfu@mail.ru}}

\vzmscaption

Пусть $\Omega$ -- прямоугольная область,
занимаемая срединной поверхностью изотропной пластины, $\Gamma$ -- граница
области $\Omega$,
$\rho=\rho(x)$ -- плотность материала,
$D=D(x)=Ed^{3}/12(1-\nu^{2})$ -- цилиндрическая жёсткость
пластины, $E=E(x)$ -- модуль Юнга,  $\nu=\nu(x)$ -- коэффициент Пуассона,
$d=d(x)$ -- толщина пластины в точке $x\in\Omega$. Предположим,
что
точка пластины
$\xi\in\Omega$ закреплена упруго с
коэффициентом упругости $K$,
в точке пластины
$\xi\in\Omega$ жёстко присоединён груз массой $M$.

Обозначим через $w(x,t)$ нормальные перемещения точки $x\in\Omega$
срединной поверхности пластины в момент времени $t$. Тогда
собственные колебания системы пластина-груз-пружина характеризуются
гармонической во времени функцией $w(x,t)$ вида
\begin{equation*}
w(x,t)=u(x)v(t),\quad x\in\Omega,
\eqno{(1)}
%\tag{1}
\end{equation*}
где
$v(t)=a_{0}\,{\rm cos}\sqrt{\lambda}t+b_{0}\,{\rm sin}\sqrt{\lambda}t$,
$t>0$;
$a_{0}$, $b_{0}$, $c_{i}$, $\lambda$ --
постоянные величины.
Число $\sqrt{\lambda}$ определяет частоту собственного колебания
сис\-темы пластина-груз-пружина,
функция $u(x)$ задаёт форму собственного колебания частотой $\sqrt{\lambda}$.

Функция (1) удовлетворяет уравнению колебания пластины
\begin{equation*}
Lw(x,t)+\rho(x)d(x)w_{tt}(x,t)+f(x,t)=0,
\quad x\in\Omega,
\eqno{(2)}
%\tag{2}
\end{equation*}
и граничным условиям
\begin{equation*}
w(x,t)=\partial_nw(x,t)=0,\quad
x\in\Gamma,
\eqno{(3)}
%\tag{3}
\end{equation*}
где $t>0$,
$f(x,t)=Mw_{tt}(x,t)\delta(x-\xi)$,
$
Lw=\partial_{11}D(\partial_{11}w+$ $\nu\partial_{22}w)+
\partial_{22}D(\partial_{22}w+\nu\partial_{11}w)+
2\partial_{12}D(1-\nu)\partial_{12}w,
$
$\partial_{i}=\partial/\partial x_{i}$,
$\partial_{ij}=\partial_{i}\partial_{j}$, $i, j=1,2$,
$(\psi(t))_t=d\psi(t)/dt$,
$\partial_n$ -- производная по внешней нормали к границе $\Gamma$,
$\delta(x)$ -- дельта-функция Дирака.

Подставляя разложение (1) в уравнения (2) и (3),
получим задачу на собственные значения:
найти числа $\lambda$ и ненулевые функции $u(x)$, $x\in\Omega$,
удовлетворяющие уравнению
\begin{equation*}
Lu=
\lambda\,(\rho d\,u+M\delta(x-\xi)u),
\quad x\in\Omega,
\eqno{(4)}
%\tag{4}
\end{equation*}
и граничным условиям
\begin{equation*}
u(x)=\partial_nu(x)=0,\quad x\in\Gamma.
\eqno{(5)}
%\tag{5}
\end{equation*}

Формулировки задач вида
(4), (5)
содержатся, например, в~[1--8].
Задача (4), (5) имеет неубывающую последовательность
положительных конечнократных собственных
значений с предельной точкой на бесконечности.
Последовательности собственных значений соответствует полная ортонормированная
система собственных функций.
В работе исследуются предельные свойства
при
$M\to\infty$
и
$M\to 0$
собственных значений и собственных
функций параметрической задачи (4), (5) с параметром  $M$.
Дифференциальная задача на собственные значения аппроксимируется
сеточной схемой метода конечных элементов с эрмитовыми бикубическими
конечными элементами на регулярной неравномерной сетке.
Доказываются оценки погрешности приближённых собственных значений
и собственных функций в зависимости от размера сетки и гладкости собственных функций.

\litlist

1.
{\it Тихонов А. Н., Самарский А. А.}
{Уравнения математической физики.}~--
М.: Наука, 1977.~-- 736~с.

2.
{\it Стрелков С. П.}
{Введение\,в\,теорию\,колебаний.}~--\,Санкт-Петербург: Издательство <<Лань>>, 2005.~-- 440~с.

3.
{\it Андреев Л. В., Дышко А. Л., Павленко И. Д.}
{Динамика пластин и оболочек с сосредоточенными массами.}~--
М.: Машиностроение, 1988.~-- 200 с.

4.
{\it Solov'ev S. I.}
{Eigenvibrations of a bar with elastically attached load}
//Differ. Equations.~--
2017.~-- V.~53.~-- No~3.~-- P.~409--423.

5.
{\it Samsonov A. A., Solov'ev S. I.}
{Eigenvibrations of a beam with load}
//Lobachevskii J. Math.~--
2017.~-- V.~38.~-- No~5.~-- P.~849--855.

6.
{\it Samsonov A. A., Solov'ev S. I., Solov'ev P. S.}
{Eigenvibra\-tions of a bar with load}
//MATEC Web Conf.~--
2017.~-- V.~129.~-- Art.~06013.~-- P.~1--4.

7.
{\it Samsonov A. A., Korosteleva D. M., Solov'ev S. I.}
{Appro\-ximation of the eigenvalue problem on eigenvibration of a load\-ed bar}
//J. Phys.: Conf. Ser.~--
2019.~-- V.~1158.~-- No~4.~-- Art. 042009.~-- P.~1--5.

8.
{\it Samsonov A. A., Korosteleva D. M., Solov'ev S. I.}
{Inves\-tigation of the eigenvalue problem on eigenvibration of a loaded string}
//J. Phys.: Conf. Ser.~--
2019.~-- V.~1158.~-- No~4.~-- Art.~042010.~-- P.~1--5.

\end{document}
