\vzmstitle[
    % \footnote{Работа выполнена при поддержки гранта}
    ]{
    НЕОБХОДИМЫЕ УСЛОВИЯ НЕПРЕРЫВНОСТИ ОТОБРАЖЕНИЙ С ($s$, $k$) - УСРЕДНЕННОЙ ХАРАКТЕРИСТИКОЙ
}
\vzmsauthor{Малютина}{А.\,Н.}
\vzmsinfo{Томск, Национальный исследовательский Томский государственный университет; {\it nmd@math.tsu.ru}}
\vzmscaption

По известной теореме вложения С. Л. Соболева [1], если $G$ ограниченная область евклидова пространства $R^{n}$ и функция $f:G\to R^{1}$,
$f\in W_{p,loc}^{1} \left(G\right),p>n$, то она непрерывна в $G$. Если $1<p\leqslant n$ этого свойства, вообще говоря, может и не быть. В настоящей работе мы обобщаем результаты,
полученные в [2] и находим необходимые условия, при которых подкласс отображений с $s($$s$, $k$) - усреднённой характеристикой [3] $1<s\leqslant n$ и функция $k\left(t\right)$ определена при $t>0$,
положительна, не возрастает и ${\mathop{\lim}\limits_{t\to 0+}} k\left(t\right)=+\infty$ функция будет непрерывными. Примеры подклассов таких отображений, с указанными выше свойствами приведены в [3,4]. 

\textbf{Определение.} Пусть область $G\subset R^{n}$, $f:G\to R^{n}$, $f\in W_{n}^{1} \left(G\right)$, $1<s\leqslant n$, пусть для любого $y\in G$ выполняется неравенства 
\begin{equation} \label{GrindEQ__1_} 
    \int_{G}\left(\lambda (x,f)\right)^{s} k\left(\left|x-y\right|\right)d\sigma _{x} < M,
\end{equation} 
и
\begin{equation} \label{GrindEQ__2_} 
    \int _{G}\left(\lambda ^{*} (x,f)\right)^{s^{*} }  k\left(\left|x-y\right|\right)d\sigma _{x} <M^{*},
\end{equation} 
В случае \ref{GrindEQ__1_} будем говорить, что отображение с ($s$, $k$) - усреднённой характеристикой,
а в случае \ref{GrindEQ__2_} -- отображение с ($s^{*} ,k$) - усреднённой характеристикой, где функция\linebreak$f\in W_{n}^{1} \left(G,k,M\right)$, $1<s<n$ [3].

\textbf{Теорема.} Пусть $f$ "--- отображение с ($s^{*} ,k$)- усреднённой характеристикой и выполнено неравенство 
\begin{equation} \label{GrindEQ__3_} 
    \int_{0}^{0}k^{\frac{1}{s^{*} } } \left(t\right) t^{\frac{n}{s^{*} } } dt<+\infty, \ a>n-p.
\end{equation}
Если $f\in W_{n,loc}^{1} \left(G\right),f^{-1} \in W_{n,loc}^{1} \left(G\right)$ и для $1<s\leqslant n$ и любой точки $y\in G$выполняется неравенство 
\begin{equation} \label{GrindEQ__4_}
    I\left(\int _{G}\left(\frac{\left|\Delta f\right|^{n} }{J\left(x,f\right)} \right)^{s} \left\| x-y\right\| ^{-\alpha } d\sigma _{x} \right)<M,
\end{equation}
если $\alpha >n-s$, то на любом компакте $A$ из области $G$ функция $f$ эквивалентна некоторой непрерывной функции.

\textit{Доказательство}. Доказательство теоремы следует из\linebreakтеоремы Арцела. Для этого построим равностепенно непрерывную и равномерно ограниченную на $K$ последовательность функций,
сходящуюся к функции $f$ почти везде в $G$. Рассмотрим последовательность $\varepsilon$ "--- усредненний функции $f$ по С.Л. Соболеву при достаточно малых $\varepsilon$ [1] называется функция 
$$
f_{\varepsilon } =\varepsilon ^{-n} \int _{R^{n} }\varphi \left(\frac{x-u}{\varepsilon } \right) f\left(u\right)du=
$$
$$
=\varepsilon ^{-n} \int _{B\left(0,\varepsilon \right)}\varphi \left(\frac{u}{\varepsilon } \right) f\left(x-u\right)du.
$$
Из [1] следует, что вне области $G$ функция $f_{\varepsilon }=0$ и функция $f_{\varepsilon }$ бесконечно дифференцируема в
${\mathcal{R}}^n$ и ${\left\|f_{\varepsilon }-f\right\|}_p,{\mathcal{R}}^n\rightarrow0$ при $\varepsilon \rightarrow0$ и что
$\frac{\partial f_{\varepsilon } }{\partial x_{i} } =\left(\frac{\partial f}{\partial x_{i} } \right)_{\varepsilon } $.
Существуют открытые множества $G_1\ \textrm{и}\ G_2$ такие, что компакт $K\subset G_1\subset G_2$, ${\overline{G}}_1\subset G_2,\ {\overline{G}}_2\subset G$,
где $\bar{G}_{i} $ замыкание множества $G_i,\ i=1,2.$ Покажем, что для достаточно малых $\varepsilon$
$$
I\left(\int _{G}\frac{\left|\Delta f\right|^{n} }{J\left(x,f\right)} \left\| x-y\right\| ^{-\alpha }  d\sigma _{x} \right)<M,y\in G_{2}.
$$
Используя обобщённое неравенство Минковского [1] и условие \ref{GrindEQ__1_} получим 
\begin{equation}\label{GrindEQ__5_} 
    \begin{gathered}
        I\left(\int _{G_{2} }\frac{\left|\Delta f\right|^{n} }{J\left(x,f\right)} \left\| x-y\right\| ^{-\alpha }  d\sigma _{x} \right) \\
        =\left[\int _{G_{2} }\varepsilon ^{-n} \int _{B\left(0,\varepsilon \right)}\varphi \left(\frac{u}{\varepsilon } \right)
        \frac{\left|\Delta f\left(x-u\right)\right|^{n} }{J\left(x,f\right)} du\left\| x-y\right\| ^{\alpha } d\sigma _{x}  \right]^{\frac{1}{s} } \leqslant \\
        \leqslant \varepsilon ^{-n}\!\!\int _{B\left(0,\varepsilon \right)}\left[\int _{G_{2} }\varphi \left(\frac{u}{\varepsilon } \right)
        \left(\frac{\left|\Delta f\left(x-u\right)\right|^{n} }{J\left(x,f\right)} \right)^{s} \left\| x-y\right\| ^{\alpha } d\sigma _{x} \right]^{\frac{1}{s}}\!du\!\leqslant \\
        \leqslant \varepsilon ^{-n} M\int _{B\left(0,\varepsilon \right)}\varphi \left(\frac{u}{\varepsilon } \right) =M,
    \end{gathered}
\end{equation}
если $\left(y-u\right)\in G$, т.е. при $\varepsilon <{\varepsilon }_0$, где ${\varepsilon }_0$ меньше расстояния от границы множества $G_2$ до границы $G$.
Из неравенства \ref{GrindEQ__5_} следует, что $\forall y\in G_2$ выполнено неравенство:
$$
\left(\int _{G}\frac{\left|\Delta f\right|^{n} }{J\left(x,f\right)} \left\| x-y\right\| ^{-\alpha }  d\sigma _{x} \right)^{s} <n^{\frac{n}{2} } M,
$$
если $\mathcal{B}(y,r)\subset G_2$.

Из \ref{GrindEQ__3_} следует, что непрерывные функции $f_{\varepsilon }$ при $\varepsilon <{\varepsilon }_0$ удовлетворяет условию леммы Ч.~Морри [3],
поэтому для любых точек $x$, $y$ таких, что шар $B\left(\frac{x+y}{2}, \ \frac{3}{2} \left|x-y\right|\right)\subset G_{2}$; $\left|f_{\varepsilon } \left(x\right)-f_{\varepsilon } \left(y\right)\right|<N\left|x-y\right|^{\beta } $,
где $\beta =\frac{\alpha -n+s}{s}$, $N$ зависит от $M$, $n, s$, $\alpha$.

Таким образом, семейство функций $f_{\varepsilon }$ при $\varepsilon <{\varepsilon }_0$ на $K$ равностепенно непрерывно. Покажем, что функции $f_{\varepsilon }$ при $\varepsilon <{\varepsilon }_0$ на $K$ ограничены одним числом.
Существует функция $\eta \in D$ такая, что её носитель лежит в $G_2$ и $\eta \left(x\right)=1$ для $x\in G_1$ [3].

Функция $f\eta \in W^1_p({\mathcal{R}}^n)$. Доопределим $f\equiv 0$ вне области $G$. Для $\varphi \in D$ имеем
\begin{equation} \label{GrindEQ__6_}
    \begin{gathered}
    \int_{R^{n} }\left(\frac{\partial \eta }{\partial x_{i} } f+\eta \frac{\partial f}{\partial x_{i} } \right) \varphi dx
    = \int _{G}\left(\frac{\partial \eta }{\partial x_{i} } f+\eta \frac{\partial f}{\partial x_{i} } \right) \varphi dx= \\
    = \int _{G}\left(\frac{\partial \eta }{\partial x_{i} } f\varphi dx+\int _{G}\eta \varphi \frac{\partial f}{\partial x_{i} } dx \right) = \\
    = \int _{G}\left(\frac{\partial \eta }{\partial x_{i} } f\varphi dx-\int _{G}f\eta \varphi \frac{\partial \left(\eta \varphi \right)}{\partial x_{i} } dx \right)
    = -\int _{G}f\eta \varphi \frac{\partial \left(\varphi \right)}{\partial x_{i}}dx.
    \end{gathered}
\end{equation}
Из \ref{GrindEQ__6_} следует, что обобщённая производная 
\begin{equation} \label{GrindEQ__7_}
    \begin{gathered}
        \frac{\partial \left(\eta ,f\right)}{\partial x_i}=\frac{\partial \eta }{\partial x_i}f+\eta \frac{\partial f}{\partial x_i}
    \end{gathered}
\end{equation}
Покажем, что для функции $\eta f$ следует
$$
I\left(\int _{G}\left|\frac{\left|\Delta \eta f\right|^{n} }{J\left(x,f\right)} \right|^{s} \left\| x-y\right\| ^{-\alpha }  d\sigma _{x} \right)<M\, \, \, \, \forall y\in K.
$$
Из \ref{GrindEQ__7_} следует, что
\begin{equation} \label{GrindEQ__8_}
    \begin{gathered}
        I\left(\int _{G}\left|\frac{\left|\Delta \eta f\right|^{n} }{J\left(x,\eta f\right)} \right|^{s} \left\| x-y\right\| ^{-\alpha }  d\sigma _{x} \right) \leqslant \\
        \leqslant I\left(\int _{G}\left|\frac{\left|\Delta \eta \right|^{n} }{J\left(x,\eta f\right)} f\right|^{s} \left\| x-y\right\| ^{-\alpha }  d\sigma _{x} \right)+ \\
        + I\left(\int _{G}\left|\eta \frac{\left|\Delta f\right|^{n} }{J\left(x,\eta f\right)} \right|^{s} \left\| x-y\right\| ^{-\alpha }  d\sigma _{x} \right)\leqslant \\
        \leqslant I\left(\int _{G}\left|\frac{\left|\Delta \eta \right|^{n} }{J\left(x,\eta f\right)} f\right|^{s} \left\| x-y\right\| ^{-\alpha } d\sigma _{x} \right)_{y\in G_{2} } + \\
        + I\left(\int _{G}\left|\eta \frac{\left|\Delta f\right|^{n} }{J\left(x,f\right)} \right|^{s} \left\| x-y\right\| ^{-\alpha }  d\sigma _{x} \right)_{y\in G_{2} } \leqslant \\
        \leqslant Ad^{\frac{-\alpha }{s} } B+CM=M_{1} ,
    \end{gathered}
\end{equation}
где $A=max\frac{\partial \eta }{\partial x_{i} }$ $x\in G_{2}$, $C=max\eta \left(x\right)$ $x\in G_{2} $, $d$ "--- расстояние между границами
$G_1\ \mathrm{\textrm{и}}\ G_2$; $=\left\| f\right\| _{p,G} $. Известно [6], что функция $\varphi \in D$ удовлетворяет условиям теоремы и, применяя условие Гёльдера и оценки \ref{GrindEQ__6_} получаем.
$$
\left|\eta \left(x\right)f_{\varepsilon } \left(x\right)\right|=\left|\frac{1}{\omega _{n-1} } \sum _{i=1}^{n}\int _{R^{n} }\frac{\partial \left(\eta f_{\varepsilon } \right)}{\partial x_{i} } \left(x-y\right)\frac{y_{i} }{\left|y\right|^{n} } dy\sigma _{y}   \right|\leqslant
$$
$$
\leqslant\frac{1}{\omega _{n-1} } \sum _{i=1}^{n}\int _{G_{2} }\frac{\partial \left(\eta f_{\varepsilon } \right)}{\partial y_{i} } \left(y\right)\frac{\partial \sigma _{y} }{\left|y-x\right|^{n-1} }\leqslant
$$
$$
\leqslant \frac{1}{\omega _{n-1} } \sum _{i=1}^{n}\left(\int _{R^{n} }\left|\frac{\partial \left(\eta f_{\varepsilon } \right)}{\partial y_{i} } \left(y\right)\right|^{p} \left|y-x\right|^{-\alpha } d\sigma _{y}  \right) ^{\frac{1}{p} } \times
$$
$$
\times\left(\int _{G_{2} }\left|y-x\right|^{m} d\sigma _{y}  \right)^{\frac{p-1}{p} } \leqslant M_{2},
$$

\noindent где $m=\frac{np-p-a}{p-1}$, $m+n>0$, $M_2$ завасит от $M_1$, $n$ и диаметра области $G_2$.

Для $x\in K$ $\eta (x)\nabla f_{\varepsilon } (x)=f_{\varepsilon } (x)$ и $|f_{\varepsilon } |\leqslant M_{2} $, следовательно равномерно ограничена.
Таким образом, на $K$ семейство функций $f_{\varepsilon } ,<\varepsilon _{0} $ равностепенно непрерывно, равномерно ограниченно,
поэтому по теореме Арцеля из семейства можно выделить подпоследовательность функций $\left|f_n(x)\ \right|$ равномерно сходящуюся на $K$ к некоторой непрерывной функций $\Psi $.
Таким образом, получившиеся функции $f$и $\Psi $ эквивалентны.

\textbf{Замечание.} Построенные примеры показывают, что в рассматриваемых подмножествах функций класса $W^1_p(G)$ с $p=n$
существуют функции не принадлежащие ни одному из классов $W^1_l\left(G\right)$ при $l>n$.


\litlist

1. {\it Работнов Ю.Н.} Элементы наследственной механики твёрдых тел / Ю.Н. Работнов. – М.: Наука, – 1977. – 384 с.

2. {\it Бырдин А.П.} Метод рядов Вольтерра в динамических задачах  нелинейной наследственной упругости /  А.П. Бырдин, М.И. Розовский // Изв. АН Арм. ССР – 1985. Т. 38. №5. – С. 49 – 56. 
