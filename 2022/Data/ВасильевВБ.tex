\vzmstitle[
	\footnote{Работа выполнена при финансовой поддержке Минобрнауки РФ, проект {FZWG}-2020-0029  }
]{
	Эллиптические уравнения и краевые задачи в областях с негладкой границей
}
\vzmsauthor{Васильев}{В.\,Б.}
\vzmsinfo{Белгород, НИУ <<БелГУ>>; {\it vbv57@inbox.ru}}
%\vzmsauthor{Петров}{П.\,П.}
%\vzmsinfo{Урюпинск, УрГУ; {\it tractorist\_petr@example.org}}

\vzmscaption


%Текст статьи

Пусть $C^a_+$ "--- угол на плоскости
\[
C^a_+=\{x\in\mathbb R^2: x=(x_1,x_2), x_2>a|x_1|, a>0\}.
\]

Мы рассматриваем модельное уравнение вида
\begin{equation}\label{1}
(Af)(x)=g(x),~~~x\in\mathbb R^{2}\setminus C^a_+,
\end{equation}
в предположении, что символ $A(\xi)$,
 \[
 |A(\xi)|\sim(1+|\xi|)^{\alpha},~~~\alpha\in\mathbb R,
 \]
 допускает волновую факторизацию относительно $-C^a_+$ [1].  Напомним, что это специальное представление символа
\[
A(\xi)=A_{\neq}(\xi)\cdot A_=(\xi)
\]
 множителями, обладающими специальными свойствами, связанными с аналитическим продолжением в радиальные трубчатые области комплексного пространства  $\mathbb C^2$. Кратко поясним.

 Сопряжённым конусом $\stackrel{*}{C^a_+}$ к $C^a_+$  называется конус $\{x\in\mathbb R^2: x=(x_1,x_2), ax_2>|x_1|\}$, радиальная трубчатая область $T(\stackrel{*}{C^a_+})$ над конусом $\stackrel{*}{C^a_+}$ "--- это множество вида $\mathbb R^2+i\stackrel{*}{C^a_+}.$

 Волновая факторизация требует аналитическую продолжимость $A_{\neq}(\xi)$ в $T(-\stackrel{*}{C^a_+})$, и $A_=(\xi)$ "--- в $T(\stackrel{*}{C^a_+})$ с оценками
 \[
 |A^{\pm 1}_{\neq}(\xi-i\tau)|\leq c_1(1+|\xi|+|\tau|)^{\pm\ae},
 \]
 \[
 |A^{\pm 1}_=(\xi+i\tau)|\leq c_2(1+|\xi|+|\tau|)^{\pm(\alpha-\ae)},~~~\forall\tau\in\stackrel{*}{C^a_+},
 \]
и число $\ae$ называется индексом волновой факторизации.

Введём интегральный оператор [1]
\[
(G_a\tilde u)(\xi)=\frac{a}{2\pi^2}\lim\limits_{\tau\to 0+}\int\limits_{\mathbb R^2}\frac{\tilde u(\eta)d\eta}{(\xi_1-\eta_1)^2-a^2(\xi_2-\eta_2+i\tau)^2}
\]

Мы используем интегральное представление решения, приведённое в [1].

\paragraph{Теорема~1.} {\it
 Пусть символ $A(\xi)$ допускает волновую факторизацию относительно $C^a_+$ с индексом $\ae$ таким, что $|\ae-s|<1/2$. Тогда уравнение \eqref{1} имеет единственное решение в пространстве  $H^s(\mathbb R^2\setminus C^a_+)$, которое в образах Фурье даётся формулой
 \[
 \tilde u(\xi)=A^{-1}_{\neq}(\xi)(I-G_a)(A^{-1}_=(\xi)\widetilde{\ell g}(\xi)),
 \]
где $\ell g$ "--- произвольное продолжение $g$ на все $H^{s-\alpha}(\mathbb R^2)$.
}

С помощью свойств одномерных сингулярных интегральных операторов [2] оказывается возможным уточнить последнее представление решения уравнения \eqref{1} и осуществить предельный переход при $a\to\infty$.

\paragraph{Теорема~2.} {\it
В дополнение к условиям Теоремы 1 предположим, что волновая факторизация с индексом $\ae$ существует для всех достаточно больших $a$ и $g\in H^{s-\alpha}(\mathbb R^2)$. Тогда предел
\[
\lim\limits_{a\to\infty}\tilde u(\xi_1,\xi_2)
\]
существует и  принимает следующий вид
\[
\lim\limits_{a\to\infty}\tilde u\xi_1,\xi_2)=A^{-1}(\xi)\tilde g(\xi)-\frac{1}{2}A^{-1}(\xi_1,0)\tilde g(\xi_1,0).
\]
}

Стоит отметить, что здесь рассмотрен случай единственного решения. Некоторые другие варианты
с наличием граничных условий рассмотрены в [3,4].

% Оформление списка литературы

\litlist

1. {\it Васильев В. Б.} Мультипликаторы интегралов Фурье, псевдодифференциальные уравнения, волновая факторизация, краевые задачи. // М.: УРСС, 2010. 235  c.

\selectlanguage{english}

2. {\it Mikhlin S.G., Pr\"o{\ss}dorf S.}
Singular integral operators. // Berlin: Akademie-Verlag, 1986. 528 p.

3. {\it Kutaiba Sh., Vasilyev V.} On solutions of certain limit boundary value
problems // AIP Conf. Proc. - 2020. - V. 2293. - 110006.

4. {\it Vasilyev V. B.} On certain 3D limit boundary value problem // Lobachevskii J. Math. -2020. - V. 41. - \No~5. - P. 913--921.
