\vzmstitle[
%	\footnote{Работа выполнена за счёт гранта РНФ, проект 16-11-10125}
]{
	Исследование периодической задачи для нелинейного
эллиптического уравнения
}
\vzmsauthor{Мухамадиев}{Э.}
\vzmsinfo{Вологда, ВоГУ; {\it emuhamadiev@rambler.ru}}
\vzmsauthor{Наимов}{А.\,Н.}
\vzmsinfo{Вологда, ВоГУ; {\it naimovan@vogu35.ru}}

\vzmscaption

Рассмотрим разрешимость периодической задачи для нелинейного
эллиптического уравнения
\begin{equation}\label{f1}
\frac{\partial^2u}{\partial x^2}+\frac{\partial^2u}{\partial
y^2}+(k_0^2+l_0^2)u=\mu F(x,y,u,\mu), \quad (x, y)\in \Pi,
\end{equation}
\begin{equation}\label{f2}
u(0,y)=u(2\pi ,y), \quad u(x,0)=u(x,2\pi), \quad x, y\in [0,
2\pi],
\end{equation}
где $\Pi=(0,2\pi)\times (0,2\pi)$, $k_0$, $l_0$ "--- фиксированные натуральные числа, $\mu\in
(-\mu_0, \mu_0)$, $F : \overline{\Pi}\times \mathbb{C}\times
[-\mu_0, \mu_0]\mapsto \mathbb{C}$ "--- непрерывное отображение,
$\mathbb{C}$ "--- комплексная плоскость. Решением задачи (\ref{f1}),
(\ref{f2}) назовём комплекснозначную функцию $u\in
C(\overline{\Pi})$, которая вместе с частными производными второго
порядка из $L_2(\Pi)$ удовлетворяют уравнению (\ref{f1}) и
условиям (\ref{f2}).

Исследование задачи (\ref{f1}), (\ref{f2}) затруднено тем, что
главная линейная часть уравнения необратима. Для исследования
разрешимости данной задачи применён новый метод, в котором
сочетаются идея метода Понтрягина из теории автономных систем на
плоскости [1, 2] и методы вычисления вращения векторных полей [3, c.~135-157].
Идея метода Понтрягина заключается
в том, что с помощью нелинейного возмущения выделяется
определённое периодическое решение соответствующей линейной
системы и посредством этого решения доказывается существование
цикла у нелинейной автономной системы при малых значениях
параметра. Данная идея в настоящей работе реализована
применительно к периодической задаче (\ref{f1}), (\ref{f2}). В
отличие от метода Понтрягина не предполагается дифференцируемость
нелинейного отображения $F$ и применяются методы вычисления
вращения векторных полей.
Разработанный метод ранее применён авторами в работах [4, 5]
при исследовании других классов нелинейных краевых
задач для дифференциальных уравнений.

Введём следующие обозначения:
$$
\langle u , v \rangle =
\frac{1}{4\pi^2}\int_0^{2\pi}\int_0^{2\pi}u(\xi,\eta)\overline{v(\xi,\eta)}d\xi
d\eta, \quad \|v\|^2=\langle v , v \rangle,
$$
$$
\psi_{kl}(x,y)=e^{i(kx+ly)}, \quad c_{kl}(v)=\langle v , \psi_{kl}\rangle,
\quad k, l = 0, \pm 1, \pm 2, \ldots \quad ,
$$
$$
J=\{(k,l): k, l - \mbox{ целые, } k^2+l^2=k_0^2+l_0^2 \},
$$
$$
\widetilde{E}_1=\{ v\in L_2(\Pi):  c_{kl}(v)=0, (k,l)\not\in J \},
$$
$$
\widetilde{\Phi}(v)=\sum_{(k,l)\in
J}\langle F(\cdot,\cdot,v,0),\psi_{kl}\rangle\psi_{kl}.
$$

Предположим, что существуют $v_1^*\in \widetilde{E}_1$ и
$\varepsilon>0$ такие, что

1) $\widetilde{\Phi}(v_1^*)=0$ и $\widetilde{\Phi}(v_1)\neq 0$ при
$0<\|v_1-v_1^*\|\leq\varepsilon$;

2)
$\gamma(\widetilde{\Phi},\widetilde{S}_{\varepsilon}^1(v_1^*))\neq
0$, где
$\gamma(\widetilde{\Phi},\widetilde{S}_{\varepsilon}^1(v_1^*))$ "---
вращение конечномерного векторного поля $\widetilde{\Phi} :
\widetilde{E}_1 \mapsto \widetilde{E}_1$ на сфере
$$
 \widetilde{S}_{\varepsilon}^1(v_1^*)= \{v\in \widetilde{E}_1
: \|v_1-v_1^*\|=\varepsilon\}.
$$

Справедлива следующая теорема.

\paragraph{Теорема.}
{\it
	Пусть выполнены условия 1) и 2). Тогда  периодическая задача
(\ref{f1}), (\ref{f2}) разрешима при всех $\mu\in (-\mu_1 ,
\mu_1)$, где $\mu_1$ "--- фиксированное положительное число.
}

\vspace{0.3 cm}

В качестве функции $F$, удовлетворяющей условиям 1) и 2),
 можно взять, например, следующую
$$
F(x,y,v,\mu)=\left(v-e^{i(k_0x+l_0y)}\right)+ F_1(x,y,v,\mu)
$$
$$
+\sum_{\nu=2}^m
d_{\nu}\left(v-e^{i(k_0x+l_0y)} \right)^{\nu} ,
$$
где $d_{\nu}$, $\nu=\overline{2, m}$ "--- комплексные числа, функция
$F_1(x,y,v,\mu)$ непрерывна по совокупности переменных и
$F_1(x,y,v,0)\equiv 0$. В этом случае, полагая
$v_1^*(x,y)=\exp{(i(k_0x+l_0y))}$ и учитывая конечномерность
$\widetilde{E}_1$, легко проверить, что при малом фиксированном
$\varepsilon>0$ и при всех $v_1^*\in \widetilde{E}_1$,
$0<\|v_1-v_1^*\|\leq\varepsilon$, имеет место неравенство
$
\langle\widetilde{\Phi}(v_1), v_1-v_1^*\rangle  \geq  \alpha
\|v_1-v_1^*\|^2,
$
где $\alpha>0$ и не зависит от $v_1$. Из этого неравенства
следует, что выполнено условие 1) и
$\gamma(\widetilde{\Phi},\widetilde{S}_{\varepsilon}^1(v_1^*))=1$,
так как  векторное поле $\widetilde{\Phi}$ на сфере
$\widetilde{S}_{\varepsilon}^1(v_1^*)$ линейно гомотопно
векторному полю $v_1-v_1^*$ и
$\gamma(v_1-v_1^*,\widetilde{S}_{\varepsilon}^1(v_1^*))=1$.


\vspace{0.3 cm}


\litlist

1. {\it Понтрягин Л.\,С.}
О динамических системах, близких к гамильтоновым  / Л.\,С.
Понтрягин // ЖЭТФ. "--- 1934. "--- Т.~4, \No~8. "--- С.~234--238.

2. {\it Баутин Н.\,Н. }
Методы и приёмы качественного исследования динамических систем на
плоскости / Н.\,Н. Баутин, Е.\,А. Леонтович "--- М.: Наука, 1990.
-- 486 с.

3. {\it Красносельский М.\,А. }
Геометрические методы нелинейного анализа / М.\,А. Красносельский,
П.\,П. Забрейко "--- М.: Наука, 1975. "--- 512 с.

4. {\it Мухамадиев Э. }
Исследование разрешимости одного класса нелинейных уравнений с
малым параметром  / Э. Мухамадиев, А.\,Б. Назимов, А.\,Н. Наимов
// Вестник Вологодского государственного университета. Серия:
Технические науки. "--- 2019. "--- Т.~3, \No~1. "--- С.~50--53.

5. {\it Мухамадиев Э. }
О разрешимости одной нелинейной краевой задачи с малым параметром
/ Э. Мухамадиев, А.\,Н. Наимов, А.\,Х. Сатторов
// Дифференциальные уравнения. "--- 2019. "--- Т.~55, \No~8. "--- С.~1127--1137.
