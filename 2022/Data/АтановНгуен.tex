\vzmstitle[
	\footnote{Работа выполнена при финансовой поддержке РФФИ (проект \No~20-01-00497).}
]{
	Примеры однородных индефинитно"=вырожденных гиперповерхностей в~$\mathbb{R}^4$
}
\vzmsauthor{Атанов}{А.\,В.}
\vzmsinfo{Воронеж, ВГУ; {\it atanov.cs@gmail.com}}
\vzmsauthor{Нгуен}{Т.\,Т.\,З.}
\vzmsinfo{Дананг, Вьетнам, Университет Дананга; {\it nttduong@ued.udn.vn}}

\vzmscaption

Будем рассматривать вещественно"=аналитические гиперповерхности пространства $\mathbb{R}^4$, описываемые вблизи начала координат уравнениями вида
\begin{equation*}
	x_4 =  F_2(x_1,x_2) + F_3(x_1,x_2, x_3) + F_4(x_1,x_2, x_3)+ \ldots
	\eqno{(1)}
\end{equation*}

Индефинитно"=вырожденный характер такой поверхности обеспечивается видом квадратичной формы $F_2 =  x_1x_2$ в уравнении (1).

\paragraph{Предложение~1.} {\it
	Если поверхность (1) c $F_2 =  x_1x_2$ аффинно однородна, то аффинным преобразованием можно привести многочлен $F_3$ из этого уравнения к виду
	\begin{equation*}
		F_3 = a_1 x_1^3 + a_2 x_2^3 + (a_3 x_1^2 + a_4 x_1 x_2 + a_5 x_2^2)x_3
		\eqno (2)
	\end{equation*}
с некоторыми вещественными коэффициентами $a_1, \ldots, a_5$.
}

Это утверждение и большое количество примеров аффинно однородных гиперповерхностей в $\mathbb{R}^4$ можно получить за счёт рассмотрения 3-мерных алгебр Ли аффинных векторных полей, касательных к изучаемым поверхностям вида (1). Каждое такое поле представляется в виде $(5\times 5)$-матрицы, и для тройки базисных полей любой из обсуждаемых алгебр изучаются коммутаторы таких матриц, разлагающиеся по исходному базису алгебры.

Возникающие системы квадратичных уравнений относительно неизвестных элементов трёх базисных матриц удаётся исследовать средствами символьной математики. Таким способом в [1--3] получены примеры нескольких семейств однородных поверхностей в $\mathbb{R}^4$, относящихся к положительно полуопределенному и индефинитному случаям многочлена $F_2$ и многочлена (2), содержащего несколько слагаемых.

\paragraph{Теорема~1.} {\it
	Существует восемь типов решений квадратичной системы уравнений, отвечающих паре многочленов
	$
		(F_2, F_3) = (x_1 x_2, x_1^2 x_3)
    $
из приведённых уравнений орбит соответствующих алгебр Ли.
}

Каждое из получаемых таким образом решений квадратичной системы представляет собой базис некоторого семейства алгебр Ли аффинных векторных полей. Однако многие из таких алгебр сводятся друг к другу за счёт матричных подобий, что означает аффинную эквивалентность их орбит.

\paragraph{Предложение~2.} {\it Базис одного из 8 семейств алгебр Ли, упомянутых в теореме 1, имеет вид}

\begin{equation*}
	\begin{gathered}
		e_1 = \begin{pmatrix}
			a & 0 & 0 & 0 & 0\\
			0 & a+b & 0 & 0 & 0\\
			0 & 0 & -b & 0 & 0\\
			0 & 0 & 0 & 2a + b & 0\\
			0 & 0 & 0 & 0 & 0
		\end{pmatrix}, \\
		e_2 = \begin{pmatrix}
			0 & s & 0 & 0 & 0\\
			0 & 0 & 0 & 0 & 0\\
			0 & 0 & 0 & 0 & -54t\\
			0 & 0 & 0 & 0 & 0\\
			0 & 0 & 0 & 0 & 0
		\end{pmatrix},
		e_3 = \begin{pmatrix}
			0 & s & 0 & 0 & 54t\\
			0 & 0 & 0 & 0 & 0\\
			0 & 0 & 0 & 0 & -54t\\
			0 & -s & 0 & 0 & 0\\
			0 & 0 & 0 & 0 & 0
		\end{pmatrix}.
	\end{gathered}
	\eqno{(2)}
\end{equation*}
{\it и включает 4 вещественных параметра. Орбиты алгебр из этого семейства описываются уравнением
\begin{equation*}
	x_4 = x_2^2x_3 + x_1x_2 + x_2^{\alpha}, \quad \alpha \in \mathbb{R},
	\eqno{(3)}
\end{equation*}
зависящим лишь от одного параметра.
}

\paragraph{Замечание 1.} {\it
	Уравнение (3) обобщает уравнение
	\begin{equation*}
		x_4 = x_1x_3^2 + x_2x_3 + x_3^4
	\end{equation*}
из классификации [4], где степень $4$ заменяется произвольным вещественным параметром $\alpha$.
}

\paragraph{Замечание 2.} {\it
	Ещё два семейства решений квадратичной системы приводят к однородным индефинитно"=вырожденным гиперповерхностям
	\begin{equation*}
		x_4 = x_2 x_3 + x_1 x_3^2 \quad \text{и} \quad x_4 = x_1^2 + x_2 x_3 + x_1 x_3^2 + \alpha x_3^4,  \quad \alpha \in \mathbb{R},
	\end{equation*}
	имеющимся в списке [4] и обладающим, тем самым, более чем 3-мерными алгебрами симметрий.
}

Приведём пример ещё одной орбиты, имеющей в своём уравнении (1) полином $F_3 = x_1^3 + x_1^2x_3$: \begin{equation*}
	x_2 = x_3^{\alpha} - x_4x_3^2 - x_1x_3, \quad \alpha \in \mathbb{R}.
\end{equation*}

\litlist

\selectlanguage{english}

1. {\it Loboda A.V., Sukovykh V.I.} Using Symbolic Computation in the Description
Problem for Degenerate Homogeneous Hy\-per\-sur\-fa\-ces in $\mathbb{C}^4$ // 22nd Workshop on Computer Algebra : Dubna, Russia, May 24–25, 2021.~-- P.~21.

\selectlanguage{russian}

2. {\it Лобода А. В., Нгуен Т. Т. З.} О компьютерных алгоритмах описания однородных гиперповерхностей в $\mathbb{R}^4$ // Уфимская осенняя математическая школа -- 2021.~-- Уфа: Аэтерна, 2021.~-- Т.~1.~-- С.~132-134.

3. {\it Атанов А. В.} Однородные гиперповерхности в $\mathbb{R}^4$: компьютерные алгоритмы описания // Воронежская весенняя математическая школа «Понтрягинские чтения -- {XXXII}».~-- 2021.~-- С.~19-21.

4. {\it Можей Н. П.} Однородные подмногообразия в четырёхмерной аффинной и проективной геометрии // Изв. вузов. Матем.~-- 2000.~-- №~7.~-- С.~41-52.
