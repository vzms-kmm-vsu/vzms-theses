\vzmstitle[
]{
	Об интеграле Римана операторной функции операторного переменного
}
\vzmsauthor{Фомин}{В.\,И.}
\vzmsinfo{Тамбов, ТГУ им. Г.Р.Державина; {\it vasiliyfomin@bk.ru}}

\vzmscaption

Пусть $E$ "--- вещественное банахово пространство; $L(E)$ "--- полная
нормированная алгебра ограниченных линейных операторов, действующих в
пространстве $E$; $A,B \in L(E)$; $A \ne B$. Рассмотрим операторный отрезок
${\left[ {A,B} \right]} =$ \\ ${\left\{ {\Lambda = (1 - t)A + tB:0 \le t \le 1}
\right\}}$. Элементы отрезка ${\left[ {A,B} \right]}$ условимся называть
операторными точками. Оператор из ${\left[ {A,B} \right]}$, получаемый при
$t = t_{\ast}  $, будем обозначать через $\Lambda _{\ast}  $. Введём на
${\left[ {A,B} \right]}$ отношение порядка: $\Lambda _{\ast}  \preceq \Lambda
_{\ast \ast}  $, если $t_{\ast}  \le t_{\ast \ast}  $ (выполнимость свойств
рефлективности, транзитивности и антисимметричности очевидна). Заметим, что
$A \preceq \Lambda \preceq B$ для любого $\Lambda \in {\left[ {A,B} \right]}$.
Если $AB = BA$, то $\Lambda _{1} \Lambda _{2} = \Lambda _{2} \Lambda _{1} $
для любых $\Lambda _{1} ,\Lambda _{2} \in {\left[ {A,B} \right]}$. Если
$\Lambda _{\ast}  ,\Lambda _{\ast \ast}   \in {\left[ {A,B} \right]}$,
$\Lambda _{\ast}  \prec \Lambda _{\ast \ast}  $, то ${\left[ {\Lambda _{\ast
} ,\Lambda _{\ast \ast}}   \right]} \subseteq {\left[ {A,B} \right]}$. Пусть
задана функция $Y = \Phi (\Lambda )$, $\Lambda \in D(\Phi )$, $D(\Phi )
\subseteq L(E)$; $R(\Phi ) \subseteq L(E)$ и существуют такие $A,B \in
D(\Phi )$, что ${\left[ {A,B} \right]} \subseteq D(\Phi )$. Рассмотрим
произвольное разбиение отрезка ${\left[ {0,1} \right]}$ на $n$ частей:
${\left[ {0,1} \right]} = {\bigcup\limits_{i = 1}^{n} {{\left[ {t_{i - 1}
,t_{i}}  \right]}}} $, где $t_{0} = 0$, $t_{n} = 1$, $t_{i - 1} < t_{i} $,
$i = \overline {1,n} $. Это разбиение порождает разбиение отрезка ${\left[
{A,B} \right]}$ на $n$ частей: ${\left[ {A,B} \right]} = {\bigcup\limits_{i
= 1}^{n} {{\left[ {\Lambda _{i - 1} ,\Lambda _{i}}  \right]}}} $, где
$\Lambda _{0} = A$, $\Lambda _{n} = B$, $\Lambda _{i - 1} \prec \Lambda _{i}
$, $i = \overline {1,n} $. Рассмотрим диаметры полученных разбиений $\mu =
{\mathop {\max} \limits_{1 \le i \le n}} \delta t_{i} $, $\nu = {\mathop
{\max} \limits_{1 \le i \le n}} {\left\| {\Delta \Lambda _{i}}  \right\|}$,
где $\delta t_{i} = t_{i} - t_{i - 1} $, $\Delta \Lambda _{i} = \Lambda _{i}
- \Lambda _{i - 1} $, $i = \overline {1,n} $. Заметим, что $\nu = {\left\|
{B - A} \right\|}\mu $, следовательно, $\nu \to 0 \Leftrightarrow \mu \to
0$. Выберем на каждой части разбиения отрезка ${\left[ {0,1} \right]}$
произвольным образом по одной точке: $\psi _{i} \in {\left[ {t_{i - 1}
,t_{i}}  \right]}$, $i = \overline {1,n} $. Это индуцирует выбор на каждой
части разбиения отрезка ${\left[ {A,B} \right]}$ по одной операторной точке:
$\Psi _{i} \in {\left[ {\Lambda _{i - 1} ,\Lambda _{i}}  \right]}$, $i =
\overline {1,n} $, где $\Psi _{i} = \left( {1 - \psi _{i}}  \right)A + \psi
_{i} B$. Соответствующая интегральная сумма функции $Y = \Phi (\Lambda )$
имеет вид
\begin{equation}
\label{eq1}
S_{n} \left( {\Lambda _{i} ,\Psi _{i}}  \right) = {\sum\limits_{i = 1}^{n}
{\Phi (\Psi _{i} )\Delta \Lambda _{i}}}  .
\end{equation}

\paragraph{Определение~1.}  Оператор $H \in L(E)$ называется пределом интегральных сумм
(\ref{eq1}) при $\nu \to 0$, если для $\forall \varepsilon > 0 \quad \exists \delta =
\delta (\varepsilon ) > 0:\forall \nu < \delta $ и любом выборе
операторных точек $\Psi _{i} $ выполняется неравенство ${\left\| {S_{n}
\left( {\Lambda _{i} ,\Psi _{i}}  \right) - H} \right\|} < \varepsilon $.
Обозначение:

\begin{equation}
\label{eq2}
H = {\mathop {\lim} \limits_{\nu \to 0}} S_{n} \left( {\Lambda _{i} ,\Psi
_{i}}  \right).
\end{equation}

\paragraph{Определение~2.}  Операторная функция $Y = \Phi (\Lambda )$ называется
интегрируемой по Риману на отрезке ${\left[ {A,B} \right]}$, если существует
предел вида (\ref{eq2}), при этом оператор $H$ называется интегралом Римана от
функции $Y = \Phi (\Lambda )$ на отрезке ${\left[ {A,B} \right]}$ и
обозначается символом ${\int\limits_{A}^{B} {\Phi (\Lambda )d\Lambda}}  $.
Таким образом, по определению,

\begin{equation}
\label{eq3}
{\int\limits_{A}^{B} {\Phi (\Lambda )d\Lambda}}   = {\mathop {\lim
}\limits_{\nu \to 0}} S_{n} \left( {\Lambda _{i} ,\Psi _{i}}  \right){\rm
.}
\end{equation}

Некоторые свойства интеграла Римана вещественной функции переносятся на
интеграл вида (\ref{eq3}). В частности, ${\int\limits_{A}^{B} {{\rm I}d\Lambda}}   =
B - A$; ${\int\limits_{A}^{B} {\Phi (\Lambda )d\Lambda}}   = -
{\int\limits_{B}^{A} {\Phi (\Lambda )d\Lambda}}  $; если $\Phi _{1} (\Lambda
)$, $\Phi _{2} (\Lambda )$ интегрируемы на ${\left[ {A,B} \right]}$, то
$\Phi _{1} (\Lambda ) + \Phi _{2} (\Lambda )$ интегрируема на ${\left[ {A,B}
\right]}$ и ${\int\limits_{A}^{B} {{\left[ {\Phi _{1} (\Lambda ) + \Phi _{2}
(\Lambda )} \right]}d\Lambda}}   =  \quad {\int\limits_{A}^{B} {\Phi _{1}
(\Lambda )d\Lambda}}   + {\int\limits_{A}^{B} {\Phi _{2} (\Lambda )d\Lambda
}} $; если $\Phi (\Lambda )$ интегрируема на ${\left[ {A,B} \right]}$, то
для любого $\alpha \in \mathbb{R}$  функция $\alpha \Phi (\Lambda )$ интегрируема на
${\left[ {A,B} \right]}$ и ${\int\limits_{A}^{B} {\alpha \Phi (\Lambda
)d\Lambda}}   = \alpha {\int\limits_{A}^{B} {\Phi (\Lambda )d\Lambda}}  $.

Построение интегральных сумм (\ref{eq1}) оказалось возможным в силу наличия операции
умножения элементов алгебры $L(E)$. Следовательно, предложенная выше
конструкция интеграла Римана пригодна для векторных функций векторного
переменного, действующих в вещественном нормированном пространстве,
снабжённом дополнительной операцией умножения элементов.
