\vzmstitle[
	\footnote{}
]{
	Нечёткие интегральные средние параметрических систем нечётких чисел
}
\vzmsauthor{Хацкевич}{В.\,Л.}
\vzmsinfo{Воронеж, ВУНЦ ВВА им. Н.Е. Жуковского и Ю.А. Гагарина; {\it vlkhats@mail.ru}}

\vzmscaption


Для параметрических систем нечётких чисел на основе интервального подхода введён и изучен класс интегральных операторов нечёткого осреднения для реализации задачи агрегирования нечёткой информации.  Отметим отличие нашего подхода от  широко распространённого (нечёткого) интеграла Шоке (см., напр., [1], [2]), где агрегируются функции принадлежности.

Как известно ([3], гл. 2), нечёткое число $\tilde{z}$ с функцией принадлежности $\mu_{\tilde{z}}(x)$ порождает $\alpha$-уровни $Z_{\alpha}$, определяемые для любого $\alpha\in [0, 1]$ равенством
$$
Z_{\alpha}  = \{x| \mu_{\tilde{z}}(x)\geq \alpha\},\,\,\,(\alpha\in(0, 1]),\,\,\,Z_0 = cl\{x| \mu_{\tilde{z}}(x)>0\},
$$
где символ $cl$ означает замыкание множества.

Будем считать, что  множества $\alpha$-уровня "--- замкнутые интервалы, так что $Z_{\alpha} = [z^{-}(\alpha), z^{+}(\alpha)]$. Функции $z^{-}(\alpha)$ и $z^{+}(\alpha)$ называют, соответственно, левым и правым индексами нечёткого числа. Ниже предполагается, что они измеримы и ограничены на [0, 1]. Совокупность таких нечётких чисел будем обозначать $J$.

Рассмотрим на множестве нечётких чисел метрику, задаваемую для $\tilde{z}_1$, $\tilde{z}_2\in J$ равенством [4]
\begin{equation}
d(\tilde{z}_1, \tilde{z}_2) = \sup\limits_{\alpha\in[0,1]}\max\{|z_1^{+}(\alpha) - z_2^{+}(\alpha)|, |z_1^{-}(\alpha) - z_2^{-}( \alpha)|\}.
\end{equation}

Будем использовать следующее определение сравнения (ранжирования) нечётких чисел, заданных в интервальной форме. Будем писать $\tilde{z}\prec\tilde{w}$ для нечётких чисел $\tilde{z}$ и $\tilde{w}$, если одновременно ([5], гл. 5)
\begin{equation}
z^{-}(\alpha)\leq w^{-}(\alpha), \,\,\,\,\,z^{+}(\alpha)\leq w^{+}(\alpha)\,\,\,\,(\forall\alpha\in (0,1]).
\end{equation}

Отметим, что (2) задаёт соотношение частичного порядка на множестве $J$.

Пусть $\Omega$ "--- измеримое по Лебегу числовое множество.   Пусть $\tilde{z}(\omega)$ "--- измеримая
(относительно метрики (1))  нечёткая функция. А именно, $\tilde{z}: \Omega \rightarrow J$ с функцией принадлежности $\mu_{\tilde{z}(\omega)}(x)$. Рассмотрим $\alpha$-уровни $Z_{\alpha}(\omega) = \{x\in R: \mu_{\tilde{z}(\omega)}(x)\geq \alpha\}$. Интервал $Z_{\alpha}(\omega)$ представим в виде $[z^{-}(\omega, \alpha), z^{+}(\omega, \alpha)]$. Будем считать индексы $z^{-}(\omega, \alpha), z^{+}(\omega, \alpha)$ измеримыми и ограниченными на $\Omega\times[0,1]$ функциями. Класс таких нечётких функций обозначим $J_{\omega}$. На множестве $J_{\omega}$ зададим метрику равенством


\begin{equation*}
\rho(\tilde{z}_1(\omega), \tilde{z}_2(\omega)) = \sup\limits_{\alpha\in[0,1], \omega\in\Omega}\max\{|z_1^{+}(\omega, \alpha) - z_2^{+}(\omega, \alpha)|, $$$$|z_1^{-}(\omega, \alpha) - z_2^{-}(\omega, \alpha)|\}.
\end{equation*}




Для нечёткой функции $\tilde{z}(\omega)$ обозначим через $p(\omega)$ функцию весов (частот).  Будем считать её квадратично суммируемой на $\Omega$ и $p(\omega)>0$ при п.в. $\omega\in\Omega$.

Определим с помощью интеграла Лебега функции
\begin{equation}
z^{-}_{p}(\alpha) = \frac{1}{p_0}\int\limits_{\Omega}z^{-}(\omega, \alpha)p(\omega)d\omega,\,\,\,\,z^{+}_{p}(\alpha) = \frac{1}{p_0}\int\limits_{\Omega}z^{+}(\omega, \alpha)p(\omega)d\omega,
\end{equation}
где $p_0 = \int\limits_{\Omega}p(\omega)d\omega$.


Введём в рассмотрение нечёткое число, индексы которого задаются формулами (3). Назовём его нечётким интегральным взвешенным средним (нечётким осредняющим оператором) нечёткой функции $\tilde{z}(\omega)$ и обозначим $M_p(\tilde{z}(\omega))$.

Это нечёткий аналог непрерывного среднего арифметического~[6, гл.~{I}].


\textbf{Теорема 1.} \textit{Нечёткое интегральное среднее, определяемое посредством (3), удовлетворяет следующим условиям:}

\textit{1). Если $\tilde{z}(\omega) = \tilde{z}\,\,\,(\text{п.в.}\,\,\omega\in\Omega)$, то $M_p(\tilde{z}(\omega)) = \tilde{z}$.
}

\textit{2). Оператор осреднения $M_p: J_{\omega}\rightarrow J$ является аддитивным и однородным.
}

\textit{3). Оператор осреднения $M_p: J_{\omega}\rightarrow J$ непрерывен.
}


Для заданной нечёткой функции $\tilde{z}(\omega)$ с индексами $z^{-}(\omega, \alpha)$ и $z^{+}(\omega, \alpha)$ (которые будем считать ограниченными функциями) определим
$$
z_{inf}^{\pm}(\alpha)=\inf\limits_{\omega\in\Omega}z^{\pm}(\omega, \alpha),\,\,\,\,z_{sup}^{\pm}(\alpha)=\sup\limits_{\omega\in\Omega}z^{\pm}(\omega, \alpha).
$$

\textbf{Утверждение 1.} \textit{Для нечёткого интегрального среднего $M_p(\tilde{z}(\omega))$ имеет место соотношение }
$$
z_{inf}^{\pm}(\alpha)\leq M_p^{\pm}(\tilde{z}(\omega))(\alpha) \leq z_{sup}^{\pm}(\alpha)\,\,\forall\alpha\in[0, 1],
$$
где $M_p^{\pm}(\tilde{z}(\omega))$ "--- индексы нечёткого среднего $M_p(\tilde{z}(\omega))$.



Это характеризует нечёткое интегральное среднее $M_p$ как промежуточное значение между наименьшим и наибольшим.

\textbf{Утверждение  2.} \textit{Оператор осреднения $M_p$ является монотонным. А именно, если для двух нечётких функций $\tilde{z}_1(\omega)$ и  $\tilde{z}_2(\omega)$ выполнено условие  $\tilde{z}_1(\omega)\prec\tilde{z}_2(\omega)$\,\,\,$(\text{п.в.}\,\,\omega\in\Omega)$, то $M_p(\tilde{z}_1(\omega))\prec M_p(\tilde{z}_2(\omega))$.
}


Для фиксированной нечёткой функции $\tilde{z}(\omega)\in J_{\omega}$ рассмотрим экстремальную задачу
\begin{equation}
\int\limits_{0}^1\int\limits_{\Omega}((z^{-}(\omega, \alpha) - v^{-}(\alpha))^2 + (z^{+}(\omega, \alpha) $$$$- v^{+}(\alpha))^2)p(\omega)d\omega d\alpha\rightarrow\min\,\,\,\forall \tilde{v}\in J.
\end{equation}
Здесь $z^{\pm}(\omega, \alpha)$ и  $v^{\pm}(\alpha)$  "--- индексы нечётких  чисел $\tilde{z}(\omega)$ и  $\tilde{v}$.

\textbf{Теорема 2.} \textit{Решение задачи (4) даёт нечёткое число  $M_p(\tilde{z}(\omega))$. }


Теорема 2  характеризует определённую равномерность приближения параметрического семейства нечётких чисел $\tilde{z}(\omega)$ посредством нечёткого интегрального среднего $M_p(\tilde{z}(\omega))$.

Близкие  утверждения получены автором и в нелинейном случае.



\litlist

1. {\it Kwak K., Pedrycz W.}
Face recognition: A study in information fusion using fuzzy integral //Patt. Recog. Lett. - 2005. - Vol.26. - pp.719-733

2. {\it M. Svistula}
A note on the Choquet integral as a set function on a locally compact space //Fuzzy Sets and Systems. - 2021, https://doi.org/10.1016/j.fss.2021.07.004.

3. {\it Пегат А.}
Нечёткое моделирование и управление.  М.: Бином. 2015. - 786 с.

4. {\it Kaleva O., Seikkala S.}
On fuzzy metric spaces //Fuzzy Sets and Systems. - 1984. - V.12. - pp. 215-229.

5. {\it Смоляк С.А.}
Оценки эффективности инвестиционных проектов в условиях риска и неопределённости. "--- М.: Наука. - 2002. - 182 с.

6. {\it Джини К. }
Средние величины. "--- Москва. Статистика. -  1970. - 447 с.
