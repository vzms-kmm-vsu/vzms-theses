\vzmstitle[\footnote{Исследование выполнено за счёт гранта Российского научного фонда № 22-21-00026, https://rscf.ru/project/22-21-00026/ .}]{Распределение корней и рост целых функций экспоненциального типа}
\vzmsauthor{Хабибуллин}{Б.\,Н.}
\vzmsinfo{Уфа, БашГУ; {\it khabib-bulat@mail.ru}}
\vzmsauthor{Салимова}{А.\,Е.}
\vzmsinfo{Уфа, БашГУ; {\it anegorova94@bk.ru}}

\vzmscaption


%%Постановка основная задачи в максимальной общности в следующем.

Всюду далее $ {\sf Z}= \{{\sf z}_j\}_{j=1,2, \dots}$ и ${\sf W}= \{{\sf w}_{j}\}_{j=1,2, \dots}$ ---    распределения точек на комплексной плоскости  $ \mathbb C$ {\it конечной верхней плотности\/} $\sum_{|{\sf z}_j|\leq r}1+\sum_{|{\sf w}_j|\leq r}1\underset{r\to +\infty}{=}O(r)$.
Излагаются некоторые усиления результатов наших работ [1], [2], [3], в которых развивалась классическая теорема Мальявена\,--\,Рубела о малости роста целых функций экспоненциального типа с заданными нулями  вдоль мнимой оси $i\mathbb R$ в терминах
{\it логарифмической субмеры} подынтервалов $(r,R]\subset \mathbb R^+$
\begin{equation*}%%\label{key}
l_{\sf Z} (r,R):=
\max \Biggl\{\sum_{r< |{\sf z}_j|\leq R} {\rm Re}^+ \frac{1}{{\sf z}_j},
\sum_{r < |{\sf z}_j|\leq R} {\rm Re}^+ \frac{-1}{{\sf z}_j}\Biggr\}
\end{equation*}
для ${\sf Z}$, где верхний индекс $+$ означает положительную часть числа, функции или множества, а также в терминах $l_{\sf W} (r,R)$.

Для целой, т.е. голоморфной на $\mathbb C$, функции $f$ через ${\sf Zero}_f$ обозначаем распределение её корней, где число повторений каждого корня равно его кратности, а $f$ обращается в нуль на ${\sf Z}$, если для каждой точки $z\in \mathbb C$ число повторений  $z$ в ${\sf Z}$ не больше кратности корня $f$ в $z$ (пишем $f({\sf Z})=0$).

Целая функция $f$ называется {\it целой функцией экспоненциального типа} (пишем ц.ф.э.т), если $\ln |f(z))|\underset{z\to \infty}{\leq} O(|z|)$.

Через $\sf mes$ обозначаем {\it линейную меру Лебега на\/} $\mathbb R$.
\paragraph{Основная теорема.}
{\it Пусть    $\varepsilon \in (0,1)\subset \mathbb R^+$ и для части ${\sf Z}_\varepsilon$ всех точек из    ${\sf Z}$, попавших в пару вертикальных углов
\begin{equation}\label{KhabibullinSalimovaZe}
%%\angle_{\varepsilon}:=
\bigl\{z\in \mathbb C\bigm| |{\rm Re\,}z|< \varepsilon |z|\bigr\},
\end{equation}
найдутся  нумерация  для ${\sf Z}_\varepsilon=({\sf z}_j^{\varepsilon})_{j=1,2,\dots}$, число $c\in \mathbb R^+$, а также последовательность $({\sf m}_j)_{j=1,2,\dots}$ попарно различных целых чисел с такой  же нумерацией, для которых
$$
\sum_j\biggl|\frac{1}{{\sf z}_j^{\varepsilon}}-i\frac{c}{{\sf m}_j}\biggr|<+\infty.
$$
Тогда следующие четыре  утверждения эквивалентны:
\begin{enumerate}
\item[{\rm I.}] Для любой ц.ф.э.т $g\neq 0$ с $g({\sf W})=0$  и для любого числа $p\geq 0$ найдутся ц.ф.э.т. $f\neq 0$ с $f({\sf Z})=0$ и открытое подмножество $E\subset \mathbb R$ , для которых
		\begin{equation}\label{KhabibullinSalimova1}
|f(iy)| \leq |g(iy)| \quad\text{при  всех $y\in \mathbb R\setminus E$,}
\end{equation}
где ${\sf mes}\bigl(E\setminus [-R,R]\bigr)=o\bigl(1/R^{p}\bigr) $ при $R\to +\infty$.
		\item[{\rm II.}] Найдутся  ц.ф.э.т $g\neq 0$ с $g({\sf W})=0$  со свойством
\begin{equation*}
\sup_{1\leq r<R<+\infty}\Bigl(l_{{\sf Zero}_g}(r,R)-l_{{\sf Z}}(r,R)\Bigr)<+\infty,
\end{equation*}
 а также   ц.ф.э.т. $f\neq 0$ с $f({\sf Z})=0$, удовлетворяющие\/ \eqref{KhabibullinSalimova1}, но при более слабом требовании\/   ${\sf mes}(E)<+\infty$.

		\item[{\rm III.}] Существует число $C\in  \mathbb{R}$, для которого
$$
l_{\sf Z}(r,R)\leq l_{\sf W}(r,R)+C\quad\text{при всех\,  $ 0< r< R<+\infty$}.
$$
%%\end{enumerate}
\item[{\rm IV.}]
Существует строго возрастающая  неограниченная последовательность положит\-е\-л\-ь\-н\-ых чисел $(r_n)_{n\in \mathbb N}$, для которой $\lim\limits_{n\to\infty}{r_{n+1}}/{r_n}<+\infty$ и
\begin{equation*}%%\label{cprec}
 \limsup_{N\to  \infty}\sup\limits_{n\leq N}
\Bigl(l_{\sf Z}(r_n,r_N)-l_{\sf W}(r_n,r_N)\Bigr)<+\infty.
\end{equation*}
\end{enumerate}
Если дополнительно предположить, что  часть ${\sf W}_{\varepsilon}$ всех точек из
${\sf W}$, попавших при некотором $\varepsilon \in (0,1)$ в  пару вертикальных углов вида  \eqref{KhabibullinSalimovaZe},
удовлетворяет двустороннему условию типа Бляшке
$$
\sum_{{\sf w}\in {\sf W}_{\varepsilon}} \Bigl|{\rm Re}\frac{1}{\sf w}\Bigr|<+\infty,
$$
то в соотношении \eqref{KhabibullinSalimova1} утверждения\, {\rm I} исключительное множество  $E$ можно выбрать пустым.}

Часть результатов, не вошедшая в цикл работ [1]--[3], оформлена в виде статьи  [4].
Ещё  одно развитие результатов из  [1]--[3], имеющееся  в  [4], касается случая замены жёсткого неравенства \eqref{KhabibullinSalimova1} на более слабое требование
	\begin{equation*}
\ln |f(iy)| \leq \ln |g(iy)| +o\bigl(|y|\bigr)\quad\text{при $|y|\to +\infty$.}
\end{equation*}


\litlist

%%1. {\it Semenov E. M., Sukochev F. A.}
 %%Invariant Banach limits and applications //Journal of Functional Analysis. – 2010. – Т. 259. – №. 6. – С. 1517-1541.

%%\bibitem{SalKha20U}
 1. {\it Салимова А. Е., Хабибуллин Б. Н.\/}
Рост субгармонических функций вдоль прямой и распределение их мер Рисса //
Уфимск. матем. журн.  – 2020. – Т. 12.  –  № 2. –  С. 35--48.

2. {\it Салимова А. Е., Хабибуллин Б. Н.\/}
 Распределение нулей целых функций экспоненциального типа с ограничениями на рост вдоль прямой // Матем. заметки.  – 2020. – Т. 108.  –  № 4. –  С. 588--600.

3. {\it Салимова А. Е., Хабибуллин Б. Н.\/}
Рост целых функций экспоненциального типа и характеристики распределений точек вдоль прямой на комплексной плоскости // Уфимск. матем. журн.  – 2021. – Т. 13.  –  № 3. –  С. 116--128.

4. {\it Салимова А. Е.\/}
Версия теоремы Мальявена\,--\,Рубела для целых функций экспоненциального типа с корнями около мнимой оси // Изв. вузов. Математика.  – 2021. – С. 1--10 (статья направлена в печать).
