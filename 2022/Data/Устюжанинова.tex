\vzmstitle[
	\footnote{Работа выполнена при финансовой поддержке Российского фонда фундаментальных исследований (проект 20-01-00051).}
]{Теорема существования слабого решения задачи оптимального управления с обратной связью для модифицированной модели Кельвина-Фойгта
}
\vzmsauthor{Устюжанинова}{А.\,С.}
\vzmsinfo{Воронеж, Воронежский государственный университет; {\it nastyzhka@gmail.com}}
\vzmsauthor{Турбин}{М.\,В.}
\vzmsinfo{Воронеж, Воронежский государственный университет; {\it mrmike@mail.ru}}

\vzmscaption

В ограниченной области $\Omega\subset \mathbb{R}^n (n=2,3)$ с границей $\partial\Omega$ класса $C^3$ на промежутке времени $[0;T], 0<T<\infty$ рассматривается следующая задача:
\begin{gather}
\frac{\partial v}{\partial t}-\nu\Delta v+\sum_{i=1}^n v_i\frac{\partial v}{\partial x_i}-\varkappa\frac{\partial\Delta v}{\partial t}-\varkappa\sum_{i=1}^n v_i\frac{\partial\Delta v}{\partial x_i}+\nabla p=f; \label{Ustiuzhaninova1} \\
{\rm div}\, v=0; \label{Ustiuzhaninova2}\\
v(x,0)=a(x), \quad x\in\Omega; \label{Ustiuzhaninova3}\\
v|_{\partial\Omega\times[0,T]}=0. \label{Ustiuzhaninova4}
\end{gather}

Система (1)-(2) впервые была введена в рассмотрение В.\,А.~Павловским [1] и подтверждается экспериментальными исследованиями растворов полиэтиленоксида, полиакриламида и гуаровой смолы [2,3]. Существование слабого решения (1)-(4) на произвольном конечном отрезке $[0,T]$ доказано в работе [4].

Рассматривается вопрос о существовании слабого решения задачи оптимального управления с обратной связью для задачи (1)-(4). Введём пространство
\begin{gather*}
W=\left\{u: u\in L_\infty(0,T;V^2), u'\in L_2(0,T;V^1) \right\}
\end{gather*}
с нормой $\|u\|_{W}= \|u\|_{L_\infty(0,T;V^2)}+\|u'\|_{L_2(0,T;V^1)}.$


Предполагается, что внешняя сила (которая и является управлением) принадлежит образу некоторого многозначного отображения $\Psi:W \multimap L_2(0,T;V^0),$ которое зависит от скорости движения жидкости:
\begin{equation}
\label{Ustiuzhaninova5}
f\in \Psi (v).
\end{equation}
Отображение $\Psi$ определено на $W,$ имеет непустые, компактные, выпуклые значения, полунепрерывно сверху, компактно, глобально ограничено и слабо замкнуто.

Для доказательства существования решения рассматриваемой задачи оптимального управления с обратной связью на основе аппроксимационно"=топологического подхода сначала показывается, что существует решение задачи управления с обратной связью. После чего устанавливается, что среди слабых решений задачи управления существует решение дающее минимум заданному функционалу качества.

\paragraph{Определение~1.} {\it Пусть $a\in V^2.$ Пара функций $(v,f)\in W\times {L}_2(0,T;V^{0})$ называется слабым решением задачи \eqref{Ustiuzhaninova1}-\eqref{Ustiuzhaninova5}, если она
удовлетворяет начальному условию $v(0)= a,$ условию обратной связи $ f\in \Psi (v)$ и равенству
\begin{multline*}
\int_\Omega v'\varphi dx+\nu\int_\Omega \nabla v:\nabla\varphi dx -\sum_{i,j=1}^n \int_\Omega v_i v_j \frac{\partial\varphi_j}{\partial x_i}dx\,+\\+\varkappa\int_\Omega \nabla\left(v'\right):\nabla\varphi dx + \varkappa\sum_{i,j=1}^n\int_\Omega v_i \Delta v_j \frac{\partial\varphi_j}{\partial x_i}dx=\int_\Omega f\varphi dx
\end{multline*}
для любого $\varphi \in V^1$ и почти всех $t\in (0,T).$
}

%Правомочность начального условия следует из того, что любая функция $u\in W$ принадлежит $C([0,T],V^1),$ поскольку $u'\in L_2(0,T;V^1).$


Обозначим через $\Sigma \subset W\times L_2(0,T;V^0)$ множество всех слабых решений задачи \eqref{Ustiuzhaninova1}-\eqref{Ustiuzhaninova5}. Рассмотрим произвольный ограниченный снизу, слабо замкнутый функционал качества $\Phi: \Sigma \to \mathbb{R}.$

Основным результатом является следующая теорема о существовании оптимального решения задачи управления с обратной связью:
\paragraph{Теорема~1.} {\it
Для любых определённых выше $\Psi$ и $\Phi$ задача \eqref{Ustiuzhaninova1}-\eqref{Ustiuzhaninova5} имеет хотя бы одно слабое решение $(v_*,f_*)$ такое, что
$$
\Phi (v_*,f_*) = \inf\limits_{(v,f)\in \Sigma}\Phi (v,f).
$$
}

\litlist

1. {\it Павловский В. А.} К вопросу о теоретическом описании слабых водных растворов полимеров //ДАН СССР. - 1971. - Т. 200. - №. 4. - С. 809-812.

2. {\it Амфилохиев В. Б. и др.} Течения полимерных растворов при наличии конвективных ускорений //Тр. Ленинградск. ордена Ленина кораблестроительного института. - 1975. - Т. 96. - С. 3-9.

3. {\it Амфилохиев В. Б., Павловский В. А.} Экспериментальные данные о ламинарно"=турбулентном переходе при течении полимерных растворов в трубах //Тр. Ленинградск. ордена Ленина кораблестроительного института. - 1976. - Т. 104. - С. 3-5.

4. {\it Турбин М. В., Устюжанинова А. С.} Теорема существования слабого решения начально"=краевой задачи для системы уравнений, описывающей движение слабых водных растворов полимеров //Изв. вузов. Математика. - 2019. - №. 8. - С. 62-78.
