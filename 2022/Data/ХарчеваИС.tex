


\vzmstitle[
	\footnote{Исследование выполнено при поддержке Российского фонда фундаментальных исследований (грант {{No}~19-01-00775-а}). \\ Автор является стипендиатом фонда развития теоретической физики и математики ``Базис'' (договор \No 20-8-10-3-1).}
]{Интегрируемые круговые биллиардные книжки с произвольным потенциалом}
\vzmsauthor{Харчева}{И.\,С.}
\vzmsinfo{Москва, МГУ; {\it irina.harcheva01@gmail.com}}

\vzmscaption
Одним из важных обобщений классического математического биллиарда является \textit{биллиардная книжка}, представляющая собой биллиард на двумерном клеточном комплексе. Более точно, рассмотрим несколько занумерованных плоских областей (листов) с общими дугами границы (корешками), и каждой такой дуге припишем некоторую перестановку. Биллиардный шар, двигаясь по одному листу и отражаясь от корешка, переходит на другой лист согласно соответствующей перестановке. Если классический биллиард интегрируем на каждом из листов, их первые интегралы порождают первые интегралы на всей книжке глобально (при определённых условиях согласованности). При этом динамика полученной интегрируемой системы устроена более сложно (а значит, более богата). Понятие биллиардной книжки впервые было введено В.~В.~Ведюшкиной в [1], а многие их важные свойства изложены в [2].

В нашей работе мы рассматриваем биллиардную книжку, листами которой являются области, ограниченные единичной окружностью $x^2+y^2=1$, при этом на общей границе задана циклическая перестановка. Потребуем, что на каждом из листов книжки действует свой потенциал $P_i(x^2+y^2)$, инвариантный относительно поворотов. При этом эти потенциалы согласованы на общей границе "--- корешке, то есть $P_i(1)=P_j(1)$. Такая система является интегрируемой гамильтоновой системой. Её гамильтониан равен полной энергии "--- функции $\frac{\dot{x}^2+\dot{y}^2}{2}+P_i(x^2+y^2)$ на каждом из листов, а дополнительный первый интеграл имеет стандартный циклический вид $x\dot y-y \dot x$ на всех листах (что и обеспечивает интегрируемость). Для этой динамической системы был проведён анализ слоения Лиувилля всевозможных невырожденных изоэнергетических многообразий. А именно, были вычислены инварианты Фоменко-Цишанга (меченые молекулы) и построены бифуркационные диаграммы (более подробно см. [3]). Один из основных результатов приведён в следующей теореме.

\textbf{Теорема:}
	\textit{Для круговой биллиардной книжки с произвольными потенциалами грубый инвариант Фоменко имеет вид симметричного дерева с седловыми атомами серии $B_n$ и, возможно, одной неморсовской бифуркацией.}

% Оформление списка литературы
\litlist

1. \textit{Ведюшкина В. В., Харчева И. С.} Биллиардные книжки моделируют все трёхмерные бифуркации интегрируемых гамильтоновых систем // Математический сборник. -- 2018. -- Т. 209, № 12.

2. \textit{Ведюшкина В. В., Харчева И. С.} Биллиардные книжки реализуют все базы слоений Лиувилля интегрируемых гамильтоновых систем // Математический сборник. -- 2021. -- Т. 212, № 8. -- С. 89–150.

3. {\it Болсинов~А.В., Фоменко~А.Т.}, Интегрируемые гамильтоновы системы. Геометрия, топология, классификация. "--- Ижевск: РХД, 1999.
