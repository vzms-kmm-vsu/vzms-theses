

\vzmstitle[
	\footnote{Работа выполнена при поддержке гранта гранта РНФ №20--71--00155 в МГУ имени М.\,В.~Ломоносова.}
]{Трёхмерный биллиард внутри эллипсоида с потенциалом Гука}
\vzmsauthor{Белозеров}{Г.\,В.}
\vzmsinfo{Москва, МГУ им. М.\,В.~Ломоносова; {\it gleb0511beloz@yandex.ru}}


\vzmscaption



	Пусть $\mathcal{E}\subset \mathbb{R}^3$~--- трёхосный эллипсоид с различными полуосями. Рассмотрим следующую динамическую систему: материальная точка (шар) единичной массы движется внутри $\mathcal{E}$ под действием силы упругости (закон Гука), отражаясь от $\mathcal{E}$ абсолютно упруго.  Эта система является интегрируемой по Лиувиллю системой в кусочно"=гладком смысле. Один из её первых интегралов~--- это  полная механическая энер\-гия, то есть функция:

	\[
	H=\dfrac{1}{2}(\dot{x}^2+\dot{y}^2+\dot{z}^2)+\dfrac{k}{2}(x^2+y^2+z^2).
	\]

	Ещё два первых интеграла $F_1$ и $F_2$, функционально независимых с $H$, можно найти с помощью метода В.\,В.~Козлова, используя дополнительные первые ин\-те\-гра\-лы $I_1, I_2$ задачи без потенциала. Оказывается, что функции $H$, $F_1$ и $F_2$ находятся в инволюции относительно стандартной скобки Пуассона.

	В эллиптических координатах происходит разделение переменных этой задачи. Уравнения движения в них можно переписать так:
	\[
	\dot{\lambda_i}=\pm\dfrac{4}{\sqrt{2}(\lambda_i-\lambda_j)(\lambda_i-\lambda_k)}\sqrt{V(\lambda_i)},
	\]
	где $V(z)$ --- полином 6-ой степени, коэффициенты которого зависят только от констант $k,a,b,c$ и значений интегралов $H,F_1,F_2$. Опираясь на свойства этого полинома, была построена бифуркационная диаграмма, найдены области возможного движения и изучены прообразы точек отображения момента.

	Пусть $h$~--- небифуркационный уровень энергии. Оказывается, что тогда изонергетическая поверхность $Q_h$ является кусочно"=гладким многообразием. Более того верна следующая теорема.

	\paragraph{Теорема.} {\it Пусть $h$~--- небуркационное значение энергии $H$, тогда:
	\begin{enumerate}
		\item если $k>0$, то изоэнергетическая поверхность $Q_h$ гомеоморфна сфере $S^5$,
		\item если $k<0$, то изоэнергетическая поверхность $Q_h$ гомеоморфна
		\begin{itemize}
			\item несвязному объединению двух пятимерных сфер $S^5$ при $h \in \Big(\dfrac{ka}{2}, \dfrac{kb}{2}\Big)$;
			\item прямому произведению окружности и четырёхмерной сферы ${S^1\times S^4}$ при $h \in \Big(\dfrac{kb}{2}, \dfrac{kc}{2}\Big)$;
			\item прямому произведению двумерной и трёхмерной сфер $S^2\times S^3$ при $h\in \Big(\dfrac{kc}{2}, 0\Big)$;
			\item пятимерной сфере $S^5$ при $h \in (0, +\infty)$.
		\end{itemize}
	\end{enumerate}
    }
\litlist

1. {\it  Якоби~К.} Лекции по динамике // М.: Гостехиздат, 1936.

2. {\it Козлов~В.\,В.} Некоторые интегрируемые обобщения задачи Якоби о геодезических на эллипсоиде // Прикладная математика и механика, том 59, {\bf1} 1995.

