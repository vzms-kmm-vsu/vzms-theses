\vzmstitle{
	Обратные задачи по отысканию правой части телеграфного уравнения
}
\vzmsauthor{Сабитов}{К.\,Б.}
\vzmsinfo{Самара, СГТУ; Стерлитамак, СФ БГУ; {\it sabitov\_fmf@mail.ru}}
\vzmsauthor{Зайнуллов}{А.\,Р.}
\vzmsinfo{Стерлитамак, СФ БГУ; {\it arturzayn@mail.ru}}

\vzmscaption

Рассмотрим неоднородное телеграфное уравнение
\begin{equation}\label{q1}Lu\equiv u_{tt}-u_{xx}+bu=F(x,t)\end{equation}
в области $Q=\{0<x<l,\ 0<t<T\}$, где $b,\ l,\ T$ --- заданные действительные числа, при этом $l>0$, $T>0,\ b\not=0,$ и поставим следующие задачи.

\textbf{Задача 1 (Первая начально"=граничная задача).} Найти в области $Q$ функцию $u(x,t)$, удовлетворяющую условиям:
\begin{equation}\label{q2}u(x,t)\in C^2(D)\cap C^1(D\cup\{t=0\})\cap C(\overline{D}),\ u_{tt},\ u_{xx}\in L[0,l];\end{equation}
\begin{equation}\label{q3}Lu\equiv F(x,t),\ (x,t)\in Q;\end{equation}
\begin{equation}\label{q4}u(0,t)=u(l,t)=0,\ 0\leqslant t\leqslant T,\end{equation}
\begin{equation}\label{q5}u(x,0)=\varphi(x),\ 0\leqslant x\leqslant l,\end{equation}
\begin{equation}\label{q6}u_t(x,0)=\psi(x),\ 0\leqslant x\leqslant l,\end{equation}
где $F(x,t),\ \varphi(x)$ и $\psi(x)$ --- заданные достаточно гладкие функции.

{\bf Задача 2.} Пусть $F(x,t)=f(x)g(t)$. Найти функции $u(x,t)$ и $g(t)$, удовлетворяющие условиям (\ref{q2}) -- (\ref{q6}), и, кроме того, дополнительному условию
\begin{equation}\label{q8.1.0}g(t)\in C[0,T],\end{equation}
\begin{equation}\label{q8.1}u(x_0,t)=h(t),\ 0\leqslant t\leqslant T,\end{equation}
где $x_0$ --- заданная фиксированная точка отрезка $[0,l]$, $\varphi(x),\ \psi(x),\ h(t)$ и $f(x)$ --- заданные достаточно гладкие функции, при этом $\varphi(x_0)=h(0).$

{\bf Задача 3.} Пусть $F(x,t)=f(x)g(t)$. Найти функции $u(x,t)$ и $f(x)$, удовлетворяющие условиям (\ref{q2}) -- (\ref{q6}), и, кроме того, дополнительному условию
\begin{equation}\label{q8.2.0}f(x)\in C(0,l)\cap L[0,l],\end{equation}
\begin{equation}\label{q8.2}u(x,t_0)=\widetilde{\varphi}(x),\ 0\leqslant x\leqslant l,\end{equation}
где $t_0$ --- заданная фиксированная точка отрезка $(0,T]$, $\varphi(x),\ \psi(x),\ \widetilde{\varphi}(x)$ и $g(t)$ --- заданные достаточно гладкие функции.

Отметим, что данная статья является продолжением исследований работ [1,2], где были изучены обратные задачи по нахождению начальных условий $\varphi(x)$ и $\psi(x)$ в начально"=граничной задаче \eqref{q2} -- \eqref{q6} с дополнительным условием \eqref{q8.2}. Здесь ставятся обратные задачи по отысканию правой части $F(x,t)$ телеграфного уравнения \eqref{q1}.

Аналогичные обратные задачи для уравнения теплопроводности изучались в работах [3, с. 118 -- 120], [4, c. 248 -- 252], [5]. В работе [3, с. 123 -- 126] для одномерного уравнения теплопроводности

$$u_t-a^2u_{xx}=f(x)g(t),\ 0<x<l,\ 0\leqslant t\leqslant T,$$
с граничными и начальным условиями
$$u_x(0,t)=u_x(l,t)=0,\ 0\leqslant t\leqslant T,$$
$$u(x,0)=0,\ 0\leqslant x\leqslant l,$$
изучены обратные задачи по отысканию множителей $g(t)$ и $f(x)$ правой части с заданием дополнительного условия
\begin{equation}\label{aa1}u(x_0,t)=h(t),\ 0\leqslant t\leqslant T,\ x_0\in[0,l].\end{equation}
Для обратной задачи по определению функций $u(x,t)$ и $g(t)$ доказана теорема единственности и существования решения, когда $f(x_0)\not=0$.

Отметим также работы [6, 7], где изучены обратные задачи по определению коэффициента при неизвестной функции телеграфного уравнения.

В данной работе для уравнения \eqref{q1} изучены задачи 1 -- 3. На основании формулы решения прямой задачи 1 обратная задача 2 по нахождению сомножителя правой части, зависящей от
времени, эквивалентно редуцирована к интегральному уравнению Вольтерра второго рода. Из которого получена
теорема об однозначной разрешимости этой обратной задачи. Решение обратной задачи 3 по
определению сомножителя правой части, зависящей от пространственной координаты,
построено в виде ряда Фурье по системе собственных функций соответствующей одномерной спектральной задачи; установлен критерий единственности и доказана теорема существования решения поставленной задачи.


\litlist

1. {\it Сабитов~К.Б., Зайнуллов~А.Р.}
 Обратные задачи по определению начальных условий в смешанной задаче для телеграфного уравнения // Итоги науки и техн. Сер. Соврем. мат. и ее прил. Темат. обз. - 2017. - Т. 141. - C. 111 -- 133.

2. {\it Sabitov~K.B., Zaynullov~A.R.} Inverse problems for initial conditions of the mixed problem
 for the telegraph equation // Journal of Mathematical Sciences. - 2019. - V. 241. - No. 5. - P. 622 -- 645.

3. {\it Денисов~А.М.} Введение в теорию обратных задач. М.: Изд--во МГУ, 1994. - 208 c.

4. {\it Кабанихин~С.И.} Обратные и некорректные задачи. Новосибирск: Сибирское научное
 издательство, 2009. - 457 c.

5. {\it Сабитов~К.Б., Зайнуллов~А.Р.} Обратные задачи для уравнения теплопроводности по отысканию
 начального условия и правой части // Учен. зап. Казан. ун-та. Физ.--мат. науки. - 2019. - Т. 161. - №2. - С. 271 -- 291.

6. {\it Романов~В.Г.} Одномерная обратная задача для телеграфного уравнения // Диффер. уравнения.
 1968. - Т. 4. - №1. - С. 87 -- 101.

7. {\it Кожанов~А.И., Сафиуллова~Р.Р.} Определение параметров в телеграфном уравнении // Уфимский
 матем. журнал. 2017. - Т. 9. - №1. - С. 63 -- 74.
