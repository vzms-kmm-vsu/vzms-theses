\vzmstitle[
	\footnote{Работа выполнена при поддержке гранта РФФИ и ВАНТ, проект 21-51-54003, а также гранта Президента РФ поддержки ведущих научных школ, проект НШ-2708.2022.1.1.}
]{О начальных задачах для интегро"=дифференциальных уравнений}
\vzmsauthor{Годова}{А.\,Д.}
\vzmsinfo{Челябинск, ЧелГУ; {\it sashka\_1997\_godova55@mail.ru}}
\vzmsauthor{Федоров}{В.\,Е.}
\vzmsinfo{Челябинск, ЧелГУ; {\it kar@csu.ru}}

\vzmscaption

Рассмотрим  начальные задачи для простейших интегро"=диффе\-рен\-ци\-аль\-ных уравнений типа Римана~--- Лиувилля и типа Герасимова.

\paragraph{Интегро"=дифференциальное уравнение типа Римана~--- Лиувилля.}

Пусть $\mathcal X$ "--- банахово пространство,  $A\in\mathcal{L}(\mathcal X)$,
$K\in C([0, \infty);\mathcal{L}(\mathcal X)).$
Определим свёртку $$(J^K x)(t):=\int_0^t  K(t-s)x(s)ds$$
и  интегро"=диф\-фе\-рен\-ци\-альный оператор %типа Римана-Лиувиля
$$(D^{m,K}x)(t):=D^m(J^K x)(t):=D^m \int_0^t  K(t-s)x(s)ds,$$
где $D^m$ "--- производная порядка $m$.
Рассмотрим задачу типа Коши
$$(J^Kx)^{(k)}(0)=x_k \in\mathcal X, \;\; k=0,1, \ldots, m-1, \eqno(1)$$
для уравнения
$$(D^{m, K}x)(t)=Ax(t),\,\, t>0.\eqno(2)$$


Подействуем преобразованием Лапласа
на обе части уравнения (2) и получим
$$\widehat{D^{m, K}x}(\lambda)=\widehat{D^m J^K x}(\lambda)=\lambda^m\widehat{J^K x}(\lambda)-\sum_{k=0}^{m-1}(J^Kx)^{(k)}(0)\lambda^k=$$
%преобразование Лапласа от свёртки - это произведение преобразования Лапласа
$$=\lambda^m \widehat{K}(\lambda)\widehat{x}(\lambda)-\sum_{k=0}^{m-1}(D^{k,K}x)(0)\lambda^k=A\widehat{x}.\eqno(3) $$


Пусть $l\in\{0,1,\dots,m-1\}$, $x_k=0$ при $k\in\{0,1,\dots,m-1\}\setminus\{l\}.$ Тогда
уравнение (3) примет вид
$\lambda^m \widehat{K}(\lambda)\widehat{x}(\lambda)-A\widehat{x}(\lambda)=x_l\lambda^{l},$ отсюда
$\widehat{x}(\lambda)=(\lambda^m\widehat{K}(\lambda)I-A)^{-1}\lambda^{l}x_{l}$
при условии
$$
\lim\limits_{|\lambda|\to\infty}\lambda^m\widehat{K}(\lambda)>\|A\|_{\mathcal{L}(\mathcal X)}.\eqno(4)
$$
%нашли преобразование Лапласа. Теперь нам известен образ, а нужно найти оригинал, т.е. x(t)
Действительно, раскладывая обратный оператор в ряд Неймана, получим $$(\lambda^m\widehat{K}(\lambda)I-A)^{-1}=\sum_{k=0}^{\infty}\frac{A^k}{\lambda^{m(k+1)}\widehat{K}(\lambda)^{k+1}},$$
поскольку для  достаточно большого $R$ при $|\lambda|>R$ имеем
$\|A\|_{\mathcal{L}(\mathcal X)}<|\lambda |^m\widehat{K}(\lambda)$.

Подействуем обратным преобразованием Лапласа и получим решение $x_l(t):=X_l(t)x_l$ задачи (1), (2) при выбранных $x_k$, $k=0,1,\dots,m-1$, где
$$X_l(t)=\frac{1}{2\pi i}\int\limits_{\gamma}(\lambda^m \widehat{K}(\lambda)I-A)^{-1}\lambda^{l}e^{\lambda t}d\lambda,\quad l=0,1,\dots,m-1,$$
где $\gamma:=\gamma_R\cup\gamma_{R,+}\cup\gamma_{R,-}$~--- положительно ориентированный контур, $\gamma_R:=\{Re^{i\varphi}:\varphi\in(-\pi,\pi)\}$, $\gamma_{R,+}:=\{re^{i\pi}:r\in[R,\infty)\}$, $\gamma_{R,-}:=\{re^{-i\pi}:r\in[R,\infty)\}$.


\paragraph{Теорема~1.} {\it
Пусть $A\in\mathcal{L}(\mathcal X)$ и выполняется условие {\rm(4)}, тогда при любых
$x_0,x_1, \ldots, x_{m-1}\in\mathcal X$ существует единственное решение задачи {\rm (1), (2)}, при этом оно имеет вид
$$x(t)=\sum\limits_{k=0}^{m-1} X_k(t)x_k.$$
}

%\newpage

\paragraph{Интегро"=дифференциальное уравнение типа Герасимова}

Введём в рассмотрение интегро"=диф\-фе\-рен\-ци\-альный оператор типа Герасимова
$$(D^{K,m}x)(t):=J^K (D^m x)(t):= \int_0^t  K(t-s)x^{(m)}(s)ds.$$
Рассмотрим задачу  Коши
$$x^{(k)}(0)=x_k \in\mathcal X, \;\; k=0,1, \ldots, m-1, \eqno(5)$$
для уравнения
$$(D^{K, m}x)(t)=Ax(t),\,\, t\geqslant0.\eqno(6)$$


Рассуждая, как в предыдущем параграфе, получим равенство
$$\widehat{x}(\lambda)=(\lambda^m\widehat{K}(\lambda)I-A)^{-1}\widehat{K}(\lambda)\sum_{k=0}^{m-1}\lambda^k x^{(k)}(0)$$
при условии (4).


\paragraph{Теорема~2.} {\it
Пусть $A\in\mathcal{L}(\mathcal X)$ и выполняется условие {\rm(4)}, тогда при любых
$x_0,x_1, \ldots, x_{m-1}\mathcal X$ существует единственное решение задачи {\rm(5), (6)}, при этом оно имеет вид
$$x(t)=\sum\limits_{k=0}^{m-1} Y_k(t)x_k,$$
$$Y_k(t)=\frac{1}{2\pi i}\int\limits_{\gamma}(\lambda^m \widehat{K}(\lambda)I-A)^{-1}\lambda^{k}\widehat{K}(\lambda)e^{\lambda t}d\lambda,$$
$$k=0,1,\dots,m-1.$$
}

\paragraph{Пример.}
Возьмём $m-1<\alpha\leqslant m\in\mathbb N$,  $K_\alpha(s):=\frac{s^{\alpha-1}}{\Gamma(\alpha)}I,$ тогда $J^{K_\alpha}:=J^\alpha$~--- оператор дробного интегрирования Римана~--- Лиувилля порядка $\alpha$ (см., например, [1]), $D^{m,K_\alpha}:={}^{RL}D^\alpha$~--- оператор дробного дифференцирования  Римана~--- Лиувилля порядка $\alpha$ [1], $D^{K_\alpha,m}:={}^{GC}D^\alpha$~--- оператор дробного дифференцирования  Герасимова~--- Капуто порядка $\alpha$ [2].


% Оформление списка литературы

\litlist

1. {\it Федоров В. Е., Туров М. М.} Дефект задачи типа Коши для линейных уравнений с несколькими производными Римана – Лиувилля // Сиб. мат. журн. - 2021. - Т.62. - №.~5.~- С. 1143-1162.

2. {\it Волкова А. Р., Ижбердеева Е. М., Федоров В. Е.} Начальные задачи для уравнений с композицией дробных производных // Челяб. физ.-мат. журн. - 2021. - Т.6. - Вып.~3.~- С. 269-277.
