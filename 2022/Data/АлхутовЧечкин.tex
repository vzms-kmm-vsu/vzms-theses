\vzmstitle[
    \footnote{
		Алхутов Ю.А. поддержан грантом РНФ (проект 22-21-00292).
	}
]
{Об оценках Боярского-Мейерса для решений задачи Зарембы}

\vzmsauthor{Алхутов}{Ю.\,А.}
\vzmsinfo{Владимир, ВлГУ им. А.Г. и Н.Г. Столетовых; {\it yurij-alkhutov@yandex.ru}}
\vzmsauthor{Чечкин}{А.\,Г.}
\vzmsinfo{Москва, МГУ им. М.В. Ломоносова; {\it gregory.chechkin@gmail.com}}
\vzmscaption


Рассмотрим в ограниченной строго липшицевой области $D\subset \mathbb{R}^n$, где $n\ge 2$,  оператор 
$$
\mathcal{L}u:=\text{div} (a(x)\nabla u)
$$
с равномерно эллиптической измеримой и симметрической матрицей $a(x)=\{a_{ij}(x)\}$, то есть такой, что $a_{ij}=a_{ji}$ и
$$
\alpha^{-1}|\xi|^2\le \sum\limits_{i,j=1}^na_{ij}(x)\xi_i\xi_j\le\alpha |\xi|^2~$$
$$\mbox{для~п.~в.}~x\in D~\mbox{и~для~всех}~\xi\in \mathbb{R}^n.
$$
Заметка посвящена свойствам решений задачи Зарембы
$$
\mathcal{L}u=\text{div} f\ \text{в}\ D,\ u=0\ \text{на}\ F,\ \frac{\partial u}{\partial \nu}=0\ \text{на}\  \partial D\setminus F, \eqno{(1)}
$$
где $f\in L_2(D)$, $F\subset D$ -- замкнутое множество, а   $\partial u/\partial \nu$ означает внешнюю конормальную производную.

Обозначим через $W^1_2(D, F)$ пополнение множества бесконечно дифференцируемых в замыкании $D$ функций, равных нулю в окрестности $F$, по норме
$$
\parallel u\parallel_{W^{1}_2(D, F)}=\biggl (~\int\limits_{D} u^2\,dx+\int\limits_{D}|\nabla u|^2\,dx\biggr )^{1/2}.
$$
Под решением задачи (1) понимается функция $u \in W^1_2(D, F)$, для которой выполнено интегральное тождество
$$
\int\limits_{D} a\nabla u\cdot\nabla\varphi\,dx=\int\limits_{D} f\cdot\nabla\varphi\,dx
$$
для всех пробных функций $\varphi\in  W^1_2(D, F)$.

Нас интересует повышенная суммируемость градиента решения задачи (1) в предположении, что $f\in L_q(D)$, где $q>2$.   
Вопрос о повышенной суммируемости градиента решений эллиптических уравнений является классическим и восходит к работе [1], в которой рассмотрена задача Дирихле для линейных дивергентных равномерно эллиптических уравнений второго порядка с измеримыми коэффициентами на плоскости. Позже в многомерном случае и уравнений такого же вида повышенная суммируемость градиента решения задачи Дирихле в области с достаточно регулярной границей была установлена в [2]. После этой работы оценки повышенной суммируемости градиента решений общепринято называть оценками типа Мейерса.

При рассмотрении задачи Зарембы (1) ключевую роль играет условие на структуру множества носителя данных Дирихле $F$.
Определим для компакта $K\subset \mathbb{R}^n$ емкость $C_p(K)$, которая при $1<p<n$ определяется  равенством
$$
C_p(K)=\inf~ \biggl \{~ \int\limits_{\mathbb{R}^n}|\nabla\varphi|^p\,dx:~\varphi\in C^\infty_0 (\mathbb{R}^n),~\varphi\ge 1~\mbox{на}~K\biggr \}.
$$

Пусть $B^{x_0}_r$ --- открытый $n$-мерный шар радиуса $r$ с центром в точке $x_0$, а $mes_{n-1}(E)$ -- (n-1)-мерная мера множества $E$.
Пусть также $p=2n/(n+2)$ при $n>2$ и $p=3/2$ при $n=2$. Предполагается выполнение одного из следующих условий: для произвольной точки $x_0\in F$ при $r\le r_0$ справедливо либо неравенство
$$
C_p(F\cap \overline B^{x_0}_r)\ge c_0 r^{n-p}, \eqno{(2)}
$$
либо неравенство
$$
mes_{n -1}(F\cap \overline B^{x_0}_r)\ge c_0 r^{n-1}, \eqno{(3)}
$$
в которых положительная постоянная $c_0$ не зависит от $x_0$ и $r$.

Условие (3) более сильное, чем (2), зато является более наглядным.
Отметим, что при выполнении любого из этих условий для функций $v\in W^1_2(D, F)$ справедливо неравенство Фридрихса
$$
\int\limits_{D} v^2\,dx\le C\int\limits_{D} |\nabla v|^2\, dx,
$$
которое в силу теоремы Лакса-Мильграма влечет однозначную разрешимость задачи (1).

Основной результат состоит в следующем утверждении.

\noindent{\bf Теорема.} {\it
Если $f\in L_{2+\delta_0}(D)$, где $\delta_0>0$, то существуют положительные постоянные $\delta(n,\delta_0)<\delta_0$ и $C$ такие, что
для решения задачи Зарембы справедлива оценка
$$
\int\limits_{D}|\nabla u|^{2+\delta}dx\leq C\int\limits_{D}|f|^{2+\delta}\ dx,
$$
где $C$ зависит только от $\delta_0$, размерности пространства $n$, величин $c_0$, $r_0$ из (2) и (3), постоянной $\alpha$ из условия
равномерной эллиптичности, а также от  характеристик липшицевой области $D$}.


\litlist

1. {\it Б.В.Боярский}
Обобщенные решения системы дифференциальных уравнений первого порядка эллиптического типа с разрывными коэффициентами // Матем. сб., Т. 43(85) (4, 1957). С.  451--503.

\selectlanguage{english}

2. {\it N.G.Meyers}
An $L^p$--estimate for the gradient of solutions of second order elliptic divergence equations // Annali della Scuola Normale Superiore di Pisa, Classe di Scienze 3-e s\'erie.  T. 17, (3, 1963). P. 189--206.
