\vzmstitle[
	\footnote{Работа выполнена при финансовой поддержке РФФИ в рамках проектов
20-31-90087,
20-08-01154.}
]{Существование позитивных решений
задач на собственные значения
с нелинейной зависимостью от спектрального параметра}
\vzmsauthor{Соловьёв}{П.\,С.}
\vzmsinfo{Казань, КФУ; {\it pavel.solovev.kpfu@mail.ru}}

\vzmscaption

Пусть
$\Omega$ -- плоская область с липшицевой границей $\Gamma$,
$\overline{\Omega}$ -- замыкание области $\Omega$.
Изучается задача нахождения наименьшего собственного
значения ${\lambda\in\Lambda}$, ${\Lambda=[0,\infty)}$, и соответствующей
положительной собственной функции $u(x)$, ${x\in\Omega}$,
удовлетворяющих в обобщённом смысле однородному
дифференциальному уравнению
в частных производных
второго порядка и
однородному граничному условию Дирихле
\begin{equation*}
\begin{array}{l}
\displaystyle
-\sum_{i=1}^{2}\frac{\partial}{\partial x_i}\left(p(\lambda s(x))\frac{\partial u}{\partial x_i}\right)=
r(\lambda s(x))u,\quad
x\in\Omega,
\bigskip
\\
\displaystyle
u(x)=0,\quad
x\in\Gamma.
\end{array}
\label{sol_f1}
\end{equation*}
Предположим, что функции
$p(\mu)$,
$r(\mu)$, $\mu\in\Lambda$,
$s(x)$, $x\in\overline{\Omega}$
являются непрерывными положительными.
Задачи такого вида возникают при моделировании баланса заряженных частиц
высокочастотного индукционного разряда
пониженного давления
[1--9].

При фиксированном $\mu\in\Lambda$
через $\gamma(\mu)$ обозначим
минимальное собственное значение
параметрической линейной задачи на собственные значения
\begin{equation*}
\begin{array}{l}
\displaystyle
-\sum_{i=1}^{2}\frac{\partial}{\partial x_i}
\left(p(\mu s(x))\frac{\partial u}{\partial x_i}\right)=
\gamma(\mu)
r(\mu s(x))u,\quad
x\in\Omega,
\bigskip
\\
\displaystyle
u(x)=0,\quad
x\in\Gamma.
\end{array}
\label{sol_f1}
\end{equation*}
Тогда собственное значение исходной нелинейной задачи на собственные значения
является корнем характеристического уравнения
$\gamma(\lambda)=1.$
С помощью характеристического уравнения исследовано существование
и единственность
ведущего собственного значения и
соответствующей положительной собственной функции
обобщённой постановки
дифференциальной задачи на собственные значения
в частных производных
второго порядка
с нелинейной зависимостью от спектрального параметра.

Исходная нелинейная дифференциальная задача на собственные значения аппроксимируется
с помощью сеточной схемы метода конечных элементов с линейными
Лагранжевыми конечными элементами на регулярной треугольной
сетке.
Устанавливаются оценки погрешности приближённых собственных значений
и собственных функций в зависимости от размера треугольной сетки.

\litlist

1.
{\it Zheltukhin V. S., Solov'ev S. I., Solov'ev P. S., Chebakova V. Yu.}
{Computation of the minimum eigenvalue for a nonlinear Sturm--Liouville problem}
//Lobachevskii J. Math.~--
2014.~-- V.~35.~-- No~4.~-- P.~416--426.

2.
{\it Zheltukhin V. S., Solov'ev S. I., Solov'ev P. S., Chebakova V. Yu.}
{Existence of solutions for electron balance problem in the stationary radio-frequency induction discharges}
//IOP Conf. Ser.: Mater. Sci. Engin.~--
2016.~-- V.~158.~-- No~1.~-- Art.~012103.~-- P.~1--6.

3.
{\it Zheltukhin V. S., Solov'ev S. I., Solov'ev P. S., Chebakova V. Yu., Sidorov A. M.}
{Third type boundary conditions for steady state ambipolar diffusion equation}
//IOP Conf. Ser.: Mater. Sci. Engin.~--
2016.~-- V.~158.~-- No~1.~-- Art.~012102.~-- P.~1--4.

4.
{\it Solov'ev S. I., Solov'ev P. S.}
{Finite element approxima\-tion of the minimal eigenvalue of
a nonlinear eigenvalue problem}
//Lobachevskii J. Math.~--
2018.~-- V.~39.~-- No~7.~-- P.~949--956.

5.
{\it Samsonov A. A., Solov'ev P. S., Solov'ev S. I.}
{The bisec\-tion method for solving the nonlinear bar eigenvalue problem}
//J. Phys.: Conf. Ser.~--
2019.~-- V.~1158.~-- No~4.~-- Art.~042011.~-- P.~1--5.

6.
{\it Samsonov A. A., Solov'ev P. S., Solov'ev S. I.}
{Spectrum division for eigenvalue problems with nonlinear dependence on the parameter}
//J. Phys.: Conf. Ser.~--
2019.~-- V.~1158.~-- No~4.~-- Art.~042012.~-- P.~1--5.

7.
{\it Solov'ev S. I.}
{The error of the Bubnov--Galerkin method with perturbations
for symmetric spectral problems with a non-linearly occurring parameter}
//Comput. Math. Math. Phys.~--
1992.~-- V.~32.~-- No~5.~-- P.~579--593.

8.
{\it Solov'ev S. I.}
{Approximation of operator eigenvalue prob\-lems in a Hilbert space}
//IOP Conf. Ser.: Mater. Sci. Engin.~--
2016.~-- V.~158.~-- No~1.~-- Art.~012087.~-- P.~1--6.

9.
{\it Solov'ev S. I.}
{Quadrature finite element method for ellip\-tic eigenvalue problems}
//Lobachevskii J. Math.~--
2017.~-- V.~38. -- No~5.~-- P.~856--863.
