


\vzmstitle[
	\footnote{Работа выполнена при поддержке гранта РНФ 20-71-00155 в Московском университете имени М.В.Ломоносова}
]{
	Интегрируемый топологический биллиард в магнитном поле
}
\vzmsauthor{Пустовойтов}{С.\,Е.}
\vzmsinfo{Москва, МГУ; {\it pustovoitovse1@mail.ru}}

\vzmscaption

Рассмотрим биллиард внутри плоской компактной области с абсолютно упругим отражением от границы. Потребуем, что на биллиардный шар действует постоянное магнитное поле, ортогональное к плоскости биллиардного стола. Такая система является гамильтоновой с гамильтонианом $H$-- кинетической энергией. Известно, что в общем случае такой биллиард не является вполне интегрируемым по Лиувиллю. Более того, А.~Е.~Мироновым и М.~Бялым в [1] было доказано, что для выпуклого биллиардного стола интегрируемость будет иметь место только в случае биллиарда в круге (за исключением конечного числа значений сигнатуры магнитного поля). При этом дополнительный первый интеграл имеет следующий вид
$$F=\frac{\dot{x}^2+\dot{y}^2}{2}+\frac{k}{2}(x^2+y^2)+k(x\dot{y}-y\dot{x})$$
Оказывается, что магнитный биллиард, ограниченный двумя концентрическими окружностями, тоже интегрируем, причём первые интегралы те же самые. Нами было изучено слоение Лиувилля таких биллиардов. А именно, были вычислены инварианты Фоменко-Цишанга (см. [2]) всевозможных невырожденных многообразий $Q^3$. В частности, грубый инвариант Фоменко таких биллиардов имеет вид $A$--$A$.

Теперь рассмотрим несколько копий описанных круговых и кольцевых биллиардов и склеим их по границам таким образом, что полученное конфигурационное пространство являлось ориентируемым многообразием. В частности, по одной границе не могут быть склеены более двух листов. При отражении от границы склейки биллиардный шар переходит на соседний лист. Впервые подобные биллиардные системы были рассмотрены Ведюшкиной в [3] и называются топологическими биллиардами. Конфигурационное пространство магнитного топологического биллиарда может быть гомеоморфно одному из четырёх многообразий: цилиндру (биллиард $bC$), диску (биллиард $bD$), сфере ($bS$) или тору ($bT$). Для каждого из четырёх таких типов были вычислены инварианты Фоменко-Цишанга, а также построены бифуркационные диаграммы. В частности, имеет место следующая теорема

\textbf{Теорема}: В магнитных топологических биллиардах типа $bC$, $bD$ и $bS$ реализуются любые 3-атомы без звёздочек, имеющие критический слой, гомеоморфный критическому слою 3-атома $B_n$ для произвольного $n$. В магнитном биллиарде $bT$ также реализованы 3-атомы без звёздочек, имеющие критический слой 3-атома $C_n$.


% Оформление списка литературы
\litlist

1. {\it M. Bialy, A. E. Mironov}, “Algebraic non-integrability of magnetic billiards”,
J. Phys. A, 49:45 (2016), 455101, 18 pp.

2. {\it Болсинов~А.В., Фоменко~А.Т.}, Интегрируемые гамильтоновы системы. Геометрия, топология, классификация.
--- Ижевск: РХД, 1999.

3. {\it Фокичева~В. В.}, Топологическая классификация биллиардов в локально плоских областях, ограниченных дугами софокусных квадрик, Матем. сб., 206:10 (2015), 127-176.

