\vzmstitle[
]{
	Об операторе приращения, действующем на пространствах графовых функций
}
\vzmsauthor{Колмыкова}{Е.\,В.}
\vzmsinfo{Воронеж, ВГУ; {\it kolmykova\_kate@bk.ru}}


\vzmscaption




Рассматриваются пространства функций, определённых на множествах графов.
Вводится понятие оператора приращения (приращения функции на ребре графа).
Этот оператор обладает свойствами, схожими со свойствами оператора дифференцирования,
а обратное к нему многозначное отображение напоминает неопределённый интеграл.
Обсуждаются возможные применения этих понятий и утверждений. 

\textbf{1. Пространства графовых функций.}
Пусть  $\mathcal{T}$ "--- множество всех деревьев.
А $\mathcal{T_*}$ "---  множество всех деревьев, кроме дерева, состоящего из одной вершины.


Через $L_1$ обозначим линейное пространство, состоящее из всевозможных вещественных функций $f(X)$, где переменная $X$ пробегает множество $\mathcal{T}$.
А через $L_2$ "--- линейное пространство, состоящее из всевозможных вещественных функций $g(X,l)$, где переменная $X$ пробегает множество $\mathcal{T}_*$, а переменная $l$ пробегает множество рёбер дерева $X$.

\textit{Примеры.} 
1)В [3] введены следующие функции $\theta_n \in L_1$, которые мы будем использовать ниже:
$\theta_n(T)$ "--- это сумма $n$-ных степеней степеней вершин дерева $T$. 


2) Определим функции $\gamma_n \in L_2$.
Пусть $l$ ребро дерева $ T $, через $d_1(T,l)$ и $d_2(T,l)$ обозначим число рёбер, инцидентных одной вершине ребра $l$ и число рёбер, инцидентных другой его вершине.
Положим $\gamma_n(T,l)=d_1^n(T,l) + d_2^n(T,l)$

\textbf{2. Оператор приращения.}
Введём понятие приращения функции на ребре дерева. Пусть $s: \mathcal{T} \rightarrow \mathbb{R}$. 
Рассмотрим какое"=то дерево $T$  и какое"=то его ребро $l$.  
Удалим это ребро (все вершины дерева $T$ останутся и все другие ребра дерева $T$ останутся ). 
Получится лес, состоящий из двух деревьев.
Обозначим их $T_1$ и $T_2$. 
Приращением функции $s$ на ребре $l$ дерева $T$ назовём число $s(T)- (s(T_1)-s(T_2))$.
Обозначим его $[s, T,l]$.

Оператором приращения назовём оператор $\Delta: L_1 \rightarrow L_2$, определённый следующим образом: $\Delta (f) (T,l)= [f, T, l]$.
Таким образом, оператор $\Delta$ переводит вещественную функцию, определённую на множестве всех деревьев, в функцию, определённую на множестве пар $(T,l)$, где $T$ пробегает $\mathcal{T_*}$, а $l$ пробегает множество рёбер дерева $T$.


Например, легко увидеть, что $\Delta (\theta_2) = 2\gamma_1 +2$.


\textbf{3. Свойства операторов приращения.}
Оператор $\Delta$ является линейным. 
Вместо $\Delta(f)$ мы будем писать просто $\Delta f$. 


Через $C$ обозначим множество всех вещественных функций"=констант, определённых на $\mathcal{T}$, произведение $C \theta_0$ понимается в естественном смысле: оно состоит из всевозможных произведений $f\theta_0$,  где $f \in C$.


Оператор $\Delta : L_1 \rightarrow L_2$ не является инъективным. Обратное отображение (определённое на образе оператора $\Delta $)является многозначным, обозначим его $I$.
Его областью определения $D(I)$ является $\Delta (L_1)$.

Сформулируем свойства отображений $\Delta$ и $I$ (они напоминают свойства операций дифференцирования и интегрирования из классического анализа).



\textbf{Теорема 1.} Общим решением  уравнения $\Delta y =0$ является $y = C \theta_0$. 



\textbf{Теорема 2.} Если $\Delta F = f$, то $I (f) = F + C\theta_0$.


\textbf{Теорема 3.}
 1) Для любых $f, g \in D(I)$
справедливо равенство
$I(f+g)=I(f)+I(g)$.


2) Для любого $f\in D(I)$ и любого ненулевого $\alpha \in \mathbb{R}$ справедливо равенство
$I(\alpha f)=\alpha I(f)$.

\textit{Лемма.}
Справедливы равенства: 

$$
\Delta\theta_n= \sum\limits_{i=0}^{n-1} C_n^i \gamma_i
$$
при $n\in \mathbb{N}$.


\textbf{4. Возможные применения.}
С каждым деревом можно связать многочлен "--- характеристический многочлен преобразования Кокстера (см.[1]).
Важнейшей задачей, связанной с преобразованиями Кокстера является нахождение их спектральных свойств(см.[2]).


Коэффициенты характеристического многочлена преобразования Кокстера, связанного с деревом, – это некоторые числовые характеристики дерева.
В [3] найдены формулы, выражающие четыре коэффициента этого многочлена (при $\lambda ^0, \lambda ^1, \lambda ^2, \lambda ^3$) через более простые числовые характеристики деревьев.
Доказательства этих формул являются довольно громоздкими. 
При помощи разработанной нами техники можно эти доказательства существенно упростить, а также получать формулы для других коэффициентов. 



\litlist

1. {\it Бернштейн И.Н., Гельфанд И.М., Пономарев В.А.}
 Функторы Кокстера и
теорема Габриеля // Успехи мат. наук. - 1973. - Т.28, вып.2 (170). - C.19-33.
 
2. {\it Stekolshchik R.}
 Notes on Coxeter Transformations and the McKay Correspondence. -  Berlin Heidelberg: Springer-Verlag, 2008. - 239 p
 
3. {\it Колмыков В. А., Меньших В. В., Субботин В. Ф., Сумин М. В.}
 Четыре коэффициента характеристического многочлена преобразования Кокстера //
Мат. Замет., - 2003. – Т. 73, N 5. – С. 788-791.
