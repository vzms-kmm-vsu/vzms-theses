


\vzmstitle{
	Разработка ПО для сжатия с потерями фотографий печатного и рукописного текста
}
\vzmsauthor{Момот}{Е.\,А.}
\vzmsinfo{Воронеж, ВГУ; {\it y-kate@ukr.net}}

\vzmscaption

Проект <<{koTspect}>> "--- это программа для неравномерного сжатия с потерями фотографий учебников и конспектов [1].

Программа обрабатывает данные о цвете изображения и с их помощью определяет, какие области изображения
не содержат важной информации. Математическая модель основана на сравнении среднеквадратичных отклонений [2]
цветовых параметров {RGB} в небольших областях (плитках) с аналогичными данными всего изображения (или его части, если выбрана
раздельная обработка) и некоторыми заданными порогами.

Если среднеквадратичное отклонение по цветовым параметрам в рассматриваемой области ниже заданного порога,
это значит, что область почти однородная. Такая область считается не содержащей важной информации и закрашивается её
средним цветом.

Отдельно обрабатываются области на фото, не лежащие на листе бумаги. Как правило, они находятся с краю и имеют
не <<бумажный>> цвет, который мы определяем по средним цветовым данным изображения. Таким образом, если какой"=нибудь
цветовой параметр области отклоняется от среднего параметра изображения более чем на произведение среднеквадратичного
отклонения по этому параметру на заданный порог и при этом есть непрерывный путь по другим не <<бумажным>> плиткам к
краю изображения, область считается чужеродной и, следовательно, также не содержащей важной информации. Закрашиваются
эти плитки средним цветом ближайших бумажных плиток.

Условие на наличие непрерывного пути по не <<бумажным>> плиткам до края картинки позволяет сохранить возможные цветные
рисунки на самом листе. Закрашивание цветных плиток начинается от края листа по спирали, что позволят получить
двовольно плавные и даже почти красивые цветовые переходы.

Совместимость обработки <<{koTspect}>> с кодированием в {JPEG} и другими методами сжатия [3 -- 5] позволяет получить выигрыш в размере
до двух раз. Подробнее о том, как пользоваться программой и о возможных результатах, см. [6].

Основные функции программы написаны на языке \foreignlanguage{english}{JavaScript (NodeJS)}. В работе с файлами и настройках сервера также используются
пакеты и утилиты на других языках (напр. \foreignlanguage{english}{bash} для перезагрузки сервера). В разработке веб"=интерфейса применяется стандартный
набор инструментов: \foreignlanguage{english}{HTML+CSS+JS}.
Имеется и консольный интерфейс: программу можно скачать на компьютер и запускать обработку файлов через командную строку.

Программа бесплатна и доступна для использования по адресу: \href{http://391701-cn25543.tmweb.ru/webui/index.html}{http://391701-cn25543.tmweb.ru/webui/index.html},
а также для скачивания:\\\href{https://github.com/Aisse-258/koTspect/archive/refs/heads/master.zip}{https://github.com/Aisse-258/koTspect/archive/refs/heads/m\\aster.zip}.

Исходный код опубликован под открытой лицензией {GPL-3.0} [7] и доступен по адресу [8].


%(см. рис.~\ref{Ivanoff:fig:google-scholar})

%\begin{figure}
%	\centering
%	\includegraphics[width=\linewidth]{pic/google-scholar.png}
%	\caption{Интерфейс сервиса Google Scholar по состоянию на 6 октября 2021~г.}
%	\label{Ivanoff:fig:google-scholar}
%\end{figure}

%\paragraph{Оформление списка литературы.}

\litlist

1. {\it Момот Е. А.} О некоторых алгоритмах сжатия с потерями фотографий печатного и рукописного текста (проект <<{koTspect}>>) //Вестник факультета прикладной математики, информатики и механики. -- 2021. -- В. 15. -- С. 124--132.

2. {\it Ивченко Г. И., Медведев Ю. И.} Введение в математическую статистику. "--- М. : Издательство ЛКИ, 2010.

3. {\it Самбулов Д.} Сравнительный анализ форматов файлов электронных книг //\foreignlanguage{english}{International Journal of Open Information Technologies. -- 2013. -- V. 1. -- No. 3.}

4. {\it Al"=Ani M. S., Awad F. H.} The JPEG image compression algorithm //International Journal of Advances in Engineering \verb|&| Technology. -- 2013. -- Т. 6. -- №. 3. -- С. 1055.

5. {\it Raid A. M. et al.} Jpeg Image Compression Using Discrete Cosine Transform -- A Survey //International Journal of\\Computer Science and Engineering Survey. -- 2014. -- Т. 5. -- №. 2. -- С. 39.

6. {\it Момот Е. А.} О практической применимости некоторых алгоритмов сжатия с потерями фотографий печатного и рукописного текста (проект <<{kotspect}>>) //{XX} Всероссийская молодёжная школа"=конференция <<Лобачевские чтения "--- 2021>>. Сборник трудов. -- 2021. -- С. 67--70.

7. GNU general public license. Version 3, 29 June 2007 (Стандартная общественная лицензия {GNU}. Версия 3, от 29 июня 2007 г.)

8. Aisse-258/koTspect, URL: \href{https://github.com/Aisse-258/koTspect}{https://github.com/Aisse-258/\\koTspect} (дата обращения: 10.01.2021)

