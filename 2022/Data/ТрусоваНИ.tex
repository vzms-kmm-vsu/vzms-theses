\vzmstitle[
%	\footnote{Работа выполнена за счёт гранта РНФ, проект 16-11-10125}
]{
	Уравнения Фредгольма в пространстве функций со смешанными нормами
}
\vzmsauthor{Трусова}{Н.\,И.}
\vzmsinfo{Липецк, ЛГПУ имени П.П. Семенова-Тян-Шанского; {\it trusova.nat@gmail.com}}
%\vzmsauthor{Семенов}{Е.\,М.}
%\vzmsinfo{Воронеж, ВГУ; {\it nadezhka\_ssm@geophys.vsu.ru}}

\vzmscaption

Пусть $D=\{x:\,0<x_i<b_i\}$ --- конечный параллелепипед в $\mathbb{R}_n$, $\alpha$, ${\overline\alpha}$ --- мультииндексы, дополняющие друг друга до полного мультииндекса $(1,2,\ldots,n)$ и $D=D_{x_\alpha}\times D_{x_{\overline\alpha}}$. Через $m$ обозначим размерность
параллелепипеда $D_{x_\alpha}$, $0\leq m\leq n$. Выражение
\begin{equation}\label{1}
(K^{(m)}_{\alpha} u)(x)=\int_{D^{(m)}_{t_\alpha}} \,k_{\alpha}(x;t_\alpha) \,\,u(t_\alpha,
\,x_{\overline{\alpha}})\,\,dt_\alpha\,,
\end{equation}
$
x=(x_\alpha,x_{\overline\alpha})\,,\quad
\alpha=(\alpha_1,\ldots,\alpha_m)$
будем называть частно-интегральным оператором (сокращение ЧИ-оператор), отвечающем ядру $k_{\alpha}=k_{\alpha}(x;\, t_\alpha)$.

При $\alpha=0$ $(\Rightarrow m=0)$ в (\ref{1}) оператор $K_0$ --- оператор умножения на функцию, а при $\alpha=(1,2,\ldots,n)$ ($m=n$) --- оператор $K_{1,\cdots,n}^{(n)}$ --- интегральный оператор.
Оператор (\ref{1}) будем называть полным, если в него включены <<крайние>> операторы $K_0$ и $K_{1,\cdots,n}^{(n)}$.

Частно-интегральным уравнением Фредгольма второго рода с полным ЧИ-оператором (\ref{1}) называется уравнение
 \begin{equation}\label{2}
\varphi(x)-\lambda K_{\alpha}\varphi(x)=f(x), \quad x\in\mathbb{R}_n\,.
\end{equation}
Уравнение (\ref{2}) в $\mathbb{R}_2$ с непрерывными ядрами и в пространстве непрерывных функций  изучались в [1, 2], в пространствах функций со смешанными $\sup$-$L$ нормами и в анизотропном пространстве Лебега $L_{\bf p}$, ${\bf p}=(p_1,p_2)$ в [3].

Решение (\ref{2}) ищем методом последовательных приближений в пространстве непрерывных функций  $C(D_{x_{\overline\alpha}})$ со значением в пространстве Лебега
$ L_p(D_{x_{\alpha}})$\,, которое обозначим через $CL_p$\,.

Мы будем использовать анизотропное пространство Лебега $L_{\bf p}(D)$, ${\bf p}=(p_1,p_2,p_3),~p_i\geq 1$ следующего вида (см. [4]). Пусть $x=(x_{\alpha_1},\,x_{\alpha_2},\,x_{\alpha_3})$, тогда
$$
\|u\|_{L_{\bf p}(D{x_{\alpha_1},D_{x_{\alpha_2}},
D_{x_{\alpha_3}})}}=
$$
$$
=\Biggl(\,\, \int\limits_{D_{\alpha_3}} \Biggl(\,\, \int\limits_{D_{\alpha_2}}\biggl[\,\,
\int\limits_{D_{\alpha_1}} |\,u(x_{\alpha_1})|^{p_1} \,dx_{\alpha_1} \,\biggr]^\frac{p_2}{p_1} dx_{\alpha_2}\Biggr)^{\frac{p_3}{p_{2}}} dx_{\alpha_3}\Biggr)^{\frac{1}{p_3}} \,.
$$

\paragraph{Критерий ограниченности ЧИ-оператора в $CL_p$.}
В (\ref{1}) все переменные разбиваем на три группы: первая $t_\alpha$ --- переменные интегрирования ЧИ-оператора, вторая  $x_\beta$ --- переменные, порожденные $L_p$-нормами (где из них $x_{\beta^*}$ --- переменные, номера которых не совпадают с номерами мультииндекса $\alpha$) и третья группа $x_{\tau}$ --- переменные, по которым применяется sup-норма (где из них $x_{\tau^*}$ --- переменные, номера которых не совпадают с номерами мультииндекса $\alpha$).

\paragraph{Теорема~1.}
 {\it Пусть $p \geq1$ и $1/p+1/p'=1$\,. \\
 Для ограниченности полного оператора $K_\alpha^{(m)}$  в пространстве
$C(D_{x_\tau};\,L_{{p}}( D_{x_{\beta^*}} {\times}  D_{x_{\beta}}))=CL_{p}\,$
достаточно, чтобы
$$
u(t_\alpha,x_{\tau^*},x_{\beta^*})\in
C(D_{x_{\tau^{*}}};\,L_{(p,\, p^2)}(D_{t_\alpha}{\times} D_{x_{\beta^*}} ))\,,
$$
$$
k_\alpha=k_\alpha(x;t_\alpha)\in C(D_{x_\tau};\,L_{(p',\,p\, p',\,p)}(D_{t_\alpha}{\times} D_{x_{\beta^*}}{\times} D_{x_{\beta}} ))\,.
$$
При этом справедливо неравенство
$$
\|K_{\alpha}^{(m)\,} u\|_{CL_{p}}
{\leq}\,\, C_{\alpha} \,\,\|u\|_{C(D_{x_{\tau^*}};\,L_{(p,\,p^2)}(D_{t_\alpha}\times D_{x_{\beta^*}}))},
$$
где
$$
  C_{\alpha}= \|k_{\alpha}\|_{C(D_{x_\tau};\,\, L_{(p',\,p\, p',\, p)}(D_{t_\alpha}\times D_{x_{\beta^*}}\times D_{x_\beta}))}\,.
$$}
 Пусть
$$
A_\alpha=\sup\limits_{1\leq i\leq {\ell}}\left\{[\,\mu(D_{x_{\beta^*}})\,]^{\frac{1}{p \, (p')^\ell}}, [\,\mu(D_{x_{\beta^*}})\,]^{\frac{1}{p^2\,(p')^i}},\, [\,\mu(D_{x_{\beta^*}})\,]^{\frac{1}{p^i \, p'}},\right\}.
$$
$$
B_\alpha= \max \left\{\|k_\alpha\|_{C(D_{x_\tau};\,\, L_{(p',\, \infty,\, p)}(D_{t_\alpha,\, x_{\beta^*},\,x_{\beta}}))}\,,\right.
$$
$$
\,\left.\|k_\alpha\|_{ C(D_{x_{\tau^*}};\,\, L_{(p,\, p'\,,\, \infty)}(D_{x_\alpha,\, t_{\alpha},\,x_{\beta^*}}))}\, \right\}\,.
$$
\paragraph{Теорема~2.}
{\it Пусть
ядро $k_\alpha$ оператора (\ref{1}) и правая часть уравнения (\ref{2}) удовлетворяют условиям
$$F_\alpha=\sup \biggl\{\|k_{\alpha}\|_{C(D_{x_\tau};\,\, L_{(p',\,p^\ell\, p',\, p)}(D_{t_\alpha,\, x_{\beta^*}\,, x_\beta}))}\biggl\}_{\ell=1}^\infty<\infty,$$
 $$\|f\|_{\Lambda_{CL_p}^{\ell}}=\sup \biggl\{\|f\|_{C(D_{x_{\tau^*}};\,L_{(p,\,p^{\ell+1})}(D_{t_\alpha,\,x_{\beta^*}}))}\biggl\}_{\ell=1}^\infty<\infty,$$
и пусть    $|\lambda|\,A_\alpha\,B_\alpha<1$. Тогда в  $CL_p$ существует предел  $\Phi=\lim_{\nu\to\infty}\Phi_\nu$
функциональной последовательности
$
\Phi_\nu f= \varphi^{(\nu)}(x)=\sum\limits_{\ell=0}^\nu \lambda^\ell K_\alpha^\ell f(x)\,.
$
Оператор $\Phi$ действует ограниченно из $C(D_{x_{\tau^*}};\,L_{(p,\,\infty)}(D_{t_\alpha,\,x_{\beta^*}}))$ в
 $CL_p$  и удовлетворяет неравенству
$$
\|\Phi\|_{CL_p}\leq{\lim_{\overline{\nu\to\infty}}}\|\Phi_\nu\|_{CL_p}.
$$
 Решение уравнения Фредгольма  с ЧИ-оператором
\eqref{1} единственно и существует в виде операторного ряда Неймана
$$
\varphi(x)=\sum\limits_{\ell=0}^\infty \lambda^\ell K_\alpha^\ell f\,,
\text{причем}\,\,\,\,\|\varphi\|_{CL_{p}}\leq\frac{\|f\|_{\Lambda_{CL_p}^\ell}}{1-|\lambda| A_\alpha\,B_\alpha}\,.
$$}
Схема доказательства аналогична схеме приведенной в [3] и мы её не приводим.

%\litlist
\begin{center}
Литература
\end{center}

1. {\it Appell J. М.,\, Kalitvin A. S.,\, Zabrejko P. P.}\, Partial Integral\, Operators\, and\, Integro-Differential\, Equations.\, New York:\, Marcel Dekker, 2000. - 560 p.

2. {\it Калитвин А. С., Фролова Е. В.}
Линейные уравнения с частными интегралами. С-теория. Липецк: ЛГПУ, 2004. - 195 c.

3. {\it Lyakhov L. N., Inozemcev A. I., Trusova N. I.} About Fredgholm\, equations\, for\, partial\, integral in $\mathbb{R}_2$.\, Jornal\, Of\, Mathematical Sciences. - 2020. - Vol. 251. - № 6. - P. 839-849.

4. {\it Бессов О. В., Ильин В. П., Никольский С. М.} Интегральные представления функций и теоремы вложения. - М: Наука, 1975. - 478 c.
