\vzmstitle[
]{
	Маломерные операторы Нийенхейса и их особенности 
}
\vzmsauthor{Акпан}{Д.\,Ж.}
\vzmsinfo{Москва, МГУ им. М. В. Ломоносова; {\it dinmukhammed.akpan@math.msu.ru}}

\vzmscaption

Операторы Нийенхейса или геометрия Нийенхейса --- это новое, активно развивающееся направление современной дифференциальной геометрии, связанное с интегрируемыми системами и другими областями математики. 
Первая крупная статья вышла в 2019 году [1], где определяются и доказываются фундаментальные свойства операторов Нийенхейса. 

Мотивировка развития данной теории связана с базовыми вопросами дифференциальной геометрии об условиях интегрируемости тензорных полей на многообразиях, т.е. возможности привести данное
тензорное поле к постоянному (или какому-то другому простому) виду. Например, для римановой метрики таким условием является равенство нулю ее тензора кривизны (тензора Римана) $R^i_{jkl}$,
а для невырожденной 2-формы --- ее замкнутость. Для операторных полей на многообразии необходимым условием их интегрируемости (т.е. постоянства в некоторой системе координат) является 
равенство нулю некоторого тензора $N^i_{jk}$ типа $(1,2)$, называемого тензором Нийенхейса, который в инвариантом виде задается следующим образом:
\begin{equation*}
N_L [\xi, \eta] = L^2[\xi, \eta] + [L \xi, L \eta] - L[L \xi, \eta] - L[\xi, L \eta],
\end{equation*}
где $L$ --- рассматриваемое операторное поле, а $\xi,\eta$ --- произвольные векторные поля. Операторное поле $L$ называется нийенхейсовым (или \textit{тензором Нийенхейса}), если 
соответствующий ему тензор $N_L$ тождественно равен нулю. 

Различные свойства операторов Нийенхейса описаны в работах [1], [2], [3]. 

В работе [4] приводится список открытых вопросов и проблем по этой тематике. Результаты, обсуждаемые в данном докладе, связаны с исследованием одного из таких вопросов (номер~5.13) 
о классификации операторов Нийенхейса и их особенностей в двумерном случае. 

Пусть у нас есть двумерный оператор Нийенхейса $L$, след которого (в окрестности рассматриваемой точки) можно взять в качестве одной из координат: $tr L = x$. 
Тогда, рассматривая его определитель, как некоторую функцию от координат $f(x,y)=\det L$, можно попытаться классифицировать такие операторы Нийенхейса. 
Задача разбивается на несколько подзадач: 
%\begin{itemize}
%\item $\frac{\partial f}{\partial y} \neq 0$
%\item $\frac{\partial f}{\partial y} \equiv 0$
%\item $\frac{\partial f}{\partial y} \mid_{y = 0} = 0$
%\end{itemize}
%
$$
1)\,\frac{\partial f}{\partial y} \neq 0    \,;\qquad
2)\,\frac{\partial f}{\partial y} \equiv 0  \,;\qquad
3)\,\frac{\partial f}{\partial y} \mid_{y = 0} = 0\,.
$$

В первых двух случаях задача решается полностью и приводится ответ. 
Последний случай является более сложным, и здесь мы приводим решение данной задачи в некоторых классах функций. 
Также нами были получены семейства примеров операторов Нийенхейса в рассматриваемой ситуации. 

\paragraph{Теорема~1.}
{\it
Пусть $\det L = f(x,y)$ --- функция, производная которой по $y$ нигде не равна нулю. Тогда соответствующий оператор Нийенхейса имеет вид 
\begin{equation*}
\begin{pmatrix}
f_x - x & f_y \\
\frac{f_x(x - f_x) - f}{f_y} & -f_x
\end{pmatrix}
\end{equation*}
}

\paragraph{Теорема~2.}
{\it
Пусть $L$ --- двумерный оператор Нийенхейса с инвариантами $tr L = x$ и $\det L = f(x)$. Тогда функция $f(x)$ и оператор $L$ имеют следующий вид:
$$
f(x) = \alpha x - \alpha^2
\qquad
\text{или}
\qquad
f(x) = \frac{x^2}{4},
\qquad
L = 
\begin{pmatrix}
x - f' & 0 \\
c(x,y) & f' \\
\end{pmatrix}
$$
где $c(x,y)$ --- произвольная функция, а $\alpha$ -- произвольная константа. 
}

В третьем случае удобно сделать замену $g(x,y) = \frac{x^2}{4} - f(x,y)$, после которой дробь $\frac{f_x(x - f_x) - f}{f_y}$ приводится 
к виду $\frac{g^2_x - g}{g_y}$. Поскольку $\frac{\partial g}{\partial y} = - \frac{\partial f}{\partial y}$, исходную задачу можно переформулировать так:
описать множество функций $\lbrace g(x,y) \in C^{\infty} : g_y(0) = 0, \frac{g^2_x - g}{g_y} \in C^{\infty} \rbrace$. 

В докладе будет рассказано об этих и некоторых других примерах особенностей двумерных операторов Нийенхейса.

\litlist 
1. {\it A. Bolsinov, V. Matveev, A. Konyaev.} Nijenhuis Geometry. arXiv:1903.04603v2 [math.DG] 11 Dec 2019.

2. {\it A. Bolsinov, V. Matveev, A. Konyaev.} Applications of Nijenhuis Geometry II: maximal pencils of multihamiltonian structures of hydrodynamic type. arXiv:2009.07802v1 [math.DG] 16 Sep 2020.

3. {\it A. Bolsinov, V. Matveev, A. Konyaev.} Nijenhuis Geometry III: gl-regular Nijenhuis operators. arXiv:2007.09506v1 [math.DG] 18 Jul 2020

4. {\it A. Bolsinov, V. Matveev, A. Konyaev, E. Miranda.} Open problems, questions and challenges in finite-dimensional integrable systems. Phil. Trans. R. Soc. A 376: 20170430.
