\vzmstitle[
	%\footnote{}
]{
	Периодические решения дифференциально-операторных уравнений высших порядков
}
\vzmsauthor{Галиакберов}{Р.\,Р.}
\vzmsinfo{Иркутск, ИГУ; {\it ruslan10651999@gmail.com}}
\vzmsauthor{Орлов}{С.\,С.}
\vzmsinfo{Иркутск, ИГУ; {\it orlov{\_}sergey@inbox.ru}}
\vzmsauthor{Соколова}{Г.\,К.}
\vzmsinfo{Иркутск, ИГУ; {\it 98gal@mail.ru}}

\vzmscaption

Пусть $E$~--- банахово пространство, а ${\mathscr B}(E)$~--- банахова алгебра  ограниченных линейных операторов, действующих из $E$ в $E$. Рассмотрим дифференциальное уравнение
$$
u^{(N)}(t)-{\cal A}u(t)=f(t)\eqno(1)
$$
порядка $N\in{\mathbb N}$ с операторным коэффициентом ${\cal A}\in{\mathscr B}(E)$, где $u$~--- искомая, а $f$~--- заданная функции аргумента $t\in{\mathbb R}$ и со значениями в $E$. Изучение существования периодических решений  дифференциально-операторных уравнений обычно сводится к рассмотрению {\it периодической краевой задачи} с граничными условиями
$$
u^{(i-1)}(0)=u^{(i-1)}(T),\,\,i\in\lbrace 1,\ldots,N\rbrace,
$$
где $T$~--- предполагаемый период решения (см., например, статью~[1]). Исследование разрешимости указанной задачи естественным образом приводит к ограничениям на спектр $\sigma({\cal A})$ оператора ${\cal A}$. Единственность периодического решения изучают отдельно, а его основной период может оказаться меньше заданного $T$.

В докладе предлагается другой подход к исследованию существования периодических решений уравнения~(1), при котором рассматривается не краевая, а начальная задача с начальными условиями
$$
u^{(i-1)}(0)=u_{i-1},\,\,i\in\lbrace 1,\ldots,N\rbrace,\eqno(2)
$$
где элементы $u_{i-1}\in E$ заданы, в условиях ее однозначной разрешимости. Требуется описать множество операторных коэффициентов ${\cal A}\in{\mathscr B}(E)$, свободных функций $f:\,{\mathbb R}\to E$ и начальных значений $u_{i-1}\in E$, при котором {\it классическое} решение задачи Коши~(1), (2) окажется $T$-периодическим. Классическим решением начальной задачи~(1), (2) назовем функцию $u(t)\in C^{N}([0;+\infty);\,E)$, обращающую в тождество уравнение~(1) и удовлетворяющую начальным условиям~(2). Предлагаемый подход позволяет ослабить указанные выше ограничения на спектр ${\cal A}$.

Известно, что задача Коши~(1), (2) имеет единственное классическое решение
$$
u(t)=\sum\limits_{k=1}^{N}U_{N}^{(k-1)}(t)u_{N-k}+\int\limits_{0}^{t}U_{N}(t-s)f(s)ds,
$$
при любых $f(t)\in C([0;+\infty);\,E)$ и $u_{i-1}\in E,\,\,i\in\lbrace 1,\ldots,N\rbrace,$ (см., например, монографию [2]), где оператор-функция $U_{N}$ задается в виде операторно-функционального ряда
$$
U_{N}(t)=\sum\limits_{k=1}^{+\infty}\frac{t^{kN-1}}{(kN-1)!}\,{\cal A}^{k-1}.
$$
равномерно сходящегося в норме банаховой алгебры ${\mathscr B}(E)$ на любом компакте $[0,T]$. Также справедливо выражение
$$
U_{N}(t)=t^{N-1}{\mathrm E}_{N,N}(t^{N}{\cal A}),
$$
через двухпараметрическую {\it функцию Миттаг-Леффлера}
$$
{\mathrm E}_{\alpha,\beta}(z)=\sum\limits_{k=0}^{+\infty}\frac{z^{k}}{\Gamma(\alpha k+\beta)},\,\,\alpha,\,\beta>0,
$$
где $\Gamma$~--- гамма-функция Эйлера. Оператор-функция $U_{N}$ есть равномерно непрерывная полугруппа операторов при $N=1$ и синус оператор-функция при $N=2$, которые порождены инфинитиземальным генератором ${\cal A}$~(см. обзор~[3]). Также заметим, что при всех $t,s\in{\mathbb R}$ имеет место  соотношение
$$
U_{N}(t+s)=\sum\limits_{k=1}^{N}U_{N}^{(k-1)}(t)U_{N}^{(N-k)}(s).
$$

Непосредственно можно убедиться, что периодичность с периодом $T$ правой части  уравнения~(1) является условием, необходимым для $T$-периодичности его решения. Так как за необходимостью функция $f$ сильно непрерывна, то она либо постоянная, либо $T$-периодическая. Для последнего случая справедлив следующий критерий.

\paragraph{Теорема.} {\it Пусть функция $f(t)\in C([0;+\infty);E)$ является $T$-периодической, и линейный оператор ${\cal A}$ ограничен. Тогда для того чтобы классическое решение задачи Коши~(1), (2) было $T$-периодическим,  необходимо и достаточно, чтобы для всех $k\in\lbrace 1,\ldots,N\rbrace$ выполнялись соотношения
$$
(U_{N}^{(N-k)}(T)-{\mathbb I})u_{N-k}=-\int\limits_{0}^{T}U_{N}^{(N-k)}(T-s)f(s)ds.
$$}

% Оформление списка литературы
\litlist

1. {\it Eidelman~Y.~S., Tikhonov~I.~V.} On periodic solutions of abstract differential equations~// Abstract and Appl. Anal.~– 2001.~– Vol.~6., No.~8.~– P.~489–499.

2. {\it Sidorov~N. et al} Lyapunov–Schmidt Methods in Nonlinear Analysis and Applications.~– Dordrecht: Kluwer Acad. Publ., 2002.~– 568~p.

3. {\it Васильев~В.~В., Крейн~С.~Г., Пискарев~C.~И.} Полугруппы операторов, косинус оператор-функции и линейные дифференциальные уравнения~// Итоги науки и техники. Серия математический анализ.~– 1990.~– Т.~28.~– С.~87–202.
