
\begin{center}
    {\bf ИНТЕГРИРУЕМЫЙ БИЛЛИАРД С ПРОСКАЛЬЗЫВАНИЕМ\footnote{Исследование выполнено в рамках Программы Президента Российской Федерации для государственной поддержки ведущих научных школ РФ(грант НШ-2554.2020.1).}}\\

    {\it В.Н. Завьялов}

    (Москва; {\it vnzavyalov@mail.ru})
\end{center}

\addcontentsline{toc}{section}{Завьялов В.Н.}

Пусть дана кусочно-гладкая кривая, ограниченная дугами софокусных квадрик  (углы равны $\frac{\pi}{2}$). Определим систему биллиарда в области, ограниченной этой кривой. Шар (материальная точка) движется в области по прямолинейным отрезкам, с сохранением скорости и отражается от кривой по закону "угол падения равен углу отражения". При попадании в угол точка после отражения продолжает движение по той же прямой, по которой попала в этот угол.

В теории интегрируемых систем представляют интерес интегрируемые биллиарды, для которых вдоль траектории сохраняется дополнительная функция, функционально независимая от квадрата длины вектора скорости. Например, биллиард в окружности – любая траектория, касается меньшей окружности с тем же центром, т.е. радиус этой меньшей окружности есть интеграл. Биллиард в эллипсе также интегрируем – траектории (или их продолжения) касаются некоторого эллипса или некоторой гиперболы, софокусной с данным эллипсом [1].
Семейство софокусных квадрик можно задать уравнением

$$(b-\lambda)x^2+(a-\lambda)y^2=(a-\lambda)(b-\lambda),$$
где $a>b>0.$  Пусть эллипс, который ограничивает биллиард соответствует параметру $\lambda=0.$ Вдоль траектории биллиарда сохраняется параметр $\lambda$ квадрики, которой касаются траектории или её продолжения.

Определим новую интегрируемую систему, называемую биллиардом с проскальзыванием. Пусть даны два эллипса. При попадании на границу точка продолжает движение по второму эллипсу выходя из соответствующей ей симметричной точке на другом эллипсе под тем же углом. Аналогично можно определить систему при которой точка после отражения продолжает движение по тому же эллипсу, но выходит из диаметрально противоположной точки, "проскальзывая". Заметим, что обе системы по прежнему останутся интегрируемыми с тем же интегралом, что и классический биллиард в эллипсе. Это позволяет изучить топологию соответствующей изоэнергетической поверхности, используя теорию инвариантов интегрируемых систем Фоменко-Цишанга [2].



\smallskip \centerline {\bf Литература} \nopagebreak

1. {\it Козлов В.В., Трещев Д.В.} Генетическое введение в динамику систем с ударами. М.: Изд-во МГУ,  1991.

2. {\it Болсинов А.В., Фоменко А.Т}. Интегрируемые гамильтоновы системы. Геометрия, топология, классификация. Т.1,2, Ижевск: НИЦ “Регулярная и хаотическая динамика”,  1999.


