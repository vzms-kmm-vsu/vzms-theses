\vzmstitle[\footnote{Исследование выполнено в рамках Программы Президента Российской Федерации для государственной поддержки ведущих научных школ РФ (грант НШ-2554.2020.1).}]{ОБОБЩЕННЫЕ ГРУППЫ КОС И ИНВАРИАНТЫ ДИНАМИЧЕСКИХ СИСТЕМ}
\vzmsauthor{Федосеев}{Д.\,А.}
\vzmsinfo{Москва; {\it denfedex@yandex.ru}}
\vzmscaption

Классическая группа кос --- один из давно изучаемых и хорошо известных объектов маломерной топологии, наряду с классическими узлами. Существуют различные обобщения групп кос (например, виртуальные косы, которые точнее было бы называть ``косами с виртуальными перекрёстками'', свободные косы). Изучаются так же и некоторые важные частные случаи групп кос (самым ярким из них является группа {\em крашеных кос}, то есть кос, которым отвечает тождественная подстановка).

Особый интерес косы представляют, поскольку тесно связаны с динамическими системами движения набора из $n$ точек по двумерной плоскости. В частности, сама группа двумерных кос рассматриваться как фундаментальная группа пространства конфигураций из $n$ различных точек. При этом образующие группы --- перекрёстки нитей --- отвечают моментам времени, в которые у пары точек совпадают координаты по оси $OY$.

Естественным образом возникла идея обобщить изучаемые системы в следующем смысле: рассматривать движение объектов по некоторому конфигурационному пространству, причём потребовать, чтобы все ``вырождения'' системы (то есть моменты времени, в которые изучаемые объекты были не в общем положении) удовлетворяли бы некоторому ``хорошему'' свойству коразмерности 1, зависящему от $k<n$ точек. В этих терминах классические косы отвечали бы свойству коразмерности 1, зависящему от двух точек: ``$y-$координаты двух точек совпали''.

Данная идея оказалась плодотворной и привела к построению В.О. Мантуровым теории {\em обобщённых свободных $k-$кос} или, иначе, {\em теории групп $G_n^k$} [1].

Более точно, группа свободных $k-$кос определяется следующим образом. Рассмотрим множество $\bar{n}=\{1,\dots, n\}$. Образующие группы $G_n^k$ находятся во взаимно"=однозначном соответствии с неупорядоченными $k-$элементными подмножествами множества $\bar{n}$. Обозначим их через $a_m$, где $m$ --- неупорядоченный $k-$мультииндекс, то есть $m\subset \bar{n}, \, |m|=k$.

Группа $G_n^k$ получается из свободной группы, порождённой образующими $a_m$, факторизацией по следующим соотношениям:

\begin{enumerate}
	\item $a_{m}a_{m'}=a_{m'}a_{m}$ для любых мультииндексов $m, m'$, для которых $Card(m\cap m') < k-1$ (соотношение {\em дальней коммутативности});
	\item $a_m^2=1$ для всех мультииндексов $m$;
	\item для каждого $(k+1)$-элементного набора $U$ индексов $u_{1},\dots, u_{k+1} \in \{1,\dots, n\}$ рассмотрим $k+1$ множество $m^{j}=U\setminus \{u_{j}\}, j=1,\dots, k+1$. С каждым набором $U$ сопоставим {\em тетраэдральное} соотношение $$a_{m^1}\cdot a_{m^2}\cdots a_{m^{k+1}}= a_{m^{k+1}}\cdots a_{m^2}\cdot a_{m^1}.$$
\end{enumerate}

Само построение группы $G_n^k$ приводит к следующему принципу:

\begin{center}
	{\em если динамическая система движения $n$ точек удовлетворяет некоторому хорошему свойству коразмерности 1, зависящему от $k$ точек, данная система обладает инвариантами со значениями в группе $G_n^k$.}
\end{center}

Разумеется, мало построить инварианты некоторой динамики. Важно, чтобы эти инварианты можно было посчитать и различить между собой. Эти вопросы в случае групп $G_n^k$ в настоящее время активно изучаются. В частности, следующие результаты были получены автором совместно с В.О. Мантуровым и А.Б Карповым [2, 3]: \\

\textbf{Теорема~1.} {\it Для группы $G_4^3$ алгоритмически разрешимы проблемы равенства и сопряжённости.}

\textbf{Теорема~2.} {\it Для группы $G_5^4$ алгоритмически разрешима проблема равенства.}

Аналогичные теоремы для случая групп $G_{k+1}^k$ для $k>4$ в настоящее время не доказаны, хотя у авторов есть основания полагать, что они справедливы. Продвижения в этом вопросе были бы крайне важны для теории инвариантов динамических систем описанного типа.


\smallskip \centerline {\bf Литература} \nopagebreak


1. {\it V.\,O. Manturov}, Non--Reidemeister knot theory and its applications in dynamical systems, geometry and topology, preprint: http:// arxiv.org/abs/1501.05208

2. {\it D.A.~Fedoseev, A.B.~Karpov, V.O.~Manturov}, Word and Conjugacy Problems in Groups $G_{k+1}^{k}$; preprint: https://arxiv.org /abs/1906.04916; accepted for publishing by Lobachevskii Journal of Mathematics.

3. {\it Vassily O. Manturov, Denis A. Fedoseev, Seongjeong Kim, Igor M. Nikonov}, On Groups $G^k_n$ and $\Gamma^k_n$: A Study of Manifolds, Dynamics, and Invariants, submitted to Bulletin of Mathematical Sciences, preprint: https://arxiv.org/abs/1905.08049
