\vzmstitle[\footnote{Работа выполнена при финансовой поддержке гранта Российского научного фонда (проект №19-11-00146).}]{О РАЗРЕШИМОСТИ ОДНОЙ ТЕРМО-ЗАДАЧИ ГИДРОДИНАМИКИ}
\vzmsauthor{Звягин}{А.\,В.}
\vzmsinfo{Воронеж; {\it zvyagin.a@mail.ru}}
\vzmscaption

На $Q_T=\Omega\times[0,T]$, где $\Omega\subset{R}^n, n=2,3,$ область с достаточно гладкой границей $\partial\Omega$ и $T>0$, рассматривается следующая начально"=краевая задача:
$$
\frac{\partial v}{\partial t}+\sum_{i=1}^nu_i\frac{\partial v}{\partial x_i} +\sum_{i=1}^nv_i\frac{ \partial u}{\partial x_i} -2 \mbox{Div }(\mu(\theta)\mathcal{E}(v)) + \nabla p =f; \eqno{(1)}$$
$$ v=(I-\alpha^2\Delta)u;\qquad
	\mathrm{div}\,u=0;\eqno{(2)}$$
	$$ u|_{t=0} =u_0; \quad u|_{[0,T]\times\partial \Omega }=0;\quad \Delta u|_{[0,T]\times\partial \Omega }=0; \eqno{(3)}$$
$$ \frac{\partial \theta}{\partial t} + (u \cdot \nabla ) \theta - \chi\Delta \,\theta=2\mu(\theta)\mathcal {E}(u):\mathcal {E}(v)+g;\eqno{(4)}$$
$$
	\theta|_{t=0} =\theta_0; \quad\theta|_{[0,T]\times\partial \Omega }=0.\eqno{(5)}
$$
Здесь, $v$ "--- вектор"=функция скорости движения частицы среды, $u$ "---
вектор"=функция модифицированной скорости движения частицы среды, $\theta(t,x)$ "--- функция температуры среды, $p$ "--- функция давления, $f$ "--- функция плотности внешних сил, $g$ "--- источник внешнего тепла, $\alpha>0$ "--- скалярный параметр, $\mu(\cdot)>0$ "--- коэффициент вязкости среды, $\chi >0$ "--- коэффициент теплопроводности, $u_0$ и $\theta_0$ "--- начальные скорость и температура. Через
$
\mathcal{E}=(\mathcal{E}_{ij}(v)),$ $ \mathcal{E}_{ij}(v)=\frac{1}{2}\Big(\frac{\partial v_i}{\partial x_j}+
\frac{\partial v_j}{\partial x_i}\Big),$ $i, j=\overline{1, n},
$
обозначается тензор скоростей деформации.

Изучаемая начально"=краевая задача с постоянной вязкостью называется альфа"=моделью системы Навье\--Стокса (см. [1]). Введём пространства, в которых будет доказана разрешимость изучаемой задачи:
$$
W_1=\{u\in L_2(0, T; V^2)\cap L_\infty(0, T; V^1), u'\in L_2(0, T; V^{-2})\};
$$
$$W_{2}=\{\theta: \theta\in L_p(0,T; W^1_p(\Omega)), \theta' \in L_1(0,T; W^{-1}_{p}(\Omega)).$$

\textbf{Определение~1.} {\it
	Слабым решением задачи (1)--(5) называется пара $(u, \theta)$, где $u \in W_1$ и $\theta \in W_2,$ удовлетворяющая при всех $\varphi \in V^2$, $ \phi \in C^{\infty}_0(\Omega)$ и почти всех $t\in [0,T]$ соотношениям
$$
	\langle (J+\alpha^2A)u', \varphi\rangle - \int_\Omega\sum_{i, j=1}^nu_i((J+\alpha^2A)u)_j
	\frac{\partial\varphi_j}{\partial x_i}\,dx +$$
	$$+ \int_\Omega\sum_{i, j=1}^n((J+\alpha^2A)u)_i\frac{\partial u_i}{\partial x_j}
	\varphi_j\,dx
	- 2\int_\Omega \mu(\theta) (J+\alpha^2A)u\Delta\varphi\,dx=$$
	$$
	\langle f, \varphi\rangle,$$ $$
	\int\limits_\Omega \frac{\partial\theta}{\partial t}\phi dx-\int\limits_\Omega\sum\limits_{i,j=1}^n
	u_i\theta_j\frac{\partial \phi_j}{\partial x_i} dx+\chi\int\limits_\Omega
	\mathcal{E}(\theta):\mathcal{E}( \phi) dx=$$ $$= 2\int\limits_\Omega\big{(}{\mu}(\theta) \mathcal {E}(u): \mathcal {E}(u)\big{)}:\phi dx+\langle g,\phi\rangle,\label{f4}
$$
	и начальным условиям $u|_{t=0} = v_0$ и $\theta|_{t=0}=\theta_0.$ Здесь $J=PI$, где $P$ "--- проектор Лере, а $I$ "--- тождественный оператор.}

\textbf{Теорема~1.} {\it Пусть функция $\mu \in C^2(-\infty,+\infty)$ и $0<\mu(\theta)\leq M$ для всех $\theta \in W_2$, $f \in L_p(0,T;V^{-1})$, $g \in L_1(0,T;$ $H^{-2(1-1/p)}_p(\Omega))$, $u_0 \in V^1$, $\theta_0 \in W^{1-2/p}_p(\Omega)$. Тогда при $1<p<4/3$ для $n=2$ и для $1<p<10/9$ при $n=3$ существует слабое решение задачи (1)--(5).}

Доказательство теоремы 1 основано на последовательном применение аппроксимационно"=топологического подхода к задачам гидродинамики и на итерационном процессе, и поэтому проводится поэтапно (см. работы [2]--[6]). На первом этапе устанавливается разрешимость задачи (1)--(3) при фиксированном $\theta \in W_2$. Затем устанавливается разрешимость задачи (4)--(5) с заданной $u \in W_1$. Далее, описывается итерационный процесс, состоящий в последовательном решении вышеприведённых задач, и, наконец, доказывается сходимость последовательных приближений к решению задачи (1)--(5).




% Оформление списка литературы
\litlist

1. {\it Звягин А.В.} Оптимальное управление с обратной связью для альфа"=модели Лере и альфа"=модели Навье-Стокса // Доклады Академии Наук. --- 2019. --- Т. 486, Н. 5. --- С. 527--530.

2. {\it Звягин А.В.} Разрешимость задачи термовязкоупругости для альфа"=модели Лере // Известия ВУЗов. Математика. --- 2016. --- Н. 10. --- С. 70--75.

3. {\it Звягин А.В.} Оптимальное управление с обратной связью для термовязкоупругой модели движения жидкости Фойгта // Доклады Академии Наук. --- 2016. --- Т. 468, Н. 3. --- С. 251--253.

4. {\it Звягин А.В.} Исследование разрешимости термовязкоупругой модели, описывающей движение слабо концентрированных водных растворов полимеров // Сибирский математический журнал. --- 2018. --- Т. 59, Н. 5. --- С. 1066--1085.

5. {\it Звягин А.В.} Слабая разрешимость термовязкоупругой модели Кельвина-Фойгта // Известия ВУЗов. Математика. --- 2018. --- Н. 3. --- С. 91--95.

6. {\it Звягин А. В.} Исследование разрешимости термовязкоупругой модели движения растворов полимеров, удовлетворяющей принципу объективности // Математические Заметки. --- 2019. --- Т. 105, Н. 6. --- С. 839--856.
