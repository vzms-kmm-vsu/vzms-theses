\vzmstitle{О ВЛИЯНИИ ЛИНЕЙНО ВОЗРАСТАЮЩЕГО ВДУВА И ЛИНЕЙНОГО ТЕМПЕРАТУРНОГО ФАКТОРА НА ОБЛАСТЬ ЗНАЧЕНИЙ ФУНКЦИОНАЛОВ ГИПЕРЗВУКОВОЙ АЭРОДИНАМИКИ}
\vzmsauthor{Бильченко (мл.)}{Г.\,Г.}
\vzmsauthor{Бильченко}{Н.\,Г.}
\vzmsinfo{Казань; {\it ggbil2@gmail.com}; {\it bilchnat@gmail.com}}
\vzmscaption
%% =============================== 
%%  Докладчик:
%%
%%  Бильченко
%%  Наталья
%%  Григорьевна
%%
%%  "О  влиянии  линейно  возрастающего  вдува 
%%  и  линейного  температурного  фактора 
%%  на  область  значений  функционалов 
%%  гиперзвуковой  аэродинамики"
%%
%%  к.ф.-м.н.
%%
%%  Казанский национальный исследовательский технический университет им. А.Н.Туполева-КАИ  
%%
%%  Секция (направление) / Section:  Нелинейный анализ и математическое моделирование
%% =============================== 
%%  Содокладчик:
%%
%%  Бильченко
%%  Григорий  (мл.)
%%  Григорьевич
%%
%%  к.ф.-м.н.
%%
%%  Казанский национальный исследовательский технический университет им. А.Н.Туполева-КАИ  
%% =============================== 



%% ==============================
%% ==============================
     В  данной  работе, 
продолжающей  исследования  свойств
полученной
с  помощью  метода  
обобщённых  интегральных  соотношений 
А.~А.~Дородницына 
%%% ===
[1]  
%%% ===
математической  модели  
ламинарного  пограничного  слоя  
(ЛПС)
электропроводящего  газа  
на  проницаемых  цилиндрических  
и  сферических  поверхностях  
гиперзвуковых  летательных  аппаратов  
(ГЛА) 
%%% ===
[2{\textbf{--}}7], 
%%% ===
рассматривается  влияние  
следующего  сочетания
управляющих  воздействий:
\textbf{линейно  возрастающего}
вдува,  
\textbf{линейного}
температурного  фактора  
и  
\textbf{постоянного} 
магнитного  поля  
на  
интегральные 
характеристики  
тепломассообмена  и  трения
и  
суммарную  мощность  
системы,  
обеспечивающей  вдув.  
%%%% ====================================



%% ==============================
%% ==============================
\textbf{1.  Постановка  задачи.}
%% ==============================
%% ==============================
В  прямой  задаче  
%%% ===
[7]  
%%% ============================ \
\[
\left(m, \tau_{w}, s\right)  
\rightarrow  
\left(q, f, \eta; Q, F, N\right)  
\eqno{(1)}  
\]
%%% ============================ /
по  
заданным  управлениям: 
$m(x)$~%
{\textbf{--}}  
\textit{вдуву}  
в  ЛПС, 
где  
%%%% ================
$x 
\in
X 
=
\left[0; 1\right]$,  
%%%% ================
а%
~%
ось  
$x$  
направлена  вдоль
контура  тела;
%%%% ================
$\tau_{w}(x)=
T_{w}(x)/T_{e_{0}}$~%
{\textbf{--}}  
\textit{температурному  фактору},   
%%%% ================
где  
$T_{w}(x)$~%
{\textbf{--}}  
температура  стенки, 
%%%% ================
а  
$T_{e_{0}}$~%
{\textbf{--}}  
температура  
в  точке  
тор\-мо\-же\-ния  
$x_{0}
=
0$  
потока;
%%%% ================ 
$s(x)
=
\sigma  B_{0}^{2}(x)$~%
{\textbf{--}}  
\textit{магнитному  полю}
%%%% ================ 
требуется  рассчитать  параметры 
%%%% ================ \\
$\theta_{0}
  \left(x; m, \tau_{w}, s\right)$, 
$\theta_{1}\left(\ldots\right)$, 
$\omega_{0}\left(\ldots\right)$, 
$\omega_{1}\left(\ldots\right)$
%%%% ================ //
математической  модели  ЛПС 
%%% ===
[2,  
 7]
%%% ===
для  случаев  обтекания  
боковой  поверхности  кругового  цилиндра 
и  
поверхности  сферического  носка.
%%%
Для  
нахождения
параметров  
$\theta_{0\,}$, 
$\ldots$, 
$\omega_{1}$ 
ЛПС 
применяется  
\textbf{объединённая}  
аппроксимирующая  система  
обыкновенных  дифференциальных  уравнений  
%%%%%%%%%%%%%%%%%%%%%%%%%
%%% Page  1 / Page  2 %%%
%%%%%%%%%%%%%%%%%%%%%%%%%
(5){\textbf{--}}(8) 
%%% ===
[7]
%%% ===
с  начальными  условиями, 
полученными
из  \textbf{объединённой}
нелинейной  алгебраической  системы 
%%% ===
(10){\textbf{--}}(13) 
%%% ===
[7].
%%%% ================
После  этого  
определяются 
%%%% ================ 
локальный  тепловой  поток 
%%%% ================ \\
$q\left(x; m, \tau_{w}, s\right)$; 
%%%% ================ //
локальное  напряжение  трения
%%%% ================ \\
$f\left(x; m, \tau_{w}, s\right)$; 
%%%% ================ //
локальная\,  мощность\,  
%%%% ================ \\
$\eta\left(x; m, \tau_{w}, s\right)$
%%%% ================ //
системы, 
обеспечивающей  вдув.
%%%% ================
Затем  
(для  
$x_{k}=1$)  
определяются  
%% 
\textbf{интегральный  тепловой  поток}  
%%% ===
(2)  
%%% ===
[7]
% =============================== \\
\[
Q\left(m, \tau_{w}, s\right)
=
\int\nolimits_{0}^{x_{k}}
(2\pi  r)^{k_{4}}
\left(
    \frac{\lambda}
         {C_{p}}
    \frac{\partial  H}
         {\partial  y}
  \right)_{y = 0} 
\cdot{dx} 
\,{;}
\eqno{(2)}
\]
%% =============================== //
\textbf{суммарная  сила  трения}  Ньютона 
%%% ===
(3)  
%%% ===
[7]
%% =============================== \\
\[
F\left(m, \tau_{w}, s\right)
=
\int\nolimits_{0}^{x_{k}}
(2\pi  r)^{k_{4}}
\left(
      \mu
      \frac{\partial  u}
           {\partial  y}
  \right)_{y = 0} 
\cdot{dx} 
\,{;}
\eqno{(3)}
\]
%% =============================== //
%%% Overfull start ===\\
вычисляемая\,  
с\,  
использованием\,   
фильтрационного\,  
закона\,  
Дарси
%%% Overfull finish ===//
%%
\textbf{суммарная  мощность}  
%%% ===
(4)  
%%% ===
[7]
%% =============================== \\
\[
  N\left(m, \tau_{w}, s\right)
  =
  \int\nolimits_{0}^{x_{k}}
   (2\pi  r)^{k_{4}}
  a  v_{w}^{2}(x) 
  \cdot{dx} 
\eqno{(4)}
\]
%% =============================== //
системы, 
обеспечивающей  вдув. 
%%%%%%
В  
%%% ===
(2){\textbf{--}}(4)
%%% ===
коэффициент
$k_{4} = 0$  
для  
боковой  поверхности  цилиндра,
$k_{4} = 1$
для  
поверхности  сферического  носка  
с  
радиусом  
$r(x)$.
%%%% ====================================



%% ==============================
%% ==============================
\textbf{2.  Вычислительные   эксперименты.} 
%% ==============================
%% ==============================
Пусть  фиксированы  значения  
\textit{неизменяемых  параметров}:  
%% =============================== \\
\begingroup\belowdisplayskip=\belowdisplayshortskip  
\[
\mbox{число  Маха}  
\quad  
M_{\infty} \in [10; 40]  
\,{;}  
\eqno{(5)}  
\]
\endgroup  
%% =============================== *
\[
\mbox{высота  полёта}  
\quad  
H \in [10; 30] \,  
[\mbox{км}]  
\,{;}  
\eqno{(6)}  
\]
%% =============================== *
\[
\mbox{радиус  тела}  
\quad  
R \in [0{,}1; 1] \,  
[\mbox{м}]  
\,{.}  
\eqno{(7)}  
\]
%% =============================== //



    Пусть  диапазоны  изменения  
\textit{управляющих  параметров}  
ограничены:  
%% =============================== \\
\begingroup\belowdisplayskip=\belowdisplayshortskip  
\[
m \in  
M^{c}=[0; 1]  
\,{;}  
\eqno{(8)}  
\]
\endgroup  
%% =============================== *
%%%%%%%%%%%%%%%%%%%%%%%%%
%%% Page  2 / Page  3 %%%
%%%%%%%%%%%%%%%%%%%%%%%%%
\[
\tau_{w} \in
T^{c}_{\text{pr}}= [0{,}15; 0{,}9]
\,{,}  
\quad
T^{c}_{\text{pr}} 
\subset 
T^{c}_{\text{th}} 
= [0; 1]
\,{;}  
\eqno{(9)}  
\]
%% =============================== *
\[
s  \in  
S^{c} =  
[0 ; 
5  
\cdot
10^4]  \,  
[\mbox{Тл}/(\mbox{Ом}  
\cdot
\mbox{м})]  
\,{.}  
\eqno{(10)}  
\]
%% =============================== //
Далее  
индекс  
``$w$''
параметра  
$\tau_{w}$  
и
размерность  
$[\mbox{Тл}/(\mbox{Ом}\cdot\mbox{м})]$  
параметра  
$s$  
опущены. 
%%%%%%%%%%
Обозначим  
%% =============================== \\
\begingroup\belowdisplayskip=\belowdisplayshortskip  
\[
M^{d}_{05}  
=
\{0; 0{,}05; \ldots; 1 \}  
\subset  
M^{c}  
{;}
\eqno{(11)}  
\]
\endgroup  
%% =============================== *
\[
M^{d}_{25}  
=  
\{0; 0{,}25; 0{,}5; 0{,}75; 1 \}  
\subset  
M^{d}_{05}  
\,{;}  
\eqno{(12)}  
\]
%% =============================== *
\[
M^{d}_{25'}=
\{0{,}25; 0{,}5; 0{,}75\}  
\subset 
M^{d}_{25}
\,{;}
\eqno{(13)}
\]
%% =============================== //
%% =============================== \\
\[
T^{d}_{05,\text{th}}
\,{=}\,
\{0; 0{,}05; \ldots; 0{,}95; 1 \} 
\subset 
T^{c}_{\text{th}}
\,{;}
\eqno{(14)}
\]
%% =============================== *
\[
T^{d}_{05,\text{pr}}
\,{=}\,
\{0{,}15; 0{,}2; \ldots; 0{,}9 \}
= 
T^{d}_{05,\text{th}}
\cap
T^{c}_{\text{pr}}
\,{;}
\eqno{(15)}
\]
%% =============================== *
\[
T^{d}_{15}
\,{=}\,
\{0{,}15; 0{,}3; \ldots;  0{,}9 \}
\,{\subset}\,
T^{d}_{05,\text{pr}}
\,{;}
\eqno{(16)}
\]
%% =============================== *
\[
T^{d}_{15'}  
=  
\{0{,}15; 0{,}45; 0{,}6 \}  
\subset  
T^{d}_{15}  
\,{;}  
\eqno{(17)}  
\]
%% =============================== //
%% =============================== \\
\[
S^{d}_{25}  
=  
\{0; 
2{,}5  \cdot  10^4; 
5  \cdot  10^4 \}  
\subset  
S^{c}  
{.}  
\eqno{(18)}  
\]
%% =============================== //



     \textbf{Линейный}  
вдув  $m(x)$,  
определяемый 
законом
%%% ====
(8)
%%% ====
[8]
%% =============================== \\
\begingroup\belowdisplayskip=\belowdisplayshortskip  
\[
m(x)=  
m(x;m_{0},m_{1})
= 
m_{0}\,{\cdot}\,(1-x)+  
m_{1}\,{\cdot}\, x  
=  
{}  
\]
\endgroup  
%% =============================== *  
\[
{}
=
m_{0}+
m^{\prime}\,{\cdot}\, x
\,{,}  
\quad
\mbox{где}  
\quad
m_{0\,}, m_{1} \,{\in}\,  M^{c},
\quad
m^{\prime} = m_{1}-m_{0}
\,{,}  
\eqno{(19)}  
\]
%% =============================== //
назовём  
%%% ====
[9]
%% =============================== 
\textit{возрастающим} 
(для  $m^\prime  >  0$)
или 
\textit{убывающим} 
(для  $m^\prime  <  0$) 
\textit{слабо}  при 
$\left \vert  m^{\prime}\right \vert  
\in  (0; 0{,}3)$,
\textit{умеренно}  при 
$\left \vert  m^{\prime}\right \vert  
\in  [0{,}3; 0{,}7)$,
\textit{сильно}  при 
$\left \vert  m^{\prime}\right \vert  
\in  [0{,}7; 1]$.



    \textbf{Линейный}  
температурный  фактор  $\tau(x)$,  
определяемый 
законом
%%% ====
(8)
%%% ====
[10]
%% =============================== \\
\begingroup\belowdisplayskip=\belowdisplayshortskip  
\[
\tau(x)=  
\tau(x;\tau_{0},\tau_{1})
=  
\tau_{0}\,{\cdot}\,(1-x)+  
\tau_{1}\,{\cdot}\, x  
=
{}  
\]
\endgroup  
%% =============================== *  
\[
{}
= 
\tau_{0}+
\tau^{\prime}\,{\cdot}\, x
\,{,}  
\quad
\mbox{где}  
\quad
\tau_{0\,}, \tau_{1} \,{\in}\,  T^{c},
\quad
\tau^{\prime}  = \tau_{1}-\tau_{0}
\,{,}  
\eqno{(20)}  
\]
%% =============================== //
назовём  
%%% ====
[9]
%% =============================== 
\textit{возрастающим} 
(для  $\tau^{\prime}  >  0$)
или 
\textit{убывающим} 
(для  $\tau^{\prime}  <  0$) 
\textit{слабо} при 
$\left\vert \tau^{\prime}\right\vert  
\in  (0;0{,}25)$,
\textit{умеренно} при 
$\left\vert \tau^{\prime}\right\vert  
\in  [0{,}25;0{,}5)$,
\textit{сильно} при 
$\left\vert \tau^{\prime}\right\vert  
\in  [0{,}5;0{,}75]$.
%%%%%%%%%%%%%%%%%%%%%%%%%
%%% Page  3 / Page  4 %%%
%%%%%%%%%%%%%%%%%%%%%%%%%



На  
рис.~1{\textbf{--}}6
представлены  
(символ  ``$\diamond$'')
результаты  
вычислительных  экспериментов  
(для  удобства  сравнения  с  
%%% ===
[3{\textbf{--}}7]
%%% ===
выполненных  для  воздуха  
в  атмосфере  Земли 
при 
%% =============================== \
$H =  10~[\mbox{км}]$, 
%% ===============================
$M_{\infty}  =  10$, 
%% ===============================
$R  =  0{,}1~[\mbox{м}]$), 
%% =============================== /
для  случаев  
%%% ===
[9,  
 11]
%%% ======================== \\
\begingroup\belowdisplayskip=\belowdisplayshortskip  
\[
(m^{\prime}\,{=}\,{}+0{,}25;
\tau^{\prime}\,{=}\,{}+ 0{,}15), 
\quad
(m^{\prime}\,{=}\,{}+0{,}25;
\tau^{\prime}\,{=}\,{}- 0{,}15),
\]
\endgroup  
%%% ======================== **
\[
(m^{\prime}\,{=}\,{}+0{,}25;
\tau^{\prime}\,{=}\,{}+ 0{,}45), 
\quad
(m^{\prime}\,{=}\,{}+0{,}25;
\tau^{\prime}\,{=}\,{}- 0{,}45),
\]
%%% ======================== **
\[
(m^{\prime}\,{=}\,{}+0{,}50;
\tau^{\prime}\,{=}\,{}+ 0{,}15), 
\quad
(m^{\prime}\,{=}\,{}+0{,}50;
\tau^{\prime}\,{=}\,{}- 0{,}15)
\hphantom{,}
\]
%%% ======================== //
управлений  
%%% ===
(19),   (20)
%%% ===
при
%%% ============
$m_{0\,},  
m_{1} \,{\in}\,  M^{d}_{25\,}$,
$\tau_{0\,}, 
\tau_{1} \,{\in}\,  T^{d}_{15\,}$,
$s\equiv  0$.
%%% ============



%% =============================== \\
%\begin{figure}[h!]
\begin{center}
\includegraphics[width=0.95\textwidth]%
{%
Bilchenko_GGjr_NG_1_fig_1_p25p15.eps%
}\\
\emph{Рис.~1.}~%
{Случай  
$m^{\prime}\,{=}\,{}+0{,}25$,  
$\tau^{\prime}\,{=}\,{}+0{,}15$,
$s\equiv  0$%
}%  
\end{center}
%\end{figure}
%% =============================== **



%% =============================== **
%\begin{figure}[h!]
\begin{center}
\includegraphics[width=0.95\textwidth]%
{%
Bilchenko_GGjr_NG_1_fig_2_p25n15.eps%
}\\
\emph{Рис.~2.}~%
{Случай  
$m^{\prime}\,{=}\,{}+0{,}25$,  
$\tau^{\prime}\,{=}\,{}-0{,}15$,
$s\equiv  0$%
}%  
\end{center}
%\end{figure}
%% =============================== //



Сопоставляя  
%%% ===
[8,  
 10]
%%% ===
элементам  
множеств 
$M^{d}_{05}$  
и  
$T^{d}_{05,\text{th}}$  
буквы  латинского  алфавита 
от  ``a''  до  ``u'',
%%%
элементам  
$T^{d}_{05,\text{pr}}$~{\textbf{--}}  
буквы  
от  ``d''  до  ``s'',
%%%
отметим  
положение  
пар  $(Q,F)$  
для  
некоторых  сочетаний  
управлений.  
%%%
Например, 
``afdg''  
%%%%%%%%%%%%%%%%%%%%%%%%%
%%% Page  4 / Page  5 %%%
%%%%%%%%%%%%%%%%%%%%%%%%%
соответствует
%%% ============ \\
($m_{0}=0{,}0$;  
$m_{1}=0{,}25$;
$\tau_{0}=0{,}15$; 
$\tau_{1}=0{,}3$).
%%% ============ //
На  
%%% ===
рис.~1{\textbf{--}}6
%%% ===
представлены  
образы  линий
двух  семейств  
%%% ===
[7]
%%% ===
%% =============================== \\
\begingroup\belowdisplayskip=\belowdisplayshortskip
\[
M \,{=}\, 
  \bigl\{ \! 
      \bigl\{
          \left(  
              m, \tau, s
            \right)  
          \bigl\vert\bigr.  {\hspace{1pt}}
          m\,{=}\,Const \,{\in}\, M^{c}
        \bigr\} 
      \bigl\vert\bigr.  {\hspace{1pt}}  
      \tau\,{=}\,Const  
        \,{\in}\,  T^{d}_{05\,}, 
      {\hspace{1pt}}  
      s \,{\equiv}\, 0 
    \bigr\}
{,}
\]
\endgroup
%% =============================== **
\[
T \,{=}\, 
  \bigl\{ \! 
      \bigl\{
          \left(  
              m, \tau,  s
            \right)  
          \bigl\vert\bigr.  {\hspace{1pt}}  
          \tau\,{=}\,Const \,{\in}\, T^{c}
        \bigr\}
      \bigl\vert\bigr.  {\hspace{1pt}} 
      m\,{=}\,Const 
        \,{\in}\,  M^{d}_{05\,},
      {\hspace{1pt}}  
      s \,{\equiv}\, 0 
    \bigr\}
{.}
\]
%% =============================== //



%% =============================== \\
%\begin{figure}[h!]
\begin{center}
\includegraphics[width=0.95\textwidth]%
{%
Bilchenko_GGjr_NG_1_fig_3_p25p45.eps%
}\\
\emph{Рис.~3.}~%
{Случай  
$m^{\prime}\,{=}\,{}+0{,}25$,  
$\tau^{\prime}\,{=}\,{}+0{,}45$,
$s\equiv  0$%
}%  
\end{center}
%\end{figure}
%% =============================== **



%% =============================== **
%\begin{figure}[h!]
\begin{center}
\includegraphics[width=0.95\textwidth]%
{%
Bilchenko_GGjr_NG_1_fig_4_p25n45.eps%
}\\
\emph{Рис.~4.}~%
{Случай  
$m^{\prime}\,{=}\,{}+0{,}25$,  
$\tau^{\prime}\,{=}\,{}-0{,}45$,
$s\equiv  0$%
}%  
\end{center}
%\end{figure}
%% =============================== //



%% ==============================
%% ==============================
\textbf{3.  Замечания.} 
%% ==============================
%% ==============================
1)  Исследование  влияния  
другого  сочетания
управляющих  воздействий:
\textbf{линейно  убывающего}
вдува,  
\textbf{линейного}
температурного  фактора  
и  
\textbf{постоянного} 
магнитного  поля  
на  
интегральные 
характеристики  
тепломассообмена  и  трения
и  
суммарную  мощность  
системы,  
обеспечивающей  вдув,
проводится  в  работе  
%%% ===
[12].    
%%% ===
%%%%%%%%%%%%%%%%%%%%%%%%%
%%% Page  5 / Page  6 %%%
%%%%%%%%%%%%%%%%%%%%%%%%%



2)  Графики  
параметров
%%%% ================ \\
$\theta_{0}\left(x\right)%
, 
\ldots, 
\omega_{1}\left(x\right)%
$
%%%% ================ //
%%% Overfull start ===\\
математичес\-кой  модели  ЛПС 
%%% Overfull finish ===//
и
локальных  зависимостей  
$q(x)$
и 
$f(x)$
представлены  в  
%%% ===
[9,  
 11].
%%% ===



%% =============================== \\
%\begin{figure}[h!]
\begin{center}
\includegraphics[width=0.95\textwidth]%
{%
Bilchenko_GGjr_NG_1_fig_5_p50p15.eps%
}\\
\emph{Рис.~5.}~%
{Случай  
$m^{\prime}\,{=}\,{}+0{,}50$,  
$\tau^{\prime}\,{=}\,{}+0{,}15$,
$s\equiv  0$%
}%  
\end{center}
%\end{figure}
%% =============================== **



%% =============================== **
%\begin{figure}[h!]
\begin{center}
\includegraphics[width=0.95\textwidth]%
{%
Bilchenko_GGjr_NG_1_fig_6_p50n15.eps%
}\\
\emph{Рис.~6.}~%
{Случай  
$m^{\prime}\,{=}\,{}+0{,}50$,  
$\tau^{\prime}\,{=}\,{}-0{,}15$,
$s\equiv  0$%
}%  
\end{center}
%\end{figure}
%% =============================== //



%% =============================== 
% Оформление списка литературы
\smallskip \centerline {\bf Литература} \nopagebreak
%% =============================== 



%% ============================== \\
1.~%
\textit% 
{Дородницын~А.~А.~} 
{%
 {Об  одном  методе  решения  
 уравнений  ламинарного  пограничного  слоя}%
%	~/  
%	   {А.~А.~Дородницын}%
~/$\!$/ 
 {Прикладная  математика  
  и  техническая  физика}.~{\textbf{---}} 
  1960.~{\textbf{---}} 
  \No~3.~{\textbf{---}} 
  С.~111{\textbf{--}}118.%
  }  
%% ============================== // 



%% ============================== \\
2.~%
\textit% 
{Бильченко~Н.~Г.~} 
{% 
  {Метод  А.~А.~Дородницына  
  в  задачах  оптимального  управления 
  тепломассообменом  на  проницаемых  поверхностях 
  в  ламинарном  пограничном  слое  
  электропроводящего  газа}%
%	~/
%	   {Н.~Г.~Бильченко}%
~/$\!$/ 
  Вестник  Воронеж.  гос.  ун-та. 
  Сер.  Системный  анализ  
  и  информационные  технологии.~{\textbf{---}} 
  2016.~{\textbf{---}} 
  \No~1.~{\textbf{---}} 
  С.~5{\textbf{--}}14.%
  }
%% ============================== // 
%%%%%%%%%%%%%%%%%%%%%%%%%
%%% Page  6 / Page  7 %%%
%%%%%%%%%%%%%%%%%%%%%%%%%



%% ============================== \\
3.~% 
\textit% 
{Бильченко~Н.~Г.~} 
{%
 {Вычислительные  
  эксперименты 
  в  
  задачах  
  оптимального  
  управления  
  тепломассообменом 
  на  проницаемых  поверхностях 
  при  гиперзвуковых  режимах  полёта}%
%	~/ 
%	   {Н.~Г.~Бильченко}%
~/$\!$/ 
  Вестник  Воронеж.  гос.  ун-та. 
  Сер.  Физика.  Математика.~{\textbf{---}}
  2015.~{\textbf{---}} 
  \No~1.~{\textbf{---}} 
  С.~83{\textbf{--}}94.%
  }
%% ============================== // 



%% ============================== \\
4.~%
\textit% 
{Бильченко~Н.~Г.~}  
{%
  {Вычислительные  
   эксперименты  
   в  
   задачах  
   оптимального  
   управления 
   тепломассообменом  
   на  проницаемых  поверхностях 
   тел  вращения  
   при  гиперзвуковых  режимах  полёта}%
%	~/ 
%	   {Н.~Г.~Бильченко}%
~/$\!$/ 
  Вестник  Воронеж.  гос.  ун-та. 
  Сер.  Системный  анализ 
  и  информационные  технологии.~{\textbf{---}} 
  2015.~{\textbf{---}} 
  \No~1.~{\textbf{---}} 
  С.~5{\textbf{--}}8.%
  }
%% ============================== // 

	

%% ============================== \\
5.~%
\textit% 
{Бильченко~Н.~Г.~} 
{%
	{Вычислительные  
	эксперименты 
	в  
	задачах  
	оптимального  
	управления 
	тепломассообменом 
	на  проницаемых  поверхностях 
	при  гиперзвуковых  
	режимах  
	полёта:    сравнительный 
	анализ  применения 
	``простых''  законов  вдува}%
%	~/ 
%	   {Н.~Г.~Бильченко}%
~/$\!$/
   Вестник  Воронеж  гос.  ун-та. 
	Сер.  Физика.  Математика.~{\textbf{---}} 
	2015.~{\textbf{---}} 
	\No~1.~{\textbf{---}}
	С.~95{\textbf{--}}102.%
	}
%% ============================== // 



%% ============================== \\
6.~%
\textit% 
{Бильченко~Н.~Г.~}  
{%
  {Вычислительные  
   эксперименты  
   в  
   задачах  
   оптимального  
   управления  тепломассообменом 
   на  
%%% Overfull start ===\\
   проницаемых\,  поверхностях\,   
   в\,  ламинарном\,   пограничном\,   
   слое 
%%% Overfull finish ===//
   электропроводящего  газа}%
%	~/
%	   {Н.~Г.~Бильченко}%
~/$\!$/ 
  Вестник  Воронеж.  гос.  ун-та. 
  Сер.  Системный  анализ 
  и  информационные  технологии.~{\textbf{---}} 
  2016.~{\textbf{---}} 
  \No~3.~{\textbf{---}} 
  С.~5{\textbf{--}}11.%
  }
%% ============================== // 



%% ============================== \\
7.~%
\textit% 
{Бильченко~Г.~Г.,  
 Бильченко~Н.~Г.~}  
{%
  {Анализ  влияния  
%%% Overfull start ===\\
  постоянных  
  управляющих  
  воздействий  
  на  
  область  
  значений  
  функционалов  
  гиперзвуковой  
%%% Overfull finish ===//
  аэродинамики}%  
%	~/  
%	  {Г.~Г.~Бильченко,  
%	   Н.~Г.~Бильченко}%  
~/$\!$/  
  Вестник  Воронеж.  гос.  ун-та.  
  Сер.  Системный  анализ  
  и  информационные  технологии.~{\textbf{---}}  
  2018.~{\textbf{---}}  
  \No~2.~{\textbf{---}}  
  С.~5{\textbf{--}}13.%
  }  
%% ============================== // 



%% ============================== \\
8.~%
\textit% 
%% ============= 
{Бильченко~Г.~Г.,  
 Бильченко~Н.~Г.~}  
{%  
  {Анализ  влияния  
линейного  вдува
и  постоянного  температурного  фактора  
на  параметры  математической  модели  
и
локальные  характеристики  тепломассообмена  
и  трения 
на  проницаемых  поверхностях  ГЛА}%
%	~/
%	  {Г.\,Г.\,Биль\-чен\-ко,  
%	   Н.\,Г.\,Биль\-чен\-ко}%
~/$\!$/ 
  Вестник  Воронеж.  гос.  ун-та. 
  Сер.  Системный  анализ  
  и  информационные  технологии.~{\textbf{---}} 
  2019.~{\textbf{---}}  
  \No~2.~{\textbf{---}}  
  С.~5{\textbf{--}}14.%
  }  
%% ============================== // 
%%%%%%%%%%%%%%%%%%%%%%%%%
%%% Page  7 / Page  8 %%%
%%%%%%%%%%%%%%%%%%%%%%%%%



%% ============================== \\
9.~%
\textit% 
{Бильченко~Г.~Г.,  
 Бильченко~Н.~Г.~}  
{%
  {Анализ  влияния  
линейно  возрастающего 
вдува  
и  
линейно  возрастающего 
температурного  фактора  
на  параметры  математической  модели  и
локальные  характеристики  тепломассообмена  
и  трения 
на  проницаемых  поверхностях  ГЛА}%  
%	~/  
%	  {Г.\,Г.\,Биль\-чен\-ко,  
%	   Н.\,Г.\,Биль\-чен\-ко}%  
~/$\!$/  
  Вестник  Воронеж.  гос.  ун-та. 
  Сер.  Системный  анализ  
  и  информационные  технологии.~{\textbf{---}} 
  2019.~{\textbf{---}} 
  \No~3.~{\textbf{---}} 
  С.~53{\textbf{--}}62.%
  }
%% ============================== 



%% ============================== \\
10.~%
\textit%
%% ============= 
{Бильченко~Г.~Г.,  
 Бильченко~Н.~Г.~}  
{%
  {Ана\-лиз  влияния  
линейного  температурного  фактора  
и  постоянного  вдува
на  параметры  математической  модели  и
локальные  характеристики  тепломассообмена  
и  трения 
на  проницаемых  поверхностях  ГЛА}%
%	~/
%	  {Г.\,Г.\,Биль\-чен\-ко,  
%	   Н.\,Г.\,Биль\-чен\-ко}%
~/$\!$/ 
  Вестник  Воронеж.  гос.  ун-та. 
  Сер.  Системный  анализ  
  и  информационные  технологии.~{\textbf{---}} 
  2019.~{\textbf{---}}  
  \No~2.~{\textbf{---}}  
  С.~15{\textbf{--}}22.%
  }
%% ============================== // 



%% ============================== \\
11.~%
\textit% 
{Бильченко~Г.~Г.,  
 Бильченко~Н.~Г.~}  
{%
  {Анализ  влияния  
линейно  возрастающего 
вдува  
и  
линейно  убывающего 
температурного  фактора  
на  параметры  математической  модели  и
локальные  характеристики  тепломассообмена  
и  трения 
на  проницаемых  поверхностях  ГЛА}%  
%	~/  
%	  {Г.\,Г.\,Биль\-чен\-ко,  
%	   Н.\,Г.\,Биль\-чен\-ко}%  
~/$\!$/  
  Вестник  Воронеж.  гос.  ун-та. 
  Сер.  Системный  анализ  
  и  информационные  технологии.~{\textbf{---}} 
  2019.~{\textbf{---}} 
  \No~4.~{\textbf{---}} 
  С.~5{\textbf{--}}12.%
  }
%% ============================== 



%% ==============================
12.~%
\textit% 
{Бильченко~Г.~Г.,  
 Бильченко~Н.~Г.~}  
{%
  {О  влиянии  линейно  убывающего  вдува 
и  линейного  температурного  фактора 
на  область  значений  функционалов 
гиперзвуковой  аэродинамики}%
%	~/  
%	  {Г.\,Г.\,Биль\-чен\-ко,  
%	   Н.\,Г.\,Биль\-чен\-ко}%  
~/$\!$/  
  <<Воронежская  зимняя  математическая  школа 
  С.~Г.~Крейна~{\textbf{--}}  2020>>,  
  посвященная  100-летию  
  М.~А.~Красносельского: 
  Материалы  международной  конференции 
  (27{\textbf{--}}30  
  января  2020~г.).~{\textbf{---}} 
  Воронеж:  ИПЦ  <<Научная  книга>>, 
  2020.%~{\textbf{---}} 
  %С.~???{\textbf{--}}???.%
  } 
%% ==============================
%%%%%%%%%%%%%%%%%%%%%%%%%
%%% Page  8 / ...     %%%
%%%%%%%%%%%%%%%%%%%%%%%%%



%%%%%%%%%%%%%%%%%%%%%%%%%
