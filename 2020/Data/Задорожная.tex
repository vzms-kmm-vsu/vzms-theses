\vzmstitle{ОПРЕДЕЛЕНИЕ ГРАНИЦ РАСХОДА ЖИДКОСТИ ПРИ ОДНОМ ЧАСТНОМ СЛУЧАЕ КРАЕВЫХ УСЛОВИЙ}
\vzmsauthor{Задорожная}{Н.\,С.}
\vzmsinfo{Ростов-на-Дону; {\it simon@sfedu.ru}}
\vzmsauthor{Клодина}{Т.\,В.}
\vzmscaption

В связи с тем, что при решении задач фильтрации наиболее важным является отыскание характеристик потока,
которые зависят как от вида краевых условий, так и от формы границы области фильтрации,
возникает вопрос, как изменяются эти характеристики при изменении границы области.

Ответить на этот вопрос позволяет метод мажорантных областей. Его сущность заключается в том,
что путём деформации границы области фильтрации строятся две такие области, для которых поставленную
задачу можно решить точно. Полученные характеристики для этих областей дают заведомо верхние и нижние оценки искомых.
В работе Г.Н. Положего [1] наиболее полно изложены теоремы для различных случаев границ области фильтрации.

 Однако, на практике встречаются задачи теории фильтрации, когда область фильтрации ограничена
 одной линией тока и тремя потенциальными линиями. Задачи такого рода представляют существенную
 трудность вследствие того, что в этом случае вид годографа комплексного потенциала может быть
 заранее неизвестен.

 Предлагается определение верхней и нижней границ расхода жидкости в области фильтрации,
 ограниченной тремя потенциальными линиями AB, AD, CD и одной линией тока BC.

 Имеем плоскую или осесимметричную задачу стационарной фильтрации

$$
  div \overline V=0, \overline V=-\kappa_0 \, grad \, h,
  \eqno (1)
$$
( $\overline V=(V_x,V_y)$ - скорость фильтрации, $h(x,y)$ -
напорная функция, $\kappa_0(x,y)$ - коэффициент фильтрации).

Краевые условия имеют вид
$$
    \varphi |_{AB}=-\kappa_0 H, \varphi
    |_{CD}=0, \varphi |_{AD}=-\kappa_0 H_{0},\phi |_{BC}=0, \eqno (2)
$$
где $\varphi (x,y)$ - потенциальная функция, $\phi (x,y)$ - линия тока,
$H$ и $H_{0}$ - величины напоров на границе области фильтрации, когда
$H_0>H>0$. Полагаем, что коэффициент фильтрации $k_0(x,y)$ вместе со
своими частными производными является правильно непрерывной функцией в области
фильтрации плоскости $z$.

Авторы предлагают новую теорему об изменении расхода при решении задачи, когда варьируется
область фильтрации, ограниченная тремя потенциальными линиями и одной линией тока, при условии
$H_0>H>0$.

\textbf{Теорема.} {\it При вдавливании линии тока расход жидкости через всякую
связную часть потенциальной линии, имеющей общий конец с линией тока, уменьшается.
При выдавливании линии тока расход увеличивается.}

Искомое решение находим как среднее арифметическое найденных оценок.

Отметим, что аналогичная теорема при условии $H>H_0$ изложена в работе [2].

% Оформление списка литературы
\smallskip \centerline {\bf Литература} \nopagebreak

1. {\it Положий Г. Н.} Теория и применение $p$-аналитических
и $(p,g)$-аналитических функций. Киев : Наукова думка, 1973. 423 с.

2. {\it Клодина Т.В., Задорожная Н.С.} Теорема об изменении расхода жидкости при решении
одной задачи стационарной фильтрации. Сборник материалов Международной конференции,
посвящённой 100-летию со дня рождения С.Г. Крейна. Воронеж: Издательский дом ВГУ, 2017. — С. 116-117.
